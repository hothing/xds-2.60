%
% def.tex - common settings and useful definitions for XDS manuals
%

%%% Page Setup %%%%%%%%%%%%%%%%%%%%%%%%%%%%%%%%%%%%%%%%%%%%%%%%%%%%%%%%%

\sloppy                             % Do not cross right margin
\raggedbottom                       % No vertical justification

%\setlength{\oddsidemargin}{0 pt}
%\addtolength{\topmargin}{-36 pt}    % Top margin is 0.5"
%\addtolength{\textheight}{36 pt}
%\addtolength{\textwidth}{72 pt}     % Width is 1 inch more than default

%%% Paragraph style %%%%%%%%%%%%%%%%%%%%%%%%%%%%%%%%%%%%%%%%%%%%%%%%%%%%

\setlength{\parindent}{0 pt}
\setlength{\parskip}{6 pt}

%%% Conditional compilation identifiers %%%%%%%%%%%%%%%%%%%%%%%%%%%%%%%%

\newif\ifcomment\commentfalse % never set to true; use for commenting

\newif\ifcommon\commonfalse   % Common document for all editions


%-- Available front-ends -----------------------------------------------

% default is M2/O2
\newif\iftopspeed       % TopSpeed extensions
\newif\ifjava           % Java

\input{../frontend.tex} % should select one code generator

%-- Available code generators ------------------------------------------

\newif\ifgenc           % C code generation
\newif\ifgencode        % native code generation
\newif\ifgenintel       % Intel x86 code generation
\newif\ifgenvax         % VAX code generation
\newif\ifgendfa         % data flow analysis
\newif\ifgenefd         % Excelsior FlawDetector
\newif\ifgeneit         % Excelsior Installation Toolkit

\input{../backend.tex}  % should select one code generator

\ifgenc\else\ifgendfa\else\ifgenefd\else\ifgeneit\else\gencodetrue\fi\fi\fi\fi

%-- Available platforms ------------------------------------------------

\newif\ifmsdos          % MS-DOS
\newif\ifosii           % OS/2
\newif\ifwinnt          % WinNT & Win95
\newif\iflinux          % Linux
\newif\ifmacosx         % Mac OS X
\newif\ifunix           % Unix

\input{../platform.tex} % should select one platform

\iflinux \unixtrue \fi  % Linux is Unix
\ifmacosx \unixtrue \fi % Mac OS X is Unix

\ifjava
  \ifgencode
    \newcommand{\ProductName}{\mbox{Excelsior JET}}
  \else\ifgenefd
    \newcommand{\ProductName}{\mbox{Excelsior FlawDetector}}
  \else\ifgeneit
    \newcommand{\ProductName}{\mbox{Excelsior Installer}}
  \else
    \JNO
  \fi\fi\fi
\else
  \ifgenc
    \newcommand{\ProductName}{\mbox{XDS-C}}
  \else\ifgencode
    \ifgenintel
      \newcommand{\ProductName}{\mbox{Native XDS-x86}}
    \else\ifgenvax
      \newcommand{\ProductName}{\mbox{Native XDS-VAX}}
    \else
      \newcommand{\ProductName}{\mbox{Native XDS}}
    \fi\fi
  \else
    \JNO
  \fi\fi
\fi


\newif\ifthreads \threadsfalse
\newif\ifdll     \dllfalse
\newif\ifxlink   \xlinkfalse
\newif\ifprofile \profilefalse
\newif\ifxlib    \xlibfalse
\ifgencode
  \ifosii
    \threadstrue\dlltrue\xlinktrue\profiletrue\xlibtrue
  \else\ifwinnt
    \threadstrue\dlltrue\xlinktrue\profiletrue\xlibtrue
  \else\iflinux
  \else\ifmacosx
  \else         \JNO
  \fi\fi\fi\fi
\fi



\newcommand{\BNFsize}{\small}

\newcommand{\XDS}{XDS}
\newcommand{\xds}{XDS}
\newcommand{\Exc}{Excelsior}

%%% File names and extensions %%%%%%%%%%%%%%%%%%%%%%%%%%%%%%%%%%%%%%%%%%

\ifjava
  \newcommand{\JET}{JET}
  \newcommand{\JP}{JetPackII}
  \newcommand{\DOD}{Java Runtime Slim-Down}
  \newcommand{\EFD}{Excelsior FlawDetector}
  \newcommand{\FD}{FlawDetector}
  \newcommand{\EP}{Packager}
  \newcommand{\EI}{Excelsior Installer}
  \newcommand{\efd}{efd}
  \newcommand{\XC}{jc}
  \newcommand{\JC}{jc}
  \newcommand{\jc}{jc}
\else
  \ifgencode
    \ifgenvax
      \newcommand{\XC}{xv}
    \else
      \newcommand{\XC}{xc}
    \fi
  \else
    \newcommand{\XC}{xm}
  \fi
\fi

%!!! warning
% \index of the form \index{xm<..>@{\XC{}<...>}}
% is used in config.tex & usage.tex
% it may lead to wrong order of items in the index.
% I don't know another way to use macro in the index.

\newcommand{\xc}{{\tt \XC{}}}
\newcommand{\red}{\XC{}.red}
\newcommand{\msg}{\XC{}.msg}
\newcommand{\cfg}{\XC{}.cfg}

\ifgenc
\newcommand{\kwd}{\XC{}.kwd}
\fi

\ifgenc
  \newcommand{\Sym}{.sym}
  \newcommand{\Code}{.c}
  \newcommand{\Header}{.h}
\fi

\ifgencode
  \newcommand{\Sym}{.sym}
  \newcommand{\Bod}{.bod}
  \ifunix
    \newcommand{\Code}{.o}
  \else
    \newcommand{\Code}{.obj}
  \fi
\fi

\ifunix
  \newcommand{\EXE}{executable}
  \newcommand{\DLL}{shared library}
  \newcommand{\EXEs}{executables}
  \newcommand{\DLLs}{shared libraries}
  \ifjava
    \newcommand{\dotObj}{.obj}   % JET uses .obj even on Unix
  \else
    \newcommand{\dotObj}{.o}
  \fi
  \newcommand{\dotDll}{.so}
  \newcommand{\dotExe}{}
\else\ifwinnt
  \newcommand{\EXE}{EXE}
  \newcommand{\DLL}{DLL}
  \newcommand{\EXEs}{EXEs}
  \newcommand{\DLLs}{DLLs}
  \newcommand{\dotObj}{.obj}
  \newcommand{\dotDll}{.dll}
  \newcommand{\dotExe}{.exe}
\else\JNO\fi\fi
\ifunix
  \newcommand{\DirSep}{/}
\else
  \newcommand{\DirSep}{$\backslash$}
\fi

\newcommand{\WinUnix}[2]{\ifwinnt #1\else\ifunix #2\else\JNO\fi\fi{}}

%----------- useful commands -----------%

\newcommand{\mt}{\mbox{Modula-2}}
\newcommand{\ot}{\mbox{Oberon-2}}

\ifgenvax
\newcommand{\Seealso}{\begin{flushleft} \bf Смотри также\end{flushleft} }
\newcommand{\Example}{\paragraph{Пример}}
\newcommand{\Examples}{\paragraph{Примеры}}
\else
\newcommand{\Seealso}{\begin{flushleft} \bf See also \end{flushleft} }
\newcommand{\Example}{\paragraph{Example}}
\newcommand{\Examples}{\paragraph{Examples}}
\fi

\newcommand{\arrow}{\^{}}
\newcommand{\bs}{$\backslash$}

\ifonline\else
\newcommand{\drawing}[1]{\epsfbox{#1.eps}}
\fi


\newcommand{\maddonly}
{
\ifonline
  \ifgenvax
    {\footnotesize ЗАМЕЧАНИЕ: доступно только при включенной опции \OERef{M2ADDTYPES}.\par}
  \else
    {\footnotesize NOTE: Only valid when option \OERef{M2ADDTYPES} is set.\par}
  \fi
\else
  \begin{center}
  \begin{tabular}{|l|}
  \hline
  \ifgenvax
    \footnotesize ЗАМЕЧАНИЕ: доступно только при включенной опции \OERef{M2ADDTYPES}.\\
  \else
    \footnotesize NOTE: Only valid when option \OERef{M2ADDTYPES} is set.\\
  \fi
  \hline
  \end{tabular}
  \end{center}
\fi
}

\newcommand{\mextonly}
{
\ifonline
  \ifgenvax
    {\footnotesize ЗАМЕЧАНИЕ: доступно только при включенной опции \OERef{M2EXTENSIONS}.\par}
  \else
    {\footnotesize NOTE: Only valid when option \OERef{M2EXTENSIONS} is set.\par}
  \fi
\else
  \begin{center}
  \begin{tabular}{|l|}
  \hline
  \ifgenvax
    \footnotesize ЗАМЕЧАНИЕ: доступно только при включенной опции \OERef{M2EXTENSIONS}.\\
  \else
    \footnotesize NOTE: Only valid when option \OERef{M2EXTENSIONS} is set.\\
  \fi
  \hline
  \end{tabular}
  \end{center}
\fi
}

\newcommand{\oextonly}
{
\ifonline
  {\footnotesize NOTE: Only valid when option \OERef{O2EXTENSIONS} is set.\par}
\else
  \begin{center}
  \begin{tabular}{|l|}
  \hline
  \footnotesize NOTE: Only valid when option \OERef{O2EXTENSIONS} is set.\\
  \hline
  \end{tabular}
  \end{center}
\fi
}

\newcommand{\onumextonly}
{
\ifonline
  {\footnotesize NOTE: Only valid when option \OERef{O2NUMEXT} is set.\par}
\else
  \begin{center}
  \begin{tabular}{|l|}
  \hline
  \footnotesize NOTE: Only valid when option \OERef{O2NUMEXT} is set.\\
  \hline
  \end{tabular}
  \end{center}
\fi
}

\newcommand{\lindex}[1]{\protect\index{#1@{\bf #1}}} % !!! Has to be changed

% This command should be used to encapsulate page numbers in the index

\newcommand{\Encap}[1]{{\bf\hyperpage{#1}}}

% Modules

\newcommand{\ModuleI}{}
\newcommand{\ModuleII}{}
\newcommand{\ModuleIII}{}
\newcommand{\ModuleIV}{}
\newif\ifOneModule     % Generate label and index entry prefixed with \ModuleI
\newif\ifTwoModules    % Generate label and index entry prefixed with \ModuleII
\newif\ifFourModules   % Generate label and index entries prefixed with \ModuleIII and \ModuleIV

\newcommand{\NoModules}{
\OneModulefalse
\TwoModulesfalse
\FourModulesfalse
}
\newcommand{\OneModule}{
\OneModuletrue   \label{\ModuleI}   \protect\index{\ModuleI}
\TwoModulesfalse
\FourModulesfalse
}
\newcommand{\TwoModules}{
\OneModuletrue   \label{\ModuleI}   \protect\index{\ModuleI}
\TwoModulestrue  \label{\ModuleII}  \protect\index{\ModuleII}
\FourModulesfalse
}
\newcommand{\FourModules}{
\OneModuletrue   \label{\ModuleI}   \protect\index{\ModuleI}
\TwoModulestrue  \label{\ModuleII}  \protect\index{\ModuleII}
\FourModulestrue \label{\ModuleIII} \protect\index{\ModuleIII}
                 \label{\ModuleIV}  \protect\index{\ModuleIV}
}

\NoModules

% Framed heading

\newif\ifFrameIsDeep       % Frame is subsubsection when online
\newif\ifFrameToContents   % Frames are added to contents

\newcommand{\Frame}[2]{
  \ifonline
    \public{#1}
    \ifFrameIsDeep
      \subsubsection{#1 - #2}
    \else
      \subsection{#1 - #2}
    \fi
  \else
    \bigskip
    \noindent
    \framebox[\textwidth]{\large {\bf #1} \hfill {\sl #2}}
    \ifFrameToContents
      \ifFrameIsDeep
        \addcontentsline{toc}{subsubsection}{#1 - #2}
      \else
        \addcontentsline{toc}{subsection}{#1 - #2}
      \fi
    \fi
  \fi
  \protect\index{#1}
  \ifOneModule
  \label{\ModuleI.#1}
  \protect\index{\ModuleI!#1}
  \fi
  \ifTwoModules
  \label{\ModuleII.#1}
  \protect\index{\ModuleII!#1}
  \fi
  \ifFourModules
  \label{\ModuleIII.#1} \label{\ModuleIV.#1}
  \protect\index{\ModuleIII!#1} \protect\index{\ModuleIV.#1}
  \fi
}

% Module list

\newcommand{\ModuleList}
{\ifonline
\ifTwoModules Modules: \else Module \fi
\IdRef{\ModuleI}\ifTwoModules ,
\IdRef{\ModuleII}\fi\ifFourModules ,
\IdRef{\ModuleIII}, \IdRef{\ModuleIV} \fi
\par
\fi}

% Reference to an id
\newcommand{\IdRef}[1]{{\tt \ifonline \ref[#1]{#1}\else #1\fi}}

% Reference to an unqualified id with qualified label
\newcommand{\IdQualRef}[1]{{\tt \ifonline \ref[#1]{\ModuleI.#1}\else #1\fi}}

% Options/Equations headings

\newcommand{\OptHead}[2]{
  \ifonline
    \subsection{#1 - #2}
    \label{#1}
    \protect\index{#1}
    \protect\index{options!#1}
  \else
    \item[#1]
    \protect\index{#1@{\bf #1}|Encap}
    \protect\index{options!#1@{\bf #1}|Encap}
    \mbox{}\nopagebreak
  \fi
}

\newcommand{\EquHead}[2]{
  \ifonline
    \subsection{#1 - #2}
    \label{#1}
    \protect\index{#1}
    \protect\index{equations!#1}
  \else
    \item[#1]
    \protect\index{#1@{\bf #1}|Encap}
    \protect\index{equations!#1@{\bf #1}|Encap}
    \mbox{}\nopagebreak
  \fi
}

\newcommand{\LinkOpt}[2]{
  \ifonline
    \subsection{/#1 - #2}
    \public{#1}                % ???
    \label{link:options:#1}
    \protect\index{/#1}
    \protect\index{linker options!/#1}
  \else
%    \addcontentsline{toc}{subsection}{\protect\makebox[0.5cm]{}/#1 - #2}
    \protect\index{#1@{\bf /#1}, linker option|Encap}
    \protect\index{linker options!#1@{\bf /#1}|Encap}
    \protect\item[{\large \bf /#1}] \mbox{} \\
  \fi
}

% Reference to an option/equation
\newcommand{\OERef}[1]{\ifonline\ref[#1]{#1}\else{\bf #1}\protect\index{#1@{\bf #1}}\fi}
\newcommand{\OEName}[1]{\ifonline #1\else{\bf #1}\protect\index{#1@{\bf #1}}\fi}

\newcommand{\LORef}[1]{\ifonline\ref[/#1]{link:options:#1}\else
{\bf /#1}\protect\index{#1@{\bf /#1}, linker option}
\fi}

\newcommand{\Ref}[2]{\ifonline\ref[#1]{#2}\else#1\fi}
\newcommand{\See}[3]{\ifonline\ref[#1]{#3}\else#1 (see #2\protect\ref{#3})\fi}

\newcommand{\havetowrite}{{\bf Have to write something here.}}

\newcommand{\EXAMPLE}[2]{
\ifonline
  \par \ref{#2} \par
\else
  \paragraph{Example - #1}
\fi
}

\ifonline\else
  \newcommand{\public}[1]{\label{#1}}
\fi

\newcommand{\Picture}[3]{
\ifonline
  \includegraphics[#2]{#3}
\else
  \includegraphics[#1]{#3.pdf}
\fi
}