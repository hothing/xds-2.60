\chapter{Expressions}
\label{expr}

XD accepts expressions in various places. For instance, an expression
may be added to the Watches window, used in a conditional breakpoint,
and even entered as an initial counter value.

The expression syntax in XD closely matches \mt{} syntax, with the 
following additions and exceptions:
\begin{itemize}
\item the qualified identifier syntax is expanded to allow referencing of
      entities from scopes and components other than current 
      (see \ref{expr:entities}).
\item expressions may not contain debuggee function calls; only
      debugger \See{built-in functions}{}{expr:builtin} may be used.
\item the construct:

\verb'    Type "<" Expression ">"'

      designates an object of type \verb'Type' residing at address
      to which \verb'Expression' evaluates (see \ref{expr:casting}).
\end{itemize}

Type compatibility rules are much more relaxed in comparison
with Modula-2. Also, implicit type conversions are performed
under certain circumstances.

Set and complex types are not currently supported.

A {\it designator}\index{designator} is an expression which denotes a memory
object (in other words, represents an lvalue). In some places XD expects
a designator to be used.

\section{Constants}

\subsection{Whole constants}

\begin{verbatim}
    WholeConst   = DecimalConst | HexConst
    DecimalConst = Digit { Digit }
    HexConst     = Digit { HexDigit } "H" |
                   "0" ("x"|"X") HexDigit { HexDigit }
    Digit    = "0" | "1" | "2" | "3" | "4" |
               "5" | "6" | "7" | ""8 | "9"
    HexDigit = Digit |
               "a" | "b" | "c" | "d" | "e" | "f" |
               "A" | "B" | "C" | "D" | "E" | "F"
\end{verbatim}
A decimal constant is a sequence of decimal digits \verb"'0'..'9'".
A hexadecimal constant is a secuence of hexadecimal digits \verb"'0'..'9'",
\verb"'a'..'f'", \verb"'A'..'F'", either prefixed with "\verb'0x'"
or "\verb'0X'", or postfixed with "\verb'H'". Examples:

\verb'    0x05074  13H  1342'

The debugger stores a whole constant in 4 bytes of memory,
so the maximum value of a whole constant is \verb'0FFFFFFFFH'.

\subsection{Real constants}

\begin{verbatim}
    RealConst = DecimalConst "." [ DecimalConst ]
                [ "e" | "E" [ "+" | "-" ] DecimalConst ]
\end{verbatim}

A whole decimal constant immediately followed by the period,
is considered to be a whole part of a real constant. The period,
in turn, may be followed by another whole decimal constant
representing a fixed part. Finally, an exponent may be specified
by adding the "\verb'E'" or "\verb'e'" character and an exponent
value with an optional sign.

\verb'    1.    3.14    0.1234E-10'

Real constants are stored in 10 bytes format.

\subsection{Boolean constants}

\begin{verbatim}
    BooleanConst = "TRUE" | "FALSE"
\end{verbatim}

\subsection{Character constants}

\begin{verbatim}
    CharConst = "'" character "'" |
                OctalConst "C"
\end{verbatim}

A character constant is a single character enclosed in apostrophes:

\verb"    '!'  '''  'X'"

\ifcomment

\subsection{Complex constants}

Complex constants are not supported in the current release.

\subsection{Set constants}

Set constants are not supported in the current release.

\subsection{String constants}

String constants are not supported in the current release.

\fi

\section{CPU registers}

You may use names of CPU registers in expressions, prefixing their names
with "\verb'@'".

The debugger recognizes the following register names:

\begin{verbatim}
    EAX, AX, AL, AH, EBX, BX, BL, BH,
    ECX, CX, CL, CH, EDX, DX, DL, DH
    ESI, SI, EDI, DI
    ESP, SP, EBP, BP
    EIP
    CS, SS, DS, ES, GS, FS
    EFL
\end{verbatim}

\section{Referencing program entities}
\label{expr:entities}

It is possible to reference a named program entity in an expression, 
provided that the executable contains debug information for that entity.
You may use in expressions identifiers of global and local variables 
and procedures, types, and enumeration type elements. In addition, 
it is possible to reference DLL publics.

Entities are referenced by their identifiers. In \mt{} and \ot{}, a program
may contain more than one entity with the given identifier in different scopes.
By default, the debugger starts searching for an entity in the scope of 
the current procedure (i.e. the procedure in which the execution point 
is placed) and then goes up to the scope of the compilation module which 
contains that procedure. Therefore, it may be necessary to qualify an 
entity name:

\verb'    QualIdent = [ [ Component "$" ] Module "." ] { Proc "::" } Name'

\ifcomment !!!
Программа может состоять из нескольких модулей, но только один из них является
текущим - это тот модуль, адресное пространство которго содержит текущее
значение IP. Точно так же, модуль может состоять из нескольких процедур,
одна из них является текущей, а именно та, адресное пространство которой
содержит текущее значение IP. Для диалоговаого отладчика текущими являются
модуль и процедура, отвечающие положению пользовательского курсора, информация
о которых отображается в верхней строке главного окна.

Идентификация обьекта, соответсвующего имени, выполняется в следующей
последовательности: alias, модуль, переменная, процедура, локальный обьект,
тип, public, имя элемента перечислимого тип.

Example:

KrnExec
result
feefawfum
sunday
PCHAR
CalcSumm
InOut_WriteString

Если обьект неопределен в текущей области видимости, его можно квалифицировать
именем обьемлющего блока.

Example:

KrnExec.LoadProgram
DlgBreak.AddBreak::BreakPos


Note

Информация об обьектах программы генерируется компилятором, для этого ему
необходимо задать соответствующие опции; более подробную информацию об этом
можно получить в соответствующих разделах описания компилятора и отладчика.

!!! \fi

\section{Operators}

\subsection{Unary operators}

\begin{tabular}{lll}
\bf Operator & \bf For operand type & \bf Denotes         \\
\hline
\tt +        & numeric              &                     \\
\tt -        & numeric              & negation            \\
\tt NOT ~    & whole                & bit-wise complement (32-bit) \\
             & boolean              & logical complement
\end{tabular}

\subsection{Binary operators}

\begin{tabular}{lll}
\bf Operator & \bf For operand types & \bf Denotes         \\
\hline
\tt +        & numeric               & addition            \\
\tt -        & numeric               & subtraction         \\
\tt          & address               & difference          \\
\tt *        & numeric               & multiplication      \\
\tt /        & numeric               & division            \\
\tt DIV      & whole                 & division            \\
\tt MOD      & whole                 & modulus             \\
\tt REM      & whole                 & modulus             \\
\tt %        & numeric               & modulus             \\ % ???
\tt AND \&   & whole                 & bit-wise AND (32-bit)  \\
\tt          & boolean               & logical conjunction \\
\tt OR  |    & whole                 & bit-wise OR  (32-bit)  \\
\tt          & boolean               & logical inclusive disjunction \\
\tt XOR      & whole                 & bit-wise XOR (32-bit)  \\
\tt          & boolean               & logical exclusive disjunction
\end{tabular}

It is also possible to add whole numbers to and subtract from addresses.

\subsection{Comparison operators}

\begin{tabular}{lll}
\bf Operator & \bf For operand types & \bf Tests if the first is        \\
\hline
\tt =        & scalar                & equal     \\
\tt \#       & scalar                & not equal \\
\tt <        & ordinal or real       & less than \\
\tt <=       & ordinal or real       & less than or equal \\
\tt >        & ordinal or real       & greater than \\
\tt >=       & ordinal or real       & greater than or equal
\end{tabular}

\section{Built-in functions}
\label{expr:builtin}

\subsection{ADR - address of a memory object}

\verb'    ADR "(" Designator ")"'

Returns the address of the memory object designated by \verb'Designator'.

\begin{verbatim}
    ADR(ArrayOfRecords)
    ADR(ArrayOfRecords[6])
    ADR(ArrayOfRecords[6].FirstField)
\end{verbatim}

\subsection{SIZE - size of a memory object}

\verb'    SIZE "(" Designator ")"'

Returns the size of the memory object designated by \verb'Designator'.

Examples:

\begin{verbatim}
    SIZE(ArrayOfRecords)
    SIZE(ArrayOfRecords[6])
    SIZE(ArrayOfRecords[6].FirstField)
\end{verbatim}

% !!! What about sizes of types?

\subsection{LOW - low index of an array}

\verb'    LOW "(" Array ")"'

Returns the low index of an array designated by \verb'Array'.

\verb'    LOW(ArrayOfRecords)'

\subsection{HIGH - high index of an array}

\verb'    HIGH "(" Array ")"'

Returns the high index of an array designated by \verb'Array'.

\verb'    HIGH(ArrayOfRecords)'

\subsection{@PASS - number of passes for breakpoint}
\label{expr:builtin:pass}

% !!! New in version 1.1:

\verb'    @PASS "(" Name | Number ")"'

Returns the number of passes for the breakpoint identified by \verb'Name' or
\verb'Number' (see \ref{batch:BREAK}). Batch mode only.

\subsection{@ARG - access to command-line arguments}
\label{expr:builtin:arg}

% !!! New in version 1.1:

\verb'    @ARG "(" n ")"'

Returns \verb'n'-th command-line argument specified after control file 
name. Batch mode only.

For instance, if the \verb'breakat.xd' file contains the following
statements:

\begin{verbatim}
       LOAD @ARG(1)
       BREAK B, PROC, @ARG(2),, Dialog
       START                              
       QUIT

Dialog DIALOG
       STOP
\end{verbatim}
 
and XD is launched as

\verb'    xd /b breakat myprog M.P'

it will load the program \verb'myprog', establish a breakpoint at the 
body of the procedure \verb'P' from module \verb'M', execute
the program up to that breakpoint and switch to the dislog mode.

\section{Type casting}
\label{expr:casting}

\verb'    Type "(" Designator ")"'

This kind of type casting allows you to treat a memory object designated by
\verb'Designator' as if it had type \verb'Type'. \verb'Type' can be any type
defined in your program and present in its debug information, or one of the
predefined types.

\verb'    CARD8(File.result)' \\
\verb'    PrintModes.MODE(PrintModeBox^.selected)'

\verb'    Type "<" Expression ">"'

The result is a memory object residing at address \verb'Expression'
and having type \verb'Type'.

The result of \verb'Expression' must be compatible with the address type.

\verb'    ADDRESS<ADR(result)>' \\
\verb'    DataBase.ArrayCardinal<ADR(Array2DimInteger[3])>'

\begin{table}
\begin{center}
\begin{tabular}{lll}
\bf Name & \bf Type & \bf Size \\
\hline
BYTE8        & unsigned integer & 1  \\  % ??? what's the diff with CARD?
BYTE16       & unsigned integer & 2  \\
BYTE32       & unsigned integer & 4  \\
INT8         & signed integer   & 1  \\
INT16        & signed integer   & 2  \\
INT32        & signed integer   & 4  \\
CARD8        & unsigned integer & 1  \\
CARD16       & unsigned integer & 2  \\
CARD32       & unsigned integer & 4  \\
REAL         & real             & 4  \\
LONGREAL     & real             & 8  \\
LONGLONGREAL & real             & 10 \\
COMPLEX      & complex          & 8  \\
LONGCOMPLEX  & complex          & 16 \\
BOOL8        & boolean          & 1  \\
BOOL16       & boolean          & 2  \\
BOOL32       & boolean          & 4  \\
CHAR8        & character        & 1  \\
SET8         & bitset           & 1 \\
SET16        & bitset           & 2 \\
SET32        & bitset           & 4 \\
ADDRESS      & address          & 4
\end{tabular}
\end{center}
\caption{Predefined types}
\end{table}

