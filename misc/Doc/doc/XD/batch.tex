\chapter{Using the batch mode}
\label{batch}

When invoked in the {\em batch mode}, the debugger loads
a control file --- a text file containing XD control language
statements. Those statements perform various debugging actions:
load program, set breakpoint, add watch expression, start
program execution, etc.

During control file processing, the debugger creates a log,
which may be saved in a file and then analyzed.

There is also a statement that temporarily switches the
debugger into the dialog mode. Upon exit from it, control
file processing continues.

\section{Control file syntax}

Each line of a control file may contain a statement
or a comment (empty lines are treated as comments).
Comments start from the symbol \verb'";"'.

\begin{verbatim}
Statement =
    [ label ] name [ parameter {"," parameter}] [";" comment]
\end{verbatim}

A statement consists of a label, a statement name,
a comma-separated list of parameters and a comment.
All these parts except statement name are optional.
Statements may not be broken across two or more lines.
Statement names are not case-sensitive.

Control files are often used to prepare a
program for debugging in the dialog mode, and
usually perform the following actions:
\begin{enumerate}
\item load the program
\item establish breakpoints, breaks, and watches
\item initiate program execution
\item upon the first breakpoint hit, switch to the dialog mode
\end{enumerate}
\pagebreak
Below, is a small control file example with detailed comments.

\begin{verbatim}
       LOAD d:\xds\samples\bench\dry.exe  ;1
                                          ;2
       MODULE dry                         ;3
                                          ;4
       BREAK P0,ADDR,Proc0,Dialog         ;5
       BREAK P8,ADDR,Proc8,Dialog         ;6
                                          ;7
       WATCH Char1Glob,Array1Glob[8]      ;8
                                          ;9
       START                              ;10
       QUIT                               ;11
                                          ;12
Dialog DIALOG                             ;13
       STOP                               ;14
\end{verbatim}

In this sample, comments are used to hold the line
numbers referenced below.

{\bf Line 1:} The \verb'LOAD' statement
loads an executable file, specified as
a parameter, into the debugger.

{\bf Line 3:} The \verb'MODULE' statement sets
the current module to \verb'dry', allowing to use unqualified
indetifiers in the following statements.

{\bf Lines 5 and 6:} The \verb'BREAK' statements
establish sticky breakpoints at the entry
points of the procedures \verb'dry.Proc0' and \verb'dry.Proc8'.
\verb'Dialog' is the name of a label from which processing
of the control file will continue if the breakpoint is hit.

{\bf Line 8:} The \verb'WATCH' statement adds expressions
\verb'dry.Char1Glob' and \verb'dry.Array1Glob[8]' to the list
of watch expressions.

{\bf Line 10:} The \verb'START' statement starts the
debuggee. Processing of this statement finishes when
the debuggee terminates or a \verb'STOP' statement is
executed in a break handler.

{\bf Line 11:} The \verb'QUIT' statement terminates
the debugging session.

{\bf Line 13:} The label \verb'Dialog' was specified
in the \verb'BREAK' statements in lines 5 and 6.
If either of the breakpoints set by those statements
will be encountered during execution, in our case
during processing of the \verb'START' statement in
line 10, the debugger will process the control file
starting from this line.

The \verb'DIALOG' statement switches the debugger to
dialog mode. Breakpoints and watch expressions
remain active. Upon user-initiated exit from the
dialog mode, the debugger will return to the batch mode
and process the next statement.

{\bf Line 14:} The \verb'STOP' statement terminates
the debuggee. The debugger will continue control
file processing from the line following the \verb'START'
statement.

\section{Load program statement}

{\samepage
\Frame{LOAD}{Load a program}\label{batch:LOAD}
\begin{verbatim}
    LOAD Program [ Arguments ]
\end{verbatim}

The \verb'LOAD' statement loads an executable file
\verb'Program' into the debugger, passing optional
command-line \verb'Arguments' to it. This statement should
be executed prior to any statements which perform debugging
actions, i.e. set breakpoints, add watch expressions, etc.
} % \samepage

If a program fails to load, a message is displayed
and control file processing is terminated.

{\samepage
{\bf Possible Errors}

\begin{flushleft}
\begin{tabular}{ll}
127 & Incorrect program name \\
165 & Program name expected  \\
700 & Loading $<$...$>$: $<$...$>$
\end{tabular}
\end{flushleft}
} % \samepage
\pagebreak[1]
{\samepage
{\bf Examples}
\begin{verbatim}
    LOAD xc =p =a dry
    LOAD dry
\end{verbatim}
} % \samepage

\section{Control flow statements}

The statements described in this section allow
to change the order in which control file is processed,

\pagebreak[1]
{\samepage
\Frame{PAUSE}{Suspend control file processing}\label{batch:PAUSE}
\begin{verbatim}
    PAUSE [ Delay ]
\end{verbatim}
The \verb'PAUSE' statement suspends control file processing
for \verb'Delay' seconds, or until a key is pressed,
allowing a user to view the last debugger messages.
If \verb'Delay' is not specified, infinite delay is assumed.
} % \samepage

{\samepage
{\bf Possible Errors}
\begin{flushleft}
\begin{tabular}{ll}
126 & Incorrect parameter
\end{tabular}
\end{flushleft}
} % \samepage
\pagebreak[1]
{\samepage
{\bf Example}
\begin{verbatim}
    PAUSE
    PAUSE 10
\end{verbatim}
} % \samepage

\pagebreak[1]
{\samepage
\Frame{GOTO}{Unconditional control transfer}\label{batch:GOTO}
\begin{verbatim}
    GOTO Label
\end{verbatim}
The \verb'GOTO' statement unconditionally transfers
control to the statement associated with \verb'Label'.
} % \samepage

If the specified \verb'Label' does not exist in the control
file, an error message is issued, and control is passed to the
next statement.

{\samepage
{\bf Possible Errors}
\begin{flushleft}
\begin{tabular}{ll}
112 & Incorrect label \\
114 & Label not found \\
168 & Label expected
\end{tabular}
\end{flushleft}
} % \samepage
\pagebreak[1]
{\samepage
{\bf Example}
\begin{verbatim}
    GOTO Terminate
\end{verbatim}
} % \samepage

\pagebreak[1]
{\samepage
\Frame{IF}{Conditional control transfer}\label{batch:IF}
\begin{verbatim}
    IF Expression THEN Label
\end{verbatim}
The \verb'IF' statement evaluates \verb'Expression', which
has to be of type \verb'BOOLEAN', and, if it yields \verb'TRUE',
passes control to the statement associated with \verb'Label'.
If the result of \verb'Expression' is \verb'FALSE', the next
statement is executed.
} % \samepage

If the specified \verb'Label' does not exist in the control
file, an error message is issued, and control is passed to the
next statement.

{\samepage
{\bf Possible Errors}
\begin{flushleft}
\begin{tabular}{ll}
128 & THEN expected \\
134 & Expression expected \\
168 & Label expected \\
176 & Incorrect expression
\end{tabular}
\end{flushleft}
} % \samepage
{\samepage
{\bf Examples}
\begin{verbatim}
    IF Dry.Char1Glob='A' THEN Write
    IF (Str.Len=10)AND(Memory.Limit) THEN exit
\end{verbatim}
} % \samepage

\pagebreak[1]
{\samepage
\Frame{CALL/RETURN}{Procedure call}\label{batch:CALL}\label{batch:RETURN}
\begin{verbatim}
    CALL Label
    RETURN
\end{verbatim}
The \verb'CALL' statement transfers control to the
statement associated with \verb'Label'. The \verb'RETURN'
statement returns control to the statement following
the last executed \verb'CALL' statement. Thus,
these two statements allow to group the common parts
of a control file into subroutines.
The procedure calls may be nested.
} % \samepage

If the specified \verb'Label' does not exist in the control
file, an error message is issued, and control is passed to the
next statement. The same happens if a \verb'RETURN' statement
is encountered. which has no corresponding previously executed
\verb'CALL' statement.

{\samepage
{\bf Possible Errors}
\begin{flushleft}
\begin{tabular}{ll}
112 & Incorrect label \\
114 & Label not found \\
115 & RETURN without CALL \\
168 & Label expected
\end{tabular}
\end{flushleft}
} % \samepage
\pagebreak[1]
{\samepage
{\bf Example}

\begin{verbatim}
    CALL Write
. . .
Write   . . .
    RETURN
\end{verbatim}
} % \samepage

\pagebreak[1]
{\samepage
\Frame{SIGNAL}{Error trapping}\label{batch:SIGNAL}
\begin{verbatim}
    SIGNAL "+" ErrorNumber "," Label
    SIGNAL "-" ErrorNumber
\end{verbatim}
The \verb'SIGNAL' statement is used for error trapping.
The \verb'"+"'-form enables trapping of an error \verb'ErrorNumber'.
After execution of a statement caused this error, control
is passed to the statement associated with \verb'Label'.
An error is considered handled and the error counter is
not incremented.
} % \samepage

If the specified \verb'Label' does not exist in the control
file, an error message is issued, and control is passed to the
next statement.

The \verb'"-"'-form disables trapping of a specified error.

The \Ref{{\tt \#IMPORT}}{batch:IMPORT} directive and the SIGNAL statement
allow to implement reusable handlers for a certain group
of errors.

{\samepage
{\bf Possible Errors}
\begin{flushleft}
\begin{tabular}{ll}
112 & Incorrect label \\
114 & Label not found \\
168 & Label expected \\
208 & Error code expected \\
209 & Incorrect error list action \\
210 & Incorrect error code \\
211 & Handler for error $<$...$>$ already defined \\
212 & Handler for error $<$...$>$ not defined
\end{tabular}
\end{flushleft}
} % \samepage
\pagebreak[1]
{\samepage
{\bf Example}
\begin{verbatim}
    SIGNAL +700,ErrorLoadProgram
    SIGNAL -700
\end{verbatim}
} % \samepage

\pagebreak[1]
{\samepage
\Frame{DIALOG}{Switch to the dialog mode}\label{batch:DIALOG}
\begin{verbatim}
    DIALOG [ Message ]
\end{verbatim}
The \verb'DIALOG' statement switches the debugger to the dialog mode.
\verb'Message', if specified, is displayed in a message box.
All breaks, breakpoints, and watches set in the batch mode remain
active.
} % \samepage

Control file processing will be resumed upon exit from the dialog mode,
unless you select {\bf Stop debugging} from the {\bf File} menu.

{\samepage
{\bf Possible Errors}
\begin{flushleft}
\begin{tabular}{ll}
116 & Program not loaded \\
\end{tabular}
\end{flushleft}
} % \samepage
\pagebreak[1]
{\samepage
{\bf Example}
\begin{verbatim}
    DIALOG
\end{verbatim}
} % \samepage

\pagebreak[1]
{\samepage
\Frame{QUIT}{Terminate control file processing}\label{batch:QUIT}
\begin{verbatim}
    QUIT
\end{verbatim}
The \verb'QUIT' statement unconditionally terminates
control file processing.
} % \samepage

{\samepage
{\bf Possible Errors}

None.

} % \samepage
\pagebreak[1]
{\samepage
{\bf Example}
\begin{verbatim}
    QUIT
\end{verbatim}
} % \samepage

\pagebreak[1]
\section{Program execution statements}

Statements of this group are used to establish breaks and breakpoints,
and execute the program being debugged.

{\samepage
\Frame{START/STOP}{Start/stop execution}\label{batch:START}\label{batch:STOP}
\begin{verbatim}
    START
    STOP
\end{verbatim}
The \verb'START' statement begins execution of the debuggee. During execution,
control may be returned to the debugger, for instance, as a result of
a breakpoint hit. The \verb'STOP' statement may be used in a \See{handler}{}{batch:BREAK} to
terminate the debuggee. In this case, control file processing continues from
a statement next to the \verb'START' statement.
} % \samepage

The \verb'START/STOP' statements form a pair similar to \Ref{{\tt CALL/RETURN}}{batch:CALL},
except that nesting is not allowed. This means that a program may not be
restarted until it is explicitly stopped, so \verb'START' is not allowed within
a handler. Conversely, \verb'STOP' is not allowed outside a handler.
% !!! Which error is issued in the latter case? Does execution continues in both cases?

The \verb'STOP' statement unwinds the \Ref{subroutine stack}{batch:CALL} up to the
level from which the \verb'START' statement was issued.

{\samepage
{\bf Possible Errors}
\begin{flushleft}
\begin{tabular}{ll}
116 & Program not loaded \\
129 & Program restart not allowed
\end{tabular}
\end{flushleft}
} % \samepage
\pagebreak[1]
{\samepage
{\bf Example}
\begin{verbatim}
    START
    STOP
\end{verbatim}
} % \samepage


\ifcomment !!!
<<< ADVANCED BEGIN >>>

        Для организации запуска отлаживаемой программы и возврата в пакет
используется стек вызовов процедур. Во время выполнения оператора
START в стек заносится номер строки, непосредственно следующей за
текущей, при этом элемент стека помечается особым признаком, отличающим его от
элементов, помещенных в стек оператором CALL. При выполнении оператора
STOP вершиной стека становится помеченный элемент, после чего
номер строки, сохраненный в этом элементе стека, становится номером текущей
строки. Затем из стека удаляется помеченный элемент и выполнение пакета
продолжается дальше.
        Как было отмечено ранее, оператор START не может быть повторно
исполнен до тех пор, пока не будет выполнен оператор STOP. Из этого
следует, что стек вызовов процедур может содержать не более одного помеченного
элемента. Элементы, находящиеся в стеке до помеченного элемента, соответствуют
вызовам процедур, произошедшим до исполнения оператора START. Элементы,
находящиеся в стеке после помеченного элемента, соответствуют вызовам процедур,
произошедшим после исполнения оператора START. Эти элементы будут
удаленны из стека после выполнения оператора STOP. Таким образом,
возврат из процедур, вызов которых произошел после начала выполнения
отлаживаемой программы, становиться невозможным после  ее завершения. Такая
организация взаимодействия операторов вызова и возврата из процедур и
операторов управления выполнением отлаживаемой программы необходима для вызова
внутри каждого варианта выполнения программы процедур, необходимых только для
этого варианта, завершение которых не нужно в случае завершения самого варианта.

<<< ADVACED END >>>

\fi

\pagebreak[1]
{\samepage
\Frame{RESUME}{Resume execution}\label{batch:RESUME}
\begin{verbatim}
    RESUME
\end{verbatim}
The \verb'RESUME' statement effectively terminates a handler,
causing program execution to continue after a breakpoint hit or
a break activation.
The \verb'CALL' stack, however, is not unwound, so issuing
\verb'RETURN' later may cause control to be transferred to the
line after the \verb'CALL' statement which called the subroutine
containing the \verb'RESUME' statement.
} % \samepage

{\samepage
{\bf Possible Errors}
\begin{flushleft}
\begin{tabular}{ll}
116 & Program not loaded \\
130 & Program was not started
\end{tabular}
\end{flushleft}
} % \samepage
\pagebreak[1]
{\samepage
{\bf Example}
\begin{verbatim}
    RESUME
\end{verbatim}
} % \samepage

\pagebreak[1]
{\samepage
\Frame{BREAK}{Install an event handler}\label{batch:BREAK}
\begin{verbatim}
    BREAK (Name | Number) "," Event

    Event = StopByUser         |
            ProgramException   |
            BreakAtAddress     |
            BreakAtLine        |
            BreakAtProc        |
            WriteAccess        |
            ReadAccess         |
            BreakByCondition

    StopByUser       = USER "," Label.
    ProgramException = XCPT "," Label.
    BreakAtAddress   = ADDR "," Address ","
                       ( Label [ "," Pass] [ ",?," Condition ] ["," "/"] |
                         "*" | "." ).
    BreakAtLine      = LINE "," ModuleName "," Line "," 
                       ( Label [ "," Pass] [ ",?," Condition ] ["," "/"] |
                         "*" | "." ).
    BreakAtProc      = PROC "," ProcName "," [ "P" | "B" | "E" | "R" ] "," 
                       ( Label [ "," Pass] [ ",?," Condition ] ["," "/"] |
                         "*" | "." ).
    WriteAccess      = WRITE "," Address "," Length "," Label.
    ReadAccess       = READ "," Address "," Length "," Label.
    BreakByCondition = COND "," BooleanExpression "," Label.
\end{verbatim}
The \verb'BREAK' statement establishes an event handler. When the specified
\verb'Event' happens during execution of the debuggee, control is returned to
XD and control file processing continues from the statement associated with
\verb'Label'.  A handler is identified by an unique symbolic \verb'Name' or 
\verb'Number'.
} % \samepage

The following types of events may be detected:
\begin{description}
\item[{\tt USER}]  Execution is interrupted by user
\item[{\tt XCPT}]  An exception is raised
\item[{\tt ADDR}]  A breakpoint is hit at \verb'Address'
\item[{\tt LINE}]  A breakpoint is hit at \verb'Line' in \verb'ModuleName'
\item[{\tt PROC}]  A breakpoint is hit at \verb'P'rologue, 
                    \verb'B'ody (default), \verb'E'pilogue, or 
                    \verb'R'eturn instruction of the procedure \verb'ProcName'
\item[{\tt WRITE}] A \verb'Length' bytes block of memory at \verb'Address' was written to
\item[{\tt READ}]  A \verb'Length' bytes block of memory at \verb'Address' was read from
\item[{\tt COND}]  \verb'BooleanExpression' was evaluated to \verb'TRUE'
\end{description}

% New in 1.1:

For breakpoints, you may optionally specify:

\begin{itemize}
\item \verb'Pass' to set a \See{delayed breakpoint}{}{dialog:breaks:delayedsticky}
\item \verb'Condition' to set a \See{conditional breakpoint}{}{dialog:breaks:expression}
\item  "\verb'/'". to set a \See{non-sticky breakpoint}{}{dialog:breaks:simple}
\end{itemize}

Specify "\verb'*'", or "\verb'.'"instead of \verb'Label' to set a 
\See{watchpoint}{}{dialog:breaks:watchpoint} or a 
\See{counter}{}{dialog:breaks:counter} respectively.

If the specified \verb'Line' is not \See{executable}{}{start:general:executable},
a warning is issued, and the breakpoint is set at the first executable line which
number is greater, if such line exists.

In \verb'Condition', it is possible to use variables that are not visible
in the current context, but they have to be visible at the specified
breakpoint location.

%!!! More details

{\bf Note:} It is recommended to remove access breaks from local variables
upon return from a procedure.

{\samepage
{\bf Possible Errors}
\begin{flushleft}
\begin{tabular}{ll}
112 & Incorrect label \\
114 & Label not found \\
116 & Program not loaded \\
131 & Incorrect address \\
151 & Incorrect breakpoint number \\
152 & Breakpoint number already in use \\
153 & Incorrect event identifier \\
155 & Incorrect length \\
159 & Line number expected \\
160 & Incorrect line number \\
161 & Break already exists \\
168 & Label expected \\
169 & Breakpoint number expected \\
170 & Breakpoint type expected \\
171 & Breakpoint attributes expected \\
182 & Incorrect break \\
213 & Module not found \\
917 & This break is set but is disabled due to hardware limit\\
918 & Wrong expression in current context
\end{tabular}
\end{flushleft}
} % \samepage
\pagebreak[1]
{\samepage
{\bf Examples}
\begin{verbatim}
    BREAK 1,USER,Break
    BREAK 2,XCPT,Break
    BREAK 3,ADDR,Dry.Proc8,Break
    BREAK 4,WRITE,ADR(Dry.Char1Glob),1,Break
    BREAK 5,READ,ADR(Dry.Char1Glob),1,Break
    BREAK 6,COND,Dry.Char1Glob='A',Break
    BREAK 7,LINE,Dry,210,Break
\end{verbatim}
} % \samepage

\pagebreak[1]
{\samepage
\Frame{DEL}{Remove event handler(s)}\label{batch:DEL}
\begin{verbatim}
    DEL Number { "," Number }
\end{verbatim}
The \verb'DEL' statement removes a previoulsy installed event
handler associated with \verb'Number'.
} % \samepage

{\samepage
{\bf Possible Errors}
\begin{flushleft}
\begin{tabular}{ll}
151 & Incorrect breakpoint number \\
162 & Break number $<$...$>$ not found \\
169 & Breakpoint number expected
\end{tabular}
\end{flushleft}
} % \samepage
\pagebreak[1]
{\samepage
{\bf Examples}
\begin{verbatim}
    DEL 1
    DEL 3,4,17
\end{verbatim}
} % \samepage

\section{Data management statements}

The statements described in this section are used to change and display
values of CPU registers, variables, and memory areas.

{\samepage
\Frame{FILL}{Initialize memory/registers}\label{batch:FILL}
\begin{verbatim}
    FILL ( Address | Register ) "," Length "," WholeExpression
\end{verbatim}
The \verb'FILL' statement initializes a memory area or a set of
CPU registers by setting all its bytes to a specified value.
} % \samepage

\verb'Address' must be an address expression, so use \verb'ADR(v)'
to initialize a variable \verb'v'. \verb'Length' is a number of
bytes to fill with result of \verb'WholeExpression MOD 100H'.

It is also possible to specify CPU \verb'Register' instead of
memory \verb'Address'. In this case, registers are "mapped" onto
a byte array in the following sequence:

\verb'EAX, EBX, ECX, EDX, ESI, EDI, EBP, ESP'

starting from the less significant byte.

{\samepage
{\bf Possible Errors}
\begin{flushleft}
\begin{tabular}{ll}
107 & Parameter expected \\
116 & Program not loaded \\
126 & Incorrect parameter \\
131 & Incorrect address \\
157 & Address value expected \\
203 & Write to address $<$...$>$ failed, len $<$...$>$
\end{tabular}
\end{flushleft}
} % \samepage
\pagebreak[1]
{\samepage
{\bf Example}
\begin{verbatim}
    FILL ADR(Dry.Char1Glob),1,0x47
    FILL ADR(Dry.Array2Dim),56,0
    FILL @EAX,20,0xFA
\end{verbatim}
} % \samepage

\pagebreak[1]
{\samepage
\Frame{SET}{Change variable/register value}\label{batch:SET}
\begin{verbatim}
    SET ( Variable | Register ) "," Expression
\end{verbatim}
The \verb'SET' statement evaluates an \verb'Expression' and assigns the
result to a \verb'Variable' or loads it into a CPU \verb'Register'.

See Chapter \ref{expr} for more information about expressions in XD.

% ??? Type compatibility?
} % \samepage

{\samepage
{\bf Possible Errors}
\begin{flushleft}
\begin{tabular}{ll}
116 & Program not loaded \\
134 & Expression expected \\
187 & Write to register $<$...$>$ failed \\
200 & Variable or register expected \\
201 & Type incompatibility in assignment \\
205 & Stuctured variable not allowed in this context
%!!! ??? ref to expr errors
\end{tabular}
\end{flushleft}
} % \samepage
\pagebreak[1]
{\samepage
{\bf Example}
\begin{verbatim}
    SET Dry.Char1Glob,'A'
    SET @ESP,0xFAFBFCFD
\end{verbatim}
} % \samepage

\pagebreak[1]
{\samepage
\Frame{MOVE}{Copy memory area}\label{batch:MOVE}
\begin{verbatim}
    MOVE SourceAddress "," DestinationAddress "," Length
\end{verbatim}
The \verb'MOVE' statement copies \verb'Length' bytes of memory from
\verb'SourceAddress' to \verb'DestinationAddress'. Both addresses
may be specified as \See{expressions}{Chapter }{expr}.
Overlapped areas are copied unsafely; a warning is issued in this case.
} % \samepage

{\samepage
{\bf Possible Errors}
\begin{flushleft}
\begin{tabular}{ll}
116 & Program not loaded \\
126 & Incorrect parameter \\
155 & Incorrect length \\
157 & Address value expected \\
203 & Write to address $<$...$>$ failed, len $<$...$>$
\end{tabular}
\end{flushleft}
} % \samepage
\pagebreak[1]

{\samepage
{\bf Example}
\begin{verbatim}
    MOVE ADR(Dry.Char1Glob),ADR(Dry.Array1Dim),1
\end{verbatim}
} % \samepage

\pagebreak[1]
\section{Log control statemnts}

When operating in the batch mode, the debugger forms a log containing
executed lines or instructions, messages, etc. The statements
described in this section allow to output data to the log, redirect it
to a file, toggle log filters, and otherwise control the logging features.

{\samepage
\Frame{LOG}{Redirect log to file}\label{batch:LOG}
\begin{verbatim}
    LOG [ FileName [ "." [ FileExt ] ]
\end{verbatim}
The \verb'LOG' statement allows to redirect log to a file.
By default, the log information is written to the standard output.
After redirection to a file, only error messages are written to the
standard output.
If \verb'FileExt' is not specified, the default extension \verb'.XDL' will be used.
If no file name is specified at all, control file name will be used
with extension replaced with the default log file extension.
} % \samepage
\pagebreak[1]

{\samepage
{\bf Possible Errors}
\begin{flushleft}
\begin{tabular}{ll}
004 & Error opening log file $<$...$>$ \\
117 & Incorrect log file name
\end{tabular}
\end{flushleft}
} % \samepage
\pagebreak[1]
{\samepage
{\bf Example}
\begin{verbatim}
    LOG
    LOG result.
    LOG result.dat
\end{verbatim}
} % \samepage

\pagebreak[1]
{\samepage
\Frame{PRINT}{Output formatted data}\label{batch:PRINT}
\begin{verbatim}
    PRINT Format { ',' Expression }.

    Format       = '"' { Character | Specifier | ControlSeq } '"'
    Character    = <any printable character except '%' and '\'>
    ControlSeq   = '\' ( 'n' | 'r' | 'l' | 'f' |
                         't' | 'e' | 'g' | '%' | '\' )
    Specifier    = '%' { Modifier } Base | MemoryRegion
    Modifier     = '$' | '-' | '+' | '|' | '#' | '0' |
                   Space | Width | Precision
    Base         = 'i' | 'd' | 'u' | 'x' | 'o' | 'b' |
                   'c' | 's' | 'w' | 'f' | 'e' | 'g' |
                   'X' | 'F' | 'E  | 'G' | 'u' | '%'
    MemoryRegion = ('M'|'m') "(" "'" Specifier "'"
                   [',' Const [',' Const ] ] ")"
    Width        = Digit { Digit }
    Precision    = '.' Digit { Digit }
\end{verbatim}
The \verb'PRINT' statement performs formatted output to the log.
The format string syntax is similar to used by the \verb'printf'
family of functions available in the C standard library.
} % \samepage

\verb'Format' is a string which may contain ordinary characters, control
sequences (starting with \verb"'\'") and format specifiers (starting with \verb"'%'").
During execution of a \verb'PRINT' statement, \verb'Format' is scanned from left
to right. Ordinary characters are sent to the log unchanged, while control
sequences are replaced with certain control characters.
Upon encountering of a format specifier, result of a corresponding
\verb'Expression' is formatted according to that specifier and output
to the log. Therefore, the number of expressions has to match the
number of format specifiers. If there is a specifier for which no
expression is supplied, an error will be raised. If there are more expressions
than specifiers, a warning will be issued. A newline sequence (\verb'CR+LF')
is output upon completion of a \verb'PRINT' statement.

\pagebreak[1]
{\samepage
{\bf Possible Errors}
\begin{flushleft}
\begin{tabular}{ll}
134 & Expression expected \\
143 & Double quotes expected \\
144 & String not terminated (double quotes expected) \\
204 & Read from address $<$...$>$ failed, len $<$...$>$ \\
205 & Stuctured variable not allowed in this context \\
206 & Register is not allowed
\end{tabular}
\end{flushleft}
} % \samepage

\pagebreak[1]
{\samepage
{\bf Example}
\begin{verbatim}
    PRINT "Hello, world!"
    PRINT "Char1Glob='%c'",Char1Glob
    PRINT "------\n\t1 bell\g\n\t2 bell\g\n\t3 bell\g\n------"
    PRINT "Array1Glob: [7]=%u, [8]=%u",Array1Glob[7],Array1Glob[8]
    PRINT "Memory region: %M('%$4X\n',k4,l20)",ADR(Array1Glob[6])
\end{verbatim}
} % \samepage

\subsubsection{Control sequences}

{\samepage
\verb'ControlSeq' is a two-character sequence which causes
a certain control character to be output:

\begin{tabular}{c|l}
\bf Sequence & \bf Replaced with \\
\hline
\verb'\n' & CR+LF (15C 12C)    \\
\verb'\r' & CR  (15C)          \\
\verb'\l' & LF  (12C)          \\
\verb'\f' & FF  (14C)          \\
\verb'\t' & TAB (11C)          \\
\verb'\e' & ESC (33C)          \\
\verb'\g' & BEL (7C)           \\
\verb'\%' & percent sign (45C) \\
\verb'\\' & backslash (134C)
\end{tabular}
} % \samepage

\subsubsection{Format bases}

{\samepage
The following format bases are recognized:

\begin{tabular}{c|l|l}
\bf Base & \bf Argument & \bf Output format           \\
\hline
   "c"   & CHAR    & single character                  \\
"d", "i" & integer & signed decimal integer            \\
   "u"   & integer & unsigned decimal integer          \\
   "o"   & integer & unsigned octal integer            \\
"x", "X" & integer & unsigned hexadecimal integer      \\
   "f"   & real    & fixed-point real                  \\
"e", "E" & real    & floating-point real               \\
"g", "G" & real    & the shorter of "f" and "e"        \\
   "s"   & string  & string                            \\
   "w"   & boolean & "FALSE" or "TRUE"                 \\
         & integer & "FALSE" if zero, "TRUE" otherwise \\
%  "\{\}" & BITSET  & bitset, e.g. \verb|{1,3..5}| \\ !!!
   "n"   & any type & default \\
   "\%"   & none & "\%"
\end{tabular}

{\footnotesize
{\bf Note:} Case of hex letters and of a letter used to introduce exponent
in a floating point number matches case of the corresponding
\verb'Base' character.

} % \footnotesize
} % \samepage

\ifcomment !!!
 - "Область памяти" предназначена для печати областей заданой длины
   по указаному количеству байт в блоке по определенному формату.
   Обозначение - 'M', 'm'. При отсутствии длины блока она принимается
   равной 1, при отсутствии длины области пямяти она принимается равной 1.
   Задаваемый формат применяется для всех блоков, расположенных внутри
   заданной области. При задании некратной длины области и блока, длина
   последнего блока при печати получается меньше заданой.

!!!

%W.P(...) means:

print W blocks of memory P bytes each using format in parenthesis,

Example:

%16.4(%8X %8X %8X %8X\n)

would print something like:

0000000 0000000 0000000 FFFFFFFF
0000000 0000000 0000000 FFFFFFFF
0000000 0000000 0000000 FFFFFFFF
0000000 0000000 0000000 FFFFFFFF

\fi

\subsubsection{Modifiers}

\verb'Modifier' allow to alter the default data formatting:

\begin{tabular}{c|c|l|l}
\bf Modifier & \bf Bases & \bf Meaning                   & \bf Default \\
\hline
 \verb'"+"' & difegEG & always print sign                & only negative     \\
   space    & difegEG & print space instead of "+"       &                   \\
 \verb'"-"' & all     & left-justify the value           & right-justify     \\
 \verb'"|"' & all     & center the value                 &                   \\
 \verb'"0"', \verb'"$"' & numeric & print leading zeroes             & spaces            \\
 \verb'"#"' & oxX     & print a base character           & no base character
\end{tabular}

\subsubsection{Width}

\verb'Width' is an unsigned decimal specifying the minimum number of
character positions used for a formatted value. If the output string
is shorter than \verb'Width', blanks or zeroes are added according to
\verb'Modifiers'. If the number of characters in an output string exceeds
\verb'Width', or \verb'Width' is omitted, all characters are printed.

\ifcomment !!!
If {\tt Width} is set to an asterisk (\verb'"*"'), its actual value will be
retrieved from the argument list. It has to be specified before the value
being formatted.
\fi

\subsubsection{Precision}

\verb'Precision' is an unsigned decimal preceded by a period,
which specifies the exact number of characters to be printed or
the number of decimal places. Unlike \verb'Width', \verb'Precision'
may cause the output value to be truncated or floating-point value
to be rounded.

\ifcomment !!!
If {\tt Precision} is set to an asterisk (\verb'"*"'), its actual value will
be taken from the argument list. It has to be specified before the value being
formatted, but after the actual value for {\tt Width}, if the latter is also set
to "*".
\fi

{\tt Precision} value is interpreted depending upon the {\tt Base}:

\begin{tabular}{l|p{10 cm}}
\bf For bases & \bf Precision specifies \\
\hline
\bf i d u o x X &
    The minimum number of digits to output. Shorter values are padded on
    the left with blanks; longer values are not truncated. Default is 1. \\
\bf f F e E &
    The number of positions after the decimal point. The last digit output
    is rounded. If {\tt Precision} is 0, decimal point is not printed.
    Default is 6. \\
\bf g G &
    The maximum number of significant digits to output. By default,
    all significant digits are printed. \\
\bf c &
    The number of times the character is to output, default is 1. \\
\bf s &
    The maximum number of characters to output. By default, characters
    are output until \verb'0C' is reached. \\
\bf \{\} &
    Has no effect.
\end{tabular}


% !!! Review MODE statement
{\samepage
\Frame{MODE}{Log control}\label{batch:MODE}
\begin{verbatim}
    MODE Mode { "," Mode }

    Mode = ADDR | BAT | BREAK | DUMP | PRINT ( "+" | "-" )
           TRACE ( "+" | "-" ) [ "," ( CODE | SOURCE | MIX ) ]
           DISASM "=" ( SMALL | FULL )
           ERRORS "=" number
\end{verbatim}
The \verb'MODE' statement controls what is written to the log and in
which format.
} % \samepage
\begin{tabular}{ll}
\bf Mode & Controls \\ % !!! defaults and references
\hline
\tt ADDR   & whether memory dump lines contain addresses \\
\tt BAT    & whether executed control file statements are logged \\
\tt BREAK  & whether info about breaks is written \\
\tt DISASM & disassembly mode \\
\tt DUMP   & whether instruction codes are dumped \\
\tt ERRORS & maximum number of errors \\
\tt PRINT  & whether {\tt PRINT} statements are executed \\
\tt TRACE  & the way execution trace is logged
\end{tabular}

{\samepage
{\bf Possible Errors}
\begin{flushleft}
\begin{tabular}{ll}
126 & Incorrect parameter \\
137 & Incorrect mode \\
138 & Errors limit set to $<$...$>$ \\
172 & Mode expected
\end{tabular}
\end{flushleft}
} % \samepage
\pagebreak[1]
{\samepage
{\bf Example}
\begin{verbatim}
    MODE TRACE-
    MODE TRACE+,CODE,DUMP+,DISASM=FULL
    MODE PRINT+,ADDR-
    MODE ERRORS=10,BREAK+
\end{verbatim}
} % \samepage

\section{Miscellaneous statements}

\ifcomment !!!
{\samepage
\Frame{ALIAS}{Define an alias}\label{batch:ALIAS}
\begin{verbatim}
    ALIAS Alias [ "," "S," Expression ].
\end{verbatim}
%!!! "O" ?
The \verb'ALIAS' statement allows to define a name for an expression or
for an address in the code, which corresponds to the speicifed source line.
It may also be used to to rename a statement.
} % \samepage

\verb'Expression' is {\em not} checked for correctness. In fact, it is treated
as a string, which will be substituted instead of \verb'Alias' into any
expression encountered during the following contfol file preocessing.
The user is responsible for excluding the possibility of its misinterpreting.
The simplest way is to surround an expression with parenthesis.

If no code corresponds to the specified \verb'Line' in the given \verb'Module',
an error message will be issued and control file processing will continue.

After statement renaming, \verb'OldStatementName' is no longer valid.

To remove an alias, specify it as the only parameter.

{\samepage
{\bf Possible Errors}
\begin{flushleft}
\begin{tabular}{ll}
116 & Program not loaded \\
126 & Incorrect parameter \\
131 & Incorrect address \\
159 & Line number expected \\
163 & Incorrect alias \\
164 & Alias $<$...$>$ not defined \\
174 & Alias expected \\
175 & Alias definiton expected \\
189 & Alias type expected \\
213 & Module not found \\
216 & Keyword can not be redefined
\end{tabular}
\end{flushleft}
} % \samepage
\pagebreak[1]
{\samepage
{\bf Example}
\begin{verbatim}
    ALIAS Alias1,S,Dry.Char1Glob
    ALIAS Alias2,L,Dry,210
    ALIAS Alias3,O,LOAD
    ALIAS Alias1                 ; Undefine Alias1
\end{verbatim}
} % \samepage
\fi

\pagebreak[1]
{\samepage
\Frame{MODULE}{Set current module}\label{batch:MODULE}
\begin{verbatim}
    MODULE [ ModuleName ]
\end{verbatim}
In the dialog mode, you may use names of program objects belonging to the current
module without qualification. Similarly, in the batch mode, when execution of
the program stops and control is returned to the debugger, the current component and
module are set according to the value of the instruction pointer.
The \verb'MODULE' statement explicitly changes the current module.
If \verb'ModuleName' is not specified, unqualified identifiers recognition is disabled.
} % \samepage

{\samepage
{\bf Possible Errors}
\begin{flushleft}
\begin{tabular}{ll}
116 & Program not loaded \\
213 & Module not found
\end{tabular}
\end{flushleft}
} % \samepage
\pagebreak[1]
{\samepage
{\bf Example}
\begin{verbatim}
    MODULE Dry
    MODULE
\end{verbatim}
} % \samepage

\pagebreak[1]
{\samepage
\Frame{WATCH}{Add watch expression}\label{batch:WATCH}
\begin{verbatim}
    WATCH Expression { "," Expression }
\end{verbatim}
The \verb'WATCH' statement is used to fill up the list of
\See{watch expressions}{}{dialog:data:watches},
which is available in the dialog mode. It has no effect in the batch mode,
and usually precedes the \verb'DIALOG' statement.

Each \verb'Expression' is checked for correctness. If \verb'Expression' may not
be evaluated in the current context, a warning will be issued.
} % \samepage

{\samepage
{\bf Possible Errors}
\begin{flushleft}
\begin{tabular}{ll}
116 & Program not loaded \\
918 & Wrong expression in current context
\end{tabular}
\end{flushleft}
} % \samepage

{\samepage
{\bf Example}
\begin{verbatim}
    WATCH Dry.Char1Glob
    WATCH Dry.Array2Glob[1],Memory.Free<=128
\end{verbatim}
} % \samepage

\pagebreak[1]
{\samepage
\Frame{\#IMPORT}{Control file import}\label{batch:IMPORT}
\begin{verbatim}
    #IMPORT ControlFile
\end{verbatim}
XD control files are likely to have some common routines,
which perform, for instance, register dump, or heap tracing.
It is possible to collect these routines in separate control files and
import them into any control file which requires the routines.
} % \samepage

The \verb'#IMPORT' directive causes the debugger to load the
specified \verb'ControlFile' into memory and add its labels
to the global table of labels.

After the debugger loads the control file specified on the
command line, it scans it for labels and \verb'#IMPORT' directives,
and recursively repeats this procedure for all imported
control files.

{\bf Note:} This directive was intentionally named \verb'#IMPORT',
not \verb'#INCLUDE', in order to emphasis the fact that the
contents of the \verb'ControlFile' are {\it not} substituted
instead of the directive.

When XD reaches the end of a control file, it terminates regardless
of whether that file was imported or specified on the command line.
So make sure the imported control files are ended with an explicit
control transfer statement, such as \verb'RETURN' or \verb'GOTO'.

{\samepage
{\bf Possible Errors}
\begin{flushleft}
\begin{tabular}{ll}
007 & Batch name missed.                 \\
009 & Too many nested packages (16)      \\
010 & Package $<$...$>$ recursively imported \\
011 & Incorrect package name $<$...$>$       \\
014 & Imported package $<$...$>$ not found.  \\
018 & Unknown directive $<$...$>$.           \\
111 & Label '$<$...$>$' already defined      \\
112 & Incorrect label                    \\
113 & Label matches reserved word
\end{tabular}
\end{flushleft}
} % \samepage
\pagebreak[1]
{\samepage
{\bf Example}   % !!! A better example
\begin{verbatim}
    #IMPORT PrintRegisters
    #IMPORT SetBreakpoints
    #IMPORT ErrorHandler
\end{verbatim}
} % \samepage

\ifcomment

{\samepage
\Frame{}{}
\begin{verbatim}

\end{verbatim}

} % \samepage

{\samepage
{\bf Possible Errors}
\begin{flushleft}
\begin{tabular}{ll}

\end{tabular}
\end{flushleft}
} % \samepage
\pagebreak[1]
{\samepage
{\bf Example}
\begin{verbatim}

\end{verbatim}
} % \samepage

\fi


