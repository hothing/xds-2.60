\chapter{Configuring the compiler}\label{config}
\index{configuration}

\section{System search paths}
\index{configuration!search paths}

In order for your operating system to know where to find the executable
binary files which constitute the \xds{}  package, you must set your
operating system search paths appropriately. See the Read Me First
file from your on-line documentation.

\section{Working configuration}\label{config:work}

The core part of \xds{} is the \xc{}\index{\XC{}@{\XC{}}} utility, which combines the
project subsystem with \mt{} and \ot{} compilers, accompanied with
a set of system files\footnote{A name of a system file
is constructed from the name of
the compiler utility and the correspondent filename extension. If you rename
the \xc{} utility, you should also rename all system files.}:

\begin{tabular}{ll}
\bf \red    & \See{Search path redirection file}{}{xc:red} \\
\bf \cfg    & \See{Configuration file}{}{config:cfg}  \\
\bf \msg    & \See{Texts of error messages}{}{xc:msg}  \\
\ifgenc
\bf \kwd    & List of C/C++ keywords and reserved identifiers \\ % !!! describe?
\fi
\end{tabular}

Being invoked, \xc{} tries to locate the {\tt \red} file, first in the
current directory and then in the directory where \xc{} is placed
(so called {\it master redirection file\index{master redirection file}\index{redirection file!master}}).

Other system files are sought by paths defined in {\tt \red}.
If {\tt \red} is not found, or it does not contain paths for
a particular system file, that file is sought in the current directory
and then in the directory where the \xc{} utility resides.

A configuration file contains settings that are relevant for all
projects. Project specific settings are defined in project files (See
\ref{xc:project}). A so-called template file is used to automate
the program build process (See \ref{xc:template}).

A redirection file, a configuration file, and, optionally, a project file
and a template file constitute a working environment for a single
execution of the \xc{} utility. The compiler preprocesses files
of all these types as described in \ref{usage:cfp}.

Portable software development is one of the main goals of \XDS{}.
To achieve that goal, not only the source texts should be portable between
various platforms, but the environment also. \XDS{} introduces
a portable notation for file names
\index{file name!portable notation}
\index{portability!file names}
that may be used in all
system files and on the command line. The portable notation combines
DOS-like and Unix-like notations (file names are case sensitive):

\verb'    [ drive_letter ":" ] unix_file_name'

\Examples
\begin{verbatim}
    c:/xds/bin
    /mnt/users/alex/cur_pro
    cur_pro/sources
\end{verbatim}
Along with the {\em base directory} macro (See \ref{usage:cfp}) this portable
notation allows to write all environment files in a platform independent
and location independent manner.

\section{\XDS{} memory usage}
\label{config:memory}
\index{memory usage (compilers)}

\XDS{} compilers are written in \ot\footnote{We use
\XDS{} in most of our developments.}. As any other \ot{} program, a
compiler uses garbage collector to deallocate memory. These days,
most operating systems, including Windows and Linux,
provide virtual memory. If an \ot{} program exceeds the amount
of avaiable physical memory, the garbage collector becomes
inefficient. Thus, it is important to restrict the amount of memory
that can be used by an \ot{} program. As a rule, such restriction
is set in the configuration or project file (See the \OERef{HEAPLIMIT} equation).
You may also let the run-time system determine the proper heap size 
at run time by setting \OERef{HEAPLIMIT} to zero.

Similarly, the equation \OERef{COMPILERHEAP}
should be used to control the amount of memory used by a compiler
itself. That equation is set in the configuration file ({\tt \cfg{}}).
We recommend to set it according to the
amount of physical memory in your computer:
\begin{flushleft}
\begin{tabular}{cr}
RAM in megabytes & COMPILERHEAP \\
\hline
32-64         &   16000000   \\
64-128        &   48000000   \\
more than 128 &   96000000
\end{tabular}
\end{flushleft}
It may be necessary to increase \OERef{COMPILERHEAP} if you get
the "\verb'out of memory'" message (\Ref{F950}{MESSAGE\_950}).
It is very unlikely, if \OERef{COMPILERHEAP} is set to 16 megabytes or more.
Your compilation unit should be very large to exceed this memory limit.
{\bf Note:} if you are using Win32 or X Window API definition modules,
set \OERef{COMPILERHEAP} to at least 16 megabytes.

Vice versa, if you notice unusually intensive disk
activity when compiling your program, it may indicate that the
value of the \OERef{COMPILERHEAP} equation is too large for your
system configuration.

Set \OERef{COMPILERHEAP} to zero if would prefer the compiler to
dynamically adjust heap size in accordance with system load.

See \ref{rts:mm} for more information on \XDS{} memory management.

\section{Directory hierarchies}
\index{configuration!directories}

\xds{} compilers give you complete freedom over where you store both your source
code files and any files which compilers create for you. It
is advisable to work in a project oriented fashion --- i.e. to have a separate
directory hierarchy for each independent project.

Due to the re-usable nature of modules written in \mt{} or \ot{},
it is wise to keep a separate directory for those files which are to  be
made available to several projects. We will call such files the {\em  library}
files.

We recommend you to have a separate working directory for each project.
You can also create subdirectories to store symbol files and
generated code files. We recommend to use the supplied script
or its customized version to create all subdirectories and,
optionally, a local redirection file or a project file. Refer to the
"Read Me First" file for more information about that script.

\section{\XDS{} search paths}

Upon activation, \XC{} looks for a file called {\tt \red} ---
a {\em redirection file}. That file defines paths by which all
other files are sought. If a redirection file was not found in
the current directory, the master redirection file is loaded
from the directory where \XC{} executable is placed.

\subsection{Redirection file}\label{xc:red}
\index{configuration!redirection file}
\index{redirection file}
\index{\XC{}.red@{\XC{}.red}}

A redirection file consists of several lines
of the form\footnote{See also \ref{usage:cfp}}:

\verb'    pattern = directory {";" directory}'

{\tt pattern}
is a regular expression with which names of files \XC{} has to open
or create are compared. A pattern usually contains wildcard symbols '*'
and '?', where

\begin{tabular}{cl}
\bf Symbol & \bf Matches \\
\hline
\tt * & any (possibly empty) string \\
\tt ? & any single character.
\end{tabular}

For a full description of regular expressions see \ref{xc:regexpr}.

It is also possible to have comment lines in a redirection file.
A comment line should start with the "\verb'%'" symbol.

A \See{portable notation}{}{config:work} is used for directory names
or paths. A path may be absolute or relative, i.e. may consist  of
full names such as

\verb'    /usr/myproj/def'

or of names relative to the current directory, such as

\verb'    src/common'

denoting the directory \verb'src/common' which is a subdirectory of the
current directory. A single dot as a pathname represents the current
directory, a double dot represents the parent, i.e. the
directory which has the current directory as a
subdirectory.

The base directory macro \verb|$!|  % \index{$!@\verb'$!'} !!!
can be used in a directory name.
It denotes the path to the redirection file.
If the redirection file is placed in the \verb|/usr/alex| directory
then \verb|$!/sym| denotes the \verb|/usr/alex/sym| directory,
whereas \verb|$!/..| denotes the \verb|/usr| directory.

For any file, its name is sequentially matched with a pattern
of each line. If a match was found, the file is sought in the
first of the directories listed on that line, then in the second
directory, and so on until either the file is found, or there are
no more directories to search or there are no more patterns to match.

If \xc{} could not locate a file which is needed for correct
operation, e.g. a necessary symbol file, it terminates
with an appropriate error message.

When creating a file, \xc{} also uses redirection, and its
behavior is determined by the \OERef{OVERWRITE} option. If the
option was set ON, \xc{} first searches for the file it is
about to create using redirection. Then, if the file was found,
\xc{} overwrites it. If no file of the same name as the one which \xc{} needs
to create was found or the \OERef{OVERWRITE} option was set OFF, then the
file is be created in the directory which appears first in the
search path list which pattern matched the filename.

If no pattern matching a given filename can be found in the {\tt
\red} file, then the file will be read from (or written to) the
current working directory.

{\bf Note:}
If a pattern matching a given filename is found then \xc{}
will {\em not} look into the current directory, unless it is
explicitly specified in the search path.

The following entry in {\tt \red} would be appropriate for
searching for the symbol files (provided that symbol files have the
extension {\bf.sym}).

\ifunix
\verb'    *.sym=sym;/usr/xds/sym;.'
\else
\verb'    *.sym=sym;c:/xds/sym;.'
\fi

Given the above redirection, the compiler will first search for
symbol files in the directory \verb'sym'
which is a subdirectory of the current working directory; then in the
directory storing the \xds{} library symbol files and then in
the current directory.

\paragraph{Example of a redirection file:}
\ifgencode
\begin{verbatim}
    xc.msg = /xds/bin
    *.mod = mod
    *.def = def
    *.ob2 = oberon
    *.sym = sym; /xds/sym/x86
\end{verbatim}
\else % ---- genc
\begin{verbatim}
    xm.msg = /xds/bin
    *.mod = mod
    *.def = def
    *.ob2 = oberon
    *.c   = c
    *.h   = include;/xds/include
    *.sym = sym; /xds/sym/C
\end{verbatim}
\fi % ---- gencode

\subsection{Regular expression}\label{xc:regexpr}
\index{regular expressions}

A regular expression is a string  containing  certain  special  symbols:

\begin{center}
\begin{tabular}{cl}
\bf Sequence & \bf Denotes \\
\hline
\verb+*+     & an arbitrary sequence of any characters, possibly empty \\
             & (equivalent to \verb|{\000-\377}| expression) \\
\verb+?+     & any single character \\
             & (equivalent to \verb|[\000-\377]| expression) \\
\verb+[...]+ & one of the listed characters \\
\verb+{...}+ & an arbitrary sequence of the listed characters, possibly empty \\
\verb+\nnn+  & the ASCII character with octal code \verb|nnn|, where n is \verb|[0-7]| \\
\verb+&+     & the logical operation AND \\
\verb+|+     & the logical operation OR  \\
\verb+^+     & the logical operation NOT \\
\verb+(...)+ & the priority of operations
\end{tabular}
\end{center}

A  sequence of the  form \verb|a-b|  used  within  either
\verb|[]| or \verb|{}| brackets denotes all characters from
\verb|a| to \verb|b|.

\paragraph{Examples}
\begin{description}

\item[{\tt *.def}] \mbox{} \\
    all files which extension is {\tt .def}

\item[{\tt project.*}] \mbox{} \\
    files which name is {\tt project} with an arbitrary extension

\item[{\tt *.def|*.mod}] \mbox{} \\
    files which extension is either {\tt .def} or {\tt .mod}

\item[{\tt \{a-z\}*X.def}] \mbox{} \\
    files starting with any sequence of letters,
    ending in one final "X" and having the extension \verb|.def|.
\end{description}

\section{Options}\label{config:options}

A rich set of \xc{} options allows one to control the source language,
code generation and internal limits and settings.  We distinguish
between boolean options (or just options) and equations. An {\em option} can
be set ON (TRUE) or OFF (FALSE), while an {\em equation} value is a string.
In this chapter we describe only the syntax of setup
directive. The full list of \xc{} options and equations is provided in
the Chapter \ref{options}.

Options and equations may be set in a \See{configuration file}{}{config:cfg},
\See{on the command line}{}{xc:modes}, in a \See{project file}{}{xc:project}),
and in the source text (see \ref{m2:pragmas}).

The same syntax of a setup directive is used in configuration
and project files and on the command line. The only difference is
that arbitrary spaces are permitted in files, but not on the
command line. Option and equation names are case independent.

\begin{verbatim}
SetupDirective   = SetOption
                 | SetEquation
                 | DeclareOption
                 | DeclareEquation
                 | DeclareSynonym
SetOption        = '-' name ('+'| '-')
SetEquation      = '-' name '=' [ value ]
DeclareOption    = '-' name ':' [ '+' | '-' ]
DeclareEquation  = '-' name ':=' [ value ]
DeclareSynonym   = '-' name '::' name
\end{verbatim}
\index{-NAME+@\verb'-NAME+'}
\index{-NAME-@\verb'-NAME-'}
\index{-NAME=@\verb'-NAME='}
\index{-NAME:@\verb'-NAME:'}
\index{-NAME:=@\verb'-NAME:='}
\index{-NAME::@\verb'-NAME::'}
All options and equations used by \xc{} are predeclared.

The \verb|DeclareSynonym| directive allows one to use a different name
(e.g. shorter name) for an option or equation.

The old version of {\tt SetOption} is also supported for convenience:

\verb"OldSetOption   = '+' name | '-' name"

\Examples
\begin{flushleft}
\begin{tabular}{l|p{8.0cm}}
\bf Directive & \bf Meaning \\ \hline
\verb|-M2Extensions+| & {\bf M2EXTENSION} is set ON \\
\verb|-Oberon=o2|     & {\bf OBERON} is set to \verb|"o2"| \\
\verb|-debug:|        & {\bf DEBUG} is declared and set OFF \\
\verb|-Demo:+|        & {\bf DEMO} is declared  and set ON \\
\verb|-Vers:=1.0|     & {\bf VERS} is declared and set to \verb|"1.0"| \\
\verb|-A::genasm|     & {\bf A} is declared as a synonym for {\bf GENASM} \\
\verb|+m2extensions|  & {\bf M2EXTENSIONS} is set OFF
\end{tabular}
\end{flushleft}

\section{Configuration file}\label{config:cfg}
\index{configuration file (\cfg)}
\index{\XC{}.cfg@{\XC{}.cfg}}
\index{configuration!configuration file}

A configuration file can be used to set the default values of options and
equations (see Chapter \ref{options}) for all projects (or a set of projects).
A non-empty line of a configuration file may contain a single compiler option or
equation setup directive (see \ref{config:options}) or a comment.
Arbitrary spaces are permitted. The "\verb|%|" character indicates a comment;
it causes the rest of a line to be discarded. {\bf Note:} the comment character
can not be used when setting an equation.

The {\it master configuration file\index{master configuration file}\index{configuration file!master}},
placed along with the \xc{} utility, usually contains default settings for
the target platform and declarations of platform-specific options and
equations, which may be used in project and template files.

\begin{figure}[htb]
\begin{verbatim}
  % this is a comment line
  % Set equation:
   - BSDef = df
  % Set predeclared options:
   - RangeCheck -   % turn range checks off
   - M2EXTENSIONS + % allow Modula-2 extensions
  % Declare new options:
   -iPentium:+
   -i80486:-
   -i80386:         % is equal to -i80386:-
  % Declare synonym:
   -N :: checknil
   -N               % disallow NIL checks
  % end of configuration file
\end{verbatim}
\caption{A sample configuration file}
\end{figure}

\section{Filename extensions}\label{config:fileext}
\index{configuration!filename extensions}
\index{file name!extension}

\xc{} allows you to define what you want to be the standard extensions
for each particular type of file. For instance, you may prefer your
\ot{} source code texts to end in {\bf .o2} instead of {\bf .ob2}.

We recommend to either use the traditional extensions or at least the extensions
which describe the kind of file they refer to, and keep same extensions
across all your projects. For example, use
{\bf.def} \index{.def (See DEF)}
and {\bf.mod} \index{.mod (See MOD)}
for \mt{} modules,
{\bf .ob2} \index{.ob2 (See OBERON)}
for \ot{} modules, etc.

Certain other factors must also influence your decisions.
Traditionally, \ot{} pseudo-definition  modules (as created by a
browser) are extended with a {\bf.def}. With \xds{}, this may conflict
with the extension used for Modula-2 definition modules.
Therefore, \See{the \XDS{} browser}{}{xc:modes:browse}
uses the extension {\bf .odf} by default.

The following filename extensions are usually defined in
the configuration file:
\begin{flushleft}
\begin{tabular}{ll}
\OERef{DEF}    & extension for Modula-2 definition  modules        \\
\OERef{MOD}    & extension for Modula-2 implementation modules     \\
\OERef{OBERON} & extension for Oberon-2 modules                    \\
\OERef{BSDEF}  & extension for Oberon-2 pseudo definition modules  \\
\ifgenc
\OERef{HEADER} & extension for C header files                      \\
\fi
\OERef{CODE}   & extension for generated code files                \\
\OERef{SYM}    & extension for symbol files                        \\
\end{tabular}
\end{flushleft}
See Table \ref{table:equ:ext} for the full list of file extensions.

\paragraph{Example (file extension entries in \cfg):}

\begin{verbatim}
   -def     = def
   -mod     = mod
   -oberon  = ob2
   -sym     = sym
\end{verbatim}

\section{Customizing compiler messages}\label{xc:msg}
\index{\XC{}.msg@{\XC{}.msg}}

The file {\tt \msg} contains texts of error messages in the form

\verb'    number text'

The following is an extract from \msg{}:
\begin{verbatim}
001 illegal character
002 comment not closed; started at line %d
...
042 incompatible assignment
...
\end{verbatim}
Some messages contain format specifiers
for additional arguments. In the above example, the message 002
contains a \verb|%d| specifier used to print a line number.

To use a language other than English for compiler messages
it is sufficient to translate \msg,  preserving error
numbers and the order of format specifiers.
\ifcomment
        % Comment \else, if Russian font is not available
% \else   % or rewrite example on some other language
        %
The following is an extract from the Russian
variant of \msg{}:
\begin{verbatim}
001 недопустимый символ
002 незакрытый комментарий, начатый в строке %d
...
042 несовместимы по присваиванию
...
\end{verbatim}
\fi

\section{XDS and your C compiler}\label{config:C}

\ifgencode % ------------------------

  \xds{} allows C object modules and libraries to be used in your projects.
  Different C compilers use different alignment, naming and calling conventions.
  \iflinux
    {\bf ATTENTION!} Since \xds{} libraries on linux are built through GCC
    compiler it is {\em absolutely neccesary} to configure XDS for GCC.
  \else % not linux -------------------------------
    Thus it is necessary to configure \xds{} for your C compiler in order
    to use C libraries and modules in your program.
  \fi  % end linux
  See \ref{multilang:ccomp} for more details.
\fi  % end gencode ------------------------

\ifgenc % ------------------------

XDS-C can be used as a translator or compiler.
In the first case, the C code generated by \xc{} will be used for
further development or will be ported to another platform. In the
second case, \xc{} is used as first pass of compilation process;
to get an executable program a C compiler must be used. A C compiler,
linker, or make utility may be invoked \See{seamlessly}{}{config:seamless}

Usually, you have to configure \xc{} for your C compiler only once and
then \xc{} will do everything to get your program ready for execution.

\subsection{Building the run-time library}

The XDS run-time library is included in an \XDS{} package
in C source form, i.e. for each library module its header
file and C source file are provided. For Unix platforms, a
pre-built library file containing object files may be included.
This is not the case for the Windows platform,
because different C compilers support different calling and naming
conventions. We recommend to build the library file for your
C compiler or several library files for various C compilers
or memory models. The \verb'LIB/C' subdirectory or your
\XDS{} installation contains makefiles which can be used
for that purpose. It may be necessary to change a C compiler
name or compiler options in this file.

\subsection{Configuring XDS for seamless compilation}\label{config:seamless}

We will use the term {\em seamless compilation} if \XC{} is
configured to call a C compiler, linker or make implicitly to prepare
an executable for your program. A makefile can be produced after
successful completion of a project and a make utility can be implicitly
called then to compile and link your program.

% The following is not true anymore !!!

The compiler uses a template file (See \ref{xc:template}) to generate
a makefile. The \XDS{} distribution contains the {\tt xds.tem} template
file, which can be used for various C compilers. The equation {\bf
ENV\_TARGET} determines a target platform, i.e. a file system,
including a representation of a file name, C compiler to use, its
options, etc.  By default, the value of this equation is equal to the
value of the {\bf ENV\_HOST} equation.  The configuration file {\tt
\cfg} contains a list of supported platform, i.e. list of names, that
can be used as a value of {\bf ENV\_TARGET} equation. We recommend to
set the value of the equation in the beginning of the configuration
file. It may be necessary to change some settings for your platform or
append a new platform. All changes should be done in {\tt \cfg} and
{\tt xds.tem} files.

Note, in the current version it is not possible to append a new
platform with DOS-like file system. We recommend to change settings
for platform "MSDOS", "OS2" and "WinNT" if necessary.

See \ref{start:build} for more information.

\fi  % end genc ------------------------
