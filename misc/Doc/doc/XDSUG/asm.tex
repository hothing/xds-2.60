\chapter{Inline assembler}
\label{asm}

This chapter contains a very brief description of the inline x86 assembler.

\section{Implemented features}

The following features are implemented in the current version:
\begin{itemize}
\item Base instruction set up to and including Pentium Pro (see \ref{asm:problems}).
\item Floating point instructions up to and including Pentium Pro.
\item Pentium MMX instructions.
\item Labels and their usage in branch and call instructions.
\item \mt{}/\ot{} procedure calls and variable access (see \ref{asm:hll}).
\end{itemize}

\section{Basic syntax}

The assembler uses the same scanner as the \mt{}/\ot{} front-end,
so it is possible to use \See{conditional compilation}{}{m2:pragmas:cc}
and comments.

Language extensions have to be enabled using the \OERef{M2EXTENSIONS}
and \OERef{O2EXTENSIONS} options in order to use inline assembly facilities.

The keyword \verb'ASM' denotes the beginning of inline assembly code;
the keyword \verb'END' denotes its end.

Each line in a piece of the assembly code may contain not more than
one instruction. It is not possible to continue an instruction on the next string.

Keywords and names of instructions and registers are not case sensitive.

There are instructions one of which arguments is fixed (\verb'DIV', \verb'FCOMI').
%В процессоре Pentium Pro (а также в его предшественниках) есть команды, ???
%у которых один из аргументов всегда фиксирован (DIV, FCOMI).
These instructions are differently denoted in different assemblers. We use
the syntax described in Intel's documentation.

If size or an operand may not be determined based on instruction semantics
and/or size of another operand, it is necessary to explicitly specify it.
The following size specifiers are recognized:

\begin{center}
\begin{tabular}{ll}
\bf Specifier    & \bf Size \\
\hline
\verb'BYTE PTR'  & 1   \\
\verb'WORD PTR'  & 2   \\
\verb'DWORD PTR' & 4   \\
\verb'FWORD PTR' & 6   \\
\verb'QWORD PTR' & 8   \\
\verb'TBYTE PTR' & 10
\end{tabular}
\end{center}

{\bf Examples:}

\verb'    MOV WORD PTR [EBX],1  ' here size specifier is obligatory

\verb'    MOV [EBX],AX          ' here size is determined automatically

\section{Labels}

An instruction may be prepended with a label, delimited with a colon character:

\verb'    Save: PUSH EAX'

Labels may not match instruction names. It is also not recommended to use
assembly keywords (\verb'DWORD', \verb'EAX') as labels.

In the current version, labels may only be used in branch and call
instructions.

\section{Accessing Modula-2/Oberon-2 objects}
\label{asm:hll}

It is possible to reference \mt{}/\ot{} entities from within
the inline assembly code, namely whole constants, variables,
and procedures (in \verb'JMP' and \verb'CALL' instructions only).

Example:

\verb'    MOV j,10'

In this example, the type of \verb'j' is used to choose between
byte, word, and doubleword \verb'MOV' instructions.

{\bf Note:} In the first pre-release versions which included inline assembler,
it was necessary to specify the base register \verb'EBP' to access local variables:

\verb'    MOV j[EBP],10'

It is {\it no longer required}.

In case of nested procedures, code to access a variable from an outer procedure scope
has to be written by hand.

Record field access is not supported yet. %??? what about arrays?
There are also no operators to denote attributes of Modula-2/Oberon-2 entities.

The \verb'OFFSET' operator returns the offset of its operand, which has to be a variable:

\verb'    MOV EAX, OFFSET j'

\section{Known problems}
\label{asm:problems}

\begin{enumerate}
\item Instruction prefixes (\verb'REP', \verb'LOCK', etc.) are not supported yet.
\item Segment overriding (\verb'DS:', etc.) is not supported yet.
\item Error position precision is $+/-$ 1-2 tokens.
\item Error 3029 incorrectly positioned.
\item 16-bit addressing modes (e.g. \verb'[BX+SI]') are not supported and
      unlikely to be supported in the future.
\item \mt{}/\ot{} entities access facilities are limited.
\item It is possible to use commands like \verb'MOVSB', \verb'MOVSW', \verb'MOVSD',
      but not as in \verb'MOVS DWORD PTR [ESI]'.
\item \mt{}/\ot{} variables usage is poorly checked for correctness.
\item It is possible to use only one \verb'OFFSET' operator in a constant expression.
\end{enumerate}

\section{Potential problems}

\begin{enumerate}
\item Not all instructions were tested with all addressing modes.
\item Error diagnostics and recovery were not tested.
\end{enumerate}

