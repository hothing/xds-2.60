\chapter{Dynamic linking}
\label{dll}

The traditional way of linking programs, when all object modules are
collected into a single executable, is called {\em static linking}, because
all references between modules are resolved statically, at link time.
This approach has certain disadvantages:

\begin{itemize}
\item The resulting program always contains all the code and data
      that {\em may be required} during execution, and all that code and
      data are loaded into memory when the program is started.
      However, in a particular environment significant subsets of them
      are {\em never used} and just waste memory space.
\item When any of the object modules are modified, the entire
      executable has to be relinked.
\item If an application suite contains several executables,
      each of them contains a copy of the run-time library.
      Other code is often duplicated as well.
\end{itemize}

{\em Dynamic linking} allows several applications to use a single copy
of an executable module (dynamic link library, or DLL). The external
references to procedures and variables built into a DLL are resolved
at load time or even at run time instead of link time.
Applications may use these procedures and variables as though
they were part of the applications' executable code.

The major advantages of dynamic linking are:

\begin{itemize}
\item Memory requirements are reduced. A DLL is loaded into memory
      only once, whereas more than one application may use a single DLL
      at the moment, thus saving memory space.

\item Application support and maintenance costs are lowered. If there was
      a bug fix or a code improvement in a DLL, there is no need to
      re-ship the entire application to customers.

\item Application enhancement is simplified. If an
      application uses DLLs to support different devices, file
      formats, or languages, it is only necessary to provide an
      additional DLL to add support for a new device, another file format,
      or one more language.

      Some applications, such as Netscape browsers, may be enhanced with
      third-party plug-ins implemented as DLLs with a known interface.

\item There is no restrictions on multi-language programming.
      DLL procedures and variables
      may be utilized by an application written in any programming
      language, provided that the programming system supports calling and
      naming conventions used in the DLL.

\item Applications may control memory usage by loading DLLs on demand
      and unloading them when they are no longer needed.

\end{itemize}

The Windows operating system makes heavy use of DLLs.
For instance, system DLLs provide interfaces to the kernel and devices.

\section{How does the dynamic linking work}
\label{dll:overview}

A dynamic link library has a table which contains names of entities (procedures
and variables) which are {\em exported} --- made available to other executable
components, along with the information about their location within the DLL.
Table entries also have numbers associated with them --- so called
{\em ordinal numbers}.

In turn, an executable file (EXE or DLL) referencing the DLL contains,
for each reference, a record with the DLL name and a name or an ordinal number of
an exported entity. When the operating system loads the executable, it
also loads all the DLLs it references, and tunes up the executable's code so that
it will call DLL procedures and access DLL variables. This is called
{\em load-time} dynamic linking.

Alternatively, an application may explicitly load a DLL at run-time using
an API call. Another call retrieves an address of a procedure or variable
exported from the loaded DLL, given its name or ordinal number. This is called
{\em run-time} dynamic linking.

See \ref{dll:using} for more information on these two types of dynamic linking.

\section{Quick start}

If you want to separate one or more sets of modules from
your existing application into DLLs, perform the following steps:

\begin{enumerate}
\item Determine the list of modules which will constitute the DLL.
      Take into account that it is impossible to import entites from an executable
      to a DLL (but quite possible from one DLL to another). % ??? cyclic import - possible?
\item Move the source files for these modules to a separate directory.
      It must differ from directories containing source files for
      the executable and the other DLLs, if any.
\item Create a project file for the DLL with the following settings:
{\samepage
\begin{verbatim}
    -GENDLL+
    -USEDLL+
    -DLLEXPORT+
    -IMPLIB-
    -LOOKUP=*.mod|*.def|*.ob2=<dll source files directory>
    !module module1
    !module module2
      .  .  .
\end{verbatim}
where \verb'module1', \verb'module2', etc. are the modules which has to be visible
outside the DLL.
} % \samepage
\item Repeat the previous steps until you set up all DLLs.
\item Add the following settings to the main project file:
\begin{verbatim}
    -USEDLL+
    -IMPLIB-
\end{verbatim}
\item Remove all the object and symbol files of your application's modules.
\item Build the DLL project(s).
\item Build the main project.
\item Run your application to make sure all things work.
\end{enumerate}


\section{Using dynamically linked run-time libraries}
\label{dll:runtime}

You may link the \XDS{} run-time library into your application
either statically or dynamically. The \OERef{USEDLL} option controls
which version of the run-time library is to be used.
By default, this option is set OFF, causing the run-time library to
be statically linked. With this setting, your application is linked
and executed exactly as it was done with Native XDS-x86 v2.2x.

If your application suite contains more than one EXE file, you may wish to
use the dynamically linked version of the run-time library, in order to
save disk space and memory.

{\bf Note:} If your application itself is dynamically linked,
it is essential to use the dynamically linked run-time library to ensure that
all executable modules use the same unique instance of the run-time library
data during execution.

See Chapter \ref{redistr} for the run-time library redistribution rights.

\section{Creating your own DLL}
\label{dll:create}

\subsection{Setting up the environment}
\label{dll:create:env}

{\bf
For your dynamically linked application to function properly,
it is essential to correctly set up the related compiler options,
so please read this section carefully.
}

First of all, the \OERef{USEDLL} compiler option has to be set ON
during compilation of each module, regardless of whether it
will be linked into an EXE or a DLL, indicating that a
\See{dynamically-linked version of the run-time library}{}{dll:runtime}
is to be used.

The \OERef{GENDLL} compiler option must be set ON during compilation
of modules which are to be built into a DLL, and OFF for modules
which will constitute the EXE file.

The compiler stores the name of the DLL into which a
module will be linked in that module's symbol file. This information may be used
then for \See{linking without import libraries}{}{dll:using:load-time}.

In the \See{COMPILE}{}{xc:modes:compile} and
\See{MAKE}{}{xc:modes:make} modes, specify the target DLL name in the
\OERef{DLLNAME} equation.
In the \See{PROJECT mode}{}{xc:modes:project}, if \OERef{DLLNAME}
is not set, the name of the project file is assumed.

\begin{quote}{\footnotesize
    {\bf Warning:} Under Windows NT/95, the compiler uses information
    about DLL names when producing code for dynamically linked variables
    access. If your program crashes or incorrectly behaves trying to access
    such a variable, make sure that you comply with the above requirement.
    You may need to remove all symbol files or use the
    \See{ALL submode}{}{xc:modes:all}.
    } % end \footnotesize
\end{quote}

Depending on whether you want to use \See{import libraries}{}{dll:using:load-time},
you may also wish to specify the \OERef{IMPLIB} compiler option.
Use the \OERef{DLLEXPORT} option to control which objects are exported from
the DLL (see \ref{dll:create:export}).

To summarize, if you invoke the compiler in the \See{COMPILE mode}{}{xc:modes:compile},
you have to specify at least the following options on the command line:

When compiling a DLL module:

\verb'    -USEDLL+ -GENDLL+ -DLLNAME='{\it name of the DLL}

When compiling an EXE module:

\verb'    -USEDLL+ -GENDLL-'

To allow the \See{MAKE mode}{}{xc:modes:make} to be used, you have to
orgranize your source files and redirections as if you were using the
\Ref{PROJECT mode}{xc:modes:project} (see below).

If you prefer to use the compiler in the \See{PROJECT mode}{}{xc:modes:project},
create a project file for each executable component (EXE or DLL).
You have to place the {\em source files}, including \mt{}
definition modules, belonging to different executable components
into separate directories. Then, use \OERef{LOOKUP} directives
in your project files to ensure that during processing of
each project source files belonging to other projects are not
visible to the compiler via redirections.

\begin{quote}{\footnotesize
    {\bf Warning:} If you fail to comply with the above requirement,
    the make subsystem may include some object files into more than one
    executable module. Also, wrong code may be produced by the compiler
    for dynamically linked variables access in the Windows NT/95 environment.
    } % end \footnotesize
\end{quote}

In the \Ref{PROJECT mode}{xc:modes:project}, if the \OERef{DLLNAME} equation was
not set, the compiler substitutes the name of the project file (without path and extension,
of course). So you do not have to care about this setting, as long as your DLLs
are named after their project files.

Thus, project files for your DLLs have to contain at least the following lines:

\verb'    -LOOKUP=*.def|*.mod|*.ob2='{\it $<$this DLL sources directory$>$}\\
\verb'    -USEDLL+'\\
\verb'    -GENDLL+'\\
\verb'    -DLLNAME='{\it $<$name of the DLL$>$}

The last line is not required if the name of the DLL matches the project file name.

Projects for EXE files must include the following lines:

\verb'    -LOOKUP=*.def|*.mod|*.ob2='{\it $<$this EXE sources directory$>$}\\
\verb'    -USEDLL+'\\
\verb'    -GENDLL-'

\subsection{Controlling export from a DLL}
\label{dll:create:export}

Any public identifier (procedure or variable) from the object files
linked together to form the DLL may be made available
(exported) to other executable modules\footnote{Under Windows NT/95,
most compilers do not allow variables to be exported from DLLs. This
is a nearly unique feature of \XDS{}.}.
There are three ways to control which identifiers are to be exported:
\begin{itemize}
\item Using the \OERef{DLLEXPORT} compiler option, which affects generation
      of export definition records in object modules. A linker collects
      information from these records into the DLL export table.
      \XDS{} compiler produces export records for exported variables and
      procedures if \OERef{DLLEXPORT} is set ON.

      This option may be specified in a project file, in a module header,
      or even in-line, allowing you to precisely select objects to be exported
      from the DLL:

\begin{verbatim}
    <* DLLEXPORT - *>
    DEFINITION MODULE M;

    PROCEDURE Internal;
    (* This procedure is only visible inside the DLL. *)

    <* DLLEXPORT+ *>
    PROCEDURE Exported;
    (* This procedure is available to other executables. *)
    <* DLLEXPORT - *>

    END M.
\end{verbatim}

      The compiler always produces export rtecords for automatically
      generated entities, such as module intialization routines
      ({\it module}\verb'_BEGIN') and global variables containing
      descriptors of Oberon-2 types and modules.

\item Using an export definition file (EDF), which contains
      a list of public identifiers which are to be visible
      outside the DLL. It also allows you to assign ordinal numbers to
      that identifiers, and give them arbitrary external names
      (i.e. names under which these procedures
      and variables will be visible to other executable modules):

\begin{verbatim}
    LIBRARY MyDLL
    DESCRIPTION 'My DLL'
    EXPORTS
        Exported=M_Exported @1
        M_BEGIN             @2
\end{verbatim}

      Use the \LORef{DLL} linker option to pass the name of the
      EDF:

\verb'    xlink . . . /dll=mydll.edf'

      You may switch to this method at any time.
      Once you have built a DLL without an EDF, you may use the
      \Ref{XDS Library Manager}{xlib} to produce it
      (see \ref{xlib:modes:export}).
      Next time you link the DLL, you may use this file,
      or a modified version of it.

% !!! What about automatically generated entities?

\item Using the \LORef{EXPORT} linker option:

\verb'    xlink . . . /EXPORT=Exported.1=M_Exported,M_BEGIN.2'

      This method has the same capabilities as export definition files.
\end{itemize}

{\bf Note:} If you supply an EDF file containing an \See{EXPORT section}{}{link:EDF:syntax}
to the XDS Linker, export definition records in object files will be ignored,
so the \OERef{DLLEXPORT} option will have no effect. The export infomation supplied
via the \LORef{EXPORT} linker option is always used, but if it conflicts
with .EDF file settings, a warning will be issued, and .EDF file settings will be used.

The advantages and disadvantages of the three methods are discussed below.

Using the \OERef{DLLEXPORT} compiler option, you do not have to introduce an
extra entity (export definition files) or care about linker options;
export information is specified right in the source text,
so it never goes out of sync, and export is performed
transparently.
However, in this case export can only be performed by name;
ordinal numbers may change from one link to another, so clients
of the DLL should not rely on ordinal numbers. You may not use name mapping:
internal and external names are always the same.
Finally, if you link any third-party object files or libraries not containing export
definiton records into your DLL, you will not be able to export any entities
from these files outside the DLL.

EDF usage involves some extra work, but gives you full control
on ordinal numbers assignment and mapping of internal
names specified in object files to arbitrary external names.

The \LORef{EXPORT} linker option is rarely used, as the same result
may be achieved via EDF. You may, however, consider using it if your
DLL exports just a few entities.

In general, we would recommend to use \OERef{DLLEXPORT} during initial development
of the DLL and switch to EDF before shipping, so that you will be able to
update just the DLL without risk of compatibility problems.

\subsection{DLL initialization and termination}
\label{dll:create:initterm}

The operating system transfers control to the DLL entry point right after
it is loaded and right before it is unloaded, in order to enable the DLL to carry out
the required initialization or cleanup\footnote{Under Windows NT/95, this also happens
when a thread is about to be started or terminated in the application which uses the DLL.
DLLs built with the XDS startup code ignore these calls.}.

The default XDS DLL startup code performs all the actions required for the DLL
to function properly in the XDS run-time environment.
In addition, if there is a main module\footnote{A \mt{} program module or an \ot{} module
compiled with the \OERef{MAIN} option set ON.} linked into the DLL,
its initialization and finalization bodies will be
called after the DLL is loaded and before it is unloaded, respectively.
This feature provides a convenient way to add your own DLL initialization and
termination code and also allows run-time linked DLLs to be properly initialized
(see \ref{dll:using:run-time}).

\section{Using your DLL}
\label{dll:using}

Once you have built a DLL, you may reference the code and data contained in it
from an executable module or another DLL.

There are two types of dynamic linking: load-time and run-time, briefly
introduced in section \ref{dll:overview}.

{\em Load-time dynamic linking} is similar to static linking in that
you may reference procedures and variables in your source code regardless
of whether they will reside in the same component (EXE or DLL) or not.
Then you link the component which uses DLLs with an {\em import library},
which contains no code or data but references to DLLs where the particular
procedures or variables are located.
The resulting executable module or DLL incorporates these references,
which are resolved by the operating system when it loads your application.
The referenced DLLs are loaded into memory and stay there until your
application terminates.
If the operating system fails to resolve any of the references, it
stops loading the application and returns an error.

{\em Run-time dynamic linking} involves explicit loading of DLLs using
an operating system API call. Addresses of the procedures and variables
in the loaded DLL may only be retrieved by their names via API calls,
with no type checking. However, linking at run-time has certain advantages:

\begin{itemize}
\item DLLs may be loaded when they are needed; after the application
      finishes using a DLL, it may unload that DLL and free some memory.
\item Error recovery is possible. A missing or incorrect load-time DLL
      prevents the application from being started by the operating system.
      In case of run-time loading, the API call will return an error
      value, enabling the application to analyze it and take an
      appropriate action.
\item Names of DLLs, procedures, and variables are not hard-coded into
      the application and may be obtained from the registry,
      a configuration file, or by prompting the user at run-time.
      This, for instance, allows several DLLs with the same interface
      but different functionality to co-exist in a single application.
\item DLLs loaded at run-time may reside in any directory, whereas load-time
      DLLs are located by the operating system in a certain set
      of directories.
\end{itemize}

\subsection{Load-time dynamic linking}
\label{dll:using:load-time}

In \XDS{}, there are two ways to link with a DLL: one involves usage of import libraries,
which may only be built on a separate step, while another is based on the XDS compiler
ability to embed import definition sections in object files and thus is more convenient.
However, in some cases, discussed below, import libraries have to be used. You may use
both methods simultaneously.

The \OERef{IMPLIB} option controls whether the currently compiled project
may be linked without import libraries:

\begin{itemize}
\item If the \OERef{IMPLIB} option is set ON, the compiler expects that
      import libraries for all DLLs used by the currently compiled source
      will be available at link time. You have to use \Ref{XLIB}{xlib}
      or another library manager to create the necessary import libraries
      (see \ref{xlib:modes:import} and then specify their names to the linker.
      If you are taking advantage of
      \See{the XDS project subsystem}{}{xc:project},
      you may list them in the project file using the
      \Ref{{\tt !module} directive}{xc:project}, provided that
      the \Ref{template file}{xc:template} has appropriate support.

      In this case, import will be performed either
      by name or by ordinal (see \ref{dll:overview}),
      depending on how the import library was built.

\item If the \OERef{IMPLIB} option is set OFF, the compiler uses information
      from symbol files of the imported modules to create import definition
      records in the output object files.
      These sections will then be used by the linker to resolve references to
      code and data contained in separate DLLs.

      As symbol files do not contain ordinal numbers,
      import can only be performed \See{{\em by name}}{}{dll:overview}.

      This method requires that during creation of a symbol file the
      compiler knew the name of the DLL into which
      the correspondent object file was is linked (see \ref{dll:create:env}).
      The compiler then uses that name in import definition records.
\end{itemize}

Importing by ordinal speeds up program load,
but requires extra efforts to preserve ordinal numbers when the DLL is updated.
Win32 API function ordinals differ even between ServicePaks.

\subsection{Run-time dynamic linking}
\label{dll:using:run-time}

You may directly use Win32 API functions to handle the
DLLs your application needs to load at run-time. You may also consider
usage of the XDS run-time library calls to ensure portability.

The library module \verb'dllRTS' provides means for loading and
unloading DLLs, and obtaining addresses of procedures and data
contained in loaded DLLs. Some other useful functions are available.

If the DLL is linked at load-time, initialization and
finalization of all modules exported from it and used by other
load-time linked components is performed automatically as
if it was linked statically. In case of run-time linking,
the operating system only executes DLL startup code (see \ref{dll:create:initterm}).
You have the following options:

\begin{itemize}
\item Leave the DLL modules uninitialized.
\item After the DLL is loaded, retrieve the address of the initialization
      routine of each module from which you plan to use procedures
      or variables, and call it. These routines have type \verb'PROC' and their
      names are formed as \verb'module_BEGIN':

\begin{verbatim}
    VAR
      hmod: dllRTS.HMOD;
      init: PROC;
     .  .  .
      hmod := dllRTS.LoadModule("MyDLL");
      init := dllRTS.GetProcAdr(hmod, "MyModule_BEGIN");
      init;
     .  .  .
\end{verbatim}

\item Create a main module which imports all modules from the DLL and add it
      to the DLL. This would also allow you to write your own
      \See{DLL initialization and termination code}{}{dll:create:initterm}.
\begin{verbatim}
    MODULE MyDLLInitTerm;
    IMPORT MyModule;
      .  .  .
    BEGIN
      (* DLL initialization *)
      .  .  .
    FINALLY
      (* DLL termination *)
      .  .  .
    END MyDLLInitTerm.
\end{verbatim}
\end{itemize}

\section{DLLs intended for a foreign environment}
\label{dll:foreign}

If you develop a product which is a single DLL or a set of DLLs expected to be
used by third-party applications (e.g. a Netscape plug-in or a REXX external
function package), you have to ensure that the \mt{} and \ot{} modules which
constitute your DLL(s) are initialized. See \ref{dll:create:initterm} and
\ref{dll:using:run-time} for more information.

You also have to specify a different calling and naming convention for the procedures
exported from your DLL, which are to be called by a third-party application.
See \ref{multilang:direct} for more information on interfacing to foreign languages.
In general, unless your DLL will be used with a particular application,
avoid using the \verb'"C"' convention, as it depends slightly on the \OERef{CC} equation
setting, thus bounding your product to a particular C compiler.
Instead, use \verb'"StdCall"' on Windows.

If your product consists of exactly one DLL, you may wish to statically
link it with the \XDS{} run-time library, so that you don't have to redistribute
the dynamically linked version of the run-time library with your product.
This is the only case when you may have the \OERef{USEDLL} option switched OFF
while a DLL is built (\OERef{GENDLL} in ON).

\ifcomment
              DLL-support in the native XDS-compiler
              ---------------------------------------

    1. User level

      1.1 New XDS entity - Export DeFinition file

          Текстовый файл состоящий из разделов

             DESCRIPTION text

             LIBRARY     name

             EXPORTS     name[=internal_name] [@ordinal]

             ; comment

          Позволяет описывать интерфейс DLL, переименовывать точки входа, а также
          фиксировать нумерацию.

      1.2 Tools ( linker, xlib, his )

          linker
          ------

          Теперь умеет собирать DLL ( а не просто файл с расширением .dll как раньше )
          в исполняемых форматах LX и PE ( возможен cross-linking ). Сборка может
          осуществлятся с EDF-файлом.

          xlib
          ----

          Умеет порождать библиотеки импорта и EDF-файлы по LX и PE DLL'ищам

          his
          ---

          Работает в случае многокомпонентной программы тоже


      1.3 DLL-version of the XDS RT

          При указании опции -USEDLL+ будет использована DLL-версия XDS RT.
          Указание опции USEDLL- приводит к тому, что RT будет влинковываться, а поведение
          компилятора не будет отличаться от того, которое было до внесения DLL'ных возможностей.
          В частности SYM-файл, порожденный с опцией GENDLL+ будет восприниматься корректно
          ( если некий набор модулей оформлен и как LIB, и как DLL, нет нужды держать для них
            различные версии SYM'ов ).

            Имеется две XDS RT DLL : singlethread & multithread. Подцепление нужной
          ( библиотеки импорта ) рулится опцией MULTITHREAD.


      1.4 How to make DLL

          Если в проекте указать опцию -GENDLL+, то соберется DLL, имя которой либо
          совпадает с именем проекта, либо задается через эквшн dllname. Собирать
          DLL в режиме "make" или компилировать по частям через "compile" возможно
          только с эквшном dllname ( при этом на пользователе лежит отвественность за
          соблюдение правила компилировать разные части одной DLL с одним и тем же
          значением dllname ).

          initialization/finalization
          ---------------------------

          Если в списке модулей проекта есть главный ( M2 - MODULE xx, O2 - <* +MAIN *> ),
          то его BEGIN-часть становится инициализацией DLL, FINALLY-часть - финализацией.
          Иниц-я зовется в случае присоединения к процессу, финал-я в случае отсоединения
          от. Случаи thread attachment/detachment игнорируются.

          data segment mode
          -----------------

          Для каждого процесса создается свой экземпляр сегмента данных ( MULTIPLE,
          NONSHARED мода ).


          export
          ------
          При указании в проекте опции -DLLEXPORT+ экспортируются интерфейсы модулей
          целиком : все об'екты из DEFINITION'ов и *-помеченные из O2-модулей кроме
          констант. Возможен выборочный экспорт: если в проекте указать -DLLEXPORT-,
          а в модулях писать прагмы, то экспортироваться будут об'екты, описанные
          между <* -DLLEXPORT+ *> и <* -DLLEXPORT- *>.


          dll intented for a foreign environment
          --------------------------------------

          Если нужно собрать DLL и влинковать туда XDS RT, достаточно указать в проекте
          опцию USEDLL-, однако все заботы относительно naming & calling conventions
          ложатся на пользователя.
          WARNING: Это единственный случай, при котором разрешается собирать DLL c
          опцией USEDLL- !

          Тот кто создает такую DLL должен позаботится о том, чтобы вызывались BEGIN-части
          модулей, входящих в DLL. Варианта здесь два:

            1) Добавить в DLL главный модуль ( если его там не было ) и написать список
               импорта, включив туда все модули, входящие в DLL.

            2) Самому ( в нужное время и в нужном месте ) вызвать процедуру, соответствующую
               BEGIN-части модуля: ее имя получаетася из имени модуля путем добавления
               постфикса "_BEGIN". Параметров она не имеет. Все такие процедуры всегда
               экпортируются из DLL. Всю заботу о финализации модулей берет на себя XDS RT,
               так что в этом направлении никаких специальных усилий предпринимать не нужно.


      1.5 How to use DLL

          Run-time dynalinking
          --------------------

             В библиотеку добавлен модуль dllRTS, позволяющий подгружать/выгружать
          DLL, доставать из них адреса процедур и переменных. Кроме того, этот модуль
          предоставляет некоторый сервис, доступный через API ( получить полный
          путь по хэндлу, хэндл по имени ).
             Если возникает необходимость того, чтобы BEGIN-части модулей вызывались
          в момент загрузки DLL в run-time моде (через dllRTS), то нужно включить в DLL
          главный модуль, а в нем написать IMPORT list_of_modules.


          Load-time dynalinking
          ---------------------

             i.  with import libraries

                 Компилировать DLL и EXE нужно с опцией IMPLIB+.
                 После сотворения DLL'ины нужно напустить на нее xlib с целью
                 порождения библиотеки импорта, после чего включить эту библиотеку
                 в список модулей проекта того EXE'шника, который собирается
                 использовать эту DLL. Если интерфейс DLL меняется, то библиотеку
                 импорта нужно породить заново. В остальном использование модулей из DLL
                 (внешне) ничем не отличается от использования модулей, влинковываемых
                 в EXE.


             ii. without import libraries

                 Компилировать DLL и EXE нужно с опцией IMPLIB-.
                 Все остальное получается само.

                 WARNING: Если до того эта DLL собиралась с опцией IMPLIB+,
                 нужно потереть все ее SYM'ы.


             WARNING: по redirection'у (lookup'ам) проекта EXE'шника, не должны
             находится OBJ-файлы от DLL'ей которые он использует. Возможно это не лучший
             вариант, но на нем решили остановиться, т.к все остальное, что приходит в голову
             еще более коряво по причинам внутреннего характера ( зависимость проектной ситемы
             от back-end'а ).



\fi
