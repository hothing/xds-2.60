\chapter{\XDS{} Модула-2}\label{m2}\index{Модула-2}

В этой главе описаны детали языка
\mt{}, специфические для данной реализации. В стандартном режиме
\footnote{Когда выключены опции \OERef{M2EXTENSIONS} и \OERef{M2ADDTYPES}} 
\xds{} \mt{} соответствует Международному Стандарту ISO
(см.в \ref{m2:ISO} декларацию совместимости и некоторые ограничения).
Правила совместимости типов изложены в \ref{m2:ISO:comp}. 
В разделе \ref{m2:ISO:features} описаны различия между 
ISO \mt{} и языком, описанным в 4-м издании книги Н.Вирта
"Программирование на Модуле-2" \cite{PIM}.
Расширения языка описаны в \ref{m2:ext}.

\section{Соответствие ISO стандарту}\label{m2:ISO}

\xds{} \mt{} почти полностью соответствует требованиям стандарта
ISO 10514. Несоответствия следующие:
\begin{itemize}
\item Текущая версия содержит не все библиотеки.
\item В текущей версии есть некоторые ограничения при использовании
новых языковых конструкций.
\end{itemize}
Подробно об этом в см. в главе \ref{limits}.

\subsection{Порядок деклараций}\label{m2:dcl:order}

\xds{} компилятор с языка \mt{} --- однопроходный.
Это, в частности, означает, что все идентификаторы должны
быть описаны перед использованием.
Такой подход ({\em описание-перед-использованием}) вполне
согласуется с Международным Стандартом.
В многопроходных реализациях может быть использован
альтернативный подход 
({\em описание-перед-использованием-в-описаниях}).

Для ссылок на процедуру, описание которой находится в тексте позже,
необходимо использовать описание FORWARD.

\Example
\begin{verbatim}
PROCEDURE a(x: INTEGER); FORWARD;
(* FORWARD-описание *)

PROCEDURE b(x: INTEGER);
BEGIN
  a(x-1);
END b;

PROCEDURE a(n: INTEGER);
(* собственно описание процедуры *)
BEGIN
  b(n-1);
END a;
\end{verbatim}

Чтобы однопроходные и многопроходные реализации были совместимы
по текстам, стандарт требует, чтобы многопроходные реализации
воспринимали и корректно обрабатывали директиву
FORWARD.

%\section{Черты, обусловленные реализацией}
%
% Значение пустой строки --- 0C.

\section{Новые черты языка}\label{m2:ISO:features}

Язык, описанный в Международном Стандарте, во многом отличается
от языка, описанного в 4-м издании книги Н.Вирта
"Программирование на Модуле-2" \cite{PIM}. Важнейшие 
нововведения --- следующие:
\begin{itemize}
\item комплексные числа
\item финализация модулей
\item обработка исключений
\item конструкторы массивов и записей
\item четыре новых системных модуля
\item стандартные библиотеки
\end{itemize}

{\bf Замечание:} Системные модули (кроме модуля {\tt SYSTEM})
не включены в компилятор и обрабатываются как отдельные модули.

\subsection{Лексика}

В ISO \mt{} добавлены новые ключевые слова
\ifonline\else (таблица \ref{table:m2:ISO:keywords},
стр. \pageref{table:m2:ISO:keywords})\fi
и предзаданные идентификаторы
\ifonline\else (таблица \ref{table:m2:ISO:pervasive},
стр. \pageref{table:m2:ISO:pervasive})\fi,
а также альтернативные представления некоторых символов
\ifonline\else(таблица \ref{table:m2:ISO:alt},
стр. \pageref{table:m2:ISO:alt})\fi.
Также зафиксирован синтаксис директив, включенных в текст (или прагм):
\begin{verbatim}
pragma = "<*" pragma_body "*>"
\end{verbatim}
Стандарт не фиксирует синтаксиса для \verb|pragma_body|.
В \XDS{} прагмы используются для установки параметров компилятора
и для контроля за исходным текстом (условная компиляция).
Подробно см. в \ref{m2:pragmas:inline}.

\begin{table}[bht]
\footnotesize
\begin{center}
\begin{tabular}{p{3.4cm}p{3.4cm}p{3.4cm}}
AND             & ARRAY                     &       BEGIN           \\
BY              & CASE                      &       CONST           \\
DEFINITION      & DIV                       &       DO              \\
ELSE            & ELSIF                     &       END             \\
EXIT            & EXCEPT (\ref{m2:ISO:exc}) &       EXPORT          \\
FINALLY (\ref{m2:ISO:final}) & FOR          &       FORWARD (\ref{m2:dcl:order})  \\
FROM            & IF                        &       IMPLEMENTATION  \\
IMPORT          & IN                        &       LOOP            \\
MOD             & MODULE                    &       NOT             \\
OF              & OR                        &       PACKEDSET (\ref{m2:ISO:sets}) \\
PROCEDURE       & QUALIFIED                 &       RECORD          \\
REM (\ref{m2:ISO:rem})  & RETRY (\ref{m2:ISO:exc}) &       REPEAT          \\
RETURN          & SET                       &       THEN            \\
TO              & TYPE                      &       UNTIL           \\
VAR             & WHILE                     &       WITH            \\
\end{tabular}
\end{center}
\caption{Ключевые слова Модулы-2}\label{table:m2:ISO:keywords}
\end{table}

\begin{table}[hbtp] \tt
\footnotesize
\begin{center}
\begin{tabular}{p{5.0cm}p{5.0cm}}
ABS             & BITSET    \\
BOOLEAN         & CARDINAL  \\
CAP             & CHR             \\
CHAR            & COMPLEX (\ref{m2:ISO:complex})  \\
CMPLX (\ref{m2:ISO:complex})  &  DEC             \\
DISPOSE         & EXCL            \\
FALSE           & FLOAT     \\
HALT            & HIGH      \\
IM (\ref{m2:ISO:complex})       & INC             \\
INCL            & INT (\ref{m2:ISO:conv})      \\
INTERRUPTIBLE (\ref{m2:ISO:prot})  & INTEGER   \\
LENGTH (\ref{m2:ISO:strings})   & LFLOAT (\ref{m2:ISO:conv})         \\
LONGCOMPLEX (\ref{m2:ISO:complex})    & LONGREAL  \\
MAX             & MIN             \\
NEW             & NIL             \\
ODD             & ORD             \\
PROC            & PROTECTION (\ref{m2:ISO:prot})     \\
RE (\ref{m2:ISO:complex})       & REAL            \\
SIZE            & TRUE      \\
TRUNC           & UNINTERRUPTIBLE (\ref{m2:ISO:prot}) \\
VAL              \\
\end{tabular}
\end{center}
\caption{Предзаданные идентификаторы Модулы-2}\label{table:m2:ISO:pervasive}
\end{table}

\begin{table}[htbp]
\begin{center}
\begin{tabular}{|c|l|c|}
\hline
\bf Символ  & \bf Значение   & \bf Альтернатива  \\ \hline
  \verb+[+  & откр. квадратная скобка  & \verb+(!+        \\
  \verb+]+  & закр. квадратная скобка  & \verb+!)+        \\
  \verb+{+  & откр. фигурная скобка    & \verb+(:+        \\
  \verb+}+  & закр. фигурная скобка    & \verb+:)+        \\
  \verb+|+  & разделитель в CASE       & \verb+!+         \\
  \verb+^+  & разыменование            & \verb+@+         \\
\hline
\end{tabular}
\end{center}
\caption{Альтернативное написание символов в Модуле-2}\label{table:m2:ISO:alt}
\end{table}

\subsection{Комплексные типы}\label{m2:ISO:complex}
\index{Модула-2!комплексные типы}
\index{COMPLEX}\index{комплексные числа}

Типы COMPLEX и LONGCOMPLEX используются для представления комплексных чисел.
Они различаются областью значений и точностью.
Тип COMPLEX определен как пара (REAL,REAL), а LONGCOMPLEX ---
как пара значений типа LONGREAL.

Специального обозначения для комплексных констант не вводится.
Комплексное значение выдает стандартная функция
{\tt CMPLX} с двумя вещественными аргументами.
Если оба аргумента у {\tt CMPLX} --- вещественные константы, то
результат --- комплексная константа.
\begin{verbatim}
CONST i = CMPLX(0.0,1.0);
\end{verbatim}

Если оба аргумента типа REAL, или один типа
REAL, а другой --- вещественная константа, то результат
имеет тип COMPLEX. 
Если оба аргумента типа LONGREAL, или один типа
LONGREAL, а другой --- вещественная константа, то результат имеет тип
LONGCOMPLEX.
Все допустимые типы аргументов и тип результата приведены в таблице:
\begin{center}
\small \tt
\begin{tabular}{|l|lll|}\hline
              & REAL     & LONGREAL    & вещ.константа    \\ \hline
REAL          & REAL     & ошибка      & COMPLEX          \\
LONGREAL      & ошибка   & LONGCOMPLEX & LONGCOMPLEX      \\
вещ.константа & COMPLEX  & LONGCOMPLEX & компл.константа \\ \hline
\end{tabular}
\end{center}

Стандартные функции {\tt RE} и {\tt IM} выдают вещественную и 
мнимую часть значения комплексного типа. Обе функции имеют один
аргумент. Если он типа COMPLEX, то функции возвращают результат типа
REAL; если аргумент типа LONGCOMPLEX, то результат --- типа
LONGREAL; если аргумент --- комплексная константа, то результат ---
вещественная константа.
\begin{verbatim}
CONST one = IM(CMPLX(0.0,1.0));
\end{verbatim}

Имеется четыре арифметических операции с операндами комплексного типа:
сложение (\verb|+|), вычитание (\verb|-|), умножение (\verb|*|) и 
деление (\verb|/|). 
Возможные комбинации типов операндов и тип результата 
приведены в таблице:
\begin{center}
\small \tt
\begin{tabular}{|l|lll|}\hline
                 & COMPLEX  & LONGCOMPLEX & компл. константа \\ \hline
COMPLEX          & COMPLEX  & ошибка      & COMPLEX          \\
LONGCOMPLEX      & ошибка   & LONGCOMPLEX & LONGCOMPLEX      \\
компл. константа & COMPLEX  & LONGCOMPLEX & компл. константа \\ \hline
\end{tabular}
\end{center}

Имеется две унарных операции с операндом комплексного типа:
сложение (\verb|+|) и вычитание (\verb|-|). Тип результата 
совпадает с типом операнда.

Для операндов комплексного типа имеется две операции сравнения:
равенство и неравенство.

\Example
\begin{verbatim}
PROCEDURE abs(z: COMPLEX): REAL;
BEGIN
  RETURN RealMath.sqrt(RE(z)*RE(z)+IM(z)*IM(z))
END abs;
\end{verbatim}

\subsection{Множества и упакованные множества}\label{m2:ISO:sets}
\index{Модула-2!PACKEDSET}

Множество и упакованное множество (упакмножество) 
\footnote{Тип "упакмножество" -- нововведение стандарта.} 
это типы, множество значений которых есть множество всех
подмножеств ассоциированного натурального типа, %??? ordinal
называемого {\em базовым типом} типа множества.
\begin{verbatim}
SetType       = SET OF Type;
PackedsetType = PACKEDSET OF Type;
\end{verbatim}

Международный Стандарт не оговаривает специального представления для
типов множества. Обычно упакмножество представляется последовательностью 
битов, зависящей от данной архитектуры. Стандартный тип 
BITSET есть предопределенный тип упакмножества.

В текущей реализации \xds{} не делает различий между типами
множества и упакмножества. Базовый тип должен иметь не более
256 элементов. 

\Example
\begin{verbatim}
TYPE
  CharSet = SET OF CHAR;
  LongSet = PACKEDSET OF [-127..128];
\end{verbatim}

Все операции над множествами, такие, как объединение
(\verb|+|), разность (\verb|-|),
пересечение (\verb|*|) и симметрическая разность (\verb|/|),
применимы к аргументам типов множества и упакмножества.

\Example
\begin{verbatim}
   VAR letters, digits, alphanum: CharSet;
   ...
   letters  := CharSet{'a'..'z','A'..'Z'};
   digits   := CharSet{'0'..'9'};
   alphanum := letters + digits;
\end{verbatim}

\subsection{Строки}\label{m2:ISO:strings}
\index{Модула-2!конкатенация строк}

Для операндов --- константных 
строк определена операция конкатенации,
обозначаемая символом \verb|"+"|. 

{\bf Замечание:} константный номер символа
(напр. 15C) рассматривается как константная строка длины 1.
Пустая строка совместима с типом
CHAR и имеет значение, равное символу конца строки (0C).

\Example
\begin{verbatim}
CONST
  CR = 15C;
  LF = 12C;
  LineEnd = CR + LF;
  Greeting = "hello " + "world" + LineEnd;
\end{verbatim}

Введена новая стандартная функция LENGTH, выдающая длину аргумента
строкового типа.

\begin{verbatim}
PROCEDURE LENGTH(s: ARRAY OF CHAR): CARDINAL;
\end{verbatim}

\subsection{Конструкторы}\label{m2:ISO:aggregates}
\index{Модула-2!конструкторы}
\index{конструкторы}

Конструктор --- это выражение, задающее значение типа массива,
записи или множества. В конструкторах массивов и записей
задается список значений, т.наз. {\em структурных компонент}, 
определяющих значения всех элементов массива или всех полей записи.
Конструктор множества задает список всех элементов этого множества.

\begin{verbatim}
ValueConstructor = ArrayValue
                 | RecordValue
                 | SetValue.
ArrayValue = TypeIdentifier "{"
             ArrayComponent { "," ArrayComponent }
             "}".
ArrayComponent = Component [ BY RepeatCount ].
Component   = Expression.
RepeatCount = ConstExpression.
RecordValue = TypeIdentifier "{"
             Component { "," Component }
             "}".
\end{verbatim}
Конструкторы множеств описаны в PIM.

Общее число компонент конструктора массива (с учетом повторений)
должно быть равно количеству элементов массива. Каждая компонента
должна быть совместима по присваиванию с базовым типом массива.

Общее число компонент конструктора записи
должно быть равно количеству полей записи. Каждая компонента
должна быть совместима по присваиванию с типом соответствующего поля.

Имеются дополнительные правила для конструктора вариантной записи.
Если $n$-ое поле записи - это поле селектора варианта, %??? tag field
то $n$-ая компонента конструктора должна быть константным выражением.
Если у данного селектора нет вариантной части ELSE,
то должен существовать вариант, ассоциированный со значением 
этого выражения. Если варианта, ассоциированного со значением, 
нет, то последовательность компонент должна содержать значения полей
варианта ELSE.

Компоненты конструктора сами могут содержать списки элементов.
Такие вложенные конструкторы могут не содержать идентификатора типа.
Это послабление необходимо для многомерных массивов, в которых
типы внутренних компонент могут быть анонимными.

\Examples
\begin{verbatim}
TYPE
  String = ARRAY [0..15] OF CHAR;
  Person = RECORD
    name: String;
    age : CARDINAL;
  END;
  Vector = ARRAY [0..2] OF INTEGER;
  Matrix = ARRAY [0..2] OF Vector;

VAR
  string: String;
  person: Person;
  vector: Vector;
  matrix: Matrix;
....
  string:=String{" " BY 16};
  person:=Person{"Alex",32};
  vector:=Vector{1,2,3};
  matrix:=Matrix{vector,{4,5,6},Vector{7,8,9}};
  matrix:=Matrix{vector BY 3};
\end{verbatim}

\subsection{Многомерные открытые массивы}
\index{Модула-2!открытые массивы}

В соответствии с Международным стандартом, параметром процедуры
может быть многомерный открытый массив.

\Example
\begin{verbatim}
PROCEDURE Foo(VAR matrix: ARRAY OF ARRAY OF REAL);
  VAR i,j: CARDINAL;
BEGIN
  FOR i:=0 TO HIGH(matrix)
    FOR j:=0 TO HIGH(matrix[i])
      ... matrix[i,j] ...
    END;
  END;
END Foo;

VAR a: ARRAY [0..2],[0..2] OF REAL;

BEGIN
  Foo(a);
END ...
\end{verbatim}

\subsection{Описания типов процедур}\label{m2:ISO:proc_dcl}

Идентификатор типа процедуры можно использовать в его 
собственном описании. Эта возможность используется в 
Стандартных Библиотеках. Примеры можно найти в модулях
{\tt ConvTypes} и {\tt WholeConv}.

\Example
\begin{verbatim}
TYPE
  Scan = PROCEDURE (CHAR; VAR Scan);
  Func = PROCEDURE (INTEGER): Func;
\end{verbatim}

\subsection{Процедурные константы}\label{m2:ISO:proc_const}

Константное выражение может содержать значения процедурных типов,
или структурные значения с компонентами процедурных типов.
Процедурные константы можно использовать как способ переименования 
процедур. В определяющем модуле можно экспортировать переименованную
версию импортируемой процедуры.

\Examples
\begin{verbatim}
TYPE ProcTable = ARRAY [0..3] OF PROC;

CONST
  WS = STextIO.WriteString;
  Table = ProcTable{Up,Down,Left,Right};
\end{verbatim}

\subsection{Целое деление}\label{m2:ISO:rem}

В добавление к {\tt DIV} и {\tt MOD} Международный Стандарт
включает еще два оператора целого деления:
`\verb+/+' и {\tt REM}.

Operators 
Операции {\tt DIV} и {\tt MOD} определены только для положительного
делителя, тогда как `\verb+/+' и {\tt REM} можно использовать и
с положительным, и с отрицательным делителем.

Исключение {\tt wholeDivException} (см. \ref{m2:ISO:exc})
возбуждается, если:
\begin{itemize}
\item второй операнд равен 0 (для всех четырех операций
\item второй операнд {\tt DIV} или {\tt MOD} отрицателен.
\end{itemize}

При данных значениях {\tt lval} и {\tt rval} и при
\begin{verbatim}
  quotient  := lval / rval;
  remainder := lval REM rval;
\end{verbatim}
верны следующие утверждения (для всех ненулевых значений {\tt rval}):
\begin{itemize}
\item \verb'lval = rval * quotient + remainder'

\item {\tt remainder} либо равен 0, либо имеет тот же знак, что
{\tt lval} и абсолютное значение меньше, чем абсолютное значение 
{\tt rval}.
\end{itemize}

При данных значениях {\tt lval} и {\tt rval} и при
\begin{verbatim}
  quotient := lval DIV rval;
  modulus  := lval MOD rval;
\end{verbatim}
верны следующие утверждения (для всех положительных значений {\tt rval}):
\begin{itemize}
\item \verb'lval = rval * quotient + modulus'

\item {\tt modulus} --- неотрицательное целое число, меньшее, чем {\tt rval}.
\end{itemize}

Примеры применения операций деления:
\begin{center}
\begin{tabular}{|c|c|c|c|c|} \hline
{\em op} & 31 {\em op} 10 & 31 {\em op} (-10) & (-31) {\em op} 10 & (-31) {\em op} (-10) \\ \hline
\verb+/+ &  3  & -3         &  -3  & 3          \\ \hline
\tt REM  &  1  & 1          &  -1  & -1         \\ \hline
\tt DIV  &  3  & exception  &  -4  & exception  \\ \hline
\tt MOD  &  1  & exception  &   9  & exception  \\ \hline
\end{tabular}
\end{center}

\subsection{Преобразования тиров}\label{m2:ISO:conv}

В язык включены следующие функции преобразования типов:
{\tt CHR}, {\tt FLOAT}, {\tt INT}, {\tt LFLOAT}, {\tt
ORD}, {\tt TRUNC} и {\tt VAL}. Функции {\tt INT} и {\tt
LFLOAT} не описаны в PIM.

Все функции преобразования типов (кроме {\tt VAL}) 
имеют один параметр и могут быть выражены через функцию {\tt VAL}.
\begin{center}
\begin{tabular}{|l|l|l|l|} \hline
\bf Имя & \bf Параметр & \bf Результат & \bf Равносильно \\ \hline
CHR      & целый           & CHAR       & VAL(CHAR,x)     \\
FLOAT    & вещ. или целый  & REAL       & VAL(REAL,x)     \\
INT      & вещ. или ординальный & INTEGER    & VAL(INTEGER,x)  \\
LFLOAT   & вещ. или целый  & LONGREAL   & VAL(LONGREAL,x) \\
ORD      & ординальный     & CARDINAL   & VAL(CARDINAL,x) \\
TRUNC    & вещ.            & CARDINAL   & VAL(CARDINAL,x) \\ \hline
\end{tabular}
\end{center}

Функция VAL выдает значение заданного скалярного типа для
выражения некоторого (вообще говоря, другого) скалярного типа.
Она имеет два параметра. Первый параметр должен быть обозначением
скалярного типа. Если это тип отрезка, то VAL возвращает значение,
имеющее базовый тип этого отрезка, иначе --- значение указанного типа.

Второй параметр должен быть выражением некоторого скалярного типа.
Должно выполняться какое-либо из следующих условий:
\begin{itemize}
\item тип результата и тип выражения совпадают;
\item оба этих типа --- целые либо вещественные;
\item один из этих типов --- целый.
\end{itemize}

В следующей таблице знаком $\surd$ отмечены разрешенные, а 
знаком $\times$ --- запрещенные сочетания этих двух типов:
\begin{center}
\begin{tabular}{|lccccc|} \hline
тип         & \multicolumn{5}{c}{тип, заданный первым параметром} \\
выражения   & целый   & вещ.     & \tt CHAR   &\tt BOOLEAN  & перечисл. \\ \hline
целый тип   & $\surd$ & $\surd$  & $\surd$  & $\surd$  & $\surd$  \\
вещ. тип    & $\surd$ & $\surd$  & $\times$ & $\times$ & $\times$ \\
\tt CHAR    & $\surd$ & $\times$ & $\surd$  & $\times$ & $\times$ \\
\tt BOOLEAN & $\surd$ & $\times$ & $\times$ & $\surd$  & $\times$ \\
перечисл.   & $\surd$ & $\times$ & $\times$ & $\times$ & $\surd$  \\
\hline
\end{tabular}
\end{center}

Если значение x лежит вне области значений типа
T при вызове {\tt VAL(T,x)}, то возбуждается исключение. 
Если {\tt x} вещественного типа, то вызовы
{\tt VAL(INTEGER,x)} и {\tt VAL(CARDINAL,x)} обрезают 
значение {\tt x} до целого.

\subsection{NEW и DISPOSE}\label{m2:ISO:NEW}
\index{Модула-2!NEW и DISPOSE}
\index{NEW (M2)}\index{DISPOSE}
\index{распределение памяти}

Стандартные процедуры NEW и DISPOSE возвращены в язык.
Вызовы NEW и DISPOSE заменяются на вызовы процедур
{\tt ALLOCATE} и {\tt DEALLOCATE}, которые при этом должны быть
видимыми в момент вызова. Компилятор проверяет соответствие 
этих подставляемых процедур ожидаемым формальным типам:
\begin{verbatim}
PROCEDURE ALLOCATE(VAR a: ADDRESS; size: CARDINAL);
PROCEDURE DEALLOCATE(VAR a: ADDRESS; size: CARDINAL);
\end{verbatim}

Как правило, используются процедуры {\tt ALLOCATE} и {\tt DEALLOCATE}
из модуля {\tt Storage}. Чтобы они стали видимыми, необходимо
включить их в список импорта:
\begin{verbatim}
FROM Storage IMPORT ALLOCATE, DEALLOCATE;
\end{verbatim}

Когда разрешено использование расширений языка, процедуры
{\tt NEW} и {\tt DISPOSE} можно применять и к динамическим массивам. 
Подробнее об этом в \ref{m2:ext:NEW}.

\subsection{Финализация}\label{m2:ISO:final}
\index{финализация}

Специальный механизм, называемый {\em финализацией},
служит для совершения некоторых действий при завершении работы программы.

Описание модуля может содержать финализационную часть, которая 
будет исполняться для статических модулей при завершении работы программы
(см. \ref{m2:ISO:termination}), и для динамических - при их
финализации.

\begin{verbatim}
ModuleBody = [ BEGIN BlockBody
             [ FINALLY BlockBody ] ] END
BlockBody  = NormalPart
             [ EXCEPT ExceptionalPart ].
NormalPart = StatementSequence.
ExceptionalPart = StatementSequence.
\end{verbatim}
{\bf Замечание:} оператор RETURN можно использовать и в {\tt BlockBody}.

\Example
\begin{verbatim}
MODULE Test;

....

VAR cid: StreamFile.ChanId;

BEGIN
  StreamFile.Open(cid,"tmp",flags,res);
  Process(cid);
FINALLY
  StreamFile.Close(cid);
END Test
\end{verbatim}

Если модуль {\tt Test} описан внутри процедуры, то
его инициализационная часть будет исполняться при вызове процедуры,
а финализационная --- после тела процедуры.

Если {\tt Test} --- статический модуль, то его финализация будет
исполнена при завершении работы программы.

В любом случае, финализации исполняются в порядке, обратном
инициализациям.

В следующем примере финализация локального модуля используется
для корректного восстановления контекста:
\begin{verbatim}
VAR state: State;

PROCEDURE Foo;

  MODULE AutoSave;
    IMPORT state, State;
    VAR save: State;
  BEGIN
    save:=state; (* save state *)
    state:=fooState;
  FINALLY
    state:=save; (* restore state *)
  END AutoSave;

BEGIN
  ... обработка ...
END Foo;
\end{verbatim}

Инициализация модуля {\tt AutoSave} будет исполнена раньше, чем
любое действие в теле процедуры
{\tt Foo}, а финализация - непосредственно перед окончанием вызова
{\tt Foo}.

\subsection{Исключения}\label{m2:ISO:exc}
\index{исключения}\index{Модула-2!исключения}

Механизм обработки исключений теперь является частью языка.
Обрабатываются как языковые, так и определенные пользователем
исключения. Специального типа исключения не вводится; исключение
задается парой из источника исключения и натурального числа
(номера исключения). 
В язык добавлены два новых ключевых слова: EXCEPT и RETRY.
Основные средства для обработки исключений содержатся в
двух модулях: {\tt EXCEPTIONS} и {\tt M2EXCEPTION}.

Модуль {\tt EXCEPTIONS} содержит средства для возбуждения и
идентификации исключений, определенных пользователем, для 
их выявления и для запросов о режиме исполнения текущей
сопрограммы.

Модуль {\tt M2EXCEPTION} содержит средства для идентификации 
возбужденных языковых исключений.

Тело процедуры, инициализация или финализация модуля могут содержать
раздел действий в режиме исключения ("исключительный раздел").
\begin{verbatim}
BlockBody  = NormalPart [ EXCEPT ExceptionalPart ].
NormalPart = StatementSequence.
ExceptionalPart = StatementSequence.
\end{verbatim}

\Example
\begin{verbatim}
PROCEDURE Div(a,b: INTEGER): INTEGER;
BEGIN
  RETURN a DIV b      (* попытка деления *)
EXCEPT
  RETURN MAX(INTEGER) (* обработка исключения *)
END Fly;
\end{verbatim}

Когда исключение возбуждено (прямо или косвенно), управление 
передается "ближайшей" (в стеке процедур) исключительной части.
Каждая сопрограмма начинает свое исполнение в нормальном режиме.
При возбуждении исключения режим переключается в режим исключения.
Если исключительной части нет, то возбуждение исключения приводит 
к завершению сопрограммы
(см. \ref{m2:ISO:termination}).

Процедура, имеющая исключительную часть, исполняется в нормальном 
режиме. После исполнения блока восстанавливается прежний режим.
Процедура без исключительной части исполняется в режиме вызвавшей ее.

Если во время исключительного режима работы вновь возбуждается 
исключение, оно передается в контекст вызова. В этом случае 
финализации локальных модулей и восстановления защиты 
(см. \ref{m2:ISO:prot}) не происходит.

В исключительной части можно использовать дополнительный оператор
RETRY. Исполнение оператора RETRY состоит в перевызове основной части 
процедуры в нормальном режиме.

При исполнении оператора RETURN в исключительной части также
восстанавливается нормальный режим.

Если в исключительной части ни RETURN, ни RETRY не исполнялись,
то происходит исключительное завершение. При этом после финализации
локальных модулей (если они есть) и восстановления защиты (если нужно)
исключение вновь будет возбуждено.

\Example
\begin{verbatim}
PROCEDURE Foo;
BEGIN
  TryFoo(...);
EXCEPT
  IF CanBeRepaired() THEN
    Repair;
    RETRY;  (* вновь исполнить нормальную часть *)
  ELSIF CanBeProcessed() THEN
    Process;
    RETURN; (* исключение обработано *)
  ELSE
    (* исключение будет перевозбуждено *)
  END;
END Foo;
\end{verbatim}

\subsection{Системный модуль EXCEPTIONS}
\index{EXCEPTIONS}
\index{системные модули!EXCEPTIONS}
\index{Модула-2!EXCEPTIONS}

Модуль {\tt EXCEPTIONS} предоставляет средства для 
возбуждения исключений и для запросов о текущем состоянии.

Определенные пользователем исключения однозначно определяются 
парой (источник исключения, номер). Когда источником определенного
пользователем исключения служит отдельный модуль, это гарантирует,
что определенные в модуле исключения нельзя напрямую возбудить
помимо него. См. напр. модуль {\tt Storage}.

\begin{verbatim}
TYPE ExceptionSource;
\end{verbatim}

\index{ExceptionSource}
\index{EXCEPTIONS!ExceptionSource}
Для определения источника исключения служат значения
скрытого типа {\tt ExceptionSource}; перед использованием
они должны быть аллоцированы.

\begin{verbatim}
TYPE ExceptionNumber = CARDINAL;
\end{verbatim}

\index{ExceptionNumber}
\index{EXCEPTIONS!ExceptionNumber}
Значения типа {\tt ExceptionNumber} служат для различения разных
исключений из одного источника.

\begin{verbatim}
PROCEDURE AllocateSource(VAR newSource: ExceptionSource);
\end{verbatim}

\index{AllocateSource}
\index{EXCEPTIONS!AllocateSource}
Эта процедура создает новое значение типа
{\tt ExceptionSource}. Обычно она вызывается при инициализации модуля,
и полученное значение используется затем во всех вызовах процедуры
{\tt RAISE}. Если новое значение не удалось создать, то возбуждается
языковое исключение {\tt exException} (см. \ref{m2:m2exception}).

\begin{verbatim}
PROCEDURE RAISE(source: ExceptionSource;
                number: ExceptionNumber;
               message: ARRAY OF CHAR);
\end{verbatim}

\index{RAISE}
\index{EXCEPTIONS!RAISE}
Процедура ассоциирует с текущим контекстом заданные значения 
источника, номера и сообщения и возбуждает исключение.

Функция {\tt CurrentNumber} выдает номер текущего исключения
из данного источника.

\begin{verbatim}
PROCEDURE CurrentNumber
   (source: ExceptionSource): ExceptionNumber;
\end{verbatim}

\index{CurrentNumber}
\index{EXCEPTIONS!CurrentNumber}
Если вызывающая сопрограмма находится в режиме исключения по 
причине возбуждения исключения из источника
{\tt source}, то процедура выдает его номер. В противном случае
возбуждается исключение.

Процедура {\tt GetMessage} выдает сообщение, переданное при
возбуждении исключения. Обычно оно содержит дополнительную
информацию об исключении.

\begin{verbatim}
PROCEDURE GetMessage(VAR text: ARRAY OF CHAR);
\end{verbatim}

\index{GetMessage}
\index{EXCEPTIONS!GetMessage}
Если вызывающая сопрограмма находится в режиме исключения, то процедура
возвращает (возможно, обрезанную) строку сообщения, ассоциированную
с текущим контекстом. Иначе, в нормальном режиме, она возвращает
пустую строку.

\begin{verbatim}
PROCEDURE IsCurrentSource
        (source: ExceptionSource): BOOLEAN;
\end{verbatim}

\index{IsCurrentSource}
\index{EXCEPTIONS!IsCurrentSource}
Если вызывающая сопрограмма находится в режиме исключения по 
причине возбуждения исключения из источника
{\tt source}, то процедура возвращает TRUE, иначе FALSE.

\begin{verbatim}
PROCEDURE IsExceptionalExecution (): BOOLEAN;
\end{verbatim}

\index{IsExceptionalExecution}
\index{EXCEPTIONS!IsExceptionalExecution}
Если вызывающая сопрограмма находится в режиме исключения по 
причине возбуждения какого-либо исключения,
то процедура возвращает TRUE, иначе FALSE.

Следующий пример иллюстрирует рекомендуемую форму библиотечного модуля
и использование процедур из модуля {\tt EXCEPTIONS}.
\begin{verbatim}
DEFINITION MODULE FooLib;

PROCEDURE Foo;
(* Может возбудить исключение Foo *)

PROCEDURE IsFooException(): BOOLEAN;
(* Возвращает TRUE, если вызвавшая процедура в
   исключительном состоянии ввиду возбуждения исключения в
   Foo, иначе FALSE.
*)

END FooLib.

IMPLEMENTATION MODULE FooLib;

IMPORT EXCEPTIONS;

VAR source: EXCEPTIONS.ExceptionSource;

PROCEDURE Foo;
BEGIN
  TryFoo(...);
  IF NOT done THEN
    EXCEPTIONS.RAISE(source,0,"Foo exception");
  END;
END Foo;

PROCEDURE IsFooException(): BOOLEAN;
BEGIN
  RETURN EXCEPTIONS.IsCurrentSource(source)
END IsLibException;

BEGIN
  EXCEPTIONS.AllocateSource(source)
END FooLib.
\end{verbatim}

Если нужно различать несколько исключений, возбуждаемых в {\tt FooLib},
то мы вводим перечислимый тип и дополнительную процедуру запроса
в определяющий модуль {\tt FooLib}:
\begin{verbatim}
TYPE FooExceptions = (fault, problem);

PROCEDURE FooException(): FooExceptions;
\end{verbatim}
Процедура {\tt FooException} может быть реализована так:
\begin{verbatim}
PROCEDURE FooException(): FooExceptions;
BEGIN
  RETURN VAL(FooExceptions,
             EXCEPTIONS.CurrentNumber(source))
END FooException;
\end{verbatim}

Следующий модуль {\tt Client} иллюстрирует использование этого
библиотечного модуля:
\begin{verbatim}
MODULE Client;

IMPORT  FooLib, EXCEPTIONS, STextIO;

PROCEDURE ReportException;
  VAR s: ARRAY [0..63] OF CHAR;
BEGIN
  EXCEPTIONS.GetMessage(s);
  STextIO.WriteString(s);
  STextIO.WriteLn;
END ReportException;

PROCEDURE TryFoo;
BEGIN
  FooLib.Foo;
EXCEPT
  IF FooLib.IsFooException() THEN
    ReportException;
    RETURN; (* исключение обработано *)
  ELSE
    (* исключение будет перевозбуждено *)
  END
END TryFoo;

END Client.
\end{verbatim}

\subsection{Системный модуль M2EXCEPTION}\label{m2:m2exception}
\index{M2EXCEPTION@{\bf M2EXCEPTION}}
\index{системные модули!M2EXCEPTION}
\index{Модула-2!M2EXCEPTION}

Системный модуль {\tt M2EXCEPTION} предоставляет возможности для 
работы с языковыми исключениями. Язык (в т.ч. системные модули)
рассматривается как один источник исключений.

Модуль экспортирует один перечислимый тип, в терминах которого 
возбуждаются языковые исключения, и две процедуры-функции.

\begin{verbatim}
TYPE
  M2Exceptions =
    (indexException,      rangeException,
     caseSelectException, invalidLocation,
     functionException,   wholeValueException,
     wholeDivException,   realValueException,
     realDivException,    complexValueException,
     complexDivException, protException,
     sysException,        coException,
     exException
    );
\end{verbatim}

\begin{verbatim}
PROCEDURE IsM2Exception(): BOOLEAN;
\end{verbatim}

\index{IsM2Exception}
\index{M2EXCEPTION!IsM2Exception}
Если текущая сопрограмма находится в режиме исключения по 
причине возбуждения языкового исключения,
то процедура возвращает TRUE, иначе FALSE.

\begin{verbatim}
PROCEDURE M2Exception(): M2Exceptions;
\end{verbatim}

\index{M2Exception}
\index{M2EXCEPTION!M2Exception}
Если текущая сопрограмма находится в режиме исключения по 
причине возбуждения языкового исключения,
то процедура выдает соответствующее значение перечислимого типа. 
В противном случае возбуждается исключение.

Ниже перечислены все языковые исключения (в алфавитном порядке)
и указаны ситуации, в которых они возбуждаются.

{\bf Замечание:} Возбуждение некоторых исключений можно
контролировать опциями компилятора (см. Главу \ref{options}).

Возбуждение некоторых исключений не обязательно, но 
может происходить на некоторых платформах
(см. Главу \ref{limits}).

\begin{description}
\item[caseSelectException] \mbox{}
\index{caseSelectException}
\index{M2EXCEPTION!caseSelectException}

Не перечисленное значение селектора оператора CASE
при отсутствии ELSE.

\item[coException] \mbox{}
\index{M2EXCEPTION!coException}

Исключение системного модуля
{\bf COROUTINES} (сопрограммы):
        \begin{itemize}
        \item RETURN из не головной сопрограммы
        \item выделенное рабочее пространство меньше требуемого минимума
(см. {\tt NEWCOROUTINE})
        \item вызывающая сопрограмма не прикреплена к источнику прерываний
(см. {\tt HANDLER})
        \item переполнение рабочего пространства
        \end{itemize}

\item[complexDivException] \mbox{}
\index{M2EXCEPTION!complexDivException}

Деление на ноль в COMPLEX выражении.

\item[complexValueException] \mbox{}
\index{M2EXCEPTION!complexValueException}

Переполнение типа COMPLEX.

\item[exException] \mbox{}
\index{M2EXCEPTION!exException}

Исключение системных модулей {\bf EXCEPTIONS} и {\bf M2EXCEPTION}:
        \begin{itemize}
        \item запрос номера исключения при нормальном режиме
              (см. {\tt CurrentNumber})
        \item запрос номера исключения из неверного источника
              (см. {\tt CurrentNumber})
        \item невозможно аллоцировать новый источник исключений
              (см. {\tt AllocateSource})

        \end{itemize}

\item[functionException] \mbox{}
\index{M2EXCEPTION!functionException}

Возвращение из функции без оператора RETURN.

\item[indexException] \mbox{}
\index{M2EXCEPTION!indexException}

Выход за границы массива. См. опции
\OERef{CHECKINDEX} и \OERef{CHECKDINDEX}.

\item[invalidLocation] \mbox{}
\index{M2EXCEPTION!invalidLocation}

Попытка разыменования NIL или непроинициализированного указателя.
См. опцию \OERef{CHECKNIL}.

\item[protException] \mbox{}
\index{M2EXCEPTION!protException}

Требуемый уровень защиты менее ограничителен, чем текущий.
%??? Given protection is less restrictive than the current protection.

\item[rangeException] \mbox{}
\index{M2EXCEPTION!rangeException}

        Нарушение границ (см. опцию \OERef{CHECKRANGE}):
        \begin{itemize}
        \item присваиваемое значение выходит за границы типа
        \item значение структурной компоненты выходит за границы
        \item выражение не может быть конвертировано в данный тип
        \item включаемое/исключаемое значение вне границ базового
типа множества (см. также опцию  \OERef{CHECKSET})
        \item возвращаемое процедурой значение вне границ типа
        \item множество вне границ типа
              (см. также опцию \OERef{CHECKSET})
        \item значение тага выходит за границы (в вариантной записи)
        \end{itemize}

\item[realDivException] \mbox{}
\index{M2EXCEPTION!realDivException}

Деление на ноль в REAL выражении.

\item[realValueException] \mbox{}
\index{M2EXCEPTION!realValueException}

Переполнение типа REAL.

\item[sysException] \mbox{}
\index{M2EXCEPTION!sysException}

        Исключения системного модуля {\bf SYSTEM}.
        {\bf Замечание:} все эти исключения не обязательны.
        \begin{itemize}
        \item неверное использование {\tt ADDADR}, {\tt SUBADR} или {\tt DIFADR}
        \item результат {\tt MAKEADR} вне области адресов
        \item проблема с выравниванием в {\tt CAST}
        \item результат {\tt CAST} не является верным представлением 
целевого типа
        \end{itemize}

\item[wholeDivException] \mbox{}
\index{M2EXCEPTION!wholeDivException}

Исключение целого деления:
        \begin{itemize}
        \item деление на ноль выражении целого типа
        \item делитель в DIV или MOD отрицателен
              (см. опцию \OERef{CHECKDIV})
        \end{itemize}

\item[wholeValueException] \mbox{}
\index{M2EXCEPTION!wholeValueException}

Целое переполнение.
\end{description}

\paragraph{Пример обработки языкового исключения}
\begin{verbatim}
PROCEDURE Div(a,b: INTEGER): INTEGER;
BEGIN
  RETURN a DIV b
EXCEPT
  IF IsM2Exception() THEN
    IF M2Exception() = wholeDivException THEN
      IF a < 0 THEN RETURN MIN(INTEGER)
      ELSE          RETURN MAX(INTEGER)
      END;
    END;
  END;
END Div;
\end{verbatim}

\subsection{Завершение}\label{m2:ISO:termination}

При завершении работы программы финализаторы всех инициализированных
статических модулей вызываются в порядке, обратном порядку их
инициализации (см. также \ref{m2:ISO:final}). 
Статические модули --- это программный модуль, реализующие модули
и локальные модули, описанные в этих модулях.

Завершение программы инициируется одним из следующих событий:
\begin{enumerate}
\item конец тела головного модуля
\item оператор RETURN в теле головного модуля
\item вызов стандартной процедуры HALT
\item возбуждено и не обработано исключение
\end{enumerate}

Системный модуль {\tt TERMINATION}
\index{Модула-2!TERMINATION}
\index{системные модули!TERMINATION}
\index{TERMINATION@{\bf TERMINATION}}
позволяет делать запросы о завершающих событиях.

\begin{verbatim}
PROCEDURE IsTerminating(): BOOLEAN;
\end{verbatim}

Возвращает TRUE, если какая-либо сопрограмма запустила
завершение работы программы; иначе FALSE.

\begin{verbatim}
PROCEDURE HasHalted(): BOOLEAN;
\end{verbatim}

Возвращает TRUE, если вызвана процедура HALT; иначе FALSE.

\subsection{Сопрограммы}\label{m2:ISO:coroutines}

Системный модуль {\tt COROUTINES} предоставляет средства для
создания сопрограмм, для явной передачи управления между ними и для
обработки прерываний.
\index{системные модули!COROUTINES}
\index{Модула-2!COROUTINES}
\index{COROUTINES@{\bf COROUTINES}}

{\bf Замечание:} Некоторые описанные средства не реализованы 
в настоящей версии.
См. подробнее в Главе \ref{limits}.

Значение типа {\tt COROUTINE} создается динамически, путем вызова
процедуры {\tt NEWCOROUTINE}, и служит идентификатором сопрограммы
во всех последующих операциях. Каждая сопрограмма идентифицируется 
одним и тем же значением этого типа в течение всего времени жизни.

\begin{verbatim}
TYPE COROUTINE;
\end{verbatim}
В PIM, соответствующий тип назывался {\tt PROCESS}.
Начиная с третьего издания PIM, для идентификации
процессов использовался тип {\tt ADDRESS}.

\begin{verbatim}
PROCEDURE NEWCOROUTINE(
                procBody: PROC;
                workspace: SYSTEM.ADDRESS;
                size: CARDINAL;
                VAR cr: COROUTINE
                [; initProtection: PROTECTION]);
\end{verbatim}
Создает новую сопрограмму на основе процедуры {\tt procBody}, 
и возвращает ее идент в переменную {\tt cr}. 
{\tt workspace} --- это указатель на рабочую область, аллоцированную
для сопрограммы;
{\tt size} указывает размер этой области в единицах SYSTEM.LOC. 
{\tt initProtection} --- необязательный параметр, задающий начальный 
уровень защиты сопрограммы.

Исключение возбуждается (см. {\tt coException}), если значение 
параметра {\tt size} меньше, чем минимальный размер рабочей области.

Если параметр initProtection опущен, то начальный уровень защиты сопрограммы
полагается равным уровню вызвавшей ее сопрограммы.

Созданный сопрограмма инициализируется таким образом, чтобы
при первой передаче управления на него процедура, заданная параметром
{\tt procBody}, вызывалась в нормальном режиме. Исключение
{\tt coException} возбуждается при попытке {\tt procBody}
вернуть управление вызвавшей ее сопрограмме.
%??? Since the caller has no... поскольку? если?
Поскольку вызвавшая сопрограмма не имеет обработчика этого исключения,
его возбуждение приводит к событию завершения.

Для передачи управления между сопрограммами служит процедура {\tt TRANSFER}.

\begin{verbatim}
PROCEDURE TRANSFER (VAR from: COROUTINE; to: COROUTINE);
\end{verbatim}
Возвращает в переменную {\tt from} идент вызвавшей сопрограммы
и передает управление сопрограмме {\tt to}.

\begin{verbatim}
PROCEDURE CURRENT (): COROUTINE;
\end{verbatim}
Возвращает идент вызвавшей сопрограммы.

\subsubsection{Обработка прерываний}
\index{обработка прерываний}

Тип {\tt INTERRUPTSOURCE} служит для идентификации прерываний.
\begin{verbatim}
TYPE INTERRUPTSOURCE = INTEGER;
\end{verbatim}
Программы, использующие обработчик прерываний, могут оказаться
непереносимыми, поскольку этот тип зависит от реализации.

\begin{verbatim}
PROCEDURE ATTACH(source: INTERRUPTSOURCE);
\end{verbatim}
Ассоциирует указанный источник прерываний с вызывающей сопрограммой.
С одной и той же сопрограммой могут быть ассоциированы несколько 
источников прерываний.

\begin{verbatim}
PROCEDURE DETACH(source: INTERRUPTSOURCE);
\end{verbatim}
Отвязывает указанный источник прерываний от вызывающей сопрограммы.
Если сопрограмма не связана с указанным источником, то ничего не делает.

\begin{verbatim}
PROCEDURE IsATTACHED(source: INTERRUPTSOURCE): BOOLEAN;
\end{verbatim}
Возвращает TRUE, если указанный источник прерываний 
в момент вызова связан с вызывающей сопрограммой; иначе
возвращает FALSE.

\begin{verbatim}
PROCEDURE HANDLER(source: INTERRUPTSOURCE): COROUTINE;
\end{verbatim}
Возвращает идент сопрограммы, связанной с с указанным источником
прерываний. Если такой сопрограммы нет, то результат неопределен.

\begin{verbatim}
PROCEDURE IOTRANSFER(VAR from: COROUTINE;
                           to: COROUTINE);
\end{verbatim}

Записывает идент вызвавшей сопрограммы в переменную 
{\tt from}
и передает управление сопрограмме
{\tt to}. При возникновении прерывания, связанного с вызвавшей сопрограммой,
управление передается обратно ей, 
и в переменную {\tt from} записывается идент прерванной сопрограммы.
Если вызывающая сопрограмма не связана с источниками прерываний, то
возбуждается исключение.

\subsubsection{Защита}\label{m2:ISO:coroutines:prot}

Описание типа PROTECTION см. в \ref{m2:ISO:prot}.

\begin{verbatim}
PROCEDURE LISTEN(prot: PROTECTION);
\end{verbatim}

Меняет защиту вызвавшей сопрограммы на {\tt prot},
обычно понижая ее, чтобы позволить
удовлетворение запроса на прерывание.

\begin{verbatim}
PROCEDURE PROT(): PROTECTION;
\end{verbatim}
Выдает защиту вызвавшей сопрограммы.

\subsection{Защита}\label{m2:ISO:prot}

Заголовок любого модуля (программного, реализующего или локального)
может содержать защиту, %??? поле защиты
указывающую на то, что исполнение операторов этого модуля защищено.

\begin{verbatim}
ModuleHeading = MODULE ident [ Protection ] ";".
Protection    = [ ConstExpression ]. %??? "[" "]"
\end{verbatim}

Модуль с указанной в заголовке защитой называется явно защищенным
модулем. Явно защищенная процедура --- это процедура,
экспортируемая из явно защищенного модуля.

Защита модуля обеспечивается тем, что доступные извне процедуры
и тело модуля окружаются вызовами специальных процедур контроля
доступа. Значение параметра защиты передается этим процедурам 
как актуальный параметр.

Выражение, задающее защиту, должно иметь тип PROTECTION. 
PROTECTION --- это простой тип, имеющий как минимум два значения:
{\tt INTERRUPTIBLE} и {\tt UNINTERRUPTIBLE}.

Операторы \verb+<+, \verb+>+, \verb+<=+ и \verb+>=+ можно 
использовать для сравнения значений типа 
PROTECTION.
Любое значение $x$ типа PROTECTION удовлетворяет следующему условию:
\begin{center}
        {\tt UNINTERRUPTIBLE} $\leq x\leq$ {\tt INTERRUPTIBLE}
\end{center}

\section{Стандартные процедуры}\label{m2:stdproc}
\index{Модула-2!стандартные процедуры}
\index{стандартные процедуры!Модула-2}

\begin{table}[hbt]
\begin{center}
\begin{tabular}{|cl|p{6.5cm}|} \hline
&\bf Процедура &\bf Действие \\ \hline
$\surd$ & \tt ASSERT(x[,n]) & Завершает программу, если
                  x$\neq$TRUE (см. \ref{m2:ext:ASSERT})       \\ \index{ASSERT}
$\surd$ & \tt  COPY(x,v)     & Копирование строки: \verb|v := x|       \\ \index{COPY}
&\tt DEC(v[,n])    & \verb|v := v - n|, по умолчанию n=1             \\ \index{DEC}
&\tt DISPOSE(v)    & Деаллоцирует v\^{} (см. \ref{m2:ISO:NEW}) \\ \index{DISPOSE}
&\tt EXCL(v,n)     & \verb|v := v - {n}| \\                    \index{EXCL}
&\tt HALT          & Завершение работы программы
                 (см. \ref{m2:ext:HALT}) \\                    \index{HALT}
&\tt INC(v[,n])    & \verb|v := v + n|, по умолчанию n=1 \\         \index{INC}
&\tt INCL(v,n)     & \verb|v := v + {n}| \\                    \index{INCL}
&\tt NEW(v)        & Аллоцирует v\^{} (см. \ref{m2:ISO:NEW}) \\
$\surd$ & \tt NEW(v,$x_0$...$x_n$) & Аллоцирует v\^{} длины
        $x_0$...$x_n$  (см. \ref{m2:ext:NEW}) \\  % \ldots !!!
\hline
\end{tabular}
\end{center}
\caption{Стандартные процедуры Модулы-2}\label{table:m2:stdproc}
\end{table}

\begin{table}[htb]
\begin{center}
\begin{tabular}{|cl|p{7.5cm}|} \hline
&\bf Функция  &\bf Действие \\ \hline
&\tt ABS(x)   & Абсолютное значение $x$ \\              \index{ABS}
$\surd$&\tt ASH(x,n) & Арифметический сдвиг \\              \index{ASH}
&\tt CAP(x)   & Соответствующая прописная буква \\  \index{CAP}
&\tt CHR(x)   & Символ с номером $x$ \\  \index{CHR}
&\tt CMPLX(x,y) & Комплексное число с вещ. частью $x$
                  и мнимой частью $y$           \\ \index{CMPLX}
$\surd$&\tt ENTIER(x) & Наибольшее целое $\leq x$ \\ \index{ENTIER}
&\tt FLOAT(x) &\tt  VAL(REAL,x) \\                     \index{FLOAT}
&\tt HIGH(v)  & Верхняя граница индексов $v$  \\    \index{HIGH}
&\tt IM(x)    & Мнимая часть комплексного $x$ \\         \index{IM}
&\tt INT(x)   &\tt VAL(INTEGER,x)          \\         \index{INT}
$\surd$&\tt LEN(v[,n]) & Длина массива
           в $n$-й координате (по умолчанию в 0-й) \\    \index{LEN}
&\tt LENGTH(x) & Длина строки \\         \index{LENGTH}
&\tt LFLOAT(x) &\tt VAL(LONGREAL,x) \\                 \index{LFLOAT}
&\tt MAX(T)   &  Максимальное значение типа $T$ \\         \index{MAX}
&\tt MIN(T)   &  Минимальное значение типа $T$ \\         \index{MIN}
&\tt ODD(x)   & $x$ MOD 2 = 1 \\                      \index{ODD}
&\tt ORD(x)   &\tt VAL(CARDINAL,x)    \\               \index{ORD (M2)}
&\tt RE(x)    & Вещ. часть комплексного $x$ \\         \index{RE}
&\tt SIZE(T)  & Количество единиц памяти, занимаемых переменной
                типа $T$  \\ \index{SIZE}
&\tt TRUNC(x) & Обрезание до целой части      \\ \index{TRUNC}
&\tt VAL(T,x) & Преобразование типа            \index{VAL (M2)}
\\ \hline
\end{tabular}
\end{center}
\caption{Модула-2 процедуры-функции}\label{table:m2:stdfunc}
\end{table}

В этой части дается краткое описание стандартных процедур и функций.
Некоторые из этих процедур и функций не описаны в
Международном Стандарте и доступны лишь при включенной опции
\OERef{M2EXTENSIONS}.
Процедура {\tt HALT} может иметь дополнительный параметр,
если расширения языка разрешены.
(см. \ref{m2:ext:HALT}).

В таблицах \ref{table:m2:stdproc} и
\ref{table:m2:stdfunc} использованы обозначения: $v$ для десигнатора,
$x$, $y$ и $n$ для выражений, $T$ для типа.
Нестандартные процедуры отмечены символом $\surd$.

Процедура {\tt COPY} и функции {\tt ASH}, {\tt ENTIER} и
{\tt LEN} описаны в {\em Сообщении об Обероне-2}.      % !!!

\section{Совместимость}\label{m2:ISO:comp}

В этом разделе описаны правила совместимости между объектами 
различных типов. Есть три формы совместимости:
\begin{itemize}
\item совместимость в выражениях (определяет, какие типы можно
совместно использовать в выражениях);
\item совместимость по присваиванию (определяет, значения какого
типа могут быть присвоены переменной);
\item совместимость при передаче (определяет тип фактического параметра,
который может быть передан в формальный параметр).
Есть две формы совместимости при передаче: совместимость по значению
и совместимость по VAR. %??? по имени?
\end{itemize}
Правила совместимости при передаче смягчены для формальных
параметров типа системной памяти. 
Такая вариация называется системной 
совместимостью при передаче.

В основном правила совместимости совпадают с описанными в
PIM. Тем не менее, мы сформулируем здесь все правила в явном виде.

\subsection{Совместимость в выражениях}\label{m2:ISO:comp:expr}

Выражения $a$ и $b$ типов $T_a$ и $T_b$
{\em совместимы в выражениях}, если выполняется одно из 
следующих утверждений:
\renewcommand{\theenumi}{\alph{enumi}}
\begin{enumerate}
\item Типы $T_a$ и $T_b$ идентичны.
        {\bf Замечание:} Если тип является отрезком, то имеет 
значение только его основной тип; типы разных отрезков одного
и того же типа совместимы в выражениях друг с другом и с 
основным типом.

\item Тип одного из выражений комплексный, а другое выражение
--- компл. константа.

\item Тип одного из выражений вещественный, а другое выражение
--- вещественная константа.

\item Тип одного из выражений целый, а другое выражение
--- целая константа.

\item Тип одного из выражений символьный, а другое выражение
--- строковый литерал длины 0 или 1.
      См. также \ref{m2:ISO:strings}.

\begin{verbatim}
VAR
  char: CHAR;
  ...
  WHILE (char # '') & (char # ".") DO
  ...
\end{verbatim}

\end{enumerate}
\renewcommand{\theenumi}{\arabic{enumi}}

\subsection{Совместимость по присваиванию}\label{m2:ISO:comp:assign}

Выражение $e$ типа $T_e$ {\em совместимо по присваиванию}
с переменной $v$ типа $T_v$ если выполняется одно из 
следующих\footnote{Для выражения типа отрезка имеет значение только
его основной тип.}:

\renewcommand{\theenumi}{\alph{enumi}}
\begin{enumerate}
\item   $T_v$ идентичен типу $T_e$, и при этом не является типом открытого
массива.

\item   $T_v$ --- отрезок типа $T_e$.

\item   $T_v$ --- CARDINAL или отрезок типа CARDINAL,
        а $T_e$ --- INTEGER, или $e$ есть целая константа.

\item   $T_v$ --- INTEGER или отрезок типа  INTEGER,
        а $T_e$ --- CARDINAL, или $e$ есть целая константа.

\item   $T_v$ --- вещественный тип и $e$ --- вещественная константа.

\item   $T_v$ --- комплексный тип и $e$ --- компл. константа.

\item   $T_v$ --- тип указателя и $e$ есть NIL.

\item   $T_v$ --- процедурный тип и $e$ десигнатор
        процедуры, имеющей такую же структуру, как 
        процедурный тип $T_v$ и декларированной на уровне 0.

\item   $T_v$ --- символьный тип или отрезок символьного типа, а
$e$ --- строковый литерал длины 0 или 1.

\item   $T_v$ --- тип массива с компонентами символьного типа,
$e$ --- строковый литерал длины, не превосходящей количества 
элементов в массиве типа
$T_v$\footnote{Строковый литерал не совместим по присваиванию с 
массивом, компоненты которого имеют тип отрезка символьного типа.}.

\item   $T_v$ --- тип адреса, а $T_e$ --- тип указателя; или
        $T_e$ --- тип адреса, а $T_v$ --- тип указателя. 

\end{enumerate}
\renewcommand{\theenumi}{\arabic{enumi}}

\subsection{Совместимость при передаче по значению}

Формальный тип {\em совместим при передаче по значению}  
с выражением, если выполняется одно из следующих:
\renewcommand{\theenumi}{\alph{enumi}}
\begin{enumerate}
\item   Формальный тип образован из системного типа памяти
и системно совместим с данным выражением.

\item   Формальный параметр --- открытый массив, фактический
%??? актуальный
параметр имеет тип массива и тип элементов формального
параметра совместим по значению с типом элементов фактического
параметра.
        \footnote{Формальный параметр типа массива элементов типа
$T$ не совместим при передаче с фактическим параметром типа $T$.}.

\item   Формальный тип совместим по присваиванию с фактическим параметром.
\end{enumerate}
\renewcommand{\theenumi}{\arabic{enumi}}

\subsection{Совместимость при передаче по VAR}

Формальный тип {\em совместим при передаче по VAR} 
с фактической переменной, если выполняется одно из следующих:
\renewcommand{\theenumi}{\alph{enumi}}
\begin{enumerate}
\item   Формальный тип образован из типа системной памяти
и системно совместим с переменной.

\item   Формальный параметр --- открытый массив, фактический параметр
имеет тип массива и тип элемента формального параметра
совместим при передаче по VAR с типом элемента фактического параметра.

\item   Формальный тип совпадает с типом фактического параметра.
\end{enumerate}
\renewcommand{\theenumi}{\arabic{enumi}}

\subsection{Системная совместимость при передаче}\label{m2:ISO:comp:system}

Формальный тип {\em системно совместим} с фактическим параметром,
если выполняется одно из следующих:
\renewcommand{\theenumi}{\alph{enumi}}
\begin{enumerate}
\item   Формальный параметр имеет тип SYSTEM.LOC,
а фактический параметр --- любого такого типа 
$T$, что {\tt SIZE(T)=1}.

\item   Формальный параметр --- типа
\begin{verbatim}
ARRAY [0..n-1] OF SYSTEM.LOC
\end{verbatim}
а фактический параметр --- любого такого типа 
$T$, что {\tt SIZE(T)=n}.

\item   Формальный параметр имеет тип открытого массива
\begin{verbatim}
ARRAY OF SYSTEM.LOC
\end{verbatim}
а фактический параметр - любой, кроме числовой константы.

\item   Формальный параметр имеет тип многомерного открытого массива

\begin{verbatim}
ARRAY OF ARRAY [0..n-1] OF SYSTEM.LOC
\end{verbatim}
а фактический параметр --- любого такого типа 
$T$, что {\tt SIZE(T)} делится на {\tt n}.

\end{enumerate}
\renewcommand{\theenumi}{\arabic{enumi}}

\section{Модула-2 модуль SYSTEM}
\label{m2:ISO:system}
\index{Модула-2!SYSTEM}\index{SYSTEM@{\bf SYSTEM}}
\index{системные модули!SYSTEM (M2)}

Модуль SYSTEM предоставляет низкоуровневые средства для работы с 
адресами и хранением переменных в памяти, для адресной арифметики
и для манипулирования представлением значений. Программы,
использующие эти средства, могут оказаться непереносимыми.

Этот модуль не существует в том же смысле, что обычные библиотеки:
он вшит в сам компилятор. Однако, чтобы получить доступ к его
средствам, идентификаторы должны быть импортированы обычным образом.

Некоторые процедуры модуля SYSTEM являются генерическими, и не могут
быть обычным образом декларированы; например, они могут быть применимы 
ко многим различным типам операндов или иметь несколько различных форм
списка параметров. 

Модуль  SYSTEM  --- единственный модуль, определенный в
Международном Стандарте, который разрешается расширять при
реализации языка.
В \XDS{}, модуль SYSTEM содержит дополнительные 
типы и процедуры.

{\bf Замечание:} В Обероне-2 модуль SYSTEM другой.
Подробно см. \ref{o2:system}.

{\small
\begin{verbatim}
DEFINITION MODULE SYSTEM;

CONST
  BITSPERLOC  = 8;
  LOCSPERWORD = 4;
  LOCSPERBYTE = 1;

TYPE
  LOC;
  ADDRESS = POINTER TO LOC;
  WORD = ARRAY [0 .. LOCSPERWORD-1] OF LOC;
  BYTE = LOC;

PROCEDURE ADDADR(addr: ADDRESS; offset: CARDINAL): ADDRESS;
PROCEDURE SUBADR(addr: ADDRESS; offset: CARDINAL): ADDRESS;
PROCEDURE DIFADR(addr1, addr2: ADDRESS): INTEGER;

PROCEDURE MAKEADR(val: <целый тип>): ADDRESS;

PROCEDURE ADR(VAR v: <любой тип>): ADDRESS;

PROCEDURE REF(VAR v: <любой тип>): POINTER TO <тип параметра>;

PROCEDURE ROTATE(val: <тип упакмножества>;
                 num: INTEGER): <тип первого параметра>;

PROCEDURE SHIFT(val: <тип упакмножества>;
                 num: INTEGER): <тип первого параметра>;

PROCEDURE CAST(<целевой тип>;
               val: <любой тип>): <целевой тип>;

PROCEDURE TSIZE (<тип>; ... ): CARDINAL;

(*------------------------------------------------------- *)
(* --------------  нестандартная часть  ----------------- *)

TYPE
  INT8   = <целый 8-битовый тип>;
  INT16  = <целый 16-битовый тип>;
  INT32  = <целый 32-битовый тип>;
  CARD8  = <натуральный 8-битовый тип>;
  CARD16 = <натуральный 16-битовый тип>;
  CARD32 = <натуральный 32-битовый тип>;
  BOOL8  = <булевский 8-битовый тип>;
  BOOL16 = <булевский 16-битовый тип>;
  BOOL32 = <булевский 32-битовый тип>;
  INDEX  = <тип индекса>
  DIFADR_TYPE = <тип, возвращаемый функцией DIFADR>

TYPE (* для Оберона-2 *)
  INT  = <Модула-2 INTEGER тип>;
  CARD = <Модула-2 CARDINAL тип>;

TYPE (* для интерфейса с C *)
  int      = <C int тип>;
  unsigned = <C unsigned тип>;
  size_t   = <C size_t тип>;
  void     = <C void тип>;

PROCEDURE MOVE(src,dest: ADDRESS; size: CARDINAL);
PROCEDURE FILL(adr : ADDRESS; val : BYTE; size : CARDINAL;);

PROCEDURE GET(adr: ADDRESS; VAR var: SimpleType);
PROCEDURE PUT(adr: ADDRESS; var: SimpleType);

PROCEDURE CC(n: CARDINAL): BOOLEAN;

END SYSTEM.
\end{verbatim}
} % small

\subsection{Системные типы}\label{m2:system:types}
\index{системные типы}

\begin{description}
\item[LOC] \mbox{}  \index{LOC}

Значение типа {\tt LOC} --- это никак не интерпретируемое содержимое
наименьшей адресуемой единицы памяти. Поэтому значение
{\tt TSIZE(LOC)} по определению равно 1.

Тип {\tt LOC} был введен для разрешения проблем с типами
{\tt  BYTE}  и  {\tt  WORD}.
Его введение позволяет корректно использовать оба этих типа,
и позволяет вводить другие типы, похожие на тип {\tt WORD}, напр.:
\begin{verbatim}
TYPE WORD16 = ARRAY [0..1] OF SYSTEM.LOC;
\end{verbatim}

Единственная операция, разрешенная с объектами типа {\tt LOC} 
--- это присваивание.
Специальные правила определяют совместимость при передаче параметров
для системных типов памяти.
См. подробнее в \ref{m2:ISO:comp:system}.

\item[BYTE] \mbox{}  \index{BYTE}

{\tt BYTE} определяется как {\tt LOC} и имеет такие же свойства, как тип
{\tt LOC}.

\item[WORD] \mbox{} \index{WORD}

        Тип {\tt WORD} определяется так:
\begin{verbatim}
CONST LOCSPERWORD = 4;
TYPE WORD = ARRAY [0..LOCSPERWORD-1] OF LOC;
\end{verbatim}
и значение {\tt TSIZE(WORD)} по определению равно {\tt LOCSPERWORD}.

Единственная операция, разрешенная с объектами типа {\tt WORD} 
--- это присваивание.
Специальные правила определяют совместимость при передаче параметров
для системных типов памяти.
См. подробнее в \ref{m2:ISO:comp:system}.

\item[ADDRESS] \mbox{} \index{ADDRESS}

        Тип ADDRESS определяется как
\begin{verbatim}
  TYPE ADDRESS = POINTER TO LOC;
\end{verbatim}

Тип ADDRESS совместим по присваиванию со всеми типами указателей, и наоборот
(см. \ref{m2:ISO:comp:assign}). Формальный VAR-параметр типа 
%??? A formal variable parameter of
ADDRESS совместим с фактическим параметром любого типа указателя.

Переменные типа ADDRESS уже не совместимы в выражениях с типом
CARDINAL (как это было в PIM) и не могут напрямую появляться
в выражениях, содержащих арифметические операторы.
Для адресной арифметики вводятся функции {\tt ADDADR}, 
{\tt SUBADR} и {\tt DIFADR}.

\item[Целые системные типы] \mbox{}   \index{типы фиксированного размера}

        Типы {\tt INT8}, {\tt CARD8}, {\tt INT16}, {\tt CARD16},
        {\tt  INT32},  {\tt  CARD32} имеют фиксированный размер 8,
        16 и 32  бита соответственно.

Эти типы введены для облегчения интерфейса с библиотеками
на других языках (см. Главу \ref{multilang}).  
Типы  {\tt  SHORTINT}, {\tt LONGINT},
{\tt  SHORTCARD},  {\tt  LONGCARD} --- синонимы типов 
{\tt INT8},   {\tt   INT32},   {\tt   CARD8},   {\tt   CARD32}
соответственно (см.  также опцию  \OERef{M2ADDTYPES}).
Типы {\tt  INTEGER}  и {\tt CARDINAL} --- синонимы типов
{\tt   INT16}/{\tt  INT32},  {\tt  CARD16}/{\tt  CARD32},
в зависимости от платформы.

Эти типы не описаны в Международном Стандарте.

\item[Булевы системные типы] \mbox{}

Типы {\tt BOOL8}, {\tt BOOL16}, и {\tt BOOL32} 
содержат 8,16 и 32 бита соответственно. По умолчанию, компилятор
использует тип {\tt BOOL8} в качестве типа BOOLEAN.
Однако в некоторых случаях
(напр. в интерфейсе к OS/2 или Windows API) 
следует использовать {\tt BOOL16} или {\tt BOOL32}.

\item[Целые типы Модулы-2] \mbox{}

Типы {\tt INT} и {\tt CARD} совпадают с \mt{} типами {\tt INTEGER}
        и {\tt CARDINAL} соответственно. 
Эти типы можно использовать в языке
\ot{} для переносимого использования
\mt{}-процедур.
См. подробнее в \ref{multilang:m2o2}.

\item[Интерфейс с C] \mbox{}

Типы {\tt int},  {\tt unsigned}, \verb|size_t| и 
{\tt void} введены для облегчения написания интерфейса к
C-библиотекам. См. подробнее в \ref{multilang:C}.

\end{description}

\subsection{Системные функции}\label{addr:arithmetic}
\index{Модула-2!системные функции}

\index{адресная арифметика}

\begin{verbatim}
PROCEDURE ADDADR(addr: ADDRESS;
                 offs: CARDINAL): ADDRESS;
\end{verbatim}
\index{SYSTEM!ADDADR}
Возвращает адрес, полученный как {\tt (addr + offs)}.
Последующее использование этого адреса может привести к возбуждению
исключения.

\begin{verbatim}
PROCEDURE SUBADR(addr: ADDRESS;
                 offs: CARDINAL): ADDRESS;
\end{verbatim}
\index{SYSTEM!SUBADR}
Возвращает адрес, полученный как {\tt (addr - offs)}.
Последующее использование этого адреса может привести к возбуждению
исключения.

\ifgenc
Результат функций {\tt ADDADR} и {\tt SUBADR} будет осмысленным,
если {\tt addr} указывает в какой-либо объект, и вычисленный адрес
указывает либо в тот же объект, либо на следующий байт после объекта.
\fi

\begin{verbatim}
PROCEDURE DIFADR(addr1,addr2: ADDRESS): INTEGER;
\end{verbatim}
\index{SYSTEM!DIFADR}
Возвращает разность адресов {\tt (addr1 - addr2)}.

\ifgenc
Результат этой функции будет осмысленным, если
\verb|addr1| и \verb|addr2| указывают в один и тот же объект
или на следующий байт после этого объекта.
\fi

\begin{verbatim}
PROCEDURE MAKEADR(val: <whole type>): ADDRESS;
\end{verbatim}

Эта функция конструирует значение типа ADDRESS 
из значения целого типа \footnote{Международный Стандарт
не специфицирует количество и типы параметров этой функции.
Программы, использующие ее, могут оказаться непереносимыми.}.

\begin{verbatim}
PROCEDURE ADR(VAR v: <any type>): ADDRESS;
\end{verbatim}

\index{SYSTEM!ADR (M2)}
Возвращает адрес переменной {\tt v}.

\begin{verbatim}
PROCEDURE REF(VAR v: <anytype>):
              POINTER TO <type of the parameter>;
\end{verbatim}

\index{SYSTEM!REF (M2)}

Возвращает указатель на переменную {\tt v}. См. также \ref{multilang:parmcomp}.

\begin{verbatim}
PROCEDURE CAST(<type>; x: <any type>): <type>;
\end{verbatim}

\index{SYSTEM!CAST}
Функция {\tt CAST} (как функция преобразования типа)
может быть использована для интерпретации значения любого типа
(кроме числового литерала) как значение другого типа
\footnote{Международный Стандарт запрещает использование
преобразований типов в стиле PIM, как напр.
{\tt CARDINAL(x)}.}.

Результатом функции {\tt  CAST(Type,val)} является неконтролируемое
преобразование {\tt val} к типу {\tt Type}.
Если {\tt SIZE(val) = TSIZE(Type)}, то
битовое представление результата совпадает с битовым представлением
{\tt  val};
иначе результат и значение {\tt val}
будут иметь одинаковое битовое представление на размере, равном 
меньшему из двух (в единицах памяти).

Некоторые комбинации типов параметров могут оказаться запрещенными.

{\bf Замечание:} В Oberon-2 модуле SYSTEM аналогичная процедура
называется \verb'VAL'.

\begin{verbatim}
PROCEDURE TSIZE(Type; ... ): CARDINAL;
\end{verbatim}
\index{SYSTEM!TSIZE}
Возвращает количество единиц памяти (LOC), используемое для
хранения значения данного типа. Дополнительные параметры, если
они есть, используются для выбора варианта в вариантной записи,
и должны быть константными выражениями
\footnote{В текущей версии эти константные выражения игнорируются.}.

\Example
\begin{verbatim}
TYPE
  R = RECORD
    CASE i: INTEGER OF
      |1: r: REAL;
      |2: b: BOOLEAN;
    END;
  END;

  ... TSIZE(R,1) ...

\end{verbatim}

Значение {\tt TSIZE(T)} равно {\tt SIZE(T)}.

\subsubsection{Функции на упакмножествах}

Значения типа PACKEDSET являются последовательностями битов
\footnote{В текущей версии типы множеств и упакмножеств не различаются.}.
Бит с номером 0 --- это наименьший значащий бит данной платформы.
Выполняются следующие равенства 
(где {\tt v} --- переменная типа {\tt CARDINAL}):
\begin{verbatim}
        CAST(CARDINAL,BITSET{0}) = VAL(CARDINAL,1)
        SHIFT(CAST(BITSET,v),1)  = v * 2
        SHIFT(CAST(BITSET,v),-1) = v DIV 2
\end{verbatim}

Функции {\tt ROTATE} и {\tt SHIFT} можно применять
к множествам размера, не превосходящего размера типа
{\tt BITSET}.

\begin{verbatim}
PROCEDURE ROTATE(x: T; n: integer): T;
\end{verbatim}
\index{SYSTEM!ROTATE}
Возвращает значение \verb|x|, циклически сдвинутого на \verb|n| 
бит влево (при положительном \verb|n|) 
или вправо (при отрицательном \verb|n|).

\begin{verbatim}
PROCEDURE SHIFT(x: T; n: integer): T;
\end{verbatim}
\index{SYSTEM!SHIFT}
Возвращает значение \verb|x|, логически сдвинутого на \verb|n| 
бит влево (при положительном \verb|n|) 
или вправо (при отрицательном \verb|n|).

\subsubsection{Нестандартные функции}

\begin{verbatim}
PROCEDURE CC(n: whole constant): BOOLEAN;
\end{verbatim}
\index{SYSTEM!CC}

\ifcomment
Возвращает TRUE, если выставлен флаг соответствующего условия
($0\leq n\leq 15$).
См. подробнее таблицу \ref{table:cc}.
\begin{center}
\begin{table}
\begin{flushleft}
\begin{tabular}{|r|l|}\hline
\bf Условие    &\bf Значение \\ \hline
 0 &   переполнение                    \\
 1 &   нет переполнения                \\
 2 &   перенос                         \\
 3 &   беззнаковое больше или равно    \\
 4 &   равно                           \\
 5 &   не равно                        \\
 6 &   беззнаковое меньше или равно    \\
 7 &   беззнаковое больше              \\
 8 &   знак                            \\
 9 &   минус знак                      \\
10 &   четность                        \\
11 &   нечетность                      \\
12 &   знаковое меньше                 \\
13 &   знаковое больше или равно       \\
14 &   знаковое меньше или равно       \\
15 &   знаковое больше                 \\
\hline
\end{tabular}
\end{flushleft}
\caption{Флаги условий}\label{table:cc}
\end{table}
\end{center}
\else
Возвращает TRUE, если выставлен флаг соответствующего условия.
В текущей версии функция не реализована.
\fi

\begin{verbatim}
PROCEDURE BIT(adr: T; bit: INTEGER): BOOLEAN;
\end{verbatim}

\index{SYSTEM!BIT}
Возвращает значение бита \verb|n| в \verb|Mem[adr]|.
\verb|T| --- либо ADDRESS, либо целый тип.

\subsection{Системные процедуры}
\index{Модула-2!Системные процедуры}

{\bf Замечание:} все эти процедуры нестандартны.

\begin{verbatim}
PROCEDURE MOVE (src, dst: ADDRESS; size: CARDINAL);
\end{verbatim}

\index{SYSTEM!MOVE}
Копирует \verb|size| байт из места в памяти по адресу
\verb|src| в место по адресу \verb|dst|.

\begin{verbatim}
PROCEDURE FILL(adr : ADDRESS; val : BYTE; size : CARDINAL);
\end{verbatim}
Заполняет участок памяти размера \verb'size', начиная с адреса
\verb'adr', значением \verb'val',
\ifgencode
используя очень эффективный код.
\else
используя библиотечную функцию \verb'memset' в C.
\fi


\begin{verbatim}
PROCEDURE GET (adr: ADDRESS; VAR v: SimpleType);
PROCEDURE PUT (adr: ADDRESS;     x: SimpleType);
\end{verbatim}

\index{SYSTEM!GET}\index{SYSTEM!PUT}
Получает/кладет значение по адресу \verb|adr|.
Второй параметр не может быть записью или массивом.

\begin{verbatim}
VAR i: INTEGER;

  GET (128, i);   (* get system cell value *)
  i := i+20;      (* change it             *)
  PUT (128, i);   (* и put back          *)
\end{verbatim}

\begin{verbatim}
PROCEDURE CODE(...);
\end{verbatim}

\index{SYSTEM!CODE}
\ifgencode
Процедура служит для вставки последовательности команд
непосредственно в порожденный код. В настоящей версии
не реализована.
\fi
\ifgenc
Процедура CODE принимает в качестве параметра последовательность
строк. Процедура вставляет конкатенацию этих строк в генерируемый
C текст.

\Example
\begin{verbatim}
PROCEDURE disable;      (* disable interrupts *)
BEGIN
  SYSTEM.CODE ("asm di;");
END disable;
\end{verbatim}
будет выдавать
\begin{verbatim}
  asm di;
\end{verbatim}
\fi

%----------------------------------------------------------------

\section{Расширения языка}\label{m2:ext}

{\bf ПРЕДУПРЕЖДЕНИЕ:} Использование расширений может привести к проблемам
при переносе программы на другие компиляторы.

По умолчанию, \xds{} \mt{} компилятор совместим со стандартом ISO
(см. \ref{m2:ISO}). 
Опции \OERef{M2EXTENSIONS} и \OERef{M2ADDTYPES} позволяют использовать
некоторые расширения языка.

Основные цели поддержания расширений языка таковы:
\begin{itemize}
\item улучшить интерфейс с другими языками
      (см. Главу \ref{multilang})
\item упростить переход с \mt{} на \ot{}
\item реализовать некоторые важные средства, не включенные в ISO \mt{}
\item обеспечить обратную совместимость с предыдущими версиями
\end{itemize}

\subsection{Лексические расширения}
\index{Модула-2!лексические расширения}

\subsubsection{Комментарии}

\mextonly

Наряду с \verb|(**)| вводится еще один формат для комментариев
в тексте программы. Часть строки от
``\verb|--|'' до конца строки рассматривается как комментарий.

\Example
\begin{verbatim}
  VAR i: INTEGER; -- это комментарий
--(*
  i:=0; (* эта строка будет скомпилирована *)
--*)
\end{verbatim}

\subsubsection{Числовые константы}

\mextonly

Разрешается использовать синтаксические правила обоих языков 
\mt{} и \ot{} для представления числовых и литерных констант.

\begin{verbatim}
число      = ["+"|"-"]целое|вещ.
целое      = цифра{цифра}
            |octalЦифра{octalЦифра}"B"
            |цифра{hexЦифра}"X".
вещ        = цифра{цифра}"."{цифра}[МасштФактор].
МасштФактор= ("E"|"D")["+"|"-"]цифра{цифра}.

character   ='"'char'"'|"'"char"'"
            |цифра{hexЦифра}"H"
            |octalЦифра{octalЦифра}"C".
\end{verbatim}

\Examples
\begin{verbatim}
1991               1991 (decimal)
0DH                13   (decimal)
15B                13   (decimal)
41X                "A"
101C               "A"
\end{verbatim}

{\bf Замечание:} идентификатор D в МасштФакторе означает
число типа LONGREAL.

\subsection{Дополнительные числовые типы}\label{m2:ext:numeric}
\index{Модула-2!числовые типы}

\maddonly

Опция компилятора \OERef{M2ADDTYPES} позволяет использовать 
следующие дополнительные числовые типы:

\ifonline
\begin{flushleft}
\begin{tabular}{lll}
1. &\bf SHORTINT  & целые от $-128$ до $127$          \\
2. &\bf LONGINT   & целые от $-2**31$ до $2**31-1$  \\
3. &\bf SHORTCARD & беззнаковые целые от $0$ до $255$        \\
4. &\bf LONGCARD  & беззнаковые целые от $0$ до $2**32-1$   \\
\end{tabular}
\end{flushleft}
\else
\begin{flushleft}
\begin{tabular}{lll}
1. &\bf SHORTINT  & целые от $-128$ до $127$          \\
2. &\bf LONGINT   & целые от $-2^{31}$ до $2^{31}-1$  \\
3. &\bf SHORTCARD & беззнаковые целые от $0$ до $255$        \\
4. &\bf LONGCARD  & беззнаковые целые от $0$ до $2^{32}-1$   \\
\end{tabular}
\end{flushleft}
\fi

Мы будем использовать следующие термины для групп типов:
\begin{flushleft}
{\em Вещественные типы\/} для (REAL, LONGREAL)                    \\
{\em Целочисленные типы\/} для (SHORTINT, INTEGER, LONGINT)        \\
{\em Кардинальные типы\/} для (SHORTCARD, CARDINAL, LONGCARD)     \\
{\em Целые типы\/} для {\em целочисленных\/} и {\em кардинальных типов} \\
{\em Числовые типы\/} для {\em целых\/} и {\em вещественных типов}          \\
\end{flushleft}

Все целочисленные типы реализованы как отрезки одного 
внутреннего для компилятора
целочисленного типа. Поэтому, в соответствии с правилами
совместимости (см. \ref{m2:ISO:comp}), в выражениях можно 
смешивать значения различных целочисленных типов. То же самое верно
и для кардинальных типов. Смешивание целочисленных и кардинальных
типов  в выражениях запрещено.
Как и в языке \ot{}, числовые типы образуют иерархию, и большие
типы включают (т.е. могут принимать значения) меньших типов:
$$
\mbox{LONGREAL} \subseteq \mbox{REAL} \subset \mbox{\em целые типы}
$$

Совместимость типов  в выражениях расширена следующими правилами
(см. \ref{m2:ISO:comp:expr}):
\begin{itemize}
\item   
Тип результата арифметической операции %??? или отношения
есть наименьший тип, включающий типы обоих операндов.

\item   Перед исполнением операции оба операнда конвертируются к типу
результата.
\end{itemize}

Например, если определены следующие переменные:
\begin{verbatim}
     s: SHORTCARD;
     c: CARDINAL;
     i: INTEGER;
     l: LONGINT;
     r: REAL;
     lr: LONGREAL;
\end{verbatim}
то
\begin{center}
\begin{tabular}{|l|l|l|} \hline
\bf Выражение &\bf Значение &\bf Тип результата \\ \hline
\tt s + c        &\tt VAL(CARDINAL,s) + c       & CARDINAL \\
\tt l * i        &\tt l * VAL(LONGINT,i)        & LONGINT \\
\tt r + 1        &\tt r + VAL(REAL,1)           & REAL    \\
\tt r = s        &\tt r = VAL(REAL,s)           & BOOLEAN \\
\tt r + lr       &\tt VAL(LONGREAL,r) + lr      & LONGREAL \\
\tt c + i        &\tt запрещено                 &           \\
\hline
\end{tabular}
\end{center}

Правила совместимости по присваиванию также расширены
(см. \ref{m2:ISO:comp:assign}),  так что выражение $e$ типа $T_e$
совместимо по присваиванию с переменной $v$ типа $T_v$ если $T_e$
и $T_v$ числовых типов и $T_v$ включает $T_e$. 
Кардинальные и целочисленные типы совместимы по присваиванию.
При необходимости компилятор генерирует проверки границ.

\paragraph{Примеры (см. декларации выше):}
\begin{center}
\begin{tabular}{|l|p{7cm}|} \hline
\bf Действие &\bf Комментарий \\ \hline
\tt i:=c;     & {\tt INTEGER} и {\tt CARDINAL} совместимы по присваиванию \\
\tt i:=s;     & {\tt INTEGER} и {\tt SHORTCARD} совместимы по присваиванию \\
\tt l:=i;     & {\tt LONGINT} и {\tt INTEGER} --- отрезки одного базового типа                                              \\
\tt r:=i;     & {\tt REAL}$\subset$ {\tt INTEGER}                           \\
\tt r:=c;     & {\tt REAL}$\subset$ {\tt CARDINAL}                          \\
\tt lr:=r;    & {\tt LONGREAL}$\subseteq$ {\tt REAL}                          \\
\hline
\end{tabular}
\end{center}

\subsection{Приведение типов}

\mextonly

В ISO Модуле-2, второй параметр процедуры
\verb'SYSTEM.CAST' не может быть числовым литералом.
\XDS{} предоставляет такую возможность как расширение языка:

\begin{verbatim}
    VAR
      c: CARDINAL;
    BEGIN
      (* Ok if M2EXTENSIONS is ON *)
      c := SYSTEM.CAST(CARDINAL,-1); 
\end{verbatim}

\subsection{Совместимость по присваиванию с типом BYTE}

\mextonly

Выражения типов {\tt CHAR}, {\tt BOOLEAN}, {\tt SHORTINT},
и {\tt SYSTEM.CARD8} можно присваивать переменным типа
{\tt BYTE}  или передавать как фактические параметры в формальные
параметры типа {\tt BYTE}.

\subsection{Динамические массивы}\label{m2:ext:dynarrs}
\index{Модула-2!динамические массивы}

\mextonly

\xds{} разрешает использовать динамические массивы в стиле
\ot{}  по правилам языка \ot{}.

Открытый массив --- это тип массива, в котором не заданы 
верхняя и нижняя границы индекса, т.е.
ARRAY OF Тип. 
Открытые массивы можно использовать только в списках параметров
процедуры, или как базовый тип указателя.
\begin{verbatim}
TYPE String = POINTER TO ARRAY OF CHAR;
\end{verbatim}
Ни переменные, ни поля записи не могут иметь тип открытого массива.

Если формально тип некоторого обозначения --- открытый массив,
то с ним разрешается делать только две операции: индексировать
и передавать как параметр в процедуру.

Расширенные версии стандартных процедур {\tt NEW} и
{\tt DISPOSE} могут создавать и удалять динамические массивы
(см. \ref{m2:ext:NEW}).



\Example
\begin{verbatim}
TYPE
  VECTOR = ARRAY OF REAL;
    (* 1-мерный открытый массив *)
  Vector = POINTER TO VECTOR;
    (* указатель на открытый массив *)
  MATRIX = ARRAY OF VECTOR;
    (* 2-мерный открытый массив *)
  Matrix = POINTER TO MATRIX;
    (* указатель на него *)

VAR
   v: Vector;
   m: Matrix;

PROCEDURE ClearVector(VAR v: VECTOR);
  VAR i: CARDINAL;
BEGIN
  FOR i := 0 TO HIGH (v) DO v[i] := 0 END;
END ClearVector;

PROCEDURE ClearMatrix(VAR m: Matrix);
  VAR i: CARDINAL;
BEGIN
  FOR i := 0 TO HIGH (m) DO ClearVector(m[i]) END;
END ClearMatrix;

PROCEDURE Test;
BEGIN
  NEW(v, 10);
  NEW(m, 10, 20);
  ClearVector(v^);
  ClearMatrix(m^);
  v^[0] := 1;
  m^[1][1] := 2;
  m^[2,2] := 1000;
  DISPOSE(v);
  DISPOSE(m);
END Test;
\end{verbatim}

\subsection{Конструкторы константных массивов}\label{m2:constarrays}
\index{Модула-2!конструкторы массивов}
\mextonly

\xds{} разрешает декларацию константного массива в форме
\begin{verbatim}
        ARRAY OF QualIdent "{" ExprList "}".
\end{verbatim}
{\tt QualIdent} должен означать основной тип, отрезок или перечислимый
тип, и все выражения в
{\tt ExprList} должны быть этого типа.

{\bf Замечание:} Структурные типы и неконстантные выражения запрещены.

Типом такой константы будет \verb|ARRAY [0..n] OF QualIdent|,
где \verb|n+1| есть количество выражений в ExprList.
\Example
\begin{verbatim}
CONST table = ARRAY OF INTEGER {1, 2+3, 3};
\end{verbatim}

Константные массивы подчиняются тем же правилам, что и все константы,
и могут рассматриваться как обычные массивы.

В некоторых случаях конструкторы такого вида удобнее, чем стандартные
конструкторы ISO (см. \ref{m2:ISO:aggregates}),
поскольку не требуется описывать тип и подсчитывать количество
выражений. Однако, чтобы Ваши программы были более переносимыми,
мы советуем пользоваться стандартными средствами.

\subsection{Дополнение множества}\index{Модула-2!дополнение множества}

\mextonly

Как в языке \ot{}, унарный минус в применении к множеству означает
дополнение этого множества, т.е.
$-x$ --- это множество всех значений, не входящих в множество $x$.

\begin{verbatim}
TYPE SmallSet = SET OF [0..5];
VAR x, y: SmallSet;
BEGIN
  x := SmallSet{1,3,5};
  y := -x;                  (* y = {0, 2, 4} *)
  y := SmallSet{0..5} - x;  (* y = {0, 2, 4} *)
END;
\end{verbatim}

\subsection{Параметры по чтению}\label{m2:ext:RO_param}
\index{параметры по чтению}
\index{Модула-2!параметры по чтению}

\mextonly

В секции формальных параметров разрешается ставить \verb|"-"| 
после имени параметра по значению. Такой параметр называется
{\em параметром по чтению}; его значение нельзя менять в теле
процедуры. Параметры по чтению не требуется копировать перед активацией
процедуры, что позволяет сделать более эффективными вызовы 
процедур со структурными параметрами.

У параметров по чтению типа ARRAY и RECORD элементы 
массива и поля записи также защищены. Параметры по чтению нельзя 
указывать в определяющем модуле.

Мы рекомендуем использовать параметры по чтению с осторожностью.
Компилятор не отслеживает их изменений через другой параметр
или через глобальную переменную.

\Example
\begin{verbatim}
PROCEDURE Foo(VAR dest: ARRAY OF CHAR;
               source-: ARRAY OF CHAR);
BEGIN
  dest[0]:='a';
  dest[1]:=source[0];
END Foo;
\end{verbatim}

Вызов \verb|Foo(x,x)| приведет к неверному результату, поскольку
первый оператор изменяет значение
\verb|source[0]| (\verb|source| не копируется, и находится в том же месте
в памяти, что и \verb|dest|).

\subsection{Переменное число параметров}\label{m2:SEQ:param}
\index{SEQ параметры}
\index{Модула-2!SEQ параметры}

\mextonly

Последний формальный параметр процедуры может быть описан как
``последовательность байтов'' (SEQ-параметр). В вызове процедуры
вместо такого параметра может быть подставлена любая, в~т.ч. пустая,
последовательность фактических параметров. Фактические параметры,
соответствующие SEQ параметру, могут быть любых типов.

Разрешается только описание
\begin{verbatim}
     SEQ name: SYSTEM.BYTE
\end{verbatim}. 
Процедура может иметь только один
SEQ параметр, и он должен быть последним в списке формальных параметров.

Внутри процедуры SEQ параметры очень похожи на обычные параметры типа
ARRAY. Более точно:
\begin{itemize}
\item функция {\tt HIGH} может быть применена к такому параметру;
\item фактический SEQ параметр может быть далее передан в другую процедуру;
\item \verb|i|-й байт последовательности \verb|s|
можно получить, написав \verb|s[i]|, как элемент массива.
\end{itemize}

Массив байтов, передаваемый в процедуру как
SEQ-параметр, формируется по следующим правилам:
\begin{itemize}
\item
  значения всех фактических параметров, образующих последовательность,
представляются, как указано ниже, и собираются в один массив в их
порядке в тексте
\item целочисленные значения конвертируются в {\tt LONGINT}
\item {\tt BOOLEAN}, {\tt CHAR}, кардинальные и перечислимые значения
конвертируются в {\tt LONGCARD}
\item значения типа отрезка конвертируются в соответствии с их базовыми типами
\item вещественные значения конвертируются в {\tt LONGREAL}
\item значения скрытых типов, указателей и
процедурных типов конвертируются в {\tt ADDRESS}
\item структурные значения (записи и массивы) рассматриваются как 
массивы байтов и передаются как последовательность из:
    \begin{itemize}
    \item адреса структуры
    \item нулевого слова (резервировано для последующих расширений)
    \item размера структуры (в единицах LOC) минус один
    \end{itemize}
\end{itemize}

\ifgencode
  См. более подробно в \ref{lowlevel:seqrep}.
\fi
\ifgenc
  См. более подробно в \ref{maptoc:procdcl:seqrep}.
\fi

\subsection{Экспорт по чтению}\label{RO:export}
\index{Модула-2!экспорт по чтению}

\mextonly

Символ "-" экспорта по чтению (из языка \ot{}), после идентификатора 
переменной или поля в определяющем модуле означает, что этот
идентификатор экспортируется только по чтению. Лишь модуль, в 
котором описан этот идентификатор, может изменять его значение.

Компилятор не позволит ни явного (присваиванием), ни неявного
(передачей как VAR параметр) изменения объекта, экспортируемого 
по чтению.

У переменной типа массива или записи, экспортируемой по чтению,
все элементы массива или поля записи также экспортируются по чтению.

\paragraph{Пример (выдержка из определяющего модуля):}
\begin{verbatim}
TYPE Rec = RECORD
  n-: INTEGER;
  m : INTEGER;
END;

VAR
  in-: FILE;
  x-: Rec;
\end{verbatim}

\subsection{Переименование импортируемых модулей}
\index{Модула-2!переименование в импорте}
\mextonly

Импортируемый модуль может быть переименован в импортирующем его модуле.
Настоящее имя модуля становится невидимым.
\begin{verbatim}
Import = IMPORT [ Ident ":=" ] Ident
          { "," [ Ident ":=" ] ident } ";".
\end{verbatim}

\Example
\begin{verbatim}
MODULE test;
IMPORT vw := VirtualWorkstation;

VAR ws: vw.Station;

BEGIN
  ws := vw.open();
END test.
\end{verbatim}

\subsection{NEW и DISPOSE для динамических массивов}\label{m2:ext:NEW}
\index{Модула-2!NEW и DISPOSE}
\index{NEW (M2)}\index{DISPOSE}
\index{распределение памяти}
\index{DYNALLOCATE}
\index{DYNDEALLOCATE}

Стандартные процедуры {\tt NEW} и {\tt DISPOSE} можно применять
к переменным типа динамического массива (см. \ref{m2:ext:dynarrs}).
При этом процедуры {\tt DYNALLOCATE} и {\tt DYNDEALLOCATE} должны быть
видимы в контексте вызова:

\begin{verbatim}
PROCEDURE DYNALLOCATE(VAR a: ADDRESS;
                       size: CARDINAL;
                        len: ARRAY OF CARDINAL);
\end{verbatim}

Эта процедура должна аллоцировать динамический массив, где
\verb|size| --- это размер базового типа (размер элемента массива),
а \verb|len[i]| --- длина массива в  $i$-й размерности.

\begin{verbatim}
PROCEDURE DYNDEALLOCATE(VAR a: ADDRESS;
                     size,dim: CARDINAL);
\end{verbatim}

Эта процедура должна деаллоцировать динамический массив, где
\verb|size| --- размер элемента массива, и
\verb|dim| --- размерность массива.

{\bf Замечание:} В большинстве случаев можно использовать встроенную
реализацию этих процедур.
Опция \OERef{STORAGE} указывает,  использовать ли ее.

Динамический массив представляется как указатель на так называемый
дескриптор массива
\ifgenc (см. \ref{maptoc:dynarr}). \fi
\ifgencode (см. \ref{lowlevel:opendesc}). \fi

\subsection{HALT}\label{m2:ext:HALT}
\index{HALT}
\mextonly

У процедуры HALT допускается необязательный целочисленный параметр.
\begin{verbatim}
PROCEDURE HALT ([code: INTEGER]);
\end{verbatim}
\verb|HALT| прекращает выполнение программы, возможно, возвращая при
этом код остановки. Подробности см. в документации к операционной
системе или окружению.

\subsection{ASSERT}\label{m2:ext:ASSERT}\index{ASSERT}
\index{ASSERT}
\mextonly

Процедура ASSERT проверяет значение своего булевского параметра
и прекращает выполнение программы, если он не равен
TRUE. Второй, необязательный, параметр задает
{\em код остановки задачи}. Если он опущен, подставляется стандартное
значение.

\begin{verbatim}
PROCEDURE ASSERT(cond: BOOLEAN [; code: INTEGER]);
\end{verbatim}

Вызов \verb|ASSERT(expr,code)| эквивалентен
\begin{verbatim}
IF NOT expr THEN HALT(code) END;
\end{verbatim}

\section{Директивы в исходном тексте (прагмы)}\label{m2:pragmas}

Директивы в исходном тексте (или прагмы) служат для установки опций 
компилятора в исходном тексте и для выбора частей текста, которые нужно
компилировать (условная компиляция).
ISO \mt{} стандарт не описывает синтаксиса прагм.
\XDS{} позволяет использовать прагмы в обоих языках,
\mt{} и \ot{}. Используется синтаксис, описанный в
{\em The Oakwood Guidelines for the \ot{} Compiler Developers}

\subsection{Опции и параметры в тексте}\label{m2:pragmas:inline}
\index{опции в тексте}
\index{параметры в тексте}

Некоторые опции и параметры удобнее устанавливать непосредственно в тексте.
Такие опции компилятора, как
\OERef{MAIN}, естественно помещать в исходный текст перед 
заголовком модуля, а такие, как проверки во время исполнения ---
между операторами.

\xds{} позволяет устанавливать опции в исходном тексте с помощью
стандартных
ISO псевдо-комментариев \verb|<* ... *>|\footnote{Поддерживается также
устаревший стиль прагм,
\verb|(*\$..*)|, для обеспечения обратной совместимости,
но компилятор выдает предупреждение "устаревший синтаксис".}.

Некоторые опции разрешается помещать в тексте только до заголовка
модуля (т.е. до ключевых слов IMPLEMENTATION, DEFINITION или MODULE).
В любом другом месте текста такие опции игнорируются.
См. подробнее в \ref{opt:bool:list}.

Формат установки в тексте опций и параметров описывается следующим 
синтаксисом:
\begin{flushleft} \tt
\begin{tabular}{l@{= }p{10cm}}
Pragma      & \verb|<*| PragmaBody \verb|*>| \\
PragmaBody  & PUSH $|$ POP $|$ NewStyle $|$ OldStyle \\
NewStyle    & [ NEW ] name [ "+" $|$ "-" $|$ "=" string ] \\
OldStyle    & ("+" $|$ "-") name \\
\end{tabular}
\end{flushleft}
\index{<* *>@\verb'<* *>'}

{\tt NewStyle} предлагается как Oakwood-стандарт для языка \ot{},
{\tt OldStyle} --- это стиль, использовавшийся в прежних версиях
\xds{}. В именах опций не различаются большие и малые буквы.
Если используется {\tt OldStyle}, то не должно быть пробелов
между \verb|<*| и \verb|+| или \verb|-|.
{\tt OldStyle} не позволяет декларировать новые опции или параметры,
и не позволяет менять значения параметров.

Символ \verb|+| во всех случаях устанавливает соответствующую опцию
ON, а символ \verb|-| устанавливает ее OFF.

Ключевые слова {\tt PUSH} и {\tt POP} служат для сохранения 
и восстановления значений всех опций и параметров.

\Examples
\begin{verbatim}
PROCEDURE Length(VAR a: ARRAY OF CHAR): CARDINAL;
  VAR i: CARDINAL;
BEGIN
  <* PUSH *>              (* save state *)
  <* CHECKINDEX - *>      (* turn CHECKINDEX off *)
  i := 0;
  WHILE (i<=HIGH(a)) & (a[i]#0C) DO INC(i) END;
  <* POP *>               (* restore state *)
  RETURN i;
END Length;
\end{verbatim}
\begin{verbatim}
<* ALIGNMENT = "2" *>
TYPE
  R = RECORD      (* This record is 6 bytes long *)
        f1: CHAR;
        f2: CARDINAL;
      END;
\end{verbatim}

\subsection{Условная компиляция}\label{m2:pragmas:cc}
\index{условная компиляция}

Команды условной компиляции и в Модуле-2, и в Обероне-2\footnote
{только при выставленной опции {\bf O2ISOPRAGMA}}
используют стандартную ISO нотацию для прагм: \verb|<* *>|.
Операторы условной компиляции можно помещать в любом месте текста.
Приведем синтаксис IF-оператора условной компиляции:
\begin{verbatim}
  IfStatement      = <* IF Expression THEN *>  text
                   { <* ELSIF Expression THEN *> text }
                   [ <* ELSE *> text ]
                     <* END *>
  Expression       = SimpleExpression
                     [ ("=" | "#") SimpleExpression].
  SimpleExpression = Term { "OR" Term}.
  Term             = Factor { "&" Factor}.
  Factor           = ident | string |
                     "DEFINED" "(" ident ")" |
                     "(" Expression ")" |
                     "~" Factor | "NOT" Factor.
  ident            = option | equation.
\end{verbatim}

Операндами в выражениях могут служить имена опций или параметров, 
и константные строки. Опция имеет значение 
TRUE, если она была выставлена в файле конфигурации, в проектном файле,
в командной строке или в тексте. 
Опция имеет значение FALSE, если она в явном виде была приравнена
FALSE (т.е. выключена), или если она вообще не была определена.
Если в операторе условной компиляции использованы недекларированные
опция или параметр, то компилятор сообщит о предупреждении.

См. также часть \ref{opt:COMPILER}.

\Examples
\begin{verbatim}
  IMPORT lib :=
        <* IF  __GEN_X86__ THEN *> MyX86Lib;
        <* ELSIF __GEN_C__ THEN *> MyCLib;
        <* ELSE *> *** Unknown ***
        <* END *>
\end{verbatim}

\begin{verbatim}
  CONST Win = <* IF Windows THEN *> TRUE
              <* ELSE *> FALSE
              <* END *>;
\end{verbatim}

\begin{verbatim}
  <* IF DEFINED(Debug) & (DebugLevel = "2") THEN *>
     PrintDebugInformation;
  <* END *>;
\end{verbatim}

\begin{verbatim}
  <* IF platform = "OS2" THEN *>
    Strings.Capitalize(filename);
    <* IF NOT HPFS THEN *>
      TruncateFileName(filename);
    <* END *>
  <* END *>
\end{verbatim}

