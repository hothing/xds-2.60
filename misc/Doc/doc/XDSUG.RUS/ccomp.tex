% !!! DS_EQ_SS - throw away?
% !!! Supported C compilers for OS/2 ?

% This chapter has to be included if and only if gencode is true.
\ifgencode\else\JNO\fi

\chapter{Настройка XDS на C компилятор}\label{ccomp}

% CC equation: WATCOM/SYMANTEC/BORLAND/MSVC/OS2SYS_CALL
%       options: GENCPREF, DS_NEQ_SS, ONECODESEG

\XDS{} позволяет использовать в Ваших проектах функции и библиотеки
языка C. Различные C компиляторы придерживаются различных соглашений 
об именах и о связях (naming and calling conventions).
Поэтому необходимо сообщить \XDS{}, какой Вы используете C компилятор,
задав параметр \OERef{CC}.
\XDS{} будет генерировать вызовы всех C-функций в виде, совместимом с 
указанным в этом параметре C компилятором.
Компилятор также сам установит значения некоторых дополнительных опций
в соответствии со значением этого параметра.
\iflinux \else См. \ref{ccomp:opt}. \fi % this section not included for Linux

\iflinux
  Для системы Linux \XDS{} поддерживает компилятор GCC (ELF). 
  Соответственно, параметр \OERef{CC} должен быть установлен равным
  GCC (буквы любого регистра). Если значение параметра не установлено, 
  то по умолчанию полагается GCC. Итак, чтобы настроить 
  \XDS{} на GCC, добавьте в файл конфигурации строку \\
  \verb'-cc = GCC' \\
\fi

\ifwinnt
  Для систем Windows NT и Windows 95 \XDS{} поддерживает компиляторы 
  MSVC++ и Watcom.
  Соответствующие значения параметра {\bf CC} --- \verb'MSVC'
  и \verb'WATCOM', (буквы любого регистра). 
  Если значение параметра не установлено, то по умолчанию полагается 
  WATCOM.
  Добавьте в файл конфигурации строку

  \verb'-cc=Watcom'

  или

  \verb'-cc=MSVC'

\fi

Параметр {\bf ALIGNMENT} контролирует выравнивание в структурах данных.

\iflinux
  {\bf ВНИМАНИЕ!} 
  Библиотеки в дистрибутиве \XDS{} созданы с помощью GCC.
  Поскольку GCC обычно генерирует выровненный код, параметр
  {\bf ALIGNMENT} должен быть равен 4. Установка другого 
  значения может привести к непредсказуемым результатам.
  Не меняйте значение этого параметра без крайней необходимости!
\fi

Имена в объектном файле, генерируемом C компилятором, могут начинаться
с символа "\_" (подчерк). Если Вы собираетесь использовать C модули
и библиотеки, необходимо, чтобы \XDS{} придерживался того же соглашения 
об именах. Для этого следует установить в файле конфигурации опцию
\OERef{GENCPREF} равной ON:

\verb'+GENCPREF'

\iflinux
Однако поскольку компилятор GCC (ELF) не префиксует имена подчерком,
для него эту опцию выставлять не следует.
\fi

См. подробнее файл {\tt samples.txt} из прилагающейся к \XDS{} 
документации.

\section{Возможные проблемы}

Чтобы использовать в языках \mt{} или \ot{} C функцию или тип данных,
необходимо выразить этот тип на соответствующем языке. Обычно это
делается в иноязычном определяющем модуле (см. \ref{multilang:C}).
Текущая версия \XDS{} поддерживает не все соглашения о связях,
поэтому в ней невозможно использование некоторых функций, а именно:
\begin{itemize}
\item функции с параметром структурного типа, передаваемым по значению:
\begin{verbatim}
void foo(struct MyStruct s)
\end{verbatim}

\item функции, возвращающие результат структурного типа:
\begin{verbatim}
struct MyStruct foo(void)
\end{verbatim}

\item C функции с паскалевским соглашением о связях, возвращающие 
      вещественный тип.

\iflinux \else % not Linux
\item функции, скомпилированные с бесстековым соглашением о связях.
      {\bf Замечание:} для Watcom стековые соглашения о связях следует
      устанавливать с помощью опций
      "-3s", "-4s" или "-5s".
\fi
\end{itemize}

\iflinux \else % not Linux
\XDS{} не поддерживает структуры данных с нестандартным выравниванием.
Если параметр \OERef{ALIGNMENT} выставлен равным {\em n}, 
используйте опцию
"-zp{\em n}" для Watcom C, и "-Zp{\em n}" для MSVC.
\fi

Как \mt{}, так и C/C++ содержат средства для финализации и обработки
исключений. Если в одной программе использовать эти средства из
разных языков, то результат может оказаться непредсказуемым.

\iflinux \else % not Linux
\section{Использование не поддерживаемого C компилятора}

\XDS{} поддерживает не все существующие C компиляторы.
Для настройки  \XDS{} на не поддерживаемый компилятор нужно использовать
имеющиеся опции (см. \ref{ccomp:opt}).
Опция \OERef{DEFLIBS} должна быть при этом выключена.

Может оказаться необходимым внести некоторые изменения в систему 
поддержки исполнения, или создать специальную версию библиотек для
данного C компилятора. Все это может быть сделано в рамках 
нашей программы поддержки продукта. %???

\section{Дополнительные опции конфигурации}\label{ccomp:opt}

Для настройки  \XDS{} на не поддерживаемый компилятор используются
следующие опции (будьте внимательны и осторожны при их использовании):

\begin{itemize}
\ifcomment
\item[DS\_NEQ\_SS] \index{DS\_NEQ\_SS@{\bf DS\_NEQ\_SS}} \mbox{}

Если эта опция выставлена, то компилятор предполагает, что регистр
DS не равен регистру SS.
\fi

\item Опция \OERef{GENCPREF} указывает, нужно ли префиксовать подчерком
внешние имена в объектных файлах (ON), 
или оставлять их без изменений (OFF).

\item Опция \OERef{ONECODESEG} указывает, порождать ли один сегмент кода
для всего модуля (ON), или отдельный сегмент для каждой процедуры (OFF).
\end{itemize}

В следующей таблице указаны умолчательные значения этих опций для
поддерживаемых C компиляторов.
\begin{flushleft}
\begin{tabular}{|l|c|c|c|c|} \hline
Опция       & WATCOM &   MSVC    \\ \hline
\ifcomment
        DS\_NEQ\_SS &   ?    &    ?      \\
\fi
GENCPREF    &   OFF  &    ON     \\
ONECODESEG  &   OFF  &    ON     \\ \hline
\end{tabular}
\end{flushleft}
\fi % not Linux
