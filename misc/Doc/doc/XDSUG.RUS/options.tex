% !!! check "inline" and "header" attributes
% !!! options/equation list not full
% !!! platform-specific options (like PM and VIO)
% !!! explicitly specify BE instead of '\ifgenc'/'\ifgencode'

% Modes list

\newcommand{\MLBegin}{\ifonline {\bf Режимы:} \else [\fi}
\newcommand{\MLEnd}{\ifonline \else ]\fi}
\newcommand{\ModeC}{\ifonline \ref[COMPILE]{xc:modes:compile}\else compile\fi}
\newcommand{\ModeM}{\ifonline \ref[MAKE]{xc:modes:make}\else make\fi}
\newcommand{\ModeP}{\ifonline \ref[PROJECT]{xc:modes:project}\else project\fi}
\newcommand{\ModeB}{\ifonline \ref[BROWSE]{xc:modes:browse}\else browse\fi}
\newcommand{\ModeG}{\ifonline \ref[GEN]{xc:modes:gen}\else gen\fi}

\chapter{Опции и параметры компилятора}\label{options}

Богатый набор опций \xds{} позволяет контролировать исходный язык, 
генерацию кода и внутренние ограничения и настройки компилятора.
Различаются два вида опций: булевы опции (или просто опции) и параметры
(equations).
{\em Опция} может быть либо выставлена (значение ON, или TRUE), 
либо не выставлена (значение OFF, или FALSE). 
Значением {\em параметра} является некоторая строка.

\section{Опции}\label{opt:bool}
\index{опции}

Опции контролируют процесс компиляции, в~т.ч. расширения языков,
проверки во время исполнения и генерацию кода. Опция может быть 
либо выставлена (значение ON, или TRUE), 
либо не выставлена (значение OFF, или FALSE). 

Для установки значения и для декларирования новых опций
используются директивы установки (см. \ref{config:options}).

Опции могут устанавливаться в файле конфигурации
(см. \ref{xc:cfg}), в командной строке
(см. \ref{xc:modes}), в проектном файле (см.
\ref{xc:project}) или в исходном тексте (см. \ref{m2:pragmas}). 
В каждый момент действует последнее выставленное значение опции.

Алфавитный список всех опций см. в \ref{opt:bool:list}.
\ifonline\else
См. также таблицы
\ref{table:opt:check} (стр. \pageref{table:opt:check}),
\ref{table:opt:ext}   (стр. \pageref{table:opt:ext}),
\ref{table:opt:code}  (стр. \pageref{table:opt:code})
и
\ref{table:opt:misc}  (стр. \pageref{table:opt:misc}).
\fi

\begin{table}[htbp]
\begin{center}
\begin{tabular}{|l|l|}
\hline
\bf Опция  & \bf Значение \\
\hline
\OERef{ASSERT}       & разрешить генерацию ASSERT           \\
\OERef{CHECKDINDEX}  & проверка границ динамических массивов \\
\OERef{CHECKDIV}     & проверка положительности делителя     \\
                      & (DIV и MOD)                        \\
\OERef{CHECKINDEX}   & проверка границ статических массивов \\
\OERef{CHECKNIL}     & проверка на разыменование NIL        \\
\OERef{CHECKPROC}    & проверка вызовов формальных процедур \\
\OERef{CHECKRANGE}   & проверка границ                         \\
                      & (перечисляемые типы и отрезки)       \\
\OERef{CHECKSET}     & проверка границ в операциях с множествами \\
\OERef{CHECKTYPE}    & динамический контроль типов (только \ot{}) \\
\ifgencode
\OERef{COVERFLOW}    & проверка на переполнение натурального \\
\fi
\ifgencode
\OERef{IOVERFLOW}    & проверка на переполнение целого        \\
\fi
\hline
\end{tabular}
\end{center}
\caption{Проверки во время исполнения}
\label{table:opt:check}
\index{опции!проверки во время исполнения}
\end{table}

\begin{table}[htbp]
\begin{center}
\begin{tabular}{|l|l|}
\hline
\bf Опция         & \bf Значение \\
\hline
\OERef{M2ADDTYPES}    & добавить типы SHORT и LONG \\
\OERef{M2BASE16}      & в Модуле-2: использовать 16-битовые осн. типы \\
\OERef{M2CMPSYM}      & в Модуле-2: сравнение символьных файлов \\
\OERef{M2EXTENSIONS}  & разрешить расширения Модулы-2 \\
\OERef{O2EXTENSIONS}  & разрешить расширения Оберона-2 \\
\OERef{O2ISOPRAGMA}   & разрешить в Обероне ISO Модула-2 прагмы \\
\OERef{O2NUMEXT}      & разрешить математич. расширения Оберона-2 \\
\OERef{STORAGE}       & в Модуле-2: включение встроенного \\
                      & распределителя памяти \\
\hline
\end{tabular}
\end{center}
\caption{Опции контроля входного языка}\label{table:opt:ext}
\index{опции!контроль языка}
\end{table}

\begin{table}[htbp]
\begin{center}
\begin{tabular}{|l|l|}
\hline
\bf Опция  & \bf Значение \\
\hline
\OERef{\_\_GEN\_C\_\_}   & генерация кода на ANSI C       \\
\OERef{\_\_GEN\_X86\_\_} & генерация кода для 386/486/Pentium/PentiumPro \\
\ifgenc
  \OERef{COMMENT}        & скопировать комментарии в C-код  \\
  \OERef{CONVHDRNAME}    & в директиве \verb|#include| использовать имя файла \\
\fi
\ifgenc
  \OERef{CSTDLIB}     & определение стандартной C-библиотеки \\
\fi
\ifgencode
  \OERef{DEFLIBS}     & занести в объектный файл имена библиотек \\
\fi
\ifgenc
  \OERef{DIFADR16}    & SYSTEM.DIFADR возвращает 16-битовый результат \\
\fi
\ifgencode
  \OERef{DOREORDER}   & разрешить переупорядочение операторов\\
\fi
\ifgenc
  \OERef{GENCDIV}     & использовать C-оператор деления \\
  \OERef{GENCPP}      & генерировать код на C++             \\
  \OERef{GENCONSTENUM} & генерировать перечислимые типы как константы \\
\fi
\ifgencode
  \OERef{GENCPREF}    & добавлять префикс-подчерк \\
\fi
\ifgenc
  \OERef{GENCTYPES}   & генерировать C-типы                     \\
  \OERef{GENDATE}     & вставить дату в C файл  \\
\fi
\OERef{GENDEBUG}      & генерация в отладочном режиме        \\
\ifgencode
  \OERef{GENFRAME}    & всегда создавать область процедурных данных \\ 
\fi
\ifgenc
  \OERef{GENFULLFNAME} & директиве \verb'#lineno' выдавать полное имя файла \\
\fi
  \OERef{GENHISTORY}  & генерировать информацию для посмертной истории  \\
\ifgenc
  \OERef{GENKRC}      & генерировать код на K\&R C               \\
  \OERef{GENPROCLASS} & генерировать спецификацию процедурного класса \\
  \OERef{GENPROFILE}  & генерировать дополнительный код для профилирования \\
\fi
\ifgencode
  \OERef{GENPTRINIT}  & локальная инициализация указателей  \\
\fi
\ifgenc
  \OERef{GENSIZE}    & оценка размеров типов               \\
  \OERef{GENTYPEDEF} & генерировать typedef для типов         \\
  \OERef{INDEX16}    & 16-битовые индексы                   \\
\fi
  \OERef{LINENO}     & сохранить номера строк
                   \ifgenc в C файлах \fi
                   \ifgencode в объектных файлах \fi
                                                        \\
\ifgenc
  \OERef{NOEXTERN}   & не использовать прототипы для внешних C процедур \\
  \OERef{NOHEADER}   & не создавать C файл-заголовок    \\
\fi
\ifgenc
  \OERef{NOOPTIMIZE}  & выключить оптимизации \\
\fi
\ifgencode
  \OERef{NOPTRALIAS} & считать, что указатели не используются как алиасы \\
\fi
\ifgencode
  \OERef{ONECODESEG} & генерировать один сегмент кода \\
\fi
  \OERef{PROCINLINE} & разрешить расписывание процедур \\
\ifgencode
  \OERef{SPACE}      & важнее размер кода, чем его скорость \\
\fi
\ifgenc
  \OERef{TARGET16}   & в C тип {\tt int} 16-битовый          \\
\fi
  \OERef{VERSIONKEY} & добавить ключ версии в инициализацию модуля \\
\hline
\end{tabular}
\end{center}
\caption{Опции контроля генерации кода}\label{table:opt:code}
\index{опции!контроль кода}
\end{table}

\begin{table}[htbp]
\begin{center}
\begin{tabular}{|l|l|}
\hline
\bf Опция  & \bf Значение \\
\hline
\OERef{BSCLOSURE}  & опция управления просмотром \\
\OERef{BSREDEFINE} & опция управления просмотром \\
\OERef{CHANGESYM}  & разрешается изменить символьный файл \\
\OERef{FATFS}      & ограничить длину имен файлов до 8.3 \\
\OERef{GCAUTO}     & автоматический вызов сборки мусора \\
\OERef{LONGNAME}   & использовать полные имена в пакетных файлах  \\
\OERef{M2}         & компилировать с Модулы-2 \\
\OERef{MAIN}       & головной Оберон-2 модуль     \\
\OERef{MAKEDEF}    & генерировать определяющий модуль      \\
\OERef{MAKEFILE}   & генерировать make-файл        \\
\OERef{O2}         & компилировать с Оберона-2  \\
\OERef{OVERWRITE}  & всегда модифицировать старый файл \\
\OERef{VERBOSE}    & дополнительные сообщения компилятора \\
\OERef{WOFF}       & не сообщать о предупреждениях    \\
\OERef{XCOMMENTS}  & сохранить экспортируемые комментарии \\
\hline
\end{tabular}
\end{center}
\caption{Разные опции}\label{table:opt:misc}
\index{Опции!разные опции}
\end{table}

\pagebreak % To not break the following section with tables

\section{Справочник по опциям}\label{opt:bool:list}

В этой главе все опции перечислены в алфавитном порядке. Опции, которые
можно помещать в любое место исходных текстов, помечены как {\em текст} 
(см. также \ref{m2:pragmas}). 
Некоторые опции можно помещать в текст, но только перед заголовком
(т.е. до появления ключевых слов \verb'"DEFINITION"', \verb'"IMPLEMENTATION"'
или \verb'"MODULE"') Такие опции помечены как {\em заголовок}.
Если опция не помечена ни одним из этих слов, то результат 
помещения ее в исходный текст неопределен.

Режимы работы, на которые опция влияет, перечислены в квадратных скобках
([])  после имени опции; символ
'*' означает все режимы. Например, [browse]
означает, что опция используется компилятором только в режиме BROWSE.

{\bf Замечание:} В режимах MAKE и PROJECT, при компиляции модуля компилятор
переключается в режим COMPILE.

Все опции проверок во время исполнения по умолчанию включены.
Остальные опции, если это не оговорено явно, по умолчанию выключены.

\newcommand{\inline}{({\em текст})}
\newcommand{\header}{({\em заголовок})}

\ifonline \else
\begin{description}
\fi

\OptHead{\_\_GEN\_X86\_\_}{генерация кода для 386/486/Pentium/PentiumPro}
        \MLBegin{}\ModeC{}\MLEnd{}

        
Компилятор выставляет эту опцию при генерации кода для
386/486/Pentium/PentiumPro.

Эта опция может быть использована для того, чтобы различные 
фрагменты текста компилировать на разные платформы.
См. также \ref{m2:pragmas:cc}.

\OptHead{\_\_GEN\_C\_\_}{генерировать ANSI C}
        \MLBegin{}\ModeC{}\MLEnd{}

Компилятор выставляет эту опцию при генерации кода на C.

Эта опция может быть использована для того, чтобы различные 
фрагменты текста компилировать на разные платформы.
См. также \ref{m2:pragmas:cc}.

\OptHead{ASSERT}{включить генерацию ASSERT}
        \MLBegin{}\ModeC{}\MLEnd{} \inline

Если эта опция OFF, то компилятор игнорирует все вызовы стандартной
процедуры  ASSERT.

По умолчанию ON.


\OptHead{BSCLOSURE}{опция контроля просмотра}
        \MLBegin{}\ModeB{}\MLEnd{}

Включить все видимые методы.

Если эта опция выставлена, то при создании псевдо-определяющего модуля
в декларацию каждого типа записи будут включены
декларации всех ее определенных и унаследованных методов. 
%        If the option is set ON, the browser includes all
%        defined and inherited type-bound procedure
%        declarations with all record declarations when creating
%        a pseudo-definition   module.  See also
См. также \ref{o2:env:makedef}.

\OptHead{BSREDEFINE}{опция контроля просмотра}
        \MLBegin{}\ModeB{}\MLEnd{}

\nopagebreak
Включить все переопределенные методы.

Если эта опция выставлена, то при создании псевдо-определяющего модуля
в декларацию каждого типа записи будут включены первоначальные
декларации всех ее переопределенных методов.
См. также \ref{o2:env:makedef}.

\OptHead{CHANGESYM}{Разрешение на перезапись символьного файла}
        \MLBegin{}\ModeC{}\MLEnd{} \header

Разрешает изменить интерфейс модуля (символьный файл).

\ot{} компилятор создает временный символьный файл при каждой
компиляции \ot{} модуля. Затем он сравнивает его с имеющимся,
и при необходимости заменяет. Если эта опция выключена
(по умолчанию), то при изменении символьного файла (а значит, и 
интерфейса модуля) компилятор сообщает об ошибке и не заменяет 
старый символьный файл.

{\bf  Замечание:}  Если выставлена опция \OERef{M2CMPSYM},
то так же поступает и \mt{} компилятор при компиляции 
определяющего модуля; т.е. для изменения интерфейса модуля 
требуется выставить опцию {\bf CHANGESYM}.

\OptHead{CHECKDINDEX}{проверка границ динамических массивов}
        \MLBegin{}\ModeC{}\MLEnd{} \inline

Проверка границ динамических массивов.

Если эта опция выставлена (ON), то компилятор генерирует проверки границ
динамических массивов (POINTER TO ARRAY OF T).

По умолчанию ON.

\OptHead{CHECKDIV}{проверка положительности делителя (DIV и MOD)}
        \MLBegin{}\ModeC{}\MLEnd{} \inline

Если эта опция выставлена (ON), то компилятор генерирует проверку,
положителен ли делитель, для операторов DIV и MOD.

По умолчанию ON.

\OptHead{CHECKINDEX}{проверка границ статических массивов}
        \MLBegin{}\ModeC{}\MLEnd{} \inline

Проверка границ статических массивов.

Если эта опция выставлена (ON), то компилятор генерирует проверки границ
всех нединамических массивов (см. опцию \OERef{CHECKDINDEX}).

По умолчанию ON.

\OptHead{CHECKNIL}{проверка разыменования NIL}
        \MLBegin{}\ModeC{}\MLEnd{} \inline

Если эта опция выставлена (ON), то компилятор генерирует проверку на NIL
для каждого разыменования.

По умолчанию ON.

\OptHead{CHECKPROC}{проверка вызова формальной процедуры}
        \MLBegin{}\ModeC{}\MLEnd{} \inline

Если эта опция выставлена (ON), то компилятор генерирует проверку на NIL
при каждом вызове процедурной переменной.

По умолчанию ON.

\OptHead{CHECKRANGE}{проверка границ (отрезки и перечислимые типы)}
        \MLBegin{}\ModeC{}\MLEnd{} \inline

Если эта опция выставлена (ON), то компилятор генерирует проверку 
границ переменных типа отрезок и перечислимых типов.

По умолчанию ON.

\OptHead{CHECKSET}{проверка границ для операций на множествах}
        \MLBegin{}\ModeC{}\MLEnd{} \inline

Если эта опция выставлена (ON), то компилятор генерирует проверку 
границ для операций на множествах (INCL, EXCL, конструкторы множеств)

По умолчанию ON.

\OptHead{CHECKTYPE}{динамический контроль типов (только \ot{})}
        \MLBegin{}\ModeC{}, \ot{} only\MLEnd{} \inline

Если эта опция выставлена (ON), то компилятор генерирует 
динамический контроль типов.

По умолчанию ON.

\ifgenvax
\OptHead{CODENAMEPREFIXED}{префиксовать имя кодового сегмента}
        \MLBegin{}\ModeC{}\MLEnd{}

По умолчанию выключена, включается либо отдельно, 
либо автоматически при включении \OERef{GENDEBUG}.  
Означает, что имя кодового сегмента будет префиксовываться 
именем модуля + подчерк. Необходима для того, 
чтобы все секции программы имели разные имена при отладке, 
так как связывание физической секции с объектом (секцией) 
в отладочной информации происходит по именам. 
При штатной эксплуатации необходимо выключить, 
для получения одной кодовой секции.
\fi


\ifgenc
\OptHead{COMMENT}{копировать комментарии в C код}
        \MLBegin{}\ModeC{}\MLEnd{} \header

Если эта опция выставлена (ON), то компилятор копирует комментарии 
из текста в подходящее место генерируемого C-кода. Комментарии
из \ot{} модуля вносятся только в файл C-кода, но не в файл-заголовок.
\fi

\ifgenc
\OptHead{CONVHDRNAME}{использовать имя файла в директиве '\#include'}
        \MLBegin{}\ModeC{}\MLEnd{}

Если эта опция выставлена (ON), то компилятор использует
имя файла в директиве '\#include'. Иначе он использует имя, состоящее
из имени модуля и расширения для файла-заголовка.
\fi

\ifgenc\else
\OptHead{COVERFLOW}{проверка переполнения беззнакового целого}
        \MLBegin{}\ModeC{}\MLEnd{} \inline

Если эта опция выставлена (ON), то компилятор генерирует проверку 
переполнения для всех арифметических операторов с беззнаковыми целыми.

По умолчанию ON.
\fi

\ifgenc
\OptHead{CSTDLIB}{определяющий модуль стандартной C-библиотеки}
        \MLBegin{}\ModeC{},{\em только для иноязычных опр. модулей}\MLEnd{} \header

Эта опция имеет эффект только при компиляции определяющего модуля
для иноязычной программы; иначе она игнорируется.
Если эта опция выставлена, то компилятор использует в директиве 
\verb|#include| угловые скобки \verb+<>+ для импорта из
иноязычного модуля. В противном случае используются кавычки.
\begin{verbatim}
  #include <stdio.h>
  #include "MyLib.h"
\end{verbatim}
\fi

\ifgenvax
\OptHead{DATAASOBJECT}{генерировать все данные в одну секцию}
        \MLBegin{}\ModeC{}\MLEnd{}
Когда включена, то все данные генерируются в одну секцию,
Когда выключена, то каждый объект типа данные попадает в свою секцию
которая имеет имя самого объекта.
\fi

\ifgencode
\OptHead{DEFLIBS}{записать в объектный файл стандартные имена библиотек}
        \MLBegin{}\ModeC{}\MLEnd{}

Если эта опция выставлена, то компилятор 
записывает в объектный файл стандартные имена библиотек.

По умолчанию ON.
\fi

\ifgenc
\OptHead{DIFADR16}{SYSTEM.DIFADR возвращает 16-битовый результат}
        \MLBegin{}\ModeC{}\MLEnd{}

Если эта опция выставлена, то компилятор считает, что
разница между адресами --- всегда 16-битовое целое;
иначе он считает эту разницу 32-битовой.

Правильная установка этой опции нужна для реализации 
системных функций ADDADR, SUBADR и DIFADR.

Более подробно см. \ref{maptoc:opt:config}.
\fi

\ifgencode
\OptHead{DOREORDER}{разрешение на перестановку команд}
        \MLBegin{}\ModeC{}\MLEnd{} \header

Эта опция разрешает использовать при генерации кода для x86
механизм {\em перестановки команд}. Он переупорядочивает 
(если возможно) команды
процессора так, чтобы их можно было исполнять параллельно.

        
{\bf Замечание}: эта оптимизация значительно замедляет процесс
компиляции, но дает выигрыш в скорости исполнения примерно 5-15\%.
\fi

\OptHead{FATFS}{длина имен файлов 8.3}
        \MLBegin{}*\MLEnd{}

Заставляет компилятор ограничивать  имена файлов в соответствии с 
FAT-ограничением "8.3".

\OptHead{GCAUTO}{разрешает неявный вызов сборки мусора}
        \MLBegin{}\ModeC{},{\em только головные модули}\MLEnd{} \header

Разрешает неявный вызов сборки мусора в генерируемой программе.
Эта опция имеет эффект только при компиляции головных модулей.
Мы рекомендуем выставлять ее в проектном файле или в файле
конфигурации.

\ifgenc
\OptHead{GENCDIV}{использовать C-деление}
        \MLBegin{}\ModeC{}\MLEnd{} \header

Если эта опция выставлена, то компилятор транслирует операторы
\verb'DIV' и \verb'MOD' в операторы C \verb'/' и \verb'%'.
По умолчанию \xds{} порождает ???. %???

\OptHead{GENCPP}{генерация в C++}
        \MLBegin{}\ModeC{}\MLEnd{} \header

Если эта опция выставлена, то компилятор порождает код на C++.
По умолчанию \xds{} порождает код на ANSI C (см. также \OERef{GENKRC}).

\OptHead{GENCONSTENUM}{транслировать перечисления в константы}
        \MLBegin{}\ModeC{}\MLEnd{} \header

Если эта опция выставлена, то компилятор с Модулы-2 транслирует 
перечислимые типы в набор констант. Иначе они транслируются в
перечислимые типы C (\verb'enum').
\fi

\ifgencode
\OptHead{GENCPREF}{подчерк-префикс}
        \MLBegin{}\ModeC{}\MLEnd{} \header

Если эта опция выставлена, то компилятор префиксует символом "подчерк"
все видимые имена в объектном файле.
\fi

\ifgenc
\OptHead{GENCTYPES}{генерировать C-типы}
        \MLBegin{}\ModeC{}\MLEnd{} \header

Если эта опция выставлена, то компилятор использует стандартные имена
типов C, где возможно, иначе он использует имена, определенные в системе
поддержки исполнения.

Более подробно см. \ref{maptoc:opt:gen}.
\fi

\ifgenc
\OptHead{GENDATE}{указать дату в C файле}
        \MLBegin{}\ModeC{}\MLEnd{}

Если эта опция выставлена, то компилятор указывает в порожденном C-файле
текущую дату.

По умолчанию ON.
\fi

\OptHead{GENDEBUG}{генерация в режиме отладки}
        \MLBegin{}\ModeC{}\MLEnd{} \header

\ifgencode
Если эта опция выставлена, то компилятор добавляет в объектный файл
дополнительную информацию для отладки (в формате OMF). В настоящей версии,
эта информация включает все глобальные переменные и их типы.

В некоторых редких случаях включение этой опции ухудшает качество кода.
\fi
\ifgenc
Если эта опция выставлена, то компилятор генерирует код в отладочном
режиме. Если Ваша программа скомпилирована в этом режиме, то при
ее  аварийном завершении система поддержки исполнения выдаст стек вызова
процедур (имя файла и номер строки). 
Если при этом выставлена опция \OERef{LINENO}, то информация выводится 
в терминах исходных \ot{}/\mt{} текстов, иначе --- в терминах
порожденных транслятором C текстов.

{\bf Замечание:} Эта опция заметно увеличивает и замедляет программу.

Более подробно см. \ref{maptoc:opt:gen}.
\fi

\ifgencode
\OptHead{GENFRAME}{всегда порождать область процедурных данных}
        \MLBegin{}\ModeC{}\MLEnd{} \header

Если эта опция выставлена, то компилятор 
всегда порождает область процедурных данных. Это может оказаться
необходимым для облегчения процесса отладки.
\fi

\ifgenc
\OptHead{GENFULLFNAME}{директива '\#lineno' выдает полное имя файла}
        \MLBegin{}\ModeC{}\MLEnd{}

Если эта опция выставлена, то компилятор по директиве \verb'#lineno'  
порождает полное имя файла, включающее путь от директивы перенаправления.
\fi

\OptHead{GENHISTORY}{включить посмертную историю}
        \MLBegin{}\ModeC{}\MLEnd{} \header

Если эта опция выставлена, то при
аварийном завершении программы система поддержки исполнения 
выдаст стек вызова процедур (имя файла и номер строки).
Эту опцию надо выставить при компиляции головного модуля программы.
Тогда нужная часть системы поддержки будет добавлена к коду
программы. Все модули программы должны быть скомпилированы с
выставленной опцией
\OERef{LINENO}.

См. пример в \ref{start:debug}.

\ifgencode
{\bf Замечание:} В некоторых случаях выведенный стек процедур
может содержать неверные строки, т.е. процедуры, которые в действительности 
не вызывались (см. \ref{rts:history}).
\fi

\ifgenc
\OptHead{GENKRC}{генерация в K\&R C}
        \MLBegin{}\ModeC{}\MLEnd{} \header

Если эта опция выставлена, то транслятор генерирует код на K\&R C.
Она может понадобиться для переноса программ на платформы, для которых
нет компиляторов с ANSI C. По умолчанию
\xds{} генерирует код на ANSI C.

Более подробно см. \ref{maptoc:proc} и \ref{maptoc:opt:gen}.
\fi

\ifgenc
\OptHead{GENPROCLASS}{генерировать спецификацию проц. класса}
        \MLBegin{}\ModeC{}\MLEnd{} \header

Если эта опция выставлена, то транслятор вставляет вызов специального
макроса \verb|X2C_PROCLASS| в функции-прототипы для процедур Модулы-2 и
Оберона-2.

По умолчанию ON.

Более подробно см. в \ref{maptoc:proc} и \ref{maptoc:opt:gen}.
\fi

\ifgenc
\OptHead{GENPROFILE}{генерировать дополнительный код для профилирования}
        \MLBegin{}\ModeC{}\MLEnd{} \header

Если эта опция выставлена, то транслятор 
генерирует дополнительный код для профилирования (т.е. для контроля
за частотой исполнения различных инструкций). 
См. более подробно в \ref{maptoc:opt:gen}.
\fi

\ifgencode
\OptHead{GENPTRINIT}{инициализировать локальные указатели}
        \MLBegin{}\ModeC{}, только \ot{} \MLEnd{} \header

Если эта опция выставлена, то транслятор генерирует команды 
инициализации для всех локальных указателей, включая переменные,
поля записей и элементы массивов. Значения полей и элементов
массивов, не являющиеся указателями, не определяются.

По умолчанию ON.
\fi

\ifgenc
\OptHead{GENSIZE}{оценка размеров типов}
        \MLBegin{}\ModeC{}\MLEnd{} \header

В соответствии со стандартом языков Модула-2 и Оберон-2,
в константных выражениях можно использовать функцию SIZE.
Однако при компиляции на промежуточный язык компилятор обычно не
знает размеров большей части типов. 

Когда эта опция выключена, компилятор выдает сообщение об ошибке,
в большинстве случаев, когда функция SIZE используется в константных 
выражениях:
        \begin{verbatim}
           CONST size = SIZE(MyRecord);
        \end{verbatim}
        
{\bf Замечание:} 
размеры целых типов, BOOLEAN, CHAR, типов множеств
известны компилятору.

Подробнее см. \ref{maptoc:opt:sizeof}.
\fi

\ifgenc
\OptHead{GENTYPEDEF}{генерировать typedef для типов}
        \MLBegin{}\ModeC{}\MLEnd{} \inline

В языке C есть два способа декларации записи
({\tt struct}) --- с использованием {\tt typedef} и без него.
Обычно выбор способа несуществен. Однако, при написании 
интерфейсного модуля на "иностранном" языке
(см. Главу \ref{multilang}) желательно уметь контролировать
описания типов.

Когда эта опция включена, компилятор использует форму
{\tt typedef} во всех декларациях типов. Опцию можно устанавливать 
и в исходном тексте, напр.:
\begin{verbatim}
  <* GENTYPEDEF + *>
  TYPE FILE = RECORD END;
  <* GENTYPEDEF - *>
\end{verbatim}

Подробнее см. \ref{maptoc:opt:gen}.
\fi

\ifgenc
\OptHead{INDEX16}{16-битовые индексы}
        \MLBegin{}\ModeC{}\MLEnd{}

Если эта опция выставлена, то компилятор подразумевает, что размер
индекса на платформе -- 16 бит. По умолчанию предполагается, что
размер индекса - 32 бита.

Подробнее см. \ref{maptoc:opt:config}.
\fi

\ifgenc\else
\OptHead{IOVERFLOW}{проверка переполнения целого}
        \MLBegin{}\ModeC{}\MLEnd{} \inline

Если эта опция выставлена, то компилятор генерирует проверку на переполнение 
для всех целых арифметических операторов.

По умолчанию ON.
\fi

\OptHead{LINENO}{генерировать номера строк}
        \MLBegin{}\ModeC{}\MLEnd{} \header

\ifgenc
Если эта опция выставлена, то компилятор вставляет в C-текст
строки следующего вида:
\begin{verbatim}
  #line lineno [module]
\end{verbatim}
для каждого сгенерированного оператора, 
и в результате C-компилятор и другие средства будут ссылаться
на изначальный текст, а не на сгенерированный C-код.
\fi
\ifgencode
Если эта опция выставлена, то компилятор вставляет в объектный файл
информацию о номерах строк исходного текста. Эту опцию необходимо
выставлять для получения посмертной истории (см. опцию \OERef{GENHISTORY}), 
и для отладки.
\fi


\OptHead{LONGNAME}{использовать полные имена в пакетном файле}
        \MLBegin{}\ModeM{},\ModeP{}\MLEnd{}

Если эта опция выставлена, то компилятор добавляет полный путь как префикс
к каждому имени файла в создаваемом пакетном файле.
См. также \ref{xc:modes:batch}.

\OptHead{M2}{компилировать с Модулы-2}
        \MLBegin{}\ModeC{}\MLEnd{}

Если эта опция выставлена, то \xds{} 
вызывает \mt{} компилятор независимо от расширения имени файла.
В режимах MAKE и PROJECT эта опция игнорируется.


\OptHead{M2ADDTYPES}{добавить типы SHORT и LONG}
        \MLBegin{}\ModeC{}, только \mt{} \MLEnd{} \header

Если эта опция выставлена, то компилятор распознает типы
{\tt SHORTINT}, {\tt LONGINT}, {\tt SHORTCARD}, {\tt LONGCARD}
как предопределенные идентификаторы.

{\bf Замечание:} Использование этих дополнительных типов
может вызвать проблемы с переносом программ на другие компиляторы.

\OptHead{M2BASE16}{16-битовые основные типы в Модуле-2}
        \MLBegin{}\ModeC{}, только \mt{}\MLEnd{} \header

Если эта опция выставлена, то основные типы
{\tt INTEGER}, {\tt CARDINAL} и {\tt BITSET} в Модуле-2
считаются 16-битовыми (по умолчанию они 32-битовые).

\OptHead{M2CMPSYM}{сравнивать символьные файлы в Модуле-2}
        \MLBegin{}\ModeC{}, только \mt{}\MLEnd{}

Если эта опция выставлена, то Модула-2 компилятор сравнивает 
симфайл, порожденный для определяющего модуля, с его старой
версией (в точности как  \ot{} компилятор). Если символьные файлы 
совпадают, то сохраняется старый, иначе он обновляется, но только 
если выставлена опция \OERef{CHANGESYM}.

\OptHead{M2EXTENSIONS}{разрешить расширения Модулы-2}
        \MLBegin{}\ModeC{}, только\mt{}\MLEnd{} \header

Если эта опция выставлена, то Модула-2 компилятор разрешает
использовать расширения языка, такие, как знак комментария
\verb|"--"|, параметры по чтению, и~т.д.

{\bf Замечание:} Использование расширений языка может вызвать 
проблемы при переносе программ на другие компиляторы.

\OptHead{MAIN}{головной модуль в Обероне-2}
        \MLBegin{}\ModeC{}, только \ot{}\MLEnd{} \header

Если эта опция выставлена, то Оберон-2 компилятор создает 
головную процедуру для модуля
(функция `main')  (см. \ref{o2:env:main}).

Рекомендуется устанавливать в тексте модуля.

\OptHead{MAKEDEF}{генерация определяющего модуля}
        \MLBegin{}\ModeC{}, только \ot{}\MLEnd{}

Заставляет компилятор с Оберона-2 генерировать (псевдо-)определяющий
модуль после успешной компиляции Оберон-2-модуля. Если при этом 
выставлена опция \OERef{XCOMMENTS}, то компилятор заносит в него и 
так называемые {\em экспортируемые} комментарии
(т.е. комментарии, начинающиеся с `\verb|(**|').

См. \ref{o2:env:makedef}.

\OptHead{MAKEFILE}{генерировать make-файл}
        \MLBegin{}\ModeP{}\MLEnd{}

Заставляет \xds{} генерировать make-файл после успешной компиляции
всего проекта. См. также \ref{xc:modes:gen}.

\ifgenc
\OptHead{NOEXTERN}{не декларировать внешние C процедуры}
        \MLBegin{}\ModeC{}\MLEnd{} \inline

Если эта опция выставлена, то компилятор не генерирует C деклараций
для процедур, определенных как внешние.

Подробнее см. \ref{multilang:extproc}.
\fi

\ifgenc
\OptHead{NOHEADER}{не генерировать файл-заголовок}
        \MLBegin{}\ModeC{},\ModeM{},\ModeP{}\MLEnd{} \header


Если эта опция выставлена, то компилятор не генерирует C файл-заголовок.

См. также \ref{maptoc:opt:foreign}.
\fi

\ifgenc
\OptHead{NOOPTIMIZE}{выключить оптимизации}
        \MLBegin{}\ModeC{}\MLEnd{}

Если эта опция выключена (по умолчанию), то компилятор производит
некоторые оптимизации, такие, как вычисление константных выражений,
распространение констант, и~т.п. Если опция включена, то сгенерированный
текст менее эффективен, но более понятен.

Мы рекомендуем включать эту опцию только если вы используете
XDS как транслятор, т.е. если вы собираетесь читать или поддерживать
сгенерированный C-код.
\fi

\ifgencode
\OptHead{NOPTRALIAS}{считать, что указатели не используются как алиасы}
        \MLBegin{}\ModeC{}\MLEnd{} \header

Если эта опция выставлена, то компилятор подразумевает, что
указатели не используются как алиасы, т.е. что указатели не указывают 
на не структурные переменные. Получить указатель на переменную 
можно лишь используя низкоуровневые средства из модуля
SYSTEM. Мы рекомендуем выставлять эту опцию для всех модулей,
кроме низкоуровневых.

{\bf Замечание:} Если она выставлена, то качество кода улучшается.
\fi

\OptHead{O2}{компилировать с Оберона-2}
        \MLBegin{}\ModeC{}\MLEnd{}

Если эта опция выставлена, то \xds{} 
вызывает \ot{} компилятор независимо от расширения имени файла.
В режимах MAKE и PROJECT эта опция игнорируется.


\OptHead{O2EXTENSIONS}{разрешить расширения Оберона-2}
        \MLBegin{}\ModeC{},только \ot{}\MLEnd{} \header

Если эта опция выставлена, то Оберон-2 компилятор разрешает
использовать расширения языка (см. \ref{o2:ext}).

{\bf Замечание:} Использование расширений языка может вызвать 
проблемы при переносе программ на другие компиляторы.

\OptHead{O2ISOPRAGMA}{разрешить в Обероне ISO Модула-2 прагмы}
        \MLBegin{}\ModeC{},только \ot{}\MLEnd{}

Если эта опция выставлена, то компилятор разрешает использовать
в Обероне-2 прагмы в стиле ISO M2 (\verb|<* *>|).

        См. \ref{o2:pragmas} и \ref{m2:pragmas}.

{\bf Замечание:} Использование ISO Модула-2 прагм может вызвать 
проблемы при переносе программ на другие компиляторы.

\OptHead{O2NUMEXT}{разрешить математические расширения Оберона-2}
        \MLBegin{}\ModeC{},только \ot{}\MLEnd{} \header

Если эта опция выставлена, то компилятор разрешает использовать
в Обероне-2 математические расширения (см. \ref{o2:ext}),
в~т.ч. типы {\tt COMPLEX} и {\tt LONGCOMPLEX}, и 
оператор возведения в степень. 

{\bf Замечание:} Использование расширений языка может вызвать 
проблемы при переносе программ на другие компиляторы.

\ifgencode
\OptHead{ONECODESEG}{генерировать один сегмент кода}
        \MLBegin{}\ModeC{}\MLEnd{}

Если эта опция выставлена, то компилятор генерирует один сегмент кода,
содержащий весь код модуля, иначе он генерирует отдельный сегмент кода 
для каждой процедуры.
\fi

\OptHead{OVERWRITE}{всегда перезаписывать старый файл}
        \MLBegin{}*\MLEnd{}

Эта опция меняет директорию, в которой компилятор создает новые
файлы. Если она выключена, то файл создается в первой директории
из списка путей поиска, заданного в образце, соответствующем имени файла
Если она включена, то компилятор старается перезаписать старый файл.
См. также \ref{xc:red}.

\ifgenvax
\OptHead{PRESERVENAMECASE}{сохранять регистр в именах объектов}
        \MLBegin{}\ModeC{}\MLEnd{}

Когда эта опция включена, имена объектов сохраняют значения регистра символов
(большие/маленькие),
когда выключена - все имена приводятся в верхний регистр.
\fi

\OptHead{PROCINLINE}{разрешение раскрывать процедуры}
        \MLBegin{}\ModeC{}\MLEnd{}

Если эта опция выставлена, то компилятор может раскрывать
процедуры: заменять вызов процедуры на ее текст. Это 
приводит к экономии накладных расходов на вызов, передачу
параметров, сохранение значений регистров и~т.п. Иногда это 
приводит к дальнейшим оптимизациям, поскольку становятся 
видимыми фактические параметры вызова.

Процедура не расписывается, если:
        \begin{itemize}
        \item компилятор счел ее слишком большой или слишком сложной,
        \item она вызывается слишком часто, или
        \item процедура рекурсивная.
        \end{itemize}

\ifgencode
\OptHead{SPACE}{важнее размер кода, чем его скорость}
        \MLBegin{}\ModeC{}\MLEnd{}

Если эта опция выставлена, то компилятор оптимизирует код, стараясь
уменьшить его размер, тогда как по умолчанию он старается
произвести более быстрый код.
\fi

\OptHead{STORAGE}{встроенные процедуры выделения памяти в Модуле-2}
        \MLBegin{}\ModeC{}, только \mt{}\MLEnd{} \header

Если эта опция выставлена, то компилятор в качестве стандартных
процедур NEW и DISPOSE использует встроенные процедуры аллокации
и освобождения памяти.

{\bf Замечание:} Использование этой опции может вызвать 
проблемы при переносе программ на другие компиляторы.

\ifgenc
\OptHead{TARGET16}{В C тип 'int' 16-битовый}
        \MLBegin{}\ModeC{}\MLEnd{}

Если эта опция выставлена, то компилятор подразумевает, что
на целевой платформе тип {\tt int} в C 16-битовый.

Подробнее см. в \ref{maptoc:opt:config}.
\fi

\OptHead{VERBOSE}{выдавать дополнительные сообщения}
        \MLBegin{}\ModeM{},\ModeP{}\MLEnd{}

Если эта опция выставлена, то компилятор будет сообщать причину
перекомпиляции каждого модуля (см. \ref{xc:smart}).

\OptHead{VERSIONKEY}{установить код версии в инициализации модуля}
        \MLBegin{}\ModeC{}\MLEnd{}

Эта опция введена для проверки версии во время сборки.
Если она выставлена, то компилятор образует имя тела модуля как
конкатенацию
        \begin{itemize}
        \item имени модуля,
        \item строки \verb+"_BEGIN_"+,
        \item времени последней модификации,
\ifgenc
        \item значений опций \OERef{TARGET16}, \OERef{INDEX16} и
        \OERef{DIFADR16} в упакованном виде.
\fi
\ifgencode
        \item значения параметра \OERef{ALIGNMENT}. % ???
\fi
        \end{itemize}
Если определяющий (или Оберон-2) модуль, импортируемый
различными единицами компиляции, имеет одну и ту же версию, то
при каждом вызове тела этого модуля будет использоваться одно и
то же имя. Во всех остальных случаях во время сборки будет
обнаружена ошибка.

Если опция выключена, то компилятор порождает имя 
\verb|<module_name>_BEGIN|.

{\bf Замечание:} опцию следует выставлять при компиляции определяющего
или обероновского модуля.

\ifgenc
Подробнее см. в \ref{maptoc:idents}.
\fi

\OptHead{WOFF}{не сообщать о предупреждениях}
        \MLBegin{}*\MLEnd{}  \inline

Если выставлена опция WOFF\# (напр. WOFF301),
то компилятор не будет сообщать о предупреждении с номером
\# (301 в приведенном примере).
Номера и тексты предупреждений см. в файле {\bf xc.msg}.
Опция WOFF без параметра выключает все предупреждения.

\OptHead{XCOMMENTS}{сохранить экспортируемые комментарии}
        \MLBegin{}\ModeC{}, только \ot{}\MLEnd{}

Если эта опция выставлена, то компилятор в режиме BROWSE
включит в псевдо-определяющий модуль все
{\em экспортируемые} комментарии
(т.е. комментарии, начинающиеся с `(**')

См. также \ref{o2:env:makedef}.

\ifonline \else
\end{description}
\fi

\section{Параметры}\label{opt:equ}
\index{параметры}

{\em Параметр} задается парой ({\tt name},{\tt value}), где {\tt value} 
--- произвольная строка. Некоторые параметры могут иметь значением
пустую строку.

Для декларации нового параметра и для установки значения параметра
используются директивы установки компилятора (см. \ref{config:options}).

Параметры могут устанавливаться в файле конфигурации
(см. \ref{xc:cfg}), в командной строке
(см. \ref{xc:modes}) или в проектном файле (см.
\ref{xc:project}).
Некоторые параметры можно устанавливать в исходном тексте,
в произвольном месте (помечены словом {\em текст}) или только
в заголовке модуля (помечены словом {\em заголовок}).
В каждый момент действует последнее выставленное значение параметра.

Алфавитный список всех параметров см. в \ref{opt:equ:list}.
\ifonline\else
См. также таблицы
\ref{table:equ:ext} (стр. \pageref{table:equ:ext}),
\ref{table:equ:code} (стр. \pageref{table:equ:code}),
\ref{table:equ:misc} (стр. \pageref{table:equ:misc})
\fi

\begin{table}[htbp]
\begin{center}
\begin{tabular}{|l|c|l|}
\hline
\bf Имя          & \bf По умолчанию & \bf Значение \\
\hline
\OERef{BATEXT}   & .bat    & пакетный файл для перекомпиляции\\
\OERef{BSDEF}    & .odf    & псевдо-определяющий модуль    \\
\ifgenc
\OERef{CODE}     & \Code   & сгенерированный файл на C   \\
\fi
\ifgencode
\OERef{CODE}     & \Code  & объектный файл        \\
\fi
\OERef{DEF}      & .def    & \mt{} определяющий модуль     \\
\ifgenc
\OERef{HEADER}   & \Header & файл-заголовок на C   \\
\fi
\OERef{MKFEXT}   & .mkf    & make-файл                                     \\
\OERef{MOD}      & .mod    & \mt{} реализующий или программный модуль  \\
\OERef{OBERON}   & .ob2    & \ot{} модуль               \\
\OERef{OBJEXT}   & \dotObj & объектный файл                           \\
\OERef{PRJEXT}   & .prj    & проектный файл                           \\
\OERef{SYM}      & .sym    & символьный файл\\
\hline
\end{tabular}
\end{center}
\caption{Расширения имен файлов}\label{table:equ:ext}
\index{опции!расширения имен файлов}
\end{table}

\begin{table}[htbp]
\begin{center}
\begin{tabular}{|l|c|p{6.0cm}|}
\hline
\bf Имя          & \bf По умолчанию & \bf Значение \\
\hline
\iflinux
\OERef{ALIGNMENT}  & 4 & выравнивание данных ({\em см. ниже})\\
\else
\OERef{ALIGNMENT}  & 1 & выравнивание данных \\
\fi
\ifgencode
\iflinux
\OERef{CC}         & GCC & совместимость с C компилятором \\
\else
\OERef{CC}         & WATCOM & совместимость с C компилятором \\
\fi
\OERef{CODENAME}   & \_TEXT & имя сегмента кода \\
\OERef{CPU}        & GENERIC & оптимизировать для CPU данного типа\\
\fi
\ifgenc
\OERef{COPYRIGHT}   &       & декларация авторских прав \\
\fi
\ifgencode
\OERef{DATANAME}   & \_DATA & имя сегмента данных\\
\fi
\OERef{ENUMSIZE}    & 4   & размер перечислимых типов по умолчанию \\
\OERef{GCTHRESHOLD} &  0  & порог для сборки мусора \\
\ifgenc
\OERef{GENIDLEN}   &   30  & макс. длина идентификатора в C тексте \\
\fi
\ifgenc
\OERef{GENINDENT}  &   3   & отступ \\
\fi
\ifgenc
\OERef{GENWIDTH}   &   78  & макс. длина строки в C тексте        \\
\fi
\OERef{HEAPLIMIT}  &    0  & предельный размер области памяти программы   \\
\ifgencode
\OERef{MINCPU}     &  386  & требуемый для исполнения процессор \\
\fi
\OERef{SETSIZE}    & 4     & размер малых типов множеств по умолчанию \\
\OERef{STACKLIMIT} &    0  & предельный размер стека программы  \\
\hline
\end{tabular}
\end{center}
\caption{Параметры генерации кода}\label{table:equ:code}
\index{опции!параметры контроля кода}
\end{table}

\begin{table}[htbp]
\begin{center}
\begin{tabular}{|l|c|p{7.0cm}|}
\hline
\bf Имя          & \bf По умолчанию & \bf Значение \\
\hline
\OERef{ATTENTION} & !     & "знак внимания" в файле-шаблоне \\
\OERef{BATNAME}   & out   & имя пакетного файла                         \\
\OERef{BATWIDTH}  &  128  & макс. длина строки в пакетном файле     \\
\OERef{BSTYLE}    &  DEF  & стиль просмотра (см. \ref{o2:env:makedef}) \\
\ifcomment
\OERef{COMPILE}   &       & командная строка для C-компилятора (см. \ref{seamless:comp}) \\
\fi
\OERef{COMPILERHEAP}  &    & предельная область памяти для компилятора \\
\OERef{COMPILERTHRES} &    & порог сборки мусора для компилятора \\
\OERef{DECOR}     & hrtp   & контроль за сообщениями компилятора \\
\ifgenc
\OERef{ENV\_HOST}  &        & рабочая платформа \\
\OERef{ENV\_TARGET} &       & целевая платформа \\
\fi
\OERef{ERRFMT}    & см. \ref{opt:errfmt} & формат сообщений об ошибках  \\
\OERef{ERRLIM}    &   16  & макс. количество ошибок               \\
\OERef{LINK}      &       & командная строка для линкера      \\
\OERef{LOOKUP}    &       & директива lookup              \\
\OERef{MKFNAME}   &       & имя make-файла                 \\
\OERef{PRJ}       &       & имя проектного файла           \\
\OERef{PROJECT}   &       & имя проекта                             \\
\OERef{TABSTOP}   &   8   & размер табуляции                \\
\OERef{TEMPLATE}  &       & имя шаблона (для make-файла)              \\
\hline
\end{tabular}
\end{center}
\caption{Разные параметры}\label{table:equ:misc}
\index{опции!разные параметры}
\end{table}
\pagebreak % To not break the following section with tables

\section{Справочник по параметрам}\label{opt:equ:list}

Режимы работы, на которые влияет параметр, перечислены в квадратных скобках
([])  после имени параметра; символ
'*' означает все режимы. Например, [browse]
означает, что параметр используется компилятором только в режиме BROWSE.

{\bf Замечание:} В режимах MAKE и PROJECT, при компиляции модуля компилятор
переключается в режим COMPILE.

\ifonline \else
\begin{description}
\fi

\EquHead{ALIGNMENT}{выравнивание данных}
        \MLBegin{}\ModeC{}\MLEnd{} \inline

Значение параметра {\em выравнивания данных}. 
Разрешенные значения: 1, 2, 4, 8. 
Подробнее см.
        \ifgenc \ref{maptoc:opt:sizeof} \fi.
        \ifgencode \ref{lowlevel:recrep} \fi.

        \iflinux
        
{\bf Замечание:} Поскольку библиотеки \XDS{} 
построены с помощью
GCC, использующего 4-байтовое выравнивание, Вы должны всегда иметь
параметр ALIGNMENT, равный 4, кроме случаев, когда Вы точно знаете,
зачем его изменяете.
Подробнее см. главу \ref{ccomp}.
        \fi

\EquHead{ATTENTION}{"знак внимания" в файле-шаблоне}
        \MLBegin{}\ModeP{},\ModeG{}\MLEnd{}

Параметр задает символ, используемый в файлах-образцах как
"знак внимания". 
По умолчанию "!".
        Cм. \ref{xc:template}.

\EquHead{BATEXT}{расширение имени пакетного файла перекомпиляции}
        \MLBegin{}\ModeM{},\ModeP{},{подрежим \em batch}\MLEnd{}

Задает расширение имени пакетного файла перекомпиляции.
По умолчанию {\bf .bat}.
        Cм. \ref{xc:modes:batch}.

\EquHead{BATNAME}{имя пакетного файла}
        \MLBegin{}\ModeM{},\ModeP{},{подрежим \em batch}\MLEnd{}

Задает имя пакетного файла. 

Если этот параметр не задан, будет использовано имя проекта.
        Cм. \ref{xc:modes:batch}.

\EquHead{BATWIDTH}{макс. длина строки в пакетном файле}
        \MLBegin{}\ModeM{},\ModeP{},{подрежим \em batch}\MLEnd{}

Задает максимальную длину строки в пакетном файле.
По умолчанию 128.
        Cм. \ref{xc:modes:batch}.

\EquHead{BSDEF}{расширение имени псевдо-определяющего модуля}
        \MLBegin{}\ModeB{}\MLEnd{}

Задает расширение имени псевдо-определяющего модуля, создаваемого
в режиме BROWSE. 
По умолчанию {\bf .odf}.
        Cм. \ref{xc:modes:browse}.

\EquHead{BSTYLE}{стиль просмотра}
        \MLBegin{}\ModeB{}\MLEnd{}

Задает {\em стиль} порождаемого псевдо-определяющего модуля.
Cм. \ref{o2:env:makedef}.

\ifgencode
\EquHead{CC}{совместимость с C компилятором}
        \MLBegin{}\ModeC{}\MLEnd{}

Задает режим совместимости с указанным C компилятором.
\iflinux
        В наст. версии может иметь только значение "GCC".
\else
        В наст. версии может иметь значения "WATCOM" и "MSVC".
\fi
%  , "BORLAND" and "SYMANTEC".
Если значение не определено, принимается значение
        \iflinux "GCC" \else "WATCOM" \fi.

Компилятор дает возможность использовать C библиотеки указанного 
C компилятора.

Подробнее см. главу \ref{ccomp}.

\fi

\ifgenc
\EquHead{CODE}{расширение для C кодофайла}
\else % gencode
\EquHead{CODE}{расширение для объектного файла}
\fi
        \MLBegin{}*\MLEnd{}

Задает расширение для порожденного кодофайла.
По умолчанию {\bf \Code}.

\ifgencode
\EquHead{CODENAME}{имя сегмента кода}
        \MLBegin{}\ModeC{}\MLEnd{} \header

Задает имя сегмента кода.
\fi

\ifcomment
\EquHead{COMPILE}{командная строка для C компилятора}
        \MLBegin{}\ModeC{}\MLEnd{}

Задает командную строку, которая будет исполнена после успешной
компиляции модуля. Как правило, используется для вызова C компилятора.
См. главу \ref{seamless}.
\fi

\EquHead{COMPILERHEAP}{предельное количество памяти для компилятора}
        \MLBegin{}*\MLEnd{}

Задает максимальный объем памяти (в байтах), который может
использовать компилятор. Для систем с виртуальной памятью
мы рекомендуем устанавливать значение, меньшее, чем объем 
физической памяти.

\EquHead{COMPILERTHRES}{порог сборки мусора для компилятора}
        \MLBegin{}*\MLEnd{}

Задает порог сборки мусора (в байтах) для компилятора.
Сборщик мусора автоматически вызывается, когда объем занятой памяти превысит
этот порог. Для систем с виртуальной памятью мы рекомендуем устанавливать
значение, меньшее, чем значение параметра
        \OERef{COMPILERHEAP}.

\ifgenc
\EquHead{COPYRIGHT}{декларация copyright}
        \MLBegin{}\ModeC{}\MLEnd{}

Строка -- значение этого параметра -- будет вставлена как комментарий
во все порожденные C кодофайлы и файлы-заголовки.
Cм. \ref{maptoc:layout}, \ref{maptoc:opt:rep}.
\fi

\ifgencode
\EquHead{CPU}{оптимизация для CPU}
        \MLBegin{}\ModeC{}\MLEnd{}

Задает, на каком представителе семейства x86 
программа должна исполняться оптимально.

Разрешенные значения: "386", "486", "PENTIUM", and "PENTIUMPRO".
Значение параметра должно быть "не меньше", чем значение параметра
\OERef{MINCPU}.

Может также принимать специальное значение "GENERIC", 
означающее, что оптимизатор не должен использовать преобразования кода,
которые могли бы 
{\it значительно} понизить эффективность на каком-либо CPU.
\fi

\ifgencode
\EquHead{DATANAME}{имя сегмента данных}
        \MLBegin{}\ModeC{}\MLEnd{} \header

Задает имя сегмента данных.
\fi

\EquHead{DECOR}{контроль сообщений компилятора}
        \MLBegin{}*\MLEnd{}

Параметр управляет сообщениями утилиты \XC{}. Его значение ---
строка, содержащая произвольную комбинацию букв
"h", "t", "c", "r", "p" (прописные буквы также разрешены).
Каждая буква включает вывод, соответственно:
        \begin{description}
        \item[h]
заголовка, содержащего имя и версию переднего и заднего 
концов компилятора
        \item[p]
сообщений о ходе работы
        \item[r]
сводки компилятора: кол-во ошибок, строк и~т.п.
        \item[t]
суммарной сводки по компиляции нескольких файлов
        \end{description}

Значение по умолчанию "hrt".

\EquHead{DEF}{расширение для \mt{} определяющего модуля}
        \MLBegin{}*\MLEnd{}

Задает расширение для имени \mt{} определяющего модуля.
По умолчанию {\bf .def}.

\EquHead{ENUMSIZE}{размер перечислимых типов по умолчанию}
        \MLBegin{}\ModeC{}\MLEnd \inline

Задает размер по умолчанию перечислимых типов (в байтах).
Может быть равно 1, 2 или 4. Если указанного размера не хватает,
то выбирается наименьший из подходящих размеров.

\ifgenc                                                  ]
\EquHead{ENV\_HOST}{рабочая платформа}
        \MLBegin{}*\MLEnd{}

Задает имя рабочей платформы. См. в
{\tt \cfg} список имеющихся платформ. 
См. также \ref{config:seamless}.

\fi

\ifgenc
\EquHead{ENV\_TARGET}{целевая платформа}
        \MLBegin{}*\MLEnd{}

Задает имя целевой платформы. Платформа определяет тип файловой системы, 
используемый C компилятор, его опции и~т.п. 
См. в {\tt \cfg} список имеющихся платформ. 
См. также \ref{config:seamless}.
\fi

\EquHead{ERRFMT}{формат сообщений об ошибках}
        \MLBegin{}*\MLEnd{}

Задает формат сообщений об ошибках.
Cм. подробнее в \ref{opt:errfmt}.

\EquHead{ERRLIM}{максимальное количество ошибок}
        \MLBegin{}*\MLEnd{}

Задает максимальное количество ошибок на одну единицу компиляции.
По умолчанию 16.

\EquHead{GCTHRESHOLD}{порог сборки мусора}
        \MLBegin{}\ModeC{},{\em только головные модули}\MLEnd{}

Задает порог для сборки мусора (в байтах). Сборщик мусора
автоматически вызывается, когда объем занятой памяти 
превосходит это значение.
Может принимать значения в диапазоне 0..\OERef{HEAPLIMIT}. 
Рекомендуется обязательно устанавливать этот параметр для систем 
с виртуальной памятью.

\ifgenc
\EquHead{GENIDLEN}{макс. длина идентификатора в C коде}
        \MLBegin{}\ModeC{}\MLEnd{}

Задает максимальную длину идентификатора в порожденном C коде.
По умолчанию 30.

{\bf Замечание:} должно быть не менее 6. Малые значения этого параметра
приводят к более компактному, но менее понятному тексту.
См. также \ref{maptoc:idents}.
\fi

\ifgenc
\EquHead{GENINDENT}{отступ}
        \MLBegin{}\ModeC{}\MLEnd{}

Задает величину отступов в порожденном тексте. По умолчанию 3.
\fi

\ifgenc
\EquHead{GENWIDTH}{длина строк в порожденном C тексте}
        \MLBegin{}\ModeC{}\MLEnd{}

Задает максимальную длину строк в порожденном C тексте (код или заголовок).
По умолчанию 30.
\fi

\ifgenc
\EquHead{HEADER}{расширение C файла-заголовка}
        \MLBegin{}*\MLEnd{}

Задает расширение имени ANSI C файла-заголовка, порожденного компилятором.
По умолчанию {\bf .h}.
\fi

\EquHead{HEAPLIMIT}{макс. объем памяти программы}
        \MLBegin{}\ModeC{},{\em только головные модули}\MLEnd{}

Задает максимальный объем памяти (в байтах), 
которую может затребовать программа.

Этот параметр следует задавать при компиляции головного модуля
программы. Советуем устанавливать его в проектном файле или в
файле конфигурации.

\ifcomment
\EquHead{IMPEXT}{}
        \MLBegin{}\ModeM{},\ModeP{},\ModeG{}\MLEnd{}

Задает расширение имени файла с псевдо-реализующим модулем.
По умолчанию {\bf .imp}.
        Cм. подробнее в \ref{xc:make}.
\fi

\EquHead{LINK}{командная строка для линкера}
        \MLBegin{}\ModeP{}\MLEnd{}

Задает командную строку, которая будет исполнена после успешной
компиляции проекта. Как правило, используется для вызова линкера 
или утилиты make.

        Cм. примеры в \ref{start:run}.

\EquHead{LOOKUP}{директива поиска}
        \MLBegin{}*\MLEnd{}

Задает директиву поиска файлов:
\begin{verbatim}
  -LOOKUP = pattern = directory {";" directory }
\end{verbatim}

Этот параметр можно использовать для переопределения
путей поиска, заданных в файле перенаправления.
Файл конфигурации или проектный файл могут содержать
несколько заданий этого параметра.
        См. также \ref{xc:cfg} и \ref{xc:project}.

\ifgenvax
\EquHead{MAINNAME}{имя главной процедуры}
        \MLBegin{}\ModeC{}\MLEnd{}
 
Этот параметр задает имя процедуры, соответствующей BEGIN-части главного 
модуля проекта, по умолчанию "\verb'X2C_BEGIN'".
\fi

\ifgencode
\EquHead{MINCPU}{наименьшее CPU для исполнения}
        \MLBegin{}\ModeC{}\MLEnd{}

Указывает представителя семейства процессоров
Intel x86, который, как минимум, требуется для исполнения программы.

Возможные значения: GENERIC", "386", "486", "PENTIUM" и
        "PENTIUMPRO". Для этого параметра "GENERIC" эквивалентно
        "386". Значение параметра \OERef{CPU} должно быть 
"не меньше" значения этого параметра.
\fi

\EquHead{MKFEXT}{расширение для make-файлов}
        \MLBegin{}\ModeG{}\MLEnd{}

Задает расширение для имени порожденного make-файла. По умолчанию
{\bf.mkf}. Cм. \ref{xc:modes:gen}.

\EquHead{MKFNAME}{имя make-файла}
        \MLBegin{}\ModeG{}\MLEnd{}

Задает имя make-файла.
Cм. \ref{xc:modes:gen}.

\EquHead{MOD}{расширение для \mt{} реализующего или головного модуля}
        \MLBegin{}*\MLEnd{}

Задает расширение для \mt{} реализующего или головного модуля.
По умолчанию {\bf .mod}.

\EquHead{OBERON}{расширение для \ot{} модуля}
        \MLBegin{}*\MLEnd{}

Задает расширение для \ot{} модуля.
По умолчанию {\bf .ob2}.

\EquHead{OBJEXT}{расширение для объектного файла}
        \MLBegin{}*\MLEnd{}

Задает расширение для объектного файла.
По умолчанию {\bf \dotObj}.


\EquHead{PRJ}{проектный файл}
        \MLBegin{}\ModeC{},\ModeM{},\ModeP{}\MLEnd{}

В режимах COMPILE и MAKE параметр задает, из какого проектного файла 
читать конфигурацию.
В режиме PROJECT, компилятор устанавливает этот параметр равным
имени проектного файла, указанному в командной строке.
        Cм. \ref{xc:modes:project}.

\EquHead{PRJEXT}{расширение для проектного файла}
        \MLBegin{}\ModeC{},\ModeM{},\ModeP{}\MLEnd{}

Задает расширение для имени проектного файла.
По умолчанию {\bf .prj}. Cм. \ref{xc:modes:project}.

\EquHead{PROJECT}{имя проекта}
        \MLBegin{}\ModeC{},\ModeM{},\ModeP{}\MLEnd{}

Значение этого параметра образуется из имени проектного файла
отбрасыванием пути и расширения. Так, если имя проектного файла
---  \verb'prj/Work.prj', то значением этого параметра будет
\verb'Work'. Этот параметр можно использовать в файле-шаблоне
для задания имени исполняемого файла и~т.п.

\EquHead{SETSIZE}{размер маленьких множеств по умолчанию}
        \MLBegin{}\ModeC{}\MLEnd \inline

Задает размер (в байтах) по умолчанию для типов маленьких 
(16 элементов или меньше) множеств. Возможные значения ---
1, 2 или 4. Если тип не помещается в этот размер, то будет выбран
наименьший подходящий размер.

\EquHead{STACKLIMIT}{макс. размер стека для программы}
        \MLBegin{}\ModeC{},{\em top-level module only}\MLEnd{}

Задает (в байтах) максимальный размер стека для программы.

Этот параметр следует задавать при компиляции головного модуля
программы. Советуем устанавливать его в проектном файле или в
файле конфигурации.

\ifgencode
        {\bf Замечание:} для некоторых линкеров этот размер должен быть 
установлен как параметр линкера.
\fi

\EquHead{SYM}{расширение для символьного файла}
        \MLBegin{}*\MLEnd{}

Задает расширение для имени символьного файла.
По умолчанию {\bf \Sym}. Cм. \ref{usage:genfiles}.

\EquHead{TABSTOP}{шаг табуляции}
        \MLBegin{}\ModeG{}\MLEnd{}

При чтении текстовых файлов \xds{} заменяет символы TAB 
на подходящее число пробелов для выравнивания текста (по умолчанию
{\bf TABSTOP} равно 8). Неправильное значение этого параметра
может привести к не туда попавшему комментарию в псевдо-определяющем 
модуле, к неверному указанию места ошибки и~т.п. Мы рекомендуем
устанавливать этот параметр равным значению, которое используется в 
Вашем текстовом редакторе.

\EquHead{TEMPLATE}{имя файла-шаблона (для make-файла)}
        \MLBegin{}\ModeG{}\MLEnd{}

Задает имя файла-шаблона. Cм. \ref{xc:template}.

\ifonline \else
\end{description}
\fi

\section{Спецификация формата сообщений об ошибках}\label{opt:errfmt}
\index{формат сообщений об ошибках}

Формат сообщений \xds{} об ошибках можно изменять, пользуясь параметром
\OERef{ERRFMT} Он имеет следующий синтаксис:
\begin{verbatim}
  { строка "," [ аргумент ] ";" }.
\end{verbatim}

В {\tt строке} можно использовать любые спецификации форматов
C процедуры "printf".

\begin{center}
\begin{tabular}{|l|c|l|}
\hline
\bf Аргумент &\bf  Тип &\bf Значение \\
\hline
line         & integer  &  позиция в исходном тексте   \\
column       & integer  &  позиция в исходном тексте   \\
file         & string   &  имя исходного файла         \\
module       & string   &  имя модуля                  \\
errmsg       & string   &  описание ошибки             \\
errno        & integer  &  код ошибки                  \\
language     & string   &  Oberon-2 или Modula-2       \\
mode         & string   &  ERROR или WARNING или FAULT \\
utility      & string   &  имя утилиты                 \\
\hline
\end{tabular}
\end{center}
Прописные и строчные буквы в именах аргументов не различаются.
По умолчанию, формат сообщения таков:
\begin{center}
\begin{tabular}{lcl}
\verb+"(%s",file;+       &---& имя файла                   \\
\verb+"%d",line;+        &---& номер строки                 \\
\verb+",%d",column;+     &---& место в строке               \\
\verb+") [%.1s] ",mode;+ &---& первая буква типа ошибки  \\
\verb+"%s\n",errmsg;+    &---& описание ошибки               \\
\end{tabular}
\end{center}
Так, если в модуле {\tt test.mod} в строке 5,
столбце 6, сообщено о предупреждении, то сообщение будет выглядеть так:
\begin{verbatim}
(test.mod 5,6) [W] переменная описана, но не использована
\end{verbatim}

\section{Системный модуль COMPILER}
\label{opt:COMPILER}
\lindex{COMPILER}
\index{системные модули!COMPILER}

Системный модуль {\tt COMPILER} предоставляет две процедуры,
которые позволяют использовать в Ваших
\mt{} или \ot{} программах значения опций и параметров, используемые во
время компиляции.
\begin{verbatim}
PROCEDURE OPTION(<constant string>): BOOLEAN;
PROCEDURE EQUATION(<constant string>): <constant string>;
\end{verbatim}

Обе процедуры вычисляются во время компиляции; их можно использовать 
в константных выражениях.

{\bf Замечание:} Модуль {\tt COMPILER} не является стандартным.

\Examples

\begin{verbatim}
Printf.printf("Эта программа оптимизирована для %s CPU\n",
              COMPILER.EQUATION("CPU"));

IF COMPILER.OPTION("__GEN_C__") THEN
  ...
END;
\end{verbatim}



