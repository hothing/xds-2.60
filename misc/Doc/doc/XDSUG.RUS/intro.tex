\chapter{\XDS}
\pagenumbering{arabic}

\section{Добро пожаловать в \XDS}

Система программирования ($\mbox{XDS}^{\mbox{\tiny TM}}$) фирмы XDS --- это
профессиональная система, предоставляемая для большинства популярных
платформ, от персональных компьютеров, основанных на процессорах
Intel x86 (системы Windows 95, Windows NT, OS/2, Linux)
до различных Unix-компьютеров (Sun, HP, DEC, MIPS) и т.д. 
\xds{} предоставляет единую среду программирования для всех этих
платформ, позволяя тем самым создавать действительно переносимые
программы.

Система содержит компиляторы с языков \mt{} и \ot{}. 
Эти языки заслуженно называют {\bf ``безопасными''} и {\bf ``модульными''}. 

Язык \mt{} был первым, реализовавшим понятия модульности,
скрытия информации %???
и раздельной компиляции.

\ot{} --- это объектно-ориентированный (ОО) язык программирования,
основанный на языке \mt. Введение в язык объектно-ориентированных средств
сильно облегчает разработку на нем расширяемых проектов. Вместе с тем
\ot{} очень прост, легок в изучении и использовании --- в отличие от 
других ОО языков (C++, Smalltalk).

\xds{} \mt{}-компилятор реализует стандарт ISO языка \mt{}.
ISO библиотека стандартных модулей доступна в обоих языках
(\mt{} и \ot{}).

В основе системы \xds{} лежит независимый от платформы "входной конец"
(front-end) для обоих языков, выполняющий синтаксический и семантический
разбор исходной программы. Компилятор строит внутреннее представление
единицы компиляции в памяти и производит платформо-независимые
анализ и оптимизацию. Затем он генерирует выходной код: либо 
исполняемый код для данной платформы, либо текст на языке ANSI C.
Генерация кода на ANSI C доступна на всех платформах; набор платформ,
для которых реализована генерация исполняемого кода, постоянно растет.

Переход на новый язык программирования обычно требует переписывания
или выбрасывания существующих библиотек, возможно --- плода многих лет
работы.
Система \xds{} позволяет программисту использовать  в одном проекте
модули и библиотеки, написанные на языках \mt{}, \ot{}, C и на 
языке ассемблера.

В \xds{} входят также стандартные библиотеки ISO и PIM, и интерфейс
к стандартным библиотекам ANSI C.

\ifgenc

 Компиляторы \xds{} генерируют оптимизированный код на языке ANSI C,
 который далее компилируется C-компилятором. Однако \xds{} можно
 использовать и как транслятор с языка \mt/\ot{} в C, поскольку 
 генерируемый текст легко читаем и понимаем. К тому же, комментарии
 из исходной программы могут быть оставлены в C-коде в их первоначальном
 контексте. 

\fi
\ifgencode
   Компиляторы \xds{} генерируют сильно оптимизированный 32-битовый код и 
   отладочную информацию в формате целевой платформы.
 \ifosii
   Программы на языках \mt{} и \ot{}
   имеют полный доступ к 32-битовому API системы OS/2
   (включая Presentation Manager).
 \fi
 \ifwinnt
   Программы на языках \mt{} и \ot{}
   имеют полный доступ к Win32 API.
 \fi

\ifcomment
 Оба компилятора \xds{} используют возможности динамической загрузки ---
 предварительное связывание не обязательно, т.к. все нужные модули
 связываются динамически в момент загрузки. Отдельно предоставляется
 линкер, для сборки завершенного проекта в один исполняемый файл.
\fi

\fi % \ifgencode

\section{Используемые соглашения}

\subsection{Описание языков}

Для формального описания языковых конструкций используется
расширенная форма Бэкуса-Наура (EBNF).

Такие описания набраны шрифтом {\tt Courier}.

\begin{verbatim}
Text= Text [{Text}] | Text.
\end{verbatim}

В EBNF, квадратные скобки \verb+[+ и \verb+]+ 
означают возможность, но не обязательность заключенного в них выражения,
фигурные скобки \verb+{+ и \verb+}+ означают его повторение (может быть,
0 раз), и вертикальная линия \verb+|+ разделяет возможные альтернативы.

Все нетерминальные символы начинаются с заглавной буквы (напр. Statement).
Терминальные символы либо начинаются со строчной буквы (напр. ident), 
либо написаны заглавными буквами (напр. BEGIN), 
либо заключены в кавычки (напр. ":=").

\subsection{Фрагменты программ}

Все фрагменты исходного кода, появляющиеся в тексте или служащие 
иллюстрациями, набраны шрифтом {\tt Courier}.

\begin{verbatim}
MODULE Example;

IMPORT InOut;

BEGIN
  InOut.WriteString("This is an example");
  InOut.WriteLn;
END Example.
\end{verbatim}

