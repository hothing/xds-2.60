% !!! XRef for operation modes

\chapter{Пользование пакетом \XDS}\label{usage}

\section{Вызов \XDS}

\xds{}  \mt{} и \ot{} компиляторы вместе с некоторыми дополнительными
операциями объединены в одну утилиту \xc. 
Будучи вызвана без параметров, она выдает краткую информацию о себе.

\xc{} вызывается командной строкой следующего вида:
\index{\XC{}(.exe)@{\XC{}(.exe)}}
\begin{flushleft} \tt
\xc{} \{ РЕЖИМ РАБОТЫ $|$ ОПЦИИ $|$ ИМЯ \}
\end{flushleft}
где {\tt ИМЯ} для различных режимов может означать
имя модуля, имя файла или имя проекта.
Полное описание режимов см. в \ref{xc:modes}.

ОПЦИИ --- это директива задания параметров компилятора 
(см. \ref{config:options}).
При вызове из командной строки все опции применяются ко всем операндам.
Для некоторых платформ может оказаться необходимым заключать некоторые 
опции в кавычки:
\begin{flushleft} \tt
    \xc{} hello.mod '-checkindex+'
\end{flushleft}
В главе \ref{options} перечислены все опции и параметры компилятора.

\subsection{Приоритет установки опций}\label{precedence}
\index{приоритет опции}

Утилита \xc{} получает свои опции последовательно из:
\begin{enumerate}
\item         файла конфигурации {\bf \cfg} (см. \ref{xc:cfg}),
\item         командной строки (см. \ref{xc:modes}),
\item         проектного файла (если есть) (см. \ref{xc:project}),
\item         прагм исходного текста (не все опции могут так устанавливаться)
              (см. \ref{m2:pragmas}).
\end{enumerate}

В любой момент работы действует последнее установленное значение
каждой опции. Так, если параметр
{\bf OBERON} был установлен равным {\bf.ob2}  
в файле конфигурации, но затем изменен на 
{\bf.o2} в командной строке, то
\xds{} будет воспринимать {\bf.o2} как расширение \ot{} файлов.

\section{Режимы работы \XDS{}}\label{xc:modes}
\label{xc:mode}\index{режим работы}

\xds{} имеет следующие режимы:
\begin{center}
\begin{tabular}{|l|l|}\hline
\bf Режим   & \bf Значение                                  \\ \hline
    COMPILE & скомпилировать все модули, указанные в вызове  \\ \hline
    PROJECT & собрать все проекты, указанные в вызове        \\ \hline
    MAKE    & проверить зависимости и перекомпилировать      \\ \hline
    GEN     & породить make-файлы для всех проектов          \\ \hline
    BROWSE  & Извлечь определяющие модули из символьных файлов\\ \hline
    HELP    & Выдать HELP и завершить работу                 \\ \hline
\end{tabular}
\end{center}

Режимы PROJECT и MAKE имеют два дополнительных подрежима:
BATCH и ALL. Еще два дополнительных подрежима --
OPTIONS и EQUATIONS используются для просмотра опций и параметров
компилятора и их значений.

Режим работы компилятора задается в командной строке знаком
'=', за которым следует имя режима. Имя режима можно сокращать, если 
это не приводит к неоднозначности. Прописные и строчные буквы не 
различаются. Так,
\begin{verbatim}
        =PROJECT   эквивалентно   =p
        =BROWSE    эквивалентно   =Bro
\end{verbatim}

Режимы работы и опции можно помещать в любом месте командной строки.
Так, следующие два вызова эквивалентны:
\begin{flushleft} \tt
    \xc{} =make hello.mod =all -checknil+ \\
    \xc{} -checknil+ =a =make hello.mod
\end{flushleft}

\subsection{Режим COMPILE}\label{xc:modes:compile}
\index{режим работы!COMPILE}

\begin{flushleft} \tt
    \xc{} [=compile] \{ ИМЯ ФАЙЛА $|$ ОПЦИИ \}
\end{flushleft}
COMPILE --- режим по умолчанию, поэтому его можно вызывать, просто
указав модули, которые надо скомпилировать.
Если \xc{} вызывается без указания режима, то подразумевается
режим COMPILE. Для определения того, какой компилятор вызывать,
\xc{} смотрит на расширения имен исходных файлов. По умолчанию, 
соответствие таково:
\begin{flushleft}
\begin{tabular}{lcl}
\bf        .mod  &-& Modula-2, реализующий модуль \\
\bf        .def  &-& Modula-2, определяющий модуль \\
\bf        .ob2  &-& Oberon-2  \\
\end{tabular}
\end{flushleft}

Например, вызов
\begin{flushleft} \tt
        \xc{} hello.mod
\end{flushleft}
запустит компилятор с Модулы-2, а
\begin{flushleft} \tt
        \xc{} hello.ob2
\end{flushleft}
запустит компилятор с Оберона-2.

Пользователь может изменить соответствие расширений типам файлов
(см. \ref{config:fileext}). Возможно также отменить использование 
расширений, напрямую указав в командной строке нужный компилятор
с помощью опций \OERef{M2} и \OERef{O2} (см. \ref{opt:bool}):
\begin{flushleft} \tt%
\xc{} hello.mod +o2  (* вызывает O2 компилятор *)  \\
\xc{} hello.ob2 +m2  (* вызывает M2 компилятор *)
\end{flushleft}

{\bf Замечание:} Всюду в дальнейшем слова "режим COMPILE"
относятся также к любому случаю, когда компилятор
{\it компилирует} исходный файл, независимо от того, 
в каком режиме он был вызван (COMPILE, MAKE, или PROJECT).
Таким образом, опции и параметры, влияющие на работу компилятора в 
режиме COMPILE, существенны и в режимах MAKE и PROJECT.

\subsection{режим MAKE}\label{xc:modes:make}
\index{режим работы!MAKE}

\begin{flushleft} \tt
    \xc{} =make [=batch] [=all] \{ FILENAME $|$ OPTION \}
\end{flushleft}
В режиме MAKE компилятор определяет зависимости между модулями
(используя при этом их секции импорта), 
и затем перекомпилирует те модули, которые нужно.
Начиная с файлов, указанных в командной строке, компилятор ищет
Оберон-модуль или определяющий и реализующий модули для каждого
модуля. Затем он делает то же для каждого импортируемого в них
модуля и~т.д. до тех пор, пока не найдены все необходимые модули.
Обратите внимание, что ищутся только исходные тексты. Если исходный
текст не найден, то импортируемые модули не будут добавлены в список
модулей для перекомпиляции. Более подробное описание см. в главе
\ref{xc:make}.

После того как все модули собраны, действия \xds{} 
зависят от подрежима работы.
Если задан подрежим BATCH, то  \xds{} создает командный файл с вызовами
всех необходимых компиляций, не выполняя их
(см. \ref{xc:modes:batch}).

Если задан подрежим ALL, то перекомпилируются все собранные файлы,
иначе \xds{} вычисляет для них условия перекомпиляции и компилирует
только те файлы, которые необходимо.
Алгоритм {\em "умной" перекомпиляции} описан в \ref{xc:smart}.

Обычно достаточно указать в командной строке только программный модуль
в Модуле-2, или модуль верхнего уровня в Обероне-2.
Если параметр \OERef{LINK} задан в файле конфигурации или в 
командной строке \xc{}, то в случае успешной перекомпиляции после нее
автоматически будет вызван линкер. Это позволяет собирать простые 
программы, не создавая для них проектных файлов.

\subsection{Режим PROJECT}\label{xc:modes:project}
\index{режим работы!PROJECT}

\begin{flushleft} \tt
  \xc{} =project [=batch] [=all] \{ PROJECTFILE $|$ OPTION \}
\end{flushleft}
Режим PROJECT --- в сущности, то же самое, что и режим
MAKE, за исключением того, что модули для сборки указываются в
проектном файле. Проектный файл содержит набор опций и список 
модулей. Более подробно см. в \ref{xc:project}
Как и в режиме MAKE, можно использовать подрежимы ALL и BATCH.

Если в имени проектного файла опущено расширение, то \xds{}
использует расширение, заданное в параметре \OERef{PRJEXT}
(по умолчанию {\bf .prj}). \index{.prj (см. PRJEXT)}

Если необходимо скомпилировать отдельный файл в окружении,
заданном в проектном файле, то это можно сделать с помощью
параметра \OERef{PRJ} в режиме COMPILE. Пример:
\begin{flushleft} \tt
    \xc{} -prj=myproject MyModule.mod
\end{flushleft}

\Seealso
\begin{itemize}
\item   режим MAKE : \ref{xc:modes:make}
\item   стратегия сборки: \ref{xc:make}
\item   умная перекомпиляция: \ref{xc:smart}
\end{itemize}

\subsection{режим GEN}\label{xc:modes:gen}
\index{режим работы!GEN}
\index{файлы-шаблоны}

\begin{flushleft} \tt
    \xc{} =gen \{ PROJECTFILE $|$ OPTION \}
\end{flushleft}
Режим GEN позволяет генерировать файл, содержащий информацию
о проекте. Наиболее важное его использование --- генерация
\ifgenc
  make-файла, который затем может быть передан сборщику
  резидентного C-компилятора с тем, чтобы были скомпилированы все 
  необходимые порожденные C-программы, и создана исполняемая программа.
\else
  файла заданий линкеру (см. \ref{start:run}). %??? LINKER RESPONSE FILE
\fi

Этот режим можно использовать также для сбора дополнительной информации
о проекте, напр. список всех модулей, списки импорта, и~т.д.

В этом режиме используется так называемый файл-шаблон.
Имя файла-шаблона задается параметром \OERef{TEMPLATE}.
Файл-шаблон --- это текстовый файл, некоторые строки которого
отмечены символом присваивания. Все неотмеченные строки просто
копируются в выходной файл, а отмеченные специальным образом преобразуются
(см. \ref{xc:template}).

В результате \xds{} создаст файл с именем, заданным либо
параметром компилятора \OERef{MKFNAME}, либо именем проектного
файла. К имени файла присоединяется расширение, заданное параметром
\OERef{MKFEXT}.
\index{.mkf (См. MKFEXT)}

\subsection{Режим BROWSE}\label{xc:modes:browse}
\index{режим работы!BROWSE}

\begin{flushleft} \tt
  \xc{} =browse \{ MODULENAME $|$ OPTION \}
\end{flushleft}
Режим работы BROWSE позволяет генерировать псевдо-определяющие
модули для модулей на Обероне-2.
В этом режиме \xds{} просматривает символьный файл и генерирует
Модула-2-образный определяющий модуль, который содержит описания всех
экспортируемых объектов.

Параметр конфигурации \OERef{BSDEF} (см. \ref{opt:equ})
задает расширение для порожденного файла. Если он не установлен,
то по умолчанию используется расширение
({\bf .odf}). \index{.odf (См. BSDEF)}

Форма порождаемого файла контролируется
опциями \OERef{BSCLOSURE} и \OERef{BSREDEFINE} 

{\bf Замечание:} Параметр \OERef{BSTYLE}
(описан в \ref{o2:env:makedef}) в этом режиме игнорируется,
а стиль просмотра установлен равным DEF.

Опция \OERef{MAKEDEF} (см. \ref{o2:env:makedef}) предоставляет
еще один способ порождения псевдо-определяющих модулей, при котором
можно сохранить так называемые {\em экспортируемые комментарии}.

\subsection{Подрежим ALL}
\index{режим работы!ALL}

В режимах PROJECT и MAKE  \xds{} проверяет время последней модификации
всех встреченных файлов, и перекомпилирует лишь те из них, которые
необходимо. В подрежиме ALL эти времена игнорируются, 
и компилируются все файлы.

\subsection{Подрежим BATCH}\label{xc:modes:batch}
\index{режим работы!BATCH}

В режиме BATCH \xds{} создает командный файл для вызова
всех необходимых компиляций, не исполняя их.

Такой командный файл есть последовательность строк, каждая из
которых начинается с имени компилятора, за которым следуют 
имена файлов для перекомпиляции.

Имя создаваемого командного файла \xds{} определит по одному из следующих:
\begin{enumerate}
\item параметр компилятора \OERef{BATNAME} (см. \ref{opt:equ})
\item имя проектного файла (если задано)
\item {\bf out} (если имя не определено выше указанными способами)
\end{enumerate}

Затем к имени командного файла добавляется расширение, заданное
параметром \OERef{BATEXT}
({\bf .bat} по умолчанию).
\index{.bat (См. BATEXT)}

\Seealso
\begin{itemize}
\item опция \OERef{LONGNAME}   (\ref{opt:bool})
\item параметр \OERef{BATWIDTH} (\ref{opt:equ})
\end{itemize}

\subsection{Подрежим OPTIONS}\label{xc:modes:options}
\index{режим работы!OPTIONS}

Подрежим OPTIONS позволяет увидеть значения опций, заданных
в файле конфигурации, в проектном файле и в командной строке.
Его можно использовать вместе с режимами COMPILE, MAKE и PROJECT.

Например, следующий вызов выведет (на стандартный вывод) список
всех определенных опций, включая все предекларированные опции,
все опции, декларированные в файле конфигурации, в проектном файле
{\tt my.prj} и в командной строке (опция {\tt xyz}):
\begin{flushleft} \tt
\xc{} =options -prj=my.prj -xyz:+
\end{flushleft}

В режиме PROJECT опции будут перечислены для каждого проектного файла,
указанного в командной строке.

\subsection{Подрежим EQUATIONS}\label{xc:modes:equations}
\index{режим работы!EQUATIONS}

Подрежим EQUATIONS позволяет увидеть значения параметров, заданных
в файле конфигурации, в проектном файле и в командной строке.
Его можно использовать вместе с режимами COMPILE, MAKE и PROJECT.

\section{Файлы, порождаемые при компиляции}\label{usage:genfiles}

В результате компиляции модуля {\bf name},
компиляторы порождают следующие файлы.

\subsection{Modula-2 компилятор}

При компиляции определяющего модуля компилятор с Модулы-2 порождает
{\em символьный файл}\index{символьные файлы}\index{\Sym (См. SYM)}
\ifgenc
  ({\bf name\Sym}) и файл-заголовок на C ({\bf name\Header}).
  \index{\Header (см. HEADER)}
  Порождение файла-заголовка можно выключить, выставив
  опцию {\bf NOHEADER} (см. \ref{opt:bool}).
\else
  ({\bf name\Sym}).
\fi
Символьный файл содержит всю информацию, необходимую для компиляции
модулей, импортирующих модуль {\bf name}.

При компиляции реализующего модуля или головного модуля компилятор с
Модулы-2 порождает
\ifgencode объектный файл \fi
\ifgenc C-файл \fi
file ({\bf name\Code}).\index{\Code (см. CODE)}

\subsection{Oberon-2 компилятор}

В результате компиляции любого модуля компилятор с Оберона-2 порождает
\ifgencode
   {\em символьный файл} ({\bf name\Sym})
  и объектный файл ({\bf name\Code}).
\fi
\ifgenc
   {\em символьный файл} ({\bf name\Sym}),
   файл-заголовок на C ({\bf name\Header}) и файл кода на C ({\bf name\Code}).
\fi
\index{символьные файлы}
Символьный файл ({\bf name\Sym}) содержит всю информацию, 
необходимую для компиляции модулей, импортирующих модуль {\bf name}.
Если при компиляции компилятору нужно записать в уже существующий
символьный файл, он сделает это лишь в том случае, если выставлена 
опция \OERef{CHANGESYM}.

\Examples
\begin{flushleft}
\begin{tabular}{l|l}
\bf Командная строка             & \bf Порожденные файлы \\
\hline
\tt \xc{} Example.def            & \tt Example\Sym  \\
\ifgenc
                                 & \tt Example\Header \\
\fi
\hline
\tt \xc{} Example.mod            & \tt Example\Code  \\
\hline
\tt \xc{} Windows.ob2 +CHANGESYM & \tt Windows\Sym  \\
\ifgenc
                                 & \tt Windows\Header  \\
\fi
                                 & \tt Windows\Code \\
\end{tabular}
\end{flushleft}

\section{Проектные файлы}\label{xc:project}
\index{проектные файлы}

Проектный файл имеет следующую структуру:
\begin{verbatim}
    {ДирективаУстановки}
    {!модуль {ИмяФайла}}
\end{verbatim}
Директивы установки задают значения опций и параметров,
с которыми будут компилироваться все модули.
Синтаксис для {\tt ДирективаУстановки} см. в \ref{config:options}.

В каждой строке проектного файла может быть не более одной директивы
установки.
Символ "\verb|%|" --- это знак комментария; остаток строки после
него пропускается.

{\bf Замечание:} В строке, задающей значение параметра, знак комментария
использовать нельзя.

\xds{} дает полную свободу, в каких файлах помещать установку
опций, параметров и перенаправлений поиска. Однако мы советуем
помещать в файлы конфигурации и перенаправления лишь те установки,
которые относятся ко всем проектам, а в проектном файле задавать
опции и перенаправления, специфические для данного проекта.

Проектный файл может содержать несколько 
заданий параметра \OERef{LOOKUP}, что позволяет переопределять
пути поиска.

Каждое {\tt ИмяФайла} --- это имя файла, который должен быть
скомпилирован, слинкован, или обработан как-либо еще, при сборке
проекта, например: исходный текст, дополнительная библиотека,
ресурсный файл (для платформ OS/2 и Win32) и~т.д.
Компилятор работает только с исходными текстами на Модуле-2 или
Обероне-2. Тип файла определяется по его расширению.
%??? (Modula-2/Oberon-2 source files extension is assumed by default).
Файлы остальных типов могут использоваться в make-файле, если
применяемый им файл-шаблон это позволяет (см. \ref{xc:template}).


В проектном файле обязательно должен быть указан хотя бы один модуль.

Следует отметить, что \xds{} рекурсивно просматривает списки импорта
всех указанных программных модулей и составляет полный список всех модулей,
используемых в проекте. Поэтому обычно проектный файл для создания
исполняемой программы содержит ровно одну директиву {\tt !module},
задающую файл, содержащий головной модуль программы и, возможно,
другие директивы {\tt !module}, задающие непрограммные файлы.

\paragraph{Пример проектного файла:}
\begin{verbatim}
-mod = mod
-checkindex+
-lookup = *.mod = mod
-lookup = *.sym = sym; /xds/sym
!module hello
!module hello.res
\end{verbatim}

В этом примере, \xds{} будет искать файлы с расширениями
{\bf.mod} и {\bf.sym}, используя пути поиска, заданные в проектном файле.
Файлы со всеми остальными расширениями будут искаться по путям,
заданным в файле  перенаправления, в файле конфигурации или в 
командной строке.

В режимах работы PROJECT и GEN проектный файл задается в явном виде.
В этих режимах сначала устанавливаются все опции, а затем
\xds{} обрабатывает список модулей, чтобы найти все модули, образующие
данный проект (см. \ref{xc:make}).

В режимах MAKE и COMPILE проектный файл может быть задан с помощью
параметра \OERef{PRJ}. В этом случае список модулей из проектного файла
игнорируется, лишь выставляются все заданные там опции и параметры.

Так, следующая командная строка заставит \xds{}
скомпилировать модуль {\tt hello.mod} в режиме COMPILE,
используя опции и параметры, заданные в проектном файле
{\tt hello.prj}:
\begin{flushleft} \tt
  \xc{} hello.mod -prj=hello.prj
\end{flushleft}

\section{Стратегия сборки}\label{xc:make}

Вся информация в этой главе относится к режимам
MAKE, PROJECT и GEN. 
В этих режимах \xds{} собирает список всех модулей, входящих в проект,
начиная с модулей, указанных в проектном файле
(режимы PROJECT и GEN) или в командной строке (режим MAKE).

В большинстве примеров мы будем использовать режим MAKE,
но комментарии к ним относятся также и к режимам PROJECT и GEN.

Сначала \xds{} ищет все заданные модули, используя такую стратегию:
\begin{itemize}
\item
        Если явно заданы имя файла, расширение и путь к файлу,
        то \xds{} проверяет, существует ли этот файл.

        \begin{flushleft} \tt
        \xc{} =make mod\DirSep{}hello.mod
        \end{flushleft}

\item
        Если заданы имя и расширение, то файл ищется по путям поиска.

        \begin{flushleft} \tt
        \xc{} =make hello.mod
        \end{flushleft}

\item
        
        Если расширение не задано, то \xds{} попытается найти
        файлы с \ot{}-расширением и с \mt{}-расширениями для
        определяющих и реализующих модулей.

        \begin{flushleft} \tt
        \xc{} =make hello
        \end{flushleft}

        Если в результате будет найдено более одного программного файла,
        напр. если доступны файлы {\tt hello.ob2} и {\tt hello.mod}, то
        выдается сообщение об ошибке.
\end{itemize}

Начиная с заданных файлов, \xds{} пытается найти Оберон-модуль или пару
--- определяющий и реализующий модули --- для каждого импортируемого модуля.
Затем это проделывается для каждого найденного модуля и~т.д., пока все
модули не будут найдены. Для всех найденных модулей 
\xds{} проверяет соответствие расширений имен файлов типам модулей.

% impext.tex !!!

\section{Умная перекомпиляция}\label{xc:smart}

В режимах PROJECT и MAKE, \xds{} использует алгоритм {\em умной
перекомпиляции} всех модулей проекта, которые не согласуются с 
имеющимися их текстами.
Для этого \xds{} использует время последней модификации файла,
чтобы определить, тексты каких модулей изменились.
О каждом модуле решение (перекомпилировать или нет) принимается
после того, как это решение принято относительно всех модулей,
от которых он зависит. Файл перекомпилируется, если выполнено хотя бы 
одно из следующих условий:
\begin{description}
\item[определяющий модуль] \mbox{}

        \begin{itemize}
        \item отсутствует символьный файл
        \item символьный файл есть, но его время последней
          модификации раньше, чем у его текста или у какого-либо
          из импортируемых им символьных файлов
        \ifgenc
          \item файл-заголовок отсутствует (и опция \OERef{NOHEADER} 
          не выставлена), либо его время модификации раньше, чем у текста
        \fi
\ifcomment
        \item время модификации у текста раньше, чем у проектного файла
         (только в режиме PROJECT)
\fi
        \end{itemize}

\item[реализующий модуль] \mbox{}

        \begin{itemize}
        \item кодофайла нет
        \item кодофайл есть, но его время модификации раньше, чем у
          исходного текста или какого-либо из импортируемых
          символьных файлов (включая его собственный символьный файл)
\ifcomment
        \item время модификации у текста раньше, чем у проектного файла
         (только в режиме PROJECT)
\fi
        \end{itemize}

\item[головной модуль] \mbox{}

        \begin{itemize}
        \item кодофайла нет
        \item кодофайл есть, но его время модификации раньше, чем у
          исходного текста или какого-либо из импортируемых
          символьных файлов
\ifcomment
        \item время модификации у текста раньше, чем у проектного файла
         (только в режиме PROJECT)
\fi
        \end{itemize}

\item[Oberon-2 модуль] \mbox{}

        \begin{itemize}
        \item отсутствует символьный файл
        \item символьный файл есть, но его время последней
          модификации раньше, чем у какого-либо
          из импортируемых им символьных файлов
        \ifgenc
          \item файл-заголовок отсутствует (и опция \OERef{NOHEADER} 
          не выставлена)
        \fi
        \item кодофайла нет
        \item кодофайл есть, но его время модификации раньше, чем у
          исходного текста или какого-либо из импортируемых
          символьных файлов
\ifcomment
        \item время модификации у текста раньше, чем у проектного файла
         (только в режиме PROJECT)
\fi
        \end{itemize}
\end{description}

Если выставлена опция \OERef{VERBOSE}, то \xds{} сообщает о каждом
модуле, почему он перекомпилируется.

{\bf  Замечание:} Если при компиляции определяющего или  
\ot{}-модуля произошла ошибка, то зависимые от него модули 
не будут перекомпилироваться.

\section{Файлы-шаблоны}\label{xc:template}
\index{файлы-шаблоны}

Файл-шаблон используется для задания структуры файла, порождаемого
в режиме GEN, или в режиме PROJECT, если выставлена опция
\OERef{MAKEFILE}. \XDS{} копирует строки шаблона в выходной файл,
за исключением строк, требующих дополнительной обработки. Это строки,
начинающиеся с символа "знак внимания", заданного параметром
\OERef{ATTENTION} (по умолчанию '!')\index{! (см. ATTENTION)}.

Отмеченная строка (или заготовка) имеет следующий формат
\footnote{Такой же синтаксис используется при задании параметра
\OERef{LINK}.}(см. также \ref{xc:env}):
\begin{verbatim}
Заготовка   = { Предложение }.
Предложение = Слово { "," Слово } ";" | Итератор.
Слово       = Атом | [ Атом | "^" ] "#" [ Расширение ].
Атом        = Строка | имя.
Строка      = '"' { символ } '"'
            | "'" { символ } "'".
Расширение  = [ ">" ] Атом.
Итератор    = "{" Множество ":" { Предложение } "}".
Множество   = { КлСлово | Строка }
КлСлово     = DEF | IMP | OBERON | MAIN
            | C | HEADER | ASM | OBJ.
\end{verbatim}
{\tt имя} должно быть именем параметра. В предложении 
может быть не более трех слов. Первое слово в предложении --- это
строка, задающая формат, остальные --- аргументы.

Дистрибутив \XDS{} включает файл-шаблон {\tt \xc{}.tem}, который
можно использовать для получения
\ifgencode
файла заданий линкеру (см. \ref{start:run}).
\fi
\ifgenc
make-файла. Этот образец может быть настроен для любой имеющейся 
платформы.
\fi

\subsection{Использование значений параметров}

В простейшей форме, заготовка может быть использована для вывода
значения какого-либо параметра. Например, если файл-шаблон содержит
строку

\verb'! "Текущий проект: %s.\n",prj;'

и обрабатывается проект {\tt prj\DirSep{}test.prj},
то на выходе появится строка

\verb'Текущий проект: prj/test.prj.'

{\bf Замечание:} Заготовка

\verb'! prj;'

разрешена, но может привести к неожиданному результату в системах,
в которых символ \verb'"\"' используется как разделитель в именах
директорий (напр. OS/2 или Windows):

\verb'prj     est.prj'

поскольку \verb'"\t"' в форматной строке будет проинтерпретировано как
знак табуляции. Лучше использовать следующую форму:

\verb'! "%s",prj;'

\subsection{Построение имен файлов}

Оператор \verb|"#"| формирует имя файла из имени
\footnote{заданного либо явно, либо как значение параметра} и расширения.
Затем этот файл ищется по путям поиска
\XDS{}, и подставляется найденное полное имя (включая путь).
Например, если файл \verb'useful.lib' 
находится в директории '../mylibs'
и файл перенаправления содержит строку

\verb'*.lib = /xds/lib;../mylibs'

то строка

\verb'! "useful"#"lib"'

произведет на выходе

\verb'../mylibs/useful.lib'

Если добавлен модификатор \verb|">"|, то \XDS{} предполагает, что
нужный файл --- это файл вывода, и строит его имя в соответствии со
стратегией поиска файлов вывода
(см. \ref{xc:red} и опцию \OERef{OVERWRITE}).

Оператор \verb'"#"' используется также для определения текущего
значения итератора. В итераторах (и только в них) можно использовать
его, опуская либо имя файла, либо расширение.

В форме \verb'"^#"' этот оператор можно использовать в итераторах
второго уровня для обозначения текущего значения итератора первого
уровня.

\subsection{Итераторы}

{\em Итераторы} предназначены для того, чтобы сгенерировать
какой-либо текст для всех модулей из некоторой группы. 
Предложения, заданные внутри скобок первого уровня, повторяются
для всех модулей из заданного множества, а предложения второго 
уровня --- для всех импортируемых ими модулей. Множество модулей
задается последовательностью строк и ключевых слов. Каждая строка
задает некоторый модуль, каждое ключевое слово --- все модули
некоторого типа.

Ключевые слова имеют следующие значения:
\begin{center}
\begin{tabular}{|l|p{8cm}|} \hline
\bf КлСлово &\bf Значение  \\ \hline
\tt DEF     & определяющий модуль в Модуле-2                    \\
\tt IMP     & реализующий модуль в Модуле-2                     \\
\tt MAIN    & программный модуль в Модуле-2, или Оберон-2-модуль, помеченный \OERef{MAIN} \\
\tt OBERON  & Оберон-модуль                                    \\
\ifgenc
\tt C       & исходный текст на C                              \\
\tt HEADER  & C файл-заголовок                                 \\
\fi
\tt ASM     & исходный текст на ассемблере                     \\
\tt OBJ     & объектный файл                                   \\
%\tt PROJECT & имя проекта                                     \\
\hline
\end{tabular}
\end{center}

Любое ключевое слово, не перечисленное здесь, считается просто расширением
имен файлов. Предложение из итератора применяется ко всем файлам с этим
расширением, заданным в проектном файле. Это позволяет, например,
указывать в проектном файле дополнительные библиотеки:

\verb'! "%s","libxds"#"lib"'\\
\verb'! { lib: "+%s",#; }'

\subsection{Примеры}

Рассмотрим пример проекта, состоящего из программного модуля
{\tt A}, импортирующего модули {\tt B} и {\tt C}, и пусть
{\tt B} к тому же импортирует {\tt D} (все модули --- на языке \mt{}):
\begin{verbatim}
            A
          /   \
         B     C
         |
         D
\end{verbatim}
Вот несколько примеров работы шаблонов в этой ситуации.

Следующая заготовка перечислит те модули проекта, исходные файлы 
которых доступны:

\verb'! { imp oberon main: "%s ",#; }'

Для нашего проекта эта строка-заготовка сгенерирует строку

\verb'A.mod B.mod C.mod D.mod'

Чтобы вывести и определяющие, и реализующие модули, можно
написать так:

\verb'! { def : "%s ",#; }'\\
\verb'! { imp oberon main: "%s ",#; }'

В результате получим:

\verb'B.def C.def D.def A.mod B.mod C.mod D.mod'

Наконец, следующая заготовка перечисляет все модули проекта 
и их списки импорта:

\verb'! { imp main: "%s\n",#; { def: "  %s\n",#; } }'

Результат:

\begin{verbatim}
A.mod
  B.def
  C.def
B.mod
  D.def
C.mod
D.mod
\end{verbatim}
