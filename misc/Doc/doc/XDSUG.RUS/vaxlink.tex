\chapter{Соглашение о связях в Native XDS-VAX}
\label{vaxlink}

Эта глава описывает соглашения, которых следует придерживаться
разработчикам ПО для успешного совместного использования модулей на языке 
ассемблера для VAX и модулей, скомпилированных посредством XDS Модула-2.

\section{Введение}
\label{vaxlink:intro}

Все скомпилированные XDS M2 функции и объекты получают в процессе компиляции
имена, соответствующие описанным в разделе \ref{vaxlink:names} соглашениям 
о именовании объектов и функций. Функции, кроме того, получают соответствующие 
описанным в разделе \ref{vaxlink:calls} соглашения о вызовах.

В свою очередь, когда XDS генерирует в теле какой-либо функции вызов другой
внешней функции или ссылку на внешний объект, ожидается что эти объекты и
функции также имеют сформированные согласно правилам раздела \ref{vaxlink:names}
имена, а функции, кроме того, соответствуют соглашениям о вызовах, описанным в 
разделе \ref{vaxlink:calls}.

\section{Соглашение об именовании процедур и объектов данных}
\label{vaxlink:names}


Имена всех экспортируемых из (и импортируемых в) XDS объектов формируются
следующим образом:

\begin{enumerate}
\item Декларируемое имя объекта это то имя, которым объект именуется на 
     уровне исходного текста программы. Для BEGIN-части неглавного в 
     проекте модуля декларируемым именем считается "BEGIN".

\item Если объект был объявлен как [2] или ["C"] то берется его 
     декларируемое имя без всяких изменений.

       Если нет, то берется имя модуля в котором описан объект, в конец
     имени добавляется один символ подчерка '\_' после чего добавляется
     декларируемое имя объекта.
       
\item Если в XDS выставлена опция \OERef{GENCPREF} то перед именем, полученным
     на предыдущей стадии, добавляется один символ подчерка '\_'.

\item Если длина имени, полученного на предыдущих стадиях, превышает
     значение константы \verb'OBJ_S_C_SYMSIZ' (на сегодняшний день эта 
     константа прошита в компилятор XDS и ее значение равно 31)
     то происходит следующее:
     \begin{itemize}
     \item считается сумма номеров всех символов этого имени
     \item имя обрезается начиная с символа под номером \verb'OBJ_S_C_SYMSIZ-4'
     \item посчитанная сумма по модулю 65536 дописывается в конец
         полученного имени в виде шестнадцатеричного числа
     \end{itemize}
     Таким образом достигается соответствие длин имен объектов стандарту
     на длины имен объектов в OS VMS и в тоже время некоторая условная
     уникальность длинных имен.

\item Если в XDS не выставлена опция \OERef{PRESERVENAMECASE} то имя объекта,
     полученное на предыдущих стадиях, приводится в верхний регистр.

\item BEGIN-часть главного модуля проекта(точка входа проекта) получает
     имя, равное значению параметра(equation) \OERef{MAINNAME}. Если значение
     этого параметра неопределено или пусто, то в качестве такого имени
     используется "\verb'X2C_MAIN'". Компилятор XDS никогда не добавляет
     символ подчерка '\_', никогда не меняет регистр, и вообще никаким
     образом не изменяет этого имени.
\end{enumerate}

    Таким образом, если в ассемблерном модуле описана функция \verb'MY_FUNCTION'
  то для того чтоб ее вызвать из XDS необходимо декларировать внешнюю
  функцию \verb'PROCEDURE [2] MY_FUNCTION(...;', выключить опцию 
  \OERef{GENCPREF} и вызывать эту функцию по имени \verb'MY_FUNCTION(...);'.

  {\bf Примечание:} если в ассемблерном модуле функция описана как имеющая две
  метки \verb'MY_FUNCTION' и \verb'_MY_FUNCTION', то очевидно,что от состояния
  опции \OERef{GENCPREF} вызов этой функции зависеть не будет.

  Аналогично, если в XDS Modula-2 модуле \verb'SOMEMODULE' описаны процедуры
  \verb'PROCEDURE [2] PPP1;' и \verb'PROCEDURE PPP2;' то из ассемблерного модуля они будут 
  доступны как \verb'PPP1' и \verb'SOMEMODULE_PPP2' соответственно (следует 
  также учесть опции \OERef{GENCPREF} и \OERef{PRESERVENAMECASE}).
    
\section{Соглашения о вызовах}
\label{vaxlink:calls}

\subsection{Инструкции вызова и возврата}

    Для вызова процедур (функций) компилятором используется инструкция вызова
  процедуры стекового вида CALLS, для возврата - инструкция возврата из
  процедуры RET. Сохранение регистров на стек перед вызовом процедуры,
  восстановление их со стека после возврата из нее, а также удаление со
  стека параметров происходит в полном соответствии с описанием этих
  инструкций. (подробнее об этих инструкциях см. Приложение 1).

\subsection{Передача параметров}

    Параметры располагаются на стеке в порядке, обратном порядку их 
  перечисления в заголовке процедуры, а именно: первым на стек кладется
  последний параметр, вторым - предпоследний и т.д. Размер памяти, занимаемой
  каждым параметром на стеке, выравнивается на размер длинного слова (4 байта). 
  Для нескалярных параметров передается адрес. В случае передачи нескалярного
  параметра по значению вызываемая процедура создает на стеке его копию. В
  противном случае (при передаче нескалярного параметра по ссылке или только
  для чтения) этого не происходит.
    
\subsection{Передача возвращаемого значения}

    Возвращаемое значение скалярного типа размером до длинного слова (4 байта)
  передается через регистр R0, от длинного слова до двойного длинного слова
  (8 байт) - через пару регистров R0, R1. Для передачи возвращаемого значения
  нескалярного типа вызывающая процедура перед помещением параметров на стек
  резервирует на стеке место с размером равным размеру возвращаемого параметра,
  выравненного на размер длинного слова (4 байта), а после помещения параметров
  кладет на стек адрес зарезервированного места (длинное слово).

\section{Справка по системе команд}

\begin{verbatim}
                          Выдержка  из:


                  Операционная система МОС ВП

           АРХИТЕКТУРА И СИСТЕМА МАШИННЫХ ИНСТРУКЦИЙ

                      Справочный материал

                       00152-01 97 01-1

                          Листов 289





       4.6. Инструкции вызова процедуры

       Инструкции вызова процедуры используются для  реализа-
  ции стандартного алгоритма вызова процедуры. Две инструкции
  используются для вызова, а третья - для возврата из  проце-
  дуры. Инструкция CALLG производит вызов процедуры со  спис-
  ком аргументов, расположенных в произвольном  месте.  Инст-
  рукция CALLS производит вызов процедуры со списком аргумен-
  тов, расположенным в стеке.  После  возврата  из  процедуры
  этот список автоматически удаляется из стека. В обеих инст-
  рукциях определяется начальный адрес вызываемой  процедуры.
  Подразумевается, что по начальному адресу расположено  сло-
  во, называемое входной маской, за которым следует собствен-
  но процедура. Процедура заканчивается выполнением  инструк-
  ции RET.
       Входная  маска  определяет  используемые  в  процедуре
  регистры и разрешения внутренних прерываний  по  переполне-
  нию.
       Формат маски
        15 14 13 12 11                   0
       +--+--+-----+----------------------+
       !D !I ! MBZ !    Регистры          !
       !V !V !     !                      !
       +--+--+-----+----------------------+

       При выполнении инструкции CALL стек выравнивается  по
  границе длинного слова,  а  разряды  разрешения  внутренних
  прерываний в слове состояния процессора  устанавливаются  в
  состояние, которое обеспечивает требуемую реакцию вызванной
  процедуры при возникновении внутренних прерываний. Разреше-
  ния внутренних прерываний по целочисленному  и  десятичному
  переполнению устанавливаются в соответствии с разрядами  14
  и 15 входной маски. Разряд разрешения внутреннего  прерыва-
  ния по переполнению снизу (потеря значимости) при операциях
  с числами с плавающей запятой сбрасывается в нуль. Содержи-
  мое регистров с R0 по R11, помеченных единицами во  входной
  маске в разрядах с 0 по 11, запоминается в стеке и  восста-
  навливается при выполнении инструкций возврата  RET.  Кроме
  того, содержимое регистров PC, SP, FP и  AP  всегда  сохра-
  няется при выполнении инструкций вызова и восстанавливается
  при выполнении инструкций возврата.
       Все внешние  процедуры  вызова,  генерируемые  языками
  высокого уровня, и все межмодульные  обращения  к  основным
  подсистемам операционной системы соответствуют общим прави-
  лам вызова процедуры. При стандартном вызове процедуры тре-
  буется, чтобы все используемые регистры в процедуре с R2 по
  R11 были указаны во входной маске. При  стандартном  вызове
  процедуры не обеспечивается сохранение содержимого  регист-
  ров R0 и R1.
       Для того, чтобы сохранить  состояние  процессора,  при
  выполнении инструкции вызова в стеке формируется структура,
  которая называется кадром вызова или  кадром  стека.  Такой
  кадр содержит значение регистров, слово состояния процессо-
  ра, маску запоминания  регистров  и  несколько  управляющих
  разрядов. Кадр также содержит длинное  слово,  которое  при
  вызове процедуры обнуляется. Это длинное слово используется
  для реализации управления.
       После выполнения инструкции вызова  регистр  указателя
  кадра FP содержит адрес кадра вызова.  Инструкция  возврата
  использует содержимое указателя  кадра  при  восстановлении
  состояния процессора.
       Формат кадра вызова
   +-------------------------------------------------+
   !Адрес программы обработки состояния              ! : (FP)
   !(первоначально нуль)                             !
   +---+-+-+----------------+----------------+-------+
   !SPA!S!0!  маска<11:0>   !    PSW<15:5>   !   0   !
   +---+-+-+----------------+----------------+-------+
   !                     AP                          !
   +-------------------------------------------------+
   !                     FP                          !
   +-------------------------------------------------+
   !                     PC                          !
   +-------------------------------------------------+
   !                     R0 (...)                    !
   +-------------------------------------------------+
          .                                   .
          .                                   .
          .                                   .
   +--------------------------------------------------+
   !                     R11 (...)                    !
   +--------------------------------------------------+
       Поле SPA определяет число байтов для выравнивания  SP.
  Его значение от 0 до 3. Значение бита S зависит от вызова и
  равно
       S=1, если CALLS; S=0, если CALLG.
       Коды условий и PSW <T> обнуляются в кадре вызова.
       Содержимое разрядов PSW<3:0> в  кадре  при  выполнении
  инструкции  возврата  устанавливается  в   соответствии   с
  результатом последней инструкции в процедуре. Сходным обра-
  зом содержимое PSW<4> кадра при выполнении инструкции возв-
  рата определяет состояние разряда слежения в слове  состоя-
  ния процессора.

       В подразделе описываются следующие инструкции:
       - вызов процедуры общего вида CALLG ARGLIST.AB,  DST.AB,
         [-SP).W*];
       - вызов  процедуры  стекового  вида   CALLS   NUMARG.RL,
         DST.AB, [-(SP).W*];
       - возврат из процедуры RET [(SP)+.R*];

       CALLG        вызов процедуры общего вида
       -----        ----- --------- ------ ----
       Формат:
       код операции ARGLIST.AB, DST.AB
       Коды условий:
       N <- 0;
       Z <- 0;
       V <- 0;
       C <- 0;
       Исключительная ситуация:
       резервный операнд
       Коды операций:
    FA CALLG     вызов процедуры общего вида
       Описание.
       Значение указателя стека  запоминается  во  внутреннем
  регистре. Затем разряды  1:0  указателя  стека  обнуляются.
  Таким образом, происходит  выравнивание  стека  по  границе
  длинного слова. Содержимое входной маски процедуры последо-
  вательно опрашивается от разряда 0 до разряда 11 и содержи-
  мое регистров, номера которых соответствуют установленным в
  маске разрядам, заносится в стек в виде длинного  слова.  В
  стек в виде длинных слов заносится также содержимое регист-
  ров PC, FP и AP. Коды условий обнуляются. В стек  заносится
  длинное слово, которое состоит из  сохраненных  в  разрядах
  31:30 двух младших разрядов указателя стека, нулей в разря-
  дах 29 и 28, младших 12 разрядов входной маски процедуры  в
  разрядах 27:16 и слово состояния  процессора  с  обнуленным
  разрядом слежения в разрядах 15:0. Затем в  стек  заносится
  длинное слово с нулевым содержимым. В регистр FP  заносится
  содержимое указателя стека. В регистр AP заносится содержи-
  мое операнда списка аргументов ARGLIST. Разрешения внутрен-
  них прерываний в слове состояния процессора устанавливаются
  в указанное состояние. Разрешения внутренних прерываний  по
  целочисленному и десятичному переполнению устанавливаются в
  соответствии с содержимым разрядов 14 и 15  входной  маски.
  Бит разрешение внутреннего прерывания по переполнению  фор-
  мата чисел с плавающей запятой снизу обнуляется.  Состояние
  разряда слежения т не изменяется. В счетчик  инструкций  PC
  заносится содержимое  операнда-приемника  DST,  к  которому
  прибавляется 2. В результате управление передается к байту,
  следующему непосредственно за входной маской.
       Примечания:
       1. Если содержимое разрядов  13:12  входной  маски  не
  равно нулю, то возникает ошибка резервного операнда.
       2. При возникновении ошибки резервного операнда содер-
  жимое кодов условий является непредсказуемым.
       3. Для вызова процедуры  и  условного  управления  при
  сохранении содержимого  регистров  используют  определенные
  соглашения. Например, регистры R0 и R1  всегда  могут  быть
  использованы для значений функций  возврата  и  никогда  не
  отражаются во входной маске. Все используемые  в  процедуре
  регистры с R2 по R11 должны быть помечены во входной маске.

       CALLS     вызов процедуры стекового вида
       -----     ----- --------- --------- ----
       Формат:
       код операции NUMERG.RL, DST.AB
       Исключительная ситуация:
       резервный операнд
       Коды условий:
       N <- 0;
       Z <- 0;
       V <- 0;
       C <- 0;
       Исключительная ситуация:
       резервный операнд
       Коды операций:
    FB CALLS     вызов процедуры стекового вида
       Описание.
       Содержимое операнда NUMERG заносится  в  стек  в  виде
  длинного слова (нулевой  байт  содержит  число  аргументов;
  старшие 24 разряда используются системным программным обес-
  печением). Значение указателя стека запоминается  во  внут-
  реннем регистре. Затем разряды 1:0  указателя  стека  обну-
  ляются. Таким образом, производится выравнивание  стека  по
  границе длинного слова. Содержимое входной маски  процедуры
  последовательно опрашивается от разряда 0 до разряда  11  и
  содержимое регистров, номера которых соответствуют установ-
  ленным в маске разрядам, заносится в стек. В  стек  в  виде
  длинных слов заносится содержимое регистров PC,  FP  и  AP.
  Коды условий обнуляются. В стек  заносится  длинное  слово,
  которое состоит из сохраненных в разрядах 31:30 двух  млад-
  ших разрядов указателя стека, единицы в разряде 29 и нуля в
  разряде 28, младших 12 разрядов входной маски  процедуры  в
  разрядах 27:16 и слова состояния  процессора  с  обнуленным
  разрядом слежения в разрядах 15:0. Затем в  стек  заносится
  длинное слово с нулевым содержимым. В регистр FP  заносится
  содержимое указателя стека. В регистр AP заносится содержи-
  мое указателя стека, которое было после  занесения  в  стек
  числа аргументов. Разрешение внутренних прерываний в  слове
  состояния процессора устанавливается в указанное состояние.
  Разрешение внутренних прерываний по целочисленному и  деся-
  тичному  переполнению  устанавливается  в  соответствии   с
  содержимым разрядов 14 и 15 входной маски. Разрешение внут-
  реннего прерывания по переполнению для  чисел  с  плавающей
  запятой снизу обнуляется. Состояние разряда T не  изменяет-
  ся.  В   счетчик   инструкций   PC   заносится   содержимое
  операнда-приемника  DST,  к  которому  прибавляется  2.   В
  результате,  управление  передается  к  байту,   следующему
  непосредственно за входной маской.
       Примечания:
       1. Если содержимое разрядов  13:12  входной  маски  не
  равно нулю, то возникает ошибка резервного операнда.
       2. При возникновении ошибки резервного операнда содер-
  жимое кодов условий является непредсказуемым.
       3. Обычно список аргументов заносится в стек в  обрат-
  ном порядке перед выполнением инструкции вызова  процедуры.
  При возврате список аргументов автоматически  удаляется  из
  стека.
       4. Стандартные средства вызова процедуры при  сохране-
  нии содержимого  регистров  используют  некоторые  правила.
  Например, регистры R0 и R1 всегда могут  быть  использованы
  для возврата значений функций и никогда  не  помечаются  во
  входной маске. Все используемые в процедуре регистры  с  R2
  по R11 должны быть помечены во входной маске.

      RET       возврат из процедуры
      ---       ------- -- ---------
       Формат:
       код операции
       Коды условий:
       N <- TMP1<3>;
       Z <- TMP1<2>;
       V <- TMP1<1>;
       C <- TMP1<0>;
       Исключительная ситуация:
       резервный операнд
       Коды операций:
    04 RET     возврат из процедуры
       Описание.
       В указатель стека SP 5+носится содержимое регистра FP,
  к которому прибавляется 4. Длинное слово, содержащее  вели-
  чину выравнивания стека в разрядах  31:30,  индикатор  типа
  инструкции вызова в разряде 29, младшие 12 разрядов входной
  маски процедуры в разрядах  27:16  и  сохраненное  значение
  слова состояния процессора в  разрядах  15:0  удаляются  из
  стека и запоминается во внутреннем регистре TMP1. В регист-
  ры PC, FP и AP из стека заносится соответствующее  содержи-
  мое длинных слов. Маска восстановления регистров формирует-
  ся из разрядов 27:16 внутреннего регистра.  Последовательно
  опрашивая разряды маски восстановления с 0 по 11,  произво-
  дится копирование содержимого из стека в  регистры,  номера
  которых соответствуют установленным  в  маске  разрядам.  К
  содержимому указателя стека добавляется  значение  разрядов
  31:30 внутреннего регистра. Если разряд  29  во  внутреннем
  регистре равен единице (это  указывает  на  то,  что  вызов
  производится по инструкции CALLS), то длинное слово, содер-
  жащее число аргументов, удаляется из стека. Значение  млад-
  шего байта (без знака) этого длинного слова умножается на 4
  и прибавляется к содержимому указателя стека.
       Примечания:
       1. Ошибка резервного операнда возникает, если содержи-
  мое разрядов <15:8> промежуточного регистра не равны нулю.
       2. При возникновении ошибки резервного  операнда  сос-
  тояние кодов условий является непредсказуемым.
       3. Состояние разряда 28 внутреннего регистра не влияет
  на выполнение инструкции.
       4. При процедурах вызова и условного управления подра-
  зумевается,  что  процедуры,  которые  возвращают  значение
  функции или код состояния, делают это через регистр R0  или
  через регистры R0 и R1.

\end{verbatim}
