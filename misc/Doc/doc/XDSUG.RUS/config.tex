\chapter{Конфигурация \XDS}\label{config}
\index{конфигурация}

\section{Системные пути поиска}
\index{конфигурация!пути поиска}

Чтобы Ваша операционная система знала, где найти исполняемые файлы, 
из которых состоит пакет \xds{}, нужно подходящим образом установить системные
пути поиска файлов. Как это сделать --- смотрите в документации к Вашей
системе.

\section{Рабочая конфигурация}\label{config:work}

\ifgenc
  \xds{} работает под операционными системами
  OS/2, Windows NT, Windows 95, Unix, 
  \index{Unix} и многими другими.
\fi
Рабочая конфигурация
\ifgencode \xds{} \fi
включает в себя основную 
утилиту (\xc{}), сочетающую  \mt{} и \ot{} компиляторы,
\ifgencode
  \ifwinnt
     линкер\index{xlink}\index{линкер} ({\tt xlink}),
  \fi
  \iflinux \else
     менеджер библиотек\index{xlib}\index{менеджер библиотек} ({\tt xlib}),%???
  \fi
\fi
и набор системных файлов
\footnote{Имена системных файлов образуются из имени основной утилиты
и стандартных расширений. Если Вам нужно переименовать основную утилиту
\xc{}, то Вы должны соответственно переименовать и все системные файлы.}:
\begin{center}
\begin{tabular}{ll}
\bf \red    & Файл перенаправления путей поиска (см. \ref{xc:red}) \\
\bf \cfg    & Файл конфигурации (см. \ref{xc:cfg})  \\
\bf \msg    & Тексты сообщений об ошибках (см. \ref{xc:msg})  \\
\ifgenc
\bf \kwd    & Список ключевых слов и зарезервированных \\
            & идентификаторов в C/C++ \\
\fi
\end{tabular}
\end{center}
При запуске \xc{} \index{\XC{}(.exe)@{\XC{}(.exe)}}
сначала пытается найти файл {\tt \red} в текущей директории или
в директории, в которой находится \xc{}.

Остальные системные файлы ищутся по путям, указанным в {\tt \red}.
Если файл {\tt \red} не найден или в нем не указаны пути для какого-либо
системного файла, то файл ищется в текущей директории, а затем в
директории, содержащей \xc{}.

Файл конфигурации устанавливает параметры, общие для всех проектов.
Значения параметров, специфические для данного проекта, задаются в
проектном файле (project file; см. \ref{xc:project}). 
Так называемый файл-шаблон (template file)
используется для автоматизации процесса построения программы (см.
\ref{xc:template}).

Файлы конфигурации и перенаправления, проектный и файл-шаблон
определяют рабочее окружение пользователя \XDS{}. Общие черты всех
этих файлов описаны в \ref{xc:env}.

Одна из основных целей системы \XDS{} --- написание переносимых программ.
Для ее достижения не только исходные тексты программ, но и рабочее
окружение должны быть переносимы между различными платформами.
В \XDS{} используется переносимый формат записи имен файлов
\index{имя файла!переносимый формат}
\index{переносимость!имена файлов}
который может быть использован во всех системных файлах и в 
командной строке. Эта запись сочетает
DOS-овский и Unix-овский форматы (большие и малые буквы в именах файлов
различаются):
\begin{verbatim}
    [drive letter:] unix file name
\end{verbatim}

\Examples
\begin{verbatim}
    c:/xds/bin
    /mnt/users/alex/cur_pro
    cur_pro/sources
\end{verbatim}

В сочетании с макросом для {\em базовой директории} (см. \ref{xc:env}), 
использование переносимого формата позволяет писать все файлы окружения
независимо от платформы и позиции в файловой системе.

\section{Использование памяти системой \XDS{}}
\index{использование памяти (компиляторы)}

Компиляторы \XDS{} сами написаны на языке \ot\footnote{Мы используем
\XDS{} для всех наших разработок.}. 
Как и любая другая программа на Обероне, компиляторы используют сборщик
мусора для возвращения памяти. Большинство операционных систем
(напр. Unix, OS/2, Windows NT, Windows 95) предоставляют виртуальную
память, которая может быть значительно больше физической. Если память,
используемая \ot{} программой, больше объема физической памяти, то
сборщик мусора неэффективен. Поэтому важно ограничить объем памяти,
доступной для программы. Как правило, это ограничение задается в 
файле конфигурации или в проектном файле. (См. параметры
\OERef{HEAPLIMIT} и \OERef{GCTHRESHOLD}).

Для контроля за использованием памяти самими компиляторами следует установить 
значения параметров \OERef{COMPILERHEAP} и \OERef{COMPILERTHRES}.
Эти параметры устанавливаются в файле конфигурации ({\tt \cfg{}}).
Мы рекомендуем устанавливать их в зависимости от объема физической памяти
Вашего компьютера:
\begin{flushleft}
\begin{tabular}{c|c|c}
RAM (Мбайт)      & COMPILERHEAP & COMPILERTHRES \\ \hline
2-8              &    4000000   & 2000000  \\
8-16             &    6000000   & 3000000  \\
16 и более       &    8000000   & 4000000  \\
\end{tabular}
\end{flushleft}
Может оказаться необходимым увеличить
\OERef{COMPILERHEAP}, если Вы получили сообщение 
"{\em нехватка памяти}".
Однако, если параметр \OERef{COMPILERHEAP} установлен на 8 Мбайт, 
то это очень маловероятно. Ваша единица компиляции должна быть
чрезвычайно большой, чтобы превысить эту границу.

{\bf Замечание:} Если Вы используете определяющие модули для
OS/2 или Win32 API, установите \OERef{COMPILERHEAP} 
не менее 10 Мбайт.

С другой стороны, если вы видите слишком частые обращения к диску
при компиляции Вашей программы, это может означать, что параметр
\OERef{COMPILERHEAP} слишком велик для Вашей системной конфигурации.

\section{Иерархия директорий}
\index{конфигурация!директории}

Система \xds{} предоставляет полную свободу выбора, в каких директориях
держать исходные тексты, а также любые файлы, создаваемые самой системой.
Советуем работать в проектно-ориентированном стиле --- т.е. иметь
отдельную иерархию директорий для каждого проекта.

Поскольку модули, написанные на языках \mt{} или \ot{}, могут быть
многократно использованы, полезно иметь отдельную директорию для 
тех файлов, которые должны быть доступны во всех проектах. Мы будем
называть такие файлы {\em библиотеками}.

Мы рекомендуем иметь для каждого проекта отдельную рабочую директорию.
Например, для проекта под названием {\bf myproj}:

\verb'    mkdir myproj'

Затем установите ее текущей рабочей директорией:

\verb'    cd myproj'

Можно также создать отдельные поддиректории для символьных файлов
и для кодофайлов. Мы рекомендуем использовать прилагающийся скрипт
{\tt xdswork} или какой-либо его вариант для создания всех нужных
поддиректорий и локального файла перенаправления ({\tt \red}).

\section{Пути поиска в \XDS{}}

При запуске \xds{} ищет файл под именем {\tt \red} --- 
{\em файл перенаправления}. Этот файл содержит описание директорий,
в которых ищутся или создаются все остальные файлы.
Если он не найден в текущей директории, то он ищется в той директории,
где находится исполняемый файл \xds{}.

\subsection{Файл перенаправления}\label{xc:red}
\index{конфигурация!файл перенаправления}
\index{файл перенаправления (\red)}
\index{\XC{}.red@{\XC{}.red}}

Файл перенаправления содержит несколько строк следующего вида
\footnote{См. также \ref{xc:env}}:

\verb'    pattern = directory {";" directory}'

Можно также помещать в него строки с комментариями.
Строка-комментарий должна начинаться с символа \verb|"%"|. {\tt Pattern}
(образец) --- это регулярное выражение, с которым будут сравниваться имена
всех файлов, используемых \xds{}.
Образец может содержать спецсимволы '*' и '?', где
\begin{description}
\item[*] означает любую (возможно, пустую) цепочку символов;
\item[?] означает один произвольный символ.
\end{description}
Полное описание спецсимволов в образцах см. в \ref{xc:regexpr}.

Для имен директорий и путей используется переносимый формат
(см. \ref{config:work}).
Путь может быть абсолютным или относительным, т.е. он может
задавать полное имя директории, напр.

\verb'    /usr/myproj/def'

или ее имя относительно текущей директории, напр.

\verb'    def'

что означает поддиректорию \verb'def' текущей директории.
Отдельная точка в имени пути означает текущую директорию,
две точки --- ее родителя, т.е. директорию, в которой текущая
является поддиректорией.

Макрос для базовой директории \verb|$!|  % \index{$!@\verb'$!'} !!!
также может быть использован в именах директорий.
Он  означает путь к файлу перенаправления (точнее, к тому файлу, 
в котором он использован). %???
Например, если файл
перенаправления находится в директории \verb|/usr/alex|, то
\verb|$!/sym| означает \verb|/usr/alex/sym|, а \verb|$!/..|
означает директорию \verb|/usr|.

Имя каждого файла последовательно сравнивается с образцами в каждой строке.
Если оно соответствует образцу, то файл ищется последовательно в первой
директории из перечисленных в этой строке, затем во второй и т.д. до
тех пор, пока либо он не будет найден, либо не закончатся директории
для поиска и образцы для сравнения.

Если \xds{} не может найти файл, необходимый для правильного исполнения
(напр. нужный символьный файл), то выполнение прекращается и выдается 
сообщение об ошибке.

При создании файлов \xds{} также использует перенаправление. Его поведение
при этом определяется опцией \OERef{OVERWRITE}.
Если она выставлена и файл найден, то все нужные изменения будут 
производиться с этим найденным файлом.

Если файл с нужным именем не найден, или опция
\OERef{OVERWRITE} не выставлена, то файл будет создан в директории,
первой в списке в строке с образцом, соответствующим имени файла.

Если ни один образец в файле {\tt \red} не соответствует имени нужного
файла, то он будет читаться и записываться в текущей директории.

{\bf Замечание:}
Если образец, соответствующий имени файла, найден, то
\xds{} не будет искать файл в текущей директории, если она явно не
задана в пути поиска.

Например, вот типичная строка в {\tt \red}, задающая поиск
символьных файлов (предполагается, что они имеют расширение
{\bf.sym}):

\ifunix
\verb'    *.sym=sym;/usr/xds/sym;.'
\else
\verb'    *.sym=sym;c:/xds/sym;.'
\fi

Используя это перенаправление, компилятор будет искать символьные файлы
сначала в поддиректории \verb|sym| текущей рабочей директории,
затем в директории, содержащей символьные файлы библиотек \xds{}, 
и затем в текущей директории. Новый символьный файл будет при этом
создаваться в первой из них.

\paragraph{Пример файла перенаправления:}
\ifgencode
\begin{verbatim}
xc.msg = /xds/bin
*.mod = mod
*.def = def
*.ob2 = oberon
*.sym = sym; /xds/sym
\end{verbatim}
\else % ---- genc
\begin{verbatim}
xm.msg = /xds/bin
*.mod = mod
*.def = def
*.ob2 = oberon
*.c   = c
*.h   = include;/xds/include
*.sym = sym; /xds/sym
\end{verbatim}
\fi % ---- gencode

\subsection{Регулярные выражения}\label{xc:regexpr}
\index{регулярные выражения}

Регулярное выражение --- это строка, содержащая некоторые спецсимволы.
Их значение дано в следующей таблице:

\begin{center}
\begin{tabular}{cl}
\bf Символ   & \bf Означает \\
\hline
\verb+*+     & произвольная, возможно, пустая, последовательность символов \\
             & (эквивалентно: \verb|{\000-\377}|) \\
\verb+?+     & любой символ \\
             & (эквивалентно: \verb|[\000-\377]|) \\
\verb+[...]+ & один из перечисленных в скобках символов\\
\verb+{...}+ & произвольная, возможно, пустая, последовательность \\
             & перечисленных символов \\
\verb+\nnn+  & ASCII символ с восьмеричным кодом \verb|nnn|, где n --- это \verb|[0-7]| \\
\verb+&+     & логическое AND \\
\verb+|+     & логическое OR  \\
\verb+^+     & логическое NOT \\
\verb+(...)+ & определяют приоритет операций
\end{tabular}
\end{center}

Последовательность \verb|a-b| внутри скобок
\verb|[]| или \verb|{}| означает все символы от \verb|a| до \verb|b|.

\paragraph{Примеры}
\begin{description}

\item[\verb|*.def|] \mbox{} \\
    все файлы с расширением {\tt .def}

\item[\verb|project.*|] \mbox{} \\
    все файлы с именем {\tt project} и произвольным расширением

\item[\verb+*.def|*.mod+] \mbox{} \\
    файлы с расширением {\tt .def} или {\tt .mod}

\item[\verb'\lbrack a-z\rbrack *X.def'] \mbox{} \\ %???
    файлы с именем, начинающимся на строчную букву,
    заканчивающимся на "X", и расширением \verb|.def|.
\end{description}

\section{Опции}\label{config:options}

Богатое множество опций \xds{} позволяет контролировать исходный язык, 
генерацию кода и внутренние ограничения и настройки компилятора.
Различаются два вида опций: булевы опции (или просто опции, options) 
и параметры (equations).
{\em Опция} может быть либо выставлена (значение ON, или TRUE), 
либо не выставлена (значение OFF, или FALSE). 
Значением {\em параметра} является некоторая строка.
Здесь мы опишем только синтаксис директивы установки опций; полный список
опций и параметров \xds{} приведен в главе \ref{options}.

Опции и параметры могут быть установлены в файле конфигурации (см.
\ref{xc:cfg}), в командной строке (см. \ref{xc:modes}) и в проектном файле
(см. \ref{xc:project}). Опции также могут определяться в тексте программы
(см. \ref{m2:pragmas}).

Синтаксис, используемый в файле конфигурации, в проектном файле и в 
командной строке, один и тот же. Единственное различие в том, что в 
файлах, в отличие от командной строки, допускается произвольное 
число пробелов. Прописные и строчные буквы в именах опций и параметров 
не различаются.

\begin{verbatim}
SetupDirective   = SetOption
                 | SetEquation
                 | DeclareOption
                 | DeclareEquation
                 | DeclareSynonym
SetOption        = '-' name ('+'| '-')
SetEquation      = '-' name '=' [ value ]
DeclareOption    = '-' name ':' [ '+' | '-' ]
DeclareEquation  = '-' name ':=' [ value ]
DeclareSynonym   = '-' name '::' name
\end{verbatim}
\index{-NAME+@\verb'-NAME+'}
\index{-NAME-@\verb'-NAME-'}
\index{-NAME=@\verb'-NAME='}
\index{-NAME:@\verb'-NAME:'}
\index{-NAME:=@\verb'-NAME:='}
\index{-NAME::@\verb'-NAME::'}

Все опции и параметры, используемые \xds{}, предекларированы.

Директива \verb|DeclareSynonym| позволяет использовать другое
(напр. более короткое) имя для какой-либо опции.

Для совместимости, поддерживается также 
устаревшая версия директивы {\tt SetOption}:

\verb"OldSetOption   = '+' name | '-' name"

\Examples
\begin{flushleft}
\begin{tabular}{l|p{8.0cm}}
\bf Директива & \bf Значение \\ \hline
\verb|-M2Extensions+| & {\bf M2EXTENSION} установить ON \\
\verb|-Oberon=o2|     & {\bf OBERON} установить равным \verb|"o2"| \\
\verb|-debug:|        & {\bf DEBUG} декларировать и установить OFF \\
\verb|-DemoVersion:+| & {\bf DEMOVERSION} декларировать и установить ON \\
\verb|-Vers:=1.0|     & {\bf VERS} декларировать и приравнять \verb|"1.0"| \\
\verb|-T::CheckIndex| & {\bf T} декларировать как синоним {\bf CHECKINDEX} \\
\verb|-m2extensions|  & {\bf M2EXTENSION} установить OFF 
\end{tabular}
\end{flushleft}

\section{Файл конфигурации}\label{xc:cfg}
\index{Файл конфигурации (\cfg)}
\index{\XC{}.cfg@{\XC{}.cfg}}
\index{конфигурация!файл конфигурации}

Файл конфигурации может быть использован для установки значений
параметров и опций (см. Главу \ref{options}). 
В каждой строке файла конфигурации может содержаться не более одной
директивы установки опции или значения параметра (см. \ref{config:options}). 
Разрешается произвольное количество пробелов между лексемами.
Символ "\verb|%|" --- это знак комментария; остаток строки после
него пропускается.

{\bf Замечание:} В строке, задающей значение параметра, знак комментария
использовать нельзя.

Файл конфигурации обычно содержит умолчательные значения опций и 
параметров для данной платформы, а также декларации специфических для 
платформы дополнительных параметров, которые могут быть затем 
использованы в проектном файле и в файле-шаблоне.

\paragraph{Пример файла конфигурации:}
\begin{verbatim}
  % это комментарий
  % Задаем параметр:
   - BSDef = df
  % Задаем предекларированные опции:
   - RangeCheck -   % выключение проверки границ
   - M2EXTENSIONS + % разрешение расширений Модулы-2
  % Декларируем новые опции:
   -i80486:+
   -i80386:-
   -i80286:         % равносильно -80286:-
  % Декларируем синоним:
   -N :: checknil
   -N               % выключение проверок на NIL
  % конец файла конфигурации
\end{verbatim}

\section{Расширения имен файлов}\label{config:fileext}
\index{конфигурация!расширения имен файлов}
\index{имя файла!расширение}

\xds{} дает возможность выбирать стандартные расширения для каждого
типа файлов. Например, Вы можете выбрать в качестве стандартного 
расширения для файлов с текстами на Обероне-2 как
{\bf .o2}, так и {\bf .ob2}.

Разумно, однако, использовать традиционные расширения, или уж расширения,
которые, как и традиционные, ясно показывают, к файлам какого типа они
относятся. Например, расширения
{\bf.def} \index{.def (см. DEF)}
и {\bf.mod} \index{.mod (см. MOD)}
для модулей на Модуле-2,
{\bf .ob2} \index{.ob2 (см. OBERON)}
для модулей на Обероне-2 и~т.д.
Разумно также использовать одни и те же расширения во всех 
Ваших проектах.

На выбор расширений могут влиять различные факторы. 
Так, по традиции псевдо-определяющие модули для модулей на 
Обероне-2 (автоматически создаваемые компилятором в режиме просмотра) 
снабжаются расширением {\bf.def}.  
В системе \xds{} это привело бы к конфликту с расширением для 
определяющих модулей Модулы-2. Поэтому по умолчанию мы используем
расширение {\bf .odf}.

Файл конфигурации обычно определяет следующие параметры, значения которых 
--- расширения для имен файлов:
\begin{flushleft}
\begin{tabular}{ll}
\OERef{DEF}    & расширение для определяющего модуля в Модуле-2     \\
\OERef{MOD}    & расширение для реализующего модуля в Модуле-2      \\
\OERef{OBERON} & расширение для модуля в Обероне-2                  \\
\OERef{BSDEF}  & расширение для псевдоопределяющего модуля в Обероне-2 \\
\ifgenc
\OERef{HEADER} & расширение для файла-заголовка в C                 \\
\fi
\OERef{CODE}   & расширение для порожденного кодофайла              \\
\OERef{SYM}    & расширение для символьного файла                   \\
\end{tabular}
\end{flushleft}
В таблице \ref{table:equ:ext} приведен полный список расширений имен файлов.

\paragraph{Пример (задание расширений имен файлов в \cfg):}

\begin{verbatim}
   -def     = def
   -mod     = mod
   -oberon  = ob2
   -sym     = sym
\end{verbatim}

\section{Подробнее об окружении \XDS{}}\label{xc:env}

Окружение компилятора \XDS{} состоит из:
\begin{itemize}
\item файла перенаправления
\item файла конфигурации
\item проектного файла для каждого проекта
\item файлов-шаблонов
\end{itemize}
Все сказанное ниже в этой секции применимо к любому из этих файлов.

Каждый файл --- это последовательность строк.
Символ \verb|\| в конце строки означает ее продолжение на следующей
строке.
Во всех файлах окружения можно использовать следующие средства:
\begin{itemize}
\item переносимую нотацию для имен файлов (см. \ref{config:work}).
\item макрос для {\em базовой директории} (\verb|$!|). % \index{$!@\verb'$!'}. !!!
      Этот макрос обозначает директорию, в которой лежит файл, содержащий
      макрос.
\item набор директив, начинающихся с символа \verb|!|.
\end{itemize}
Директивы имеют следующий синтаксис (в ключевых словах прописные 
и строчные не различаются):
\begin{verbatim}
Directive = "!" "NEW" SetOption | SetEquation
          | "!" "SET" SetOption | SetEquation
          | "!" "MESSAGE" Expression
          | "!" "IF" Expression "THEN"
          | "!" "ELSIF" Expression "THEN"
          | "!" "ELSE"
          | "!" "END".
SetOption   = name ( "+" | "-" ).
SetEquation = name = string.
\end{verbatim}

Директива NEW декларирует новую опцию или параметр. Директива 
SET меняет значение опции или параметра.
Директива MESSAGE выводит значение выражения Expression.
Директива IF позволяет обрабатывать или пропускать куски файла
в зависимости от значения выражения Expression.

\begin{verbatim}
Expression = Simple [ Relation Simple ].
Simple      = Term { "+" | OR Term }.
Relation    = "=" | "#" | "<" | ">".
Term        = Factor { AND Factor }.
Factor      = "(" Expression ")".
            | String
            | NOT Factor
            | DEFINED name
            | name.
String      = "'" { character } "'"
            | '"' { character } '"'.
\end{verbatim}

Каждый операнд в выражении --- либо строка, либо имя параметра, либо
имя опции.
В случае параметра, используется его значение (строка).
В случае опции используется строка
"TRUE" или "FALSE". 
Оператор "+" означает конкатенацию строк. Операторы отношений означают
сравнение строк.
Оператор NOT может быть применен к любой строке со значением
"TRUE" или "FALSE". Оператор DEFINED возвращает "TRUE", если
опция или параметр с именем {\tt name} декларированы, и "FALSE"
в противном случае.

См. также секцию \ref{opt:COMPILER}.

\paragraph{Пример проектного файла}
\begin{verbatim}
% проверка режима проекта
!if not defined mode then
  % по умолчанию режим debug 
  !new mode = debug
!end
% вывести режим проекта
!message "Сборка проекта в режиме " + mode + "."
% установить опции для данного режима
!if mode = debug then
   - gendebug+
   - checkrange+
!else
   - gendebug-
!fi
% указать файл-шаблон
- template = $!/templates/watcom.tem
!module Main.ob2
\end{verbatim}

\section{Изменение сообщений \XDS{}}\label{xc:msg}
\index{\XC{}.msg@{\XC{}.msg}}

Файл {\tt \msg} содержит тексты сообщений об ошибках в виде

\verb'<number>       <text>'

Вот выдержка из файла \msg{}:
\begin{verbatim}
001 недопустимый символ
002 незакрытый комментарий, начатый в строке %d
...
042 несовместимы по присваиванию
...
\end{verbatim}

Некоторые сообщения содержат спецификации формата для дополнительных
аргументов. Так, приведенное выше сообщение
{\em незакрытый комментарий, начатый в строке...} содержит спецификацию 
\verb|%d| для вывода номера строки.

Чтобы компилятор сообщал об ошибках на другом языке, нужно лишь перевести
текст сообщений, сохраняя номера ошибок и количество и последовательность
спецификаций формата в каждой строке. 
В качестве примера --- соответствующая выдержка из английского варианта
файла \msg{}:
\begin{verbatim}
001 illegal character
002 comment not closed; started at line %d
...
042 incompatible assignment
...
\end{verbatim}

\section{XDS и Ваш C компилятор}\label{config:C}

\ifgencode % ------------------------
  \xds{} позволяет использование в Ваших проектах C-библиотек.
  Различные C компиляторы используют разные соглашения о
  выравнивании, об именах и о связях.
  \iflinux
    {\bf ВНИМАНИЕ!} Поскольку библиотеки \xds{} для linux 
    построены с помощью компилятора GCC,
    {\em абсолютно необходимо} сконфигурировать XDS под GCC.
  \else % not linux -------------------------------
    Поэтому необходимо сконфигурировать \xds{} под Ваш C компилятор,
    чтобы использовать имеющиеся у Вас C-библиотеки и программы.
  \fi  % end linux
  См. более подробно в главе \ref{ccomp}.
\fi  % end gencode ------------------------

\ifgenc % ------------------------

Пакет XDS-C можно использовать как транслятор и как компилятор.
В первом случае, C-код, порожденный \XDS{}, используется для
дальнейшей работы или переносится на другую платформу. Во втором случае,
\XDS{} служит как бы первым проходом компилятора: для получения 
исполняемого кода вызывается C-компилятор. Вызовы C-компилятора,
линкера и сборщика могут быть сделаны видимыми для пользователя. %???

Обычно достаточно сконфигурировать \XDS{} под Ваш C-компилятор
лишь один раз, а затем \XDS{} сам сделает все остальное для превращения
Вашей программы в исполняемый файл.

\subsection{Создание библиотеки}

Для Unix-овских платформ, пакет \XDS{} включает в себя библиотечный файл,
содержащий объектные файлы всех библиотечных модулей.
Для платформ OS/2 и Windows NT/95, библиотеки в пакете \XDS{}
предоставлены в виде исходных текстов на C, т.е. для каждого библиотечного
модуля в пакет входят его файл-заголовок и исходный текст.
Так сделано потому, что различные C-компиляторы придерживаются
различных соглашений об именах и связях. Мы рекомендуем в этом случае
скомпилировать все библиотеки и создать библиотечный файл для 
Вашего C-компилятора, или даже несколько различных библиотечных файлов
для разных компиляторов и моделей памяти.
Директория C в пакете \XDS{} содержит специальный 
make-файл, который предназначен для этой цели. В этом файле может 
потребоваться изменить лишь название C-компилятора или набор его
опций.

Если Вы предпочитаете работу в интегрированном окружении, то можно
создать проект, включить в него все C-тексты, и затем построить 
библиотеку.

\subsection{Настройка XDS для непрерывной компиляции}\label{config:seamless}

Мы будем использовать термин {\em непрерывная (seamless) компиляция},
если \xds{} сконфигурирован так, что он автоматически вызывает
C-компилятор, линкер или сборщик для получения исполняемого 
файла Вашей программы. Make-файл может быть создан, и сборщик 
автоматически вызван, после успешного завершения проекта для компиляции
и сборки Вашей программы.

Для создания make-файла компилятор использует
файл-шаблон (см. \ref{xc:template}).
Дистрибутив \XDS{} содержит файл-шаблон {\tt xds.tem}, который
может быть использован со многими C-компиляторами.
Параметр {\bf ENV\_TARGET} задает целевую платформу (target platform), 
т.е. файловую систему (в~т.ч. представление имен файлов), используемый
C-компилятор, его опции и~т.п. Значение этого параметра по умолчанию
равно значению параметра {\bf ENV\_HOST} --- рабочей платформы.
В файле конфигурации {\tt \cfg} есть список поддерживаемых платформ,
т.е. список имен, которые могут использоваться в качестве значения
параметра {\bf ENV\_TARGET}. Мы рекомендуем установить это значение 
в самом начале файла конфигурации. Если возникнет необходимость
изменить некоторые параметры для Вашей платформы, или добавить новую
платформу, то все изменения следует вносить только в файлы
{\tt \cfg} и {\tt xds.tem}.

{\bf Замечание.} В текущей версии невозможно добавить новую
платформу с 
DOS-подобной файловой системой. Мы рекомендуем при необходимости
просто изменять нужные параметры в платформах
"MSDOS", "OS2" или "WinNT".

Более подробно см. в \ref{start:run}.

\fi  % end genc ------------------------
