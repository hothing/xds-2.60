% !!! M2/O2 и GC (if a program contains O2 модуль)
% !!! Review Interfacing to C (it rlates to SysCall etc. as well)
%

\chapter{Многоязыковое программирование}\label{multilang}
\index{многоязыковое программирование}

\xds{} позволяет программисту использовать в одном проекте
модули, библиотеки и объектные файлы на языках
\mt{}, \ot{}, C и Ассемблере.

\section{Модула-2 и Оберон-2}\label{multilang:m2o2}
\index{многоязыковое программирование!Модула-2/Оберон-2}

Не обязательно сообщать компилятору об использовании объектов 
из языка \mt{} в модуле на языке
\ot{}, и наоборот. Компилятор автоматически определит язык по 
символьному файлу при обработке секции импорта.

\subsection{Основные типы}

В языке \ot{} основные типы имеют одну и ту же длину на всех платформах.
В языке
\mt{} размер типов {\tt INTEGER}, {\tt CARDINAL} и {\tt BITSET} 
может меняться, и зависит от значения опции
\OERef{M2BASE16}.
В следующей таблице приводятся размеры основных типов в обоих языках.
\begin{center}
\begin{tabular}{|l|c|c|c|c|} \hline
Тип      & Размер & Оберон-2  & \multicolumn{2}{c|}{Модула-2}     \\
         &      &           & {\tt M2BASE16+} & {\tt M2BASE16-} \\ \hline
integer  & 8    &  SHORTINT &   ---           &         ---     \\
integer  & 16   &  INTEGER  &   INTEGER       &         ---     \\
integer  & 32   &  LONGINT  &   ---           &       INTEGER   \\
cardinal & 8    &  ---      &   ---           &         ---     \\
cardinal & 16   &  ---      &   CARDINAL      &         ---     \\
cardinal & 32   &  ---      &   ---           &       CARDINAL  \\
bitset   & 16   &  ---      &   BITSET        &         ---     \\
bitset   & 32   &  SET      &   ---           &       BITSET    \\
\hline
\end{tabular}
\end{center}

Системные типы {\tt INT} и {\tt CARD} соответствуют
\mt{}-типам {\tt INTEGER} и {\tt CARDINAL}.
We recommend to use 
Мы рекомендуем при импорте из Модула-2 модулей в Оберон-2 модули
использовать типы {\tt INT} и {\tt CARD}.
Например, если процедура {\tt Foo} определена в
\mt{} определяющем модуле как
\begin{verbatim}
DEFINITION MODULE M;

PROCEDURE Foo(VAR x: INTEGER);

END M.
\end{verbatim}
то переносимое ее использование в Обероне-2 таково:
\begin{verbatim}
 VAR x: SYSTEM.INT;
  ...
  M.Foo(x);
\end{verbatim}

\subsection{Структуры данных}

\xds{} позволяет использовать в Обероне-2 любые структуры данных
языка \mt{}, даже те, которые не могут быть определены в Обероне-2
(напр. вариантные записи, отрезки, множества, перечислимые типы и~т.п.).

Однако использование модуловских типов в Обероне, и наоборот,
имеет некоторые ограничения. 
Когда это возможно, \xds{} генерирует корректный код. 
В тех случаях, когда корректная трансляция невозможна, выдается 
сообщение об ошибке:
\begin{itemize}
\item
  поле \mt{}-записи не может иметь тип \ot{}-указателя, записи или массива;
\item
  \mt{}-указатель на \ot{}-запись нельзя использовать в специфических
конструкциях языка \ot{} (методы, охрана типа и~т.п.);
\item
  скрытый тип не может быть определен как \ot{}-указатель.
\end{itemize}

Стандартные процедуры NEW и DISPOSE всегда применяются в соответствии
с родным языком типа параметра. Рассмотрим, например, такие описания 
типов в \ot{} модуле:
\begin{verbatim}
TYPE
  Rec = RECORD END;
  MP  = POINTER ["Modula"] TO Rec; (* Modula pointer *)
  OP  = POINTER TO Rec;     (* Oberon pointer *)
VAR
  m: MP;
  o: OP;
\end{verbatim}
Вызов \verb|NEW(m)| будет заменен (по умолчанию) на вызов
\mt{} процедуры {\tt ALLOCATE}, а вызов \verb|NEW(o)| будет рассматриваться
как вызов стандартной обероновской процедуры из поддержки исполнения.
См. также \ref{multilang:direct}.

Неявное освобождение памяти (сборка мусора) применяется только к 
объектам языка \ot{}. Если в
\ot{} модуле описана переменная типа модуловского указателя, 
то ее следует деаллоцировать явно. 

\paragraph{Пример: Модуловские типы данных в Обероне}
\begin{verbatim}
(* Модула-2*) DEFINITION MODULE m2;
TYPE
  Rec = RECORD  (* вариантная запись *)
    CASE tag: BOOLEAN OF
      |TRUE:  i: INTEGER;
      |FALSE: r: REAL;
    END;
  END;
  Ptr = POINTER TO Rec;

VAR
  r: Rec;
  p: Ptr;

PROCEDURE Foo(VAR r: Rec);

END m2.

(* Оберон-2 *) MODULE o2;

IMPORT m2; (* импорт Модула-2 модуля *)

VAR
  r: m2.Rec;  (* Модула-2 запись *)
  p: m2.Ptr;  (* Модула-2 указатель *)
  x: POINTER TO m2.Rec;

BEGIN
  NEW(p);     (* Модула-2 ALLOCATE *)
  NEW(x);     (* Оберон-2 NEW *)
  m2.Foo(r);
  m2.Foo(p^);
  m2.Foo(x^);
END o2.
\end{verbatim}

\subsection{Сборка мусора}

Следует помнить, что \mt{} и \ot{} по-разному подходят к проблеме
использования памяти. Если программа содержит и
\mt{}, и \ot{} модули, то используется сборка мусора.
См. также \ref{rts:mm}.

%--------------------------------------------------------------

\section{Явное указание языка}\label{multilang:direct}
\index{многоязыковое программирование!спецификация языка}
\index{C интерфейс!спецификация языка}

Компилятор должен знать язык, на котором реализован модуль,
чтобы учесть различия в семантике разных языков и сгенерировать
корректный код.

В некоторых случаях необходимо, чтобы процедура или тип данных были
реализованы по правилам не того языка, на котором написан весь модуль,
а какого-либо другого.
\xds{} позволяет в явном виде задавать родной язык для типа или объекта.
Явное указание языка\index{DLS}(DLS) разрешается, если модуль импортирует
модуль SYSTEM, или если разрешены расширения языка.

В описании типа записи, указателя или процедуры, или в декларации
процедуры, сразу после ключевых слов
RECORD, POINTER или PROCEDURE можно указать нужный язык (т.е. соглашения,
по которым это описание будет обрабатываться). Указание языка имеет вид
\verb|[n]|, где
\verb|n| --- это строка или целое константное выражение
\footnote{Рекомендуем использовать строки; числовые значения сохраняются
только для обратной совместимости.}:
\begin{center}
\begin{tabular}{|l|c|c|}
\hline
Соглашение  & Строка    & Число   \\ %??? Convention
\hline
Оберон-2   & "Oberon"  & 0       \\
Модула-2   & "Modula"  & 1       \\
C          & "C"       & 2       \\
Pascal     & "Pascal"  & 5       \\
Win32 API  & "StdCall" & 7       \\
OS/2 API   & "SysCall" & 8       \\
\hline
\end{tabular}
\end{center}

\Example
\begin{verbatim}
TYPE
  UntracedPtr = POINTER ["Modula"] TO Rec;
\end{verbatim}

Здесь {\tt UntracedPtr} определен как \mt{} указатель, поэтому 
переменные этого типа не будут отслеживаться сборщиком мусора.

Явное указание языка после имени поля, константы, типа или переменной,
означает, что имя этого объекта будет обрабатываться в соответствии
с правилами указанного языка (или конвенции).
\begin{verbatim}
TYPE
  Rec ["C"] = RECORD
    name ["C"]: INTEGER;
  END;

CONST pi ["C"] = 3.14159;

VAR buffer[]["C"]: POINTER TO INTEGER;
\end{verbatim}
ISO \mt{} позволяет задавать адрес переменной; чтобы воспользоваться этим,
следует добавить пустые квадратные скобки в качестве указания языка. 
%??? an address may be specified for a variable,
%??? the empty brackets are required to set the language.

Имя процедуры обрабатывается по правилам языка описания процедуры.
Так, в описании
\begin{verbatim}
PROCEDURE ["C"] Foo;
\end{verbatim}
и тип процедуры, и ее имя будут обрабатываться по правилам языка C.

Note: if you are using 
Замечание: если Вы используете компилятор C++, то функция {\tt Foo} 
должна быть описана
with C name mangling style. %??? 
Более подробно см. руководство по C++.

%----------------------------------------------------------------

\section{Интерфейс с C}\label{multilang:C}
\index{многоязыковое программирование!интерфейс с C}
\index{C интерфейс}

Особое внимание в \XDS{} уделено удобному интерфейсу с другими языками,
преимущественно с языком C. Основная цель этого --- предоставить доступ
к уже имеющимся программам.
Иноязычные определяющие модули предоставляют такой доступ к
стандартным библиотекам.

\subsection{Иноязычный определяющий модуль}
\index{иноязычный определяющий модуль}

Явное указание языка \verb|[n]| может находиться сразу после 
ключевых слов
DEFINITION MODULE (см. \ref{multilang:direct}). 
При этом все объекты, определенные в модуле, будут транслироваться 
по правилам указанного языка, и явное указание языка для каждого
объекта становится необязательным.

Некоторые опции компилятора специально предназначены для 
иноязычных определяющих модулей.
\ifgenc
См. в \ref{maptoc:opt:foreign} 
подробное описание опций для создания иноязычного
определяющего модуля.
\fi
\ifgencode
{\bf Замечание:} опции {\bf CSTDLIB} и {\bf NOHEADER} 
применимы только в трансляторах на язык C.
\fi

\Example
\begin{verbatim}
<*+ M2EXTENSIONS *>
<*+ CSTDLIB *>      (* C стандартная библиотека *)
<*+ NOHEADER *>     (* файл-заголовок уже есть *)
DEFINITION MODULE ["C"] string;

IMPORT SYSTEM;

PROCEDURE strlen(s: ARRAY OF CHAR): SYSTEM.size_t;
PROCEDURE strcmp(s1: ARRAY OF CHAR;
                 s2: ARRAY OF CHAR): SYSTEM.int;
END string.
\end{verbatim}

Некоторые полезные советы для  создания своего собственного иноязычного
определяющего модуля:
\begin{itemize}
\item
Если Вы пишете интерфейс к существующему файлу-заголовку,
выставляйте опцию {\bf NOHEADER}, выключающую порождение файла-заголовка.
Эта опция влияет только на трансляторы.

\item
Если файл-заголовок --- стандартный, используйте опцию {\bf CSTDLIB}.
Эта опция влияет только на трансляторы.

\item
Используйте специальные типы из модуля SYSTEM: {\tt int}, {\tt unsigned},
\verb|size_t| и {\tt void} в качестве соответствующих C типов.

\item
Параметр типа \verb|T *| следует заменять на параметр типа
{\tt ARRAY OF T}, если в него передаются только массивы. Иначе 
следует использовать {\tt POINTER TO T}.

\end{itemize}
Определяющие модули для библиотек ANSI C ({\tt stdio.def},
{\tt string.def}, и~т.д.) могут служить хорошими учебными примерами.

\subsection{Спецификация внешних процедур}\label{multilang:extproc}
\index{многоязыковое программирование!внешние процедуры}
\index{C интерфейс!внешние процедуры}

В некоторых случаях выгоднее не писать иноязычный 
определяющий модуль, а напрямую использовать имеющиеся C функции.
Компиляторы \xds{} позволяют декларировать функцию как внешнюю.

Декларация внешней процедуры состоит только из заголовка процедуры.
Перед именем процедуры в заголовке ставится символ \verb|"/"|.
\begin{verbatim}
PROCEDURE ["C"] / putchar(ch: SYSTEM.int): SYSTEM.int;
\end{verbatim}

%------------------------------------------------------------------

\section{Ослабления правил совместимости}

Все семантические проверки для объектов и типов компилятор проводит
в соответствии с их языком. Объект, языком которого является
\mt{} или \ot{}, будет подчиняться правилам совместимости соответственно
языков \mt{} или \ot{}.
Правила совместимости для объектов и типов, языками которых объявлены
"C", "Pascal", "StdCall" и "SysCall", смягчены.

\subsection{Совместимость по присваиванию}

Два объекта типа указателя совместимы по присваиванию, если

\begin{itemize}
\item они одного и того же Модула-2 или Оберон-2 типа.
\item тип хотя бы одного из них объявлен как
      "C", "Pascal", "StdCall" или "SysCall" и их базовые типы совпадают.
\end{itemize}

\begin{verbatim}
VAR
  x: POINTER TO T;
  y: POINTER TO T;
  z: POINTER ["C"] TO T;
BEGIN
  x := y;       -- error
  y := z;       -- ok
  z := y;       -- ok
\end{verbatim}

\subsection{Совместимость при передаче}
\label{multilang:parmcomp}

Для процедур, объявленных как "C", "Pascal", "StdCall" или "SysCall",
правила совместимости при передаче параметров значительно смягчены.
Если формальный параметр --- типа, описанного как {\tt POINTER TO T},
то фактический параметр может быть одним из следующих типов:

\begin{itemize}
\item 
Того же самого типа. Для обычных Модула-2/Оберон-2 процедур это ---
единственная возможность.
\item Другого типа, описанного как {\tt POINTER TO T}
\item 
Любого типа массива элементов типа {\tt T}.
В этом случае, как это происходит в C, передается адрес первого элемента
массива.
\item Самого типа {\tt T}, если {\tt T} --- тип записи. В этом случае
передается адрес записи.
\end{itemize}

Такое ослабление, в сочетании с наличием системной функции
{\tt SYSTEM.REF}, облегчает использование в языках
Модула-2/Оберон-2 вызовов C библиотек и целевой операционной системы
API, сохраняя при этом преимущества механизмов контроля типов этих языков.

\Example
\begin{verbatim}
TYPE
  Str = POINTER TO CHAR;
  Rec = RECORD ... END;
  Ptr = POINTER TO Rec;

PROCEDURE ["C"] Foo(s: Str); ... END Foo;
PROCEDURE ["C"] Bar(p: Ptr);  ... END Bar;

VAR
  s: Str;
  a: ARRAY [0..5] OF CHAR;
  p: POINTER TO ARRAY OF CHAR;
  R: Rec;
  A: ARRAY [0..20] OF REC;
  P: POINTER TO REC;

  Foo(s);   (* типы совпадают *)
  Foo(a);   (* разрешено для "C" процедуры *)
  Foo(p^);  (* разрешено для "C" процедуры *)
  Bar(R);   (* эквивалентно Bar(SYSTEM.REF(R)); *)
  Bar(A);   (* разрешено для "C" процедуры *)
  Bar(P);   (* разрешено для "C" процедуры *)
\end{verbatim}

\subsection{Игнорирование результата функции}
\index{C интерфейс!использование C функций}

В программировании на C стандартная практика --- игнорировать результат
вызова функции. Некоторые стандартные библиотечные функции спроектированы
с учетом этой практики. Например, функция копирования строки получает
строку назначения как VAR параметр (в терминах Модулы-2) и возвращает
указатель на нее:
\begin{verbatim}
extern char *strcpy(char *, const char *);
\end{verbatim}
Часто результат функции {\tt strcpy} игнорируется.

Компиляторы \xds{} разрешают игнорировать результаты функций, определенных
как "C", "Pascal" или "StdCall".
Так, функция {\tt strcpy}
определенная в иноязычном определяющем модуле {\bf string.def} как
\begin{verbatim}
PROCEDURE ["C"] strcpy(VAR d: ARRAY OF CHAR;
                           s: ARRAY OF CHAR): ADDRESS;
\end{verbatim}
может использоваться и как процедура, и как функция:
\begin{verbatim}
  strcpy(d,s);
  ptr:=strcpy(d,s);
\end{verbatim}


\ifcomment % ------------------------------------------------------
%!!!!

\section{Интерфейс с другими языками}
\index{многоязыковое программирование!другие языки}

Хотя \xds{} ориентирован на ANSI C, возможно и использование
{Pascal} и {FORTRAN} библиотек в 
\mt{}/\ot{} программах.

Большинство C компиляторов распознают специальное ключевое слово
{\tt pascal} (и/или {\tt fortran}), ставящееся непосредственно перед
именем функции.

\begin{verbatim}
void pascal PrintScreen (void);
/* pascal function declaration */
\end{verbatim}

Соглашения об именах и связях различны в 
C и в {Pascal}/{FORTRAN}, поэтому и требуется специальное ключевое слово.
Если Ваш C компилятор поддерживает такой интерфейс с иноязычными
библиотеками, то Вы можете использовать процедуры, переменные, типы
из языков {Pascal}/{FORTRAN} точно так же, как и C-объекты.

Для этого в первую очередь нужно создать соответствующий 
файл-заголовок, если его нет. См. об этом в документации к
Вашему C компилятору.

После этого нужно создать Модула-2 определяющий модуль.
Такие ключевые слова, как {\tt pascal}, {\tt far}, {\tt volatile} 
в определяющем модуле можно опустить.
Затем скомпилируйте этот определяющий модуль с опцией {\bf NOHEADER}
для получения символьного файла.

Описанию на Паскале
\begin{verbatim}
  function Foo(p: real): real;
\end{verbatim}
соответствует строка из C файла заголовка
\begin{verbatim}
  float pascal far Foo(float p);
\end{verbatim}

и определяющий модуль
\begin{verbatim}
<*+ NOHEADER *>
DEFINITION MODULE [2] Pascals;

  PROCEDURE Foo(p: REAL): REAL;
END Pascals.
\end{verbatim}

\fi % comment -----------------------------------------------------
