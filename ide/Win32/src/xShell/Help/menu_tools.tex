\section{Tools Menu}
\public{M\_TOOLS}
\index{Tools Menu}
\nominitoc

\ifenglish
The {\bf Tools} menu contains commands that you use to invoke
external tools, such as an XDS compiler.  
The following commands appear in the {\bf Tools} menu by default:
\else
Меню {\bf Tools} содержит команды, позволяющие вызывать внешние утилиты, 
такие, как XDS компилятор.
В меню {\bf Tools} по умолчанию содержатся следующие команды:
\fi
\ref[Compile]{IDM\_COMPILE}
\ifenglish
  - Compiles the file in the current edit window
 \else
  - Компилирует файл в текущем окне редактирования
 \fi
\ref[Make]{IDM\_MAKE}
\ifenglish
  - Makes the current project
  \else
  - Компонует текущий проект
  \fi
\ref[Build all]{IDM\_BUILDALL}
\ifenglish
  - Makes the current project, rebuilding all files
  \else
  - Компонует текущий проект, перестраивая все файлы
  \fi
\ref[Update file list]{IDM\_UPDATE}
\ifenglish
  - Updates source files list without compilation
  \else
  - Обновляет список исходных файлов без компиляции
  \fi
\subsection{Tools: Compile}
\public{IDM\_COMPILE}

\ifenglish
Use the {\bf Compile} command to compile a module in the
current window.

Keyboard shortcut: {\bf F9}
\else
Используйте команду {\bf Compile}, чтобы компилировать модуль в 
текущем окне.

Комбинация клавиш: {\bf F9}
\fi
\subsection{Tools: Make}
\public{IDM\_MAKE}

\ifenglish
Use the {\bf Make} command to compile all changed modules
in the current project and link it.

Keyboard shortcut: {\bf Shift+F9}
\else
Используйте команду {\bf Make}, чтобы компилировать все измененные модули
текущего проекта и связать их.

Комбинация клавиш: {\bf Shift+F9} 
\fi
\subsection{Tools: Build all}
\public{IDM\_BUILDALL}

\ifenglish
Use the {\bf Build all} command to rebuild the current project
completely.
\else
Используйте команду {\bf Build all}, чтобы полностью перестроить текущий 
проект.
\fi
\subsection{Tools: Update file list}
\public{IDM\_UPDATE}

\ifenglish
Use the {\bf Update file list} command to update the source
files list without performing any compile or link actions.
\else
Используйте команду {\bf Update file list}, чтобы обновлять 
список исходных файлов без компиляции и связывания.
\fi