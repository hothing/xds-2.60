\ifenglish
\section{Tools}
\else
\section{Утилиты}
\fi
\public{Tools}
\nominitoc

\ifenglish
{\em Tools} are external programs, such as an XDS compiler, which you
may quickly invoke from within the IDE. Use the \ref[Configure Tools]{CFGTOOLS\_DIALOG}
dialog that is displayed by selecting
\ref[Configure]{M\_CONFIGURE}:\ref[Tools]{IDM\_CONFIGURETOOLS} menu
to add and remove tools and to modify their settings.
\else
{\em Утилиты} -- это внешние программы, вроде XDS-компилятора, которые вы
можете быстро вызвать в IDE. Используйте диалог \ref[Configure Tools]{CFGTOOLS\_DIALOG},
который выводится посредством меню \ref[Configure]{M\_CONFIGURE}:\ref[Tools]{IDM\_CONFIGURETOOLS},
чтобы добавлять, удалять утилиты и менять их установки.
\fi
\ifenglish
Each tool have unique name that identifies it.
There are nine {\em predefined} tools which can not be removed:
{\bf Compile}, {\bf Compile\_NoProj}, {\bf Make}, {\bf Make\_NoProj},
{\bf Build}, {\bf Build\_NoProj}, {\bf MakeList}, {\bf MakeList\_NoProj},
and {\bf Debug}.

Tools may have items in \ref[Tools]{M\_TOOLS} menu and keyboard shortcuts
associated with them.
Predefined tools may also have toolbar buttons.
\else
Каждая утилита имеет уникальное имя, которое ее идентифицирует.
Девять {\em стандартных} утилит нельзя удалять: 
{\bf Compile}, {\bf Compile\_NoProj}, {\bf Make}, {\bf Make\_NoProj},
{\bf Build}, {\bf Build\_NoProj}, {\bf MakeList}, {\bf MakeList\_NoProj},
и {\bf Debug}.
\fi

\ifenglish
Tools may be configured to be enabled:
\else
Утилиты можно настроить так, что они будут действовать:
\fi
\begin{itemize}
\item \ifenglish
      permanently
      \else
      постоянно
      \fi
\item \ifenglish
      when some project is opened
      \else
      когда открыт какой-нибудь проект
      \fi
\item \ifenglish
      when current window is edit window
      \else
      когда текущее окно -- это окно редактирования
      \fi
\item \ifenglish
      when current window is edit window and file of specific type is loaded in it
      \else
      когда текущее окно -- это окно редактирования, и в него загружен файл 
      особенного типа
      \fi
\end{itemize}

\ifenglish
Tools are invoked on basis of {\bf Category} property. Category is a text string
defined for each tool. Several tools may share one category. When some tool
is invoked its category is first looked up. All tools with this category are
checked. The first enabled tool is actually invoked.

\else
Утилиты вызываются на основе свойства {\bf Category}. Категория -- это текстовая
строка, определенная для каждой утилиты. К одной категории можно относить несколько 
утилит. Когда запускается некоторая утилита, в первую очередь проверяется ее 
категория. Просматриваются все утилиты этой категории, запускается первая 
годная к употреблению. 
\fi
\ifenglish
Since tools are searched in top-down order it is reasonable to place the
most specific tool first and the most general last. For example, if tool {\bf A}
requires file with extension {\bf .mod}, tool {\bf B} works on files with
extensions {\bf .ob2}, tool {\bf C} on any files, tool {\bf D} doesn't require
file at all, just needs a project, and tool {\bf E} doesn't need even project,
it is reasonable to order tools like this:

\else
Поскольку поиск утилит происходит в нисходящем порядке, разумно помещать 
наиболее специфические утилиты первыми, а наиболее общие -- последними.
Например, если утилита {\bf A} требует файл с расширением {\bf .mod}, утилита
{\bf B} работает с файлами с расширениями{ \bf .ob2}, утилита {\bf C} -- 
с любыми файлами, утилите {\bf D} файлы вовсе не нужны, нужен только проект, 
а утилите {\bf E} не нужен даже проект, то имеет смысл упорядочить утилиты 
так:
\fi
 {\bf A B C D E F}

\ifenglish
or like this:
\else
или так:
\fi
 {\bf B A C D E F}

\ifenglish
Note that invocation of any tool with given category actually invokes the
first enabled tool. There is no way to invoke specific tool. If all the
tools in a category have keyboard shortcut, all the shortcuts will work
the same way (invoke first enabled tool).

The inclusion of tools into \ref[Tools]{M\_TOOLS} menu is also controlled by their
category. Each tool may have two strings associated with it: menu string for
enabled and disabled state. If any of the tools with given category is enabled
its enabled menu string (if it has any) is included into {\bf Tools} menu.
If no tool is enabled, first tool's disabled menu string (if it has any) is
included into {\bf Tools} menu in the grayed state. So at any given time no
more than one menu item exist for each category.

\else
Заметьте, что вызов любой утилиты данной категории на самом деле вызывает 
первую разрешенную к использованию утилиту. Способа запустить определенную утилиту не 
существует. Если всем утилитам в категории сопоставлена комбинация клавиш,
все комбинации будут действовать одинаково (запустят первую годную к 
употреблению утилиту).

Включение утилит в меню \ref[Tools]{M\_TOOLS} также управляется их категорией.
С каждой утилитой можно ассоциировать две строчки: строчку меню для 
действительного и недействительного состояний. Если какая-то из утилит данной
категории разрешена к использованию, в меню {\bf Tools} включается относящаяся к 
ней строчка дейсвительного состояния (если такая имеется). Если разрешенных к 
использованию утилит нет, то в меню {\bf Tools} включается относящаяся к 
первой утилите серая строчка недейсвительного состояния. Таким образом, в 
данный момент для каждой категории существует не более одного пункта меню.
\fi
\ifenglish
This procedure allows polymorphic menu. Consider three tools, {\bf Ob2Compiler},
{\bf Assembler} and {\bf Compiler}. Let {\bf Ob2Compiler} work on {\bf .ob2} files,
{\bf Assembler} on {\bf .asm} files and {\bf Compiler} on any files. Then we can assign
string

\verb'Compile module $(filename)'

to Ob2Compiler, string

\verb'Assemble $(file)'

to Assembler and

\verb'Compile file $(file)'

to Compiler. Now active menu item will show what exactly will happen if we
select it.

\else
Эта процедура допускает полиморфное меню. Возьмем три утилиты, {\bf Ob2Compiler},
{\bf Assembler} и {\bf Compiler}. Пусть {\bf Ob2Compiler} работает с файлами 
{\bf .ob2}, {\bf Assembler} с {\bf .asm} файлами, и {\bf Compiler} с любыми
файлами. Тогда мы можем сопоставить строчку

\verb'Compile module $(filename)'

утилите Ob2Compiler, строчку 

\verb'Assemble $(file)'

утилите Assembler и 

утилите Compiler. Теперь активный пункт меню покажет что в точности случится,
если его выбрать.
\fi
\ifenglish
Presence of two menu string may seem strange and unnecessary. But consider
{\bf Make} tool that has

\verb'Make $(projwholename)'

as active menu item and

\verb'Make current project'

as inactive menu item. When the project is not opened the menu item is
visible as {\bf Make current project} not as just {\bf Make}.

\else
Присутствие двух строчек меню может показаться странным и ненужным. Но 
рассмотрим утилиту {\bf Make} для которой 

\verb'Make $(projwholename)'

-- активный пункт меню, а 

\verb'Make current project'

-- неактивный пункт. Когда проект не открыт, этот пункт меню отображается
как {\bf Make current project} а не просто как  {\bf Make}.
\fi
\ifenglish
All standard tools (with exceptions of {\bf Debug}) have a companion that is
defined in the same way as original tool but doesn't require open project
to work. Tool search order leads to main tool being invoked when there is
a project opened and the companion to be invoked otherwise. The companion has
the same category as main tool and the same name with addition of suffix
{\bf \_NoProj}.

{\bf Debug} is specific tool. Most of the options are not applicable to debug tool.
You can't specify if it requires a project or a file. Debug tool is invoked from
\ref[Debug]{M\_DEBUG} menu, not from \ref[Tools]{M\_TOOLS} menu. But still
debugger is considered a tool and configured where all the tools are.

\else
Все стандартные утилиты (за исключением {\bf Debug}) имеют спутника, который 
определяется так же, как и первоначальная, но не требует для работы открытого 
проекта. Процедура поиска утилиты приводит к вызову главной утилиты, когда 
открыт проект, и к вызову спутника -- в противном случае. Спутник относится к 
той же категории, что и главная утилита, и имеет то же имя с добавлением 
суффикса {\bf \_NoProj}.

{\bf Debug} -- особенная утилита. Большинство опций неприменимо к утилите отладки. 
Нельзя указать, нуждается ли она в проекте или в файле. Утилита отладки 
вызывается посредством меню \ref[Debug]{M\_DEBUG}, а не \ref[Tools]{M\_TOOLS}.
Но все же отладчик считается утилитой и настраивается там же, где все утилиты.
\fi
\ifenglish
For each tool you can configure starting directory. For most tools it is usually
project directory, for tools that run without project - current file directory.

You may use \ref{variables} when defining tools command lines and directories.
That allows these to depend on current project and file.

\else
Для каждой утилиты вы можете указывать директорию запуска. Для большинства утилит
это обычно директория проекта, для утилит, запускаемых без проекта, -- директория
текущего файла. 

При оределении командных строк и директорий утилит вы можете использовать 
\ref[переменные]{variables}. Это позволяет делать их зависимыми от текущего 
проекта и файла.
\fi
\ifenglish
Tools can be run in two modes: in their own window and in pop-up window. The latter
is reasonable if the tool has the ability to send progress information to the IDE.
The interface is provided ({\bf xShell} module) that allows all custom-written
tools to send error messages to the shell and to display progress on
the progress indicator.

If a tool doesn't use the interface it still can supply IDE with error and warning messages.
That requires parsing of its standard output. \ref{MessageFilters} are introduced for
this purpose.

\else
Утилиты могут запускаться в двух режимах: в своем собственном окне и во всплывающем.
Последний режим имеет смысл если утилита способна передавать IDE сведения о
ходе работы. Предоставляется интерфейс (модуль {\bf xShell}), который позволяет
всем заказным утилитам посылать оболочке сообщения об ошибках и отображать 
сведения о ходе работы.

Если утилита не использует этот интерфейс, она все же может передавать IDE сообщения
об ошибках и предупреждениях. Для этого требуется проанализировать ее стандартный
вывод. Для этой цели служат \ref[фильтры сообщений]{MessageFilters}.
\fi