\ifenglish
\section{Tabulation modes}
\else
\section{Режимы табуляции}
\fi
\public {Tabulation}

\ifenglish
There are three modes of tabulation:
\else
Существует три режима табуляции:
\fi
\begin{description}
\item[{\bf Hard}]
\ifenglish
{\bf Tab} key enters the {\bf Tab Character} (0x09) which
moves cursor to the next tab stop
\else
Клавиша {\bf Tab} вводит {\bf символ табуляции} (0x09), который передвигает
курсор к следующей позиции табуляции.
\fi
\item[{\bf Spaces}]
\ifenglish
Tabulation is immediately expanded by spaces until next
tab stop
\else
Табуляция немедленно расширяется пробелами вплоть до следующей позиции 
табуляции.
\fi
\item[{\bf Smart}]
\ifenglish
Spaces are inserted until cursor positions under 
beginning of the next word in previous line (more precisely, under first non-whitespace
after first white-space after current cursor position). This mode comes from
TopSpeed editor.
\else
Пробелы вставляются до тех пор, пока курсор не окажется под 
началом следующего слова в предыдущей строке (точнее, под первым отличающимся от
пробела символом после первого пробела после текущей позиции курсора). Этот режим
происходит из редактора TopSpeed.
\fi
\end{description}
\ifenglish
The tabulation mode and tab-size can be selected in the
\ref[Tabs and Indents]{CFGEDIT\_TAB\_DIALOG} page of
\ref[Configure Editor]{CFGEDIT\_DIALOG} dialog 
that appears after \ref[Configure]{M\_CONFIGURE}:\ref[Editor]{IDM\_CONFIGUREEDITOR} menu.

Note that {\bf Tab} key is sensitive to insert/overwrite mode. In overwrite mode
it overwrites either single character with tab character, or several characters
with spaces.
\else
Режим и размер табуляции могут выбираться на закладке \ref[Tabs and Indents]{CFGEDIT\_TAB\_DIALOG} 
диалога \ref[Configure Editor]{CFGEDIT\_DIALOG}, который появляется после меню
\ref[Configure]{M\_CONFIGURE}:\ref[Editor]{IDM\_CONFIGUREEDITOR}.

Заметьте что клавиша {\bf Tab} чувствительна к режиму вставки/перезаписи.
В режиме перезаписи она либо заменяет один символ на символ табуляции, либо 
несколько символов -- на пробелы. 
\fi