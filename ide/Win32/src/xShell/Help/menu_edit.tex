\section{Edit menu}
\public{M\_EDIT}
\nominitoc

\ifenglish
The {\bf Edit} menu contains commands that you use to undo
changes you've made to file, to work with Clipboard
and to find and replace text.
The following commands appear in the
{\bf Edit} menu:
\else
Меню {\bf Edit} содержит команды, которые вы используете, чтобы отменять 
произведенные вами в файле изменения, работать с буфером обмена
и искать и заменять текст. 
Меню {\bf Edit} содержит следующие команды:
\fi
\begin{description}
\item \ref[Undo]{IDM\_UNDO} -
  \ifenglish
  Undoes the last edit action
  \else
  Отменяет последнее изменение 
  \fi
\item \ref[Cut]{IDM\_CUT} -
  \ifenglish
  Cuts selected text to the clipboard
  \else
  Перемещает отмеченный текст в буфер обмена
  \fi
\item \ref[Copy]{IDM\_COPY} -
  \ifenglish
  Copies selected text to the clipboard
  \else
  Копирует отмеченный текст в буфер обмена
  \fi
\item \ref[Add]{IDM\_ADD} -
  \ifenglish
  Adds selected text to the clipboard
  \else
  Добавляет отмеченный текст в буфер обмена
  \fi
\item \ref[Exchange]{IDM\_EXCHANGE} -
 \ifenglish
  Exchanges selected text with clipboard contents
  \else
  Меняет отмеченный текст на содержимое буфера обмена
 \fi
\item \ref[Paste]{IDM\_PASTE} -
 \ifenglish
  Inserts contents of the clipboard at the current position
  \else
  Вставляет содержимое буфера обмена на текущую позицию
  \fi
\item \ref[Find]{IDM\_FIND} -
  \ifenglish
  Searches for a string
  \else
  Ищет строку
  \fi
\item \ref[Find Next]{IDM\_FINDNEXT} -
 \ifenglish
  Searches for the next occurrence of last searched string
  \else
  Ищет следующее вхождение последней искомой строки
  \fi
\item \ref[Find Previous]{IDM\_FINDPREV} -
 \ifenglish
  Searches for the previous occurrence of last searched string
  \else
  Ищет предыдущее вхождение последней искомой строки
 \fi
\item \ref[Replace]{IDM\_REPLACE} -
 \ifenglish
 Searches for a string, replacing it with another one
 \else
  Ищет строку, заменяя ее на другую
 \fi
\item \ref[Replace Next]{IDM\_REPLACENEXT} -
 \ifenglish
  Repeats previous replace action
 \else
 Повторяет последнюю операцию замены
 \fi
\item \ref[Find in Files]{IDM\_FINDFILES}
 \ifenglish
  Searches for the string in multiple files
 \else
  Ищет строку во множестве файлов
 \fi
\item \ref[Go to line]{IDM\_GOTOLINE}
 \ifenglish
  Moves cursor to the line with specified number
 \else
  Перемещает курсор на строку с указанным номером
 \fi
\end{description}

\subsection{Edit: Undo}
\public{IDM\_UNDO}

\ifenglish
Use the {\bf Undo} command to undo the last change you've made to
the current file.
\else
Используйте команду {\bf Undo}, чтобы отменить последнее изменение, 
произведенное вами в текущем файле
\fi
\ifenglish
For this to work, for your changes the editor saves information necessary to
revert them into so-called undo buffer.
This buffer has limited size; when the buffer is full
you are prompted to abandon current change. If you refuse, the editor will
automatically drop earliest records from undo buffer in order to make room for
new records. This leads to inability to undo all the actions.

Casual editing operation such as typing of new text don't use much room in
undo buffer and so are unlikely to fill it. So you never get into mentioned
situation unless you frequently delete large blocks of text.

The undo buffer is cleared when you save file to disk, so you can't
undo changes done before saving.

Keyboard shortcuts: {\bf Alt+Backspace}, {\bf Ctrl+Z}
\else
Чтобы это срабатывало, редактор сохраняет необходимую информацию о ваших
изменениях в так называемом буфере отмены.
Этот буфер имеет ограниченный размер, когда он заполнится, вас попросят
отказаться от текущего изменения. Если вы откажетесь, редактор автоматически
выбросит более ранние сведения из буфера для того, чтобы создать место для
новых записей. Это приводит к невозможности отменить все изменения.

Нерегулярные операции редактирования, такие, как ввод нового текста, не 
занимают много места в буфере отмены, и потому вряд ли его переполнят.
Таким образом, вы никогда ни попадете в упомянутую ситуацию, если не 
удаляете часто большие участки текста.

Когда вы сохраняете файл на диск, буфер отмены очищается, поэтому вы не
можете отменить произведенные до записи изменения.
\fi
\subsection{Edit: Cut}
\public{IDM\_CUT}

\ifenglish
Use the {\bf Cut} command to delete selected text from the
file and move it to the Clipboard. You can use the \ref[Paste]{IDM\_PASTE}
command later to insert the Clipboard contents at another position
or even into another file. You can also insert the text to other applications.

Keyboard shortcuts: {\bf Shift+Delete}, {\bf Ctrl+X}
\else
Используйте команду {\bf Cut}, чтобы удалять отмеченный текст из файла и 
помещать его в буфер обмена. Затем вы можете использовать команду
\ref[Paste]{IDM\_PASTE}, чтобы вставить содержимое буфера обмена на
другую позицию или даже в другой файл. Вы также можете вставить текст
в другие приложения.         

Комбинации клавиш: {\bf Shift+Delete}, {\bf Ctrl+X}
\fi
\subsection{Edit: Copy}
\public{IDM\_COPY}

\ifenglish
Use the {\bf Copy} command to duplicate selected text. A copy of
selected text is moved to the Clipboard. You can use the \ref[Paste]{IDM\_PASTE}
command later to insert the Clipboard contents at another position
or even into another file.  You can also insert the text to other applications.

Keyboard shortcuts: {\bf Ctrl+Insert}, {\bf Ctrl+C}
\else
Используйте команду {\bf Copy}, чтобы дублировать отмеченный текст.
Копия отмеченного текста помещается в буфер обмена. Затем вы можете 
использовать команду \ref[Paste]{IDM\_PASTE}, чтобы вставить содержимое 
буфера обмена на другую позицию или даже в другой файл. Вы также можете 
вставить текст в другие приложения.         

Комбинации клавиш: {\bf Ctrl+Insert}, {\bf Ctrl+C}
\fi
\subsection{Edit: Add}
\public{IDM\_ADD}

\ifenglish
Use the {\bf Add} command to combine several portions of text together.
This command appends selected text to the Clipboard contents.
If the Clipboard is empty or doesn't contain text, the command works
exactly as \ref[Copy]{IDM\_COPY} command.
You can use the {\bf Paste}
command later to insert the resulting text into desired place.
\else                                    
Чтобы сочетать несколько участков текста, используйте команду {\bf Add}.
Эта команда добавляет отмеченный текст к содержимому буфера обмена.
Если буфер обмена пуст или не содержит текста, команда действует в точности
как и команда \ref[Copy]{IDM\_COPY}.
Затем вы можете использовать команду {\bf Paste},
чтобы вставить полученный текст на нужное место.
\fi
\ifenglish
Keyboard shortcuts: {\bf Ctrl+Shift+Insert}, {\bf Ctrl+A}

Be careful when pressing {\bf Ctrl+Shift+Insert}: if you release {\bf Shift} key
the command will be recognized as \ref[Copy]{IDM\_COPY} command and contents
of the Clipboard will be lost.
\else
Комбинации клавиш: {\bf Ctrl+Shift+Insert}, {\bf Ctrl+A}

Будьте внимательны, когда нажимаете {\bf Ctrl+Shift+Insert}, если вы отпустите 
клавишу {\bf Shift}, команда будет воспринята как \ref[Copy]{IDM\_COPY}, и 
содержимое буфера обмена будет утеряно.
\fi
\subsection{Edit: Paste}
\public{IDM\_PASTE}

\ifenglish
Use the {\bf Paste} command to insert the text from the Clipboard
at the cursor position in the current file.

Text appears in the Clipboard as a result of \ref[Cut]{IDM\_CUT}, \ref[Copy]{IDM\_COPY},
\ref[Add]{IDM\_ADD}, \ref[Exchange]{IDM\_EXCHANGE} operations, or it can be copied
from another application.

Keyboard shortcuts: {\bf Shift+Insert}, {\bf Ctrl+V}
\else
Используйте команду {\bf Paste}, чтобы вставлять текст из буфера обмена
на позицию курсора в текущем файле.

Появляющийся в буфере обмена текст есть результат операций \ref[Cut]{IDM\_CUT}, 
\ref[Copy]{IDM\_COPY}, \ref[Add]{IDM\_ADD}, \ref[Exchange]{IDM\_EXCHANGE} или
может быть пренесен из других приложений.

Комбинации клавиш: {\bf Shift+Insert}, {\bf Ctrl+V}
\fi
\subsection{Edit: Exchange}
\public{IDM\_EXCHANGE}

\ifenglish
Use the {\bf Exchange} command to exchange contents of the Clipboard
with the current selection.

The command is only available when Clipboard contain text.

This command is useful for exchanging two strings in a file. You select one
string, \ref[Cut]{IDM\_CUT} it to the Clipboard, then select another string,
{\bf Exchange} it with the Clipboard, then return to a place where first string
once was, and \ref[Paste]{IDM\_PASTE} the Clipboard.

Keyboard shortcuts: {\bf Ctrl+Shift+Delete}, {\bf Ctrl+E}

Be careful when pressing {\bf Ctrl+Shift+Delete}: if you release {\bf Ctrl} key
the command will be recognized as \ref[Cut]{IDM\_CUT} command and contents
of the Clipboard will be lost.
\else
Используйте команду {\bf Exchange}, чтобы обменять содержимое буфера обмена и 
отмеченный в данный момент текст.

Команда доступна только в случае, если буфер обмена содержит текст.

Эта команда полезна при обмене двух строк в файле. Вы отмечаете одну строку
командой \ref[Cut]{IDM\_CUT}, перемещаете ее в буфер обмена, затем отмечаете
другую строку командой {\bf Exchange}, меняете ее на содержимое буфера, затем
возвращаетесь на место, где прежде находилась первая строка, и используете
команду \ref[Paste]{IDM\_PASTE}.

Комбинации клавиш: {\bf Ctrl+Shift+Delete}, {\bf Ctrl+E}.

Будьте внимательны, когда нажимаете {\bf Ctrl+Shift+Delete}, если вы отпустите 
клавишу {\bf Ctrl}, команда будет воспринята как \ref[Cut]{IDM\_CUT}, и 
содержимое буфера обмена будет утеряно.
\fi
\subsection{Edit: Find}
\public{IDM\_FIND}

\ifenglish
Use the {\bf Find} command to locate occurrences of a string in
the current file. A dialog box will appear, allowing you to
specify the string to find and various options such as case sensitiveness,
search direction and others.

Keyboard shortcut: {\bf Ctrl+F}
\else
Используйте команду {\bf Find}, чтобы обнаружить вхождения строки в текущем 
файле. Будет выведен диалог, позволяющий вам указать искомую строку и различные
опции, такие как чувствительность к регистру, направление поиска и другие.

Комбинация клавиш: {\bf Ctrl+F}
\fi
\subsection{Edit: Find Next}
\public{IDM\_FINDNEXT}

\ifenglish
Use the {\bf Find Next} command to find the next occurrence
of the current search string.

Note that term "next" here is relative to search direction: if last search was
backwards this command will find previous occurrence of a text.

Keyboard shortcut: {\bf Ctrl+L}
\else
Чтобы найти следующее вхождение текущей искомой строки, используйте команду
{\bf Find Next}.

Заметьте, что термин "следующее" относится здесь к направлению поиска: если
последний поиск велся в обратном направлении: эта команда найдет предыдущее 
вхождение текста.

Комбинация клавиш: {\bf Ctrl+L}
\fi
\subsection{Edit: Find Previous}
\public{IDM\_FINDPREV} {\bf Ctrl+L}

\ifenglish
Use the {\bf Find Previous} command to find the previous occurrence
of the current search string.

Note that term "previous" here is relative to search direction: if last search was
backwards this command will find next occurrence of a text.

Keyboard shortcut: {\bf Ctrl+Shift+L}
\else
Для того, чтобы найти предыдущее вхождение текущей искомой строки, 
используйте команду {\bf Find Previous}. 

Заметьте, что термин "предыдущее" относится здесь к направлению поиска, если
последний поиск велся в обратном направлении, эта команда найдет следующее
вхождение текста.

Комбинация клавиш: {\bf Ctrl+Shift+L}
\fi
\subsection{Edit: Replace}
\public{IDM\_REPLACE}

\ifenglish
Use the {\bf Replace} command to locate occurrences of a string in
the current file, replacing it with another string. 
A dialog box will appear, allowing you to
specify the search string, the replacement string, and various options such as
operation scope, case-sensetiveness and others.

Keyboard shortcut: {\bf Ctrl+H}
\else
Используйте команду {\bf Replace}, чтобы обнаруживать вхождения строки
в текущем файле, заменяя их на другую строку.
Будет выведен диалог, позволяющий указать искомую строку, строку замены и 
различные опции, такие как рамки операции, чувствительность к регистру и другие.

Комбинация клавиш: {\bf Ctrl+H}
\fi
\subsection{Edit: Replace Next}
\public{IDM\_REPLACENEXT}

\ifenglish
Use the {\bf Replace Next} command to repeat the previous replace
action.
\else
Используйте команду {\bf Replace Next}, чтобы повторить предыдущую замену.
\fi
\subsection{Edit: Find in Files}
\public{IDM\_FINDFILES}

\ifenglish
Use the {\bf Find in Files} command to search a string in several files.
You can search for files in open windows, in project files or in files in
specified directory on disk.
A dialog box will appear, allowing you to
specify the search string, and various options such as search scope and
case-sensetiveness.

Keyboard shortcut: {\bf Ctrl+Shift+F}

\else
Используйте команду {\bf Find in Files}, чтобы производить поиск в нескольких
файлах. Вы можете производить поиск в файлах открытых окон, файлах проекта
или файлах в указанной директории или диске.
Будет выведен диалог, позволяющий указать искомую строку и различные опции,
вроде рамок поиска и чувствительности к регистру.

Комбинация клавиш: {\bf Ctrl+Shift+F}
\fi
\subsection{Edit: Go to Line}
\public{IDM\_GOTOLINE}

\ifenglish
Use the {\bf Go to Line} command to quickly move the cursor
to the line with a specific number.

Keyboard shortcut: {\bf Ctrl+G}
\else
Чтобы быстро перемещать курсор на строку с указанным номером, 
используйте команду {\bf Go to Line}.

Комбинация клавиш: {\bf Ctrl+G}
\fi