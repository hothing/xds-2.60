\ifenglish
\section {Configure Editor dialog}
\else
\section {диалог Configure Editor}
\fi
\public{CFGEDIT\_DIALOG}
\nominitoc

\ifenglish
XDS IDE uses its own text editor called {\bf PoorEdit} for its relatively
small set of functions and modest nature of its developers who will never
call something {\bf Rich} unless it is really rich.

The editor is not configurable in general sense of a word. The keyboard
can't be configured now and the editor doesn't have any macro language.
The editor behaves in a way standard for Windows in most cases. However,
some of its properties can be tuned to make it resemble some text editors
from other systems. This can be done in this dialog.

\else
XDS IDE использует собственный текстовый редактор, называемый {\bf PoorEdit}
за его относительно малый набор функций, и по причине природной скромности
его разработчиков, которые никогда не назовут нечто {\bf богатым}, если
оно не богато в действительности.

Редактор не является настраиваемым в общем смысле слова. На настоящий момент
нельзя настраивать клавиатуру, и в редакторе нет макроязыка Редактор ведет
себя обычным во многих случаях для Windows образом. Тем не менее,
некоторые из его свойств могут настраиваться таким образом, чтобы редактор
походил на редакторы из других систем. Это может быть сделано в этом диалоге.
\fi

\ifenglish
This dialog contains following pages:
\else
Этот диалог содержит следующие страницы:
\fi
\ref[Tabs and Indents]{CFGEDIT\_TAB\_DIALOG} 
\ifenglish
-- set tabulation size and mode and auto-indent mode
\else
-- установка режима и размера табуляции и режима авто-отступа.
\fi
\ref[Features]{CFGEDIT\_FEATURES\_DIALOG} 
\ifenglish
-- fine tune subtle details of editor commands
\else
-- тонкая настройка команд редактора.
\fi
\ref[Blocks]{CFGEDIT\_BLOCK\_DIALOG} 
\ifenglish
-- customize operations with text blocks
\else
-- операции кастомизации с текстовыми блоками.
\fi
\ref[Show]{CFGEDIT\_SHOW\_DIALOG} 
\ifenglish
-- define whether various elements of the editor are visible
\else
-- определить, видимы ли различные элементы редактора.
\fi
\ref[Encoding]{CFGEDIT\_ENCODING\_DIALOG} 
\ifenglish
-- select text encoding, end-of-file and end-of-line style
\else
-- выбор кодировки текста, вида конца файла и строки.
\fi
\ref[Files]{CFGEDIT\_FILES\_DIALOG} 
\ifenglish
-- configure how files are saved
\else
-- настройка сохранения файлов на диск.
\fi
%---------------------------------------------------------------------
\ifenglish
\subsection{Tabs and Indents}
\else
\subsection{Табуляции и отступы}
\fi
\public{CFGEDIT\_TAB\_DIALOG}

\ifenglish
This page allows to select \ref[tabulation mode]{Tabulation} to use in {\bf Tab style} group.

\else
Эта страница позволяет выбирать \ref[режим табуляции]{Tabulation}
 для использования в группе {\bf Tab style}.
\fi
\ifenglish
It also allows to select {\bf tab size} that determines where tab stops are
situated in the text. Tab stops are situated at positions whose column numbers
after subtracting 1 are divisible by tab size.

Normally pressing {\bf Enter} key invokes automatic indentation (inserting spaces
at the beginning of a new line). This option can be turned off and on by
{\bf Autoindent} checkbox.
\else
Она также позволяет выбирать {\bf размер табуляции}, который определяет, где
располагаются в тексте позиции табуляции. Позиции табуляции разполагаются там,
где номер столбца с вычтенной единицей нацело делится на размер табуляции.

Обычно нажатие клавиши {\bf Enter} приводит к автоматическому отступу 
(вставке пробелов в начале новой строки). Эта опция может переключаться с
помощью контроля {\bf Autoindent}.
\fi
%---------------------------------------------------------------------
\ifenglish
\subsection{Features}
\else
\subsection{Свойства}
\fi
\public{CFGEDIT\_FEATURES\_DIALOG}

\ifenglish
This page allows to configure various properties of editor commands such as
\else
Эта страница позволяет настраивать различные свойства команд редактора, такие
как
\fi
\begin{itemize}
\item \ifenglish
      behavior of {\bf BackSpace} key at the beginning of a line (whether it joins lines or not)
     \else
      поведение клавиши {\bf BackSpace} в начале строки (соединяет строки или нет)
     \fi
\item \ifenglish
      behavior of {\bf Delete} key at the end of line (whether it joins lines or not)
    \else 
      поведение клавиши {\bf Delete} в конце строки (соединяет строки или нет)
      \fi
\item \ifenglish
      whether {\bf Enter} key in insert mode splits line or just moves to beginning of
      the next line
      \else
      разбивает ли клавиша {\bf Enter} строку в режиме вставки, или лишь
      передвигает курсор к началу следующей строки
      \fi
\item \ifenglish
      whether {\bf Tab} key in overwrite mode should erase text or just move cursor
     \else
      должна ли клавиша {\bf Tab} в режиме перезаписи стирать текст или лишь
       сдвигать курсор
      \fi
\item \ifenglish
      the state of Insert/Overwrite mode of newly created editor
      \else
      состояние Вставки/Перезаписи вновь созданного редактора
      \fi
\end{itemize}

%---------------------------------------------------------------------
\ifenglish
\subsection{Blocks}
\else
\subsection{Блоки}
\fi
\public{CFGEDIT\_BLOCK\_DIALOG}

\ifenglish
This page provides means for configuration of editor's handling of selected blocks.

Following features can be configured:

\else
Эта страница предоставляет средства для настройки работы редактора с блоками
отметки

Могут быть настроены следующие опции:
\fi
\ifenglish
{\bf Paste when block is selected} group
-- what happens to existing and inserted block after
\ref[Paste]{IDM\_PASTE} command.
Three options are available: new block is
inserted before the block, after the block or instead of the block.
\else
Группа {\bf Paste when block is selected}
-- что происходит с существующим и вставляемым блоком после команды
\ref[Paste]{IDM\_PASTE}.
\fi
\ifenglish
{\bf Cursor after Paste operation} group
-- whether newly pasted block should be selected or, if no, where the cursor should
be positioned -- before or after the block.
\else
Группа {\bf Cursor after Paste operation}
-- должен ли вклеенный блок отмечаться, или, если нет, то где размещать курсор --
в начале или в конце блока.
\fi
\ifenglish
{\bf Typing when block is selected}
-- what happens to the block when new text is typed over it. New text can be inserted
before the block, after the block or instead of the block.
\else
{\bf Набор текста, когда блок отмечен}
-- что происходит с блоком, когда поверх него пишется новый текст. Новый текст
может вставляться перед блоком, после блока или на место блока.
\fi
\ifenglish
Note that {\bf Paste} operation is not affected by Insert/Overwrite mode.
It never overwrites text unless the text is selected and Paste command is
configured to replace block.

\else
Заметьте, что на операцию {\bf Paste} не воздействует режим Вставки/Перезаписи
Текст никогда не перезаписывается, если только текст не отмечен, и команда
Paste не настроена на замещение блока.
\fi
%---------------------------------------------------------------------
\ifenglish
\subsection{Show}
\else
\subsection{Отображение}
\fi
\public{CFGEDIT\_SHOW\_DIALOG}
\ifenglish
Editor window may have some decorations in and around it.
This page allows to select what elements to display or to hide:
\else
Окно редактирования может быть определенным образом приукрашено.
Эта страница позволяет выбрать, какие элементы вывести а какие спрятать.
\fi
\begin{itemize}
\item \ifenglish
      Hard tabs
      \else
      Жесткая табуляция
      \fi
\item \ifenglish
      Left margin
      \else
      Левые поля
      \fi
\item \ifenglish
      Right margin
      \else
      Правые поля
      \fi
\item \ifenglish
      Bookmarks
      \else
      Пометки
      \fi
\end{itemize}

\ifenglish
It also allows configuration of right margin position. Note that left margin
is physical delimiter of areas where you can type and where you can not. Right margin,
however, is just a reminder for a user that some lines might be too long.
There is no physical limit of line length.

When bookmarks display is turned on left margin becomes bigger. Bookmarks are
drawn as circles with bookmark number inside.

Margins are drawn as dotted lines, tabs as small vertical dashed line in tab
position and horizontal dashed line spanning to the next tab stop.
\else
Она также позволяет настраивать положение правого поля. Заметьте, что левые поля
являются физическим ограничителем области, где вы можете набирать текст. Правые
поля, однако, -- лишь напоминание пользователю о том, что некоторые строчки могут
оказаться слишком длинными. Физического ограничения на длину строки не существует.

Когда включено отображение пометок, левые поля становятся больше. Пометки рисуются
в виде кружков с номером метки внутри.

Поля отображаются как точечный пунктир, табуляция -- в виде небольшой вертикальной
пунктирной линии в позиции табуляции, и горизонтальной пунктирной линии вплоть до
следующей позиции табуляции.
\fi
%---------------------------------------------------------------------
\ifenglish
\subsection{Encoding}
\else
\subsection{Кодировка}
\fi
\public{CFGEDIT\_ENCODING\_DIALOG}
\ifenglish
This page allows configuration of text encoding. Following parameters can be
set:

{\bf OEM conversion} checkbox defines whether edited text is in OEM or Windows encoding.
For some national character sets such as Cyrillic these encoding really differs,
and some developers prefer texts to be in OEM encoding to work with it in console mode.
The OEM conversion mode invokes text conversion during display and typing and
affects Clipboard operations.

OEM characters also appear when user wishes to use pseudo-graphic characters.
They'll not be visible in the editor anyway, unless OEM font is used, but when
OEM encoding is turned on, something reasonable will be displayed on their place
instead of complete garbage in case it is turned off.

\else
Эта страница позволяет настраивать кодировку текста. Могут быть установлены
следующие параметры.

Контроль {\bf OEM conversion} определяет, принадлежит ли редактируемый текст
кодировке Windows, или OEM. Для некоторых национальных наборов символов, таких
как Кириллица, эти кодировки различаются, и некоторые разработчики предпочитают
тексты в OEM кодировке для работы в консольном режиме.
Режим OEM conversion преобразует текст во время его отображения и набора, и влияет
на операции с буфером обмена.

Символы OEM появляются также и когда пользователь желает воспользоваться символами
псевдо-графики. В любом случае, если в редакторе не используется шрифт OEM, они
видимыми не будут, но если включена OEM кодировка, то на их месте отобразится нечто
разумное, вместо совершенного хлама в случае, если кодировка отключена.
\fi
\ifenglish
Note that text is stored internally in its original form. It is translated during display
rather than during load/save operation. That means that even if some OEM characters
are not present in Windows character set and vice versa, it is still safe to load
OEM text in the IDE without any risk of information loss.

{\bf End of line style} group defines characters used as line separators when
files are written to disk. When files are read the IDE recognizes many character
sequences as line separators: {\bf CR}, {\bf LF}, {\bf CRLF}, and {\bf NUL}
(zero-code character). When it writes file back it replaces all of them with
one line separator defined here. Three options are available: {\bf CR}, {\bf LF}
and {\bf CR LF}.
\else
Заметьте, что текст внутри хранится в исходной форме. Он перекодируется скорее во
время отображения, чем во время операций загрузки/сохранения. Это означает, что
если даже некоторые OEM символы не присутствуют в наборе символов Windows,
и наоборот, то все еще можно спокойно загружать в IDE OEM-текст без риска
потери информации.
\fi
\ifenglish
{\bf Autodetect EOL style} checkbox turns on automatic detection of line separators
used in files read from disk. When file is read, the editor remembers the line
separator that is used in it most frequently. This line separator is used when file
is saved back to disk. This option is very useful if you work with both files that
you access from UNIX computers by a network and local files at the same time.
Turning this option on does not disable End-of-line style switch: in this case
it defines EOL style for newly created file.

{\bf Consider \^Z as EOF} checkbox instruct the editor to stop reading file when
{\bf Ctrl+Z (0x1A)} character is encountered. It was common to write this
character at the end of file in old times, and some systems still do it.

{\bf Write \^Z at EOF} checkbox tells the editor to behave like mentioned
old times systems.
\else
Контроль {\bf Autodetect EOL style} включает автоматическое выявление разделителей
строк, используемых в считываемых с диска файлах. Когда читается файл, редактор
запоминает разделитель строки, который используется в нем наиболее часто. Этот
разделитель используется, когда файл снова сохраняется на диск. Эта опция очень
полезна, когда вы работаете одновременно как с файлами, доступ к которым вы
получаете через сеть с UNIX компьютеров, так и с локальными файлами. Включение
этой опции не выключает работу опции стиля конца строки:  в этом случае она
определяет стиль EOL для вновь созданного файла.

Контроль {\bf Consider \^Z as EOF} заставляет редактор прекращать считывание файла,
когда встречается символ {\bf Ctrl+Z (0x1A)}. Раньше было принято писать этот
символ в конце файла, и некоторые системы до сих пор так и делают.

Контроль {\bf Write \^Z at EOF} заставляет редактор вести себя подобно упомянутым
старомодным системам.
\fi
%---------------------------------------------------------------------
\ifenglish
\subsection{Files}
\else
\subsection{Файлы}
\fi
\public{CFGEDIT\_FILES\_DIALOG}

\ifenglish
In this page you can configure how and when files are saved. Two options are
currently available:

{\bf Autosave} -- when window with file is closed, IDE is exited or external tool
is started, system saves all changed files to the disk. Before saving each file the
systems asks if this file should be saved. Turning this option off tell system
to save files without asking.

{\bf Create BAK files} -- tell system to keep previous copy of file when file is saved.
The name of backup copy depends on file system where the file resides.
If it allows long file names the {\bf .bak} suffix is added to file name.
Otherwise, suffix {\bf .bak} replaces existing suffix.

\else
На этой странице вы можете настроить когда и как будут сохранятся файлы.
Доступны две опции:

{\bf Autosave} -- когда закрывается окно, содержащее файл, происходит выход из
IDE, или запускается внешняя утилита, то система сохраняет все измененные файлы на
диск. Перед сохранением каждого файла система спрашивает, нужно ли его сохранять.
Включение этой опции означает сохранение без запроса.

{\bf Create BAK files} -- заставляет систему сохранять предыдущую копию файла,
когда файл сохраняется. Название резервной копии зависит от файловой системы.
Если она допускает длинные имена, то к имени файла добавляется суффикс {\bf .bak},
иначе {\bf .bak} заменяет существуюий суффикс.
\fi
%---------------------------------------------------------------------

\begin{popup}
\ifenglish
\caption{Hard}
\else
\caption{Жесткая табуляция}
\fi
\public{CFGEDIT\_TAB\_HARD}
\ifenglish
Horizontal tab character (0x09) is inserted in text each time TAB key is pressed.
This character is expanded into several spaces when displaying on the screen or
printing. You may make these characters visible by selecting {\bf Hard Tabs} checkbox
in {\bf Show} page of this dialog.

\else
Символ горизонтальной табуляции (0x09) вставляется в текст всякий раз, когда
нажимается клавиша TAB. Этот символ при отображении на экране или печати означает
несколько пробелов. Вы можете сделать эти символы видимыми, включив контроль
{\bf Hard Tabs} страницы {\bf Show} этого диалога.
\fi
\end{popup}

\begin{popup}
\ifenglish
\caption{Spaces}
\else
\caption{Пробелы}
\fi
\public{CFGEDIT\_TAB\_SPACES}
\ifenglish
TAB characters are expanded into spaces immediately as they are typed. Exact number
of spaces is determined by {\bf Tab size} parameter.

\else
Символы TAB заменяются пробелами сразу же, как они вводятся. Точное количество
пробелов определяется параметром {\bf Tab size}.
\fi
\end{popup}

\begin{popup}
\caption{Smart}
\public{CFGEDIT\_TAB\_FLEXIBLE}
\ifenglish
This tabulation style appeared initially in Top Speed system.
Spaces are inserted until cursor positions under beginning of the next
word in previous line. If cursor is beyond the length of previous line this
mode works as {\bf Spaces} mode.

\else
Этот стиль табуляции появился впервые в системе Top Speed.
Пробелы вставляются до тех пор, пока курсор не окажется под началом следующего
слова предыдущей строки. Если курсор находится дальше конца предыдущей строки,
то этот режим работает, как режим {\bf Spaces}.
\fi
\end{popup}

\begin{popup}
\ifenglish
\caption{Tab size}
\else
\caption{Размер табуляции}
\fi
\public{CFGEDIT\_TAB\_SIZE}
\ifenglish
This field determines size of displaying or expanding tabs. Cursor moves to the
next point whose column number minus 1 is divisible by this number.

Valid range for this field is [1..20]

\else
Это поле определяет размер раширения символов табуляции. Курсор передвигается к
следующей позиции, чей номер столбца с вычтенной едиицей делится нацело на это
число.

Допустимые значения для этого поля -- [1..20]
\fi
\end{popup}

\begin{popup}
\ifenglish
\caption{Autoindent}
\else
\caption{Автоотступ}
\fi
\public{CFGEDIT\_TAB\_AUTOINDENT}
\ifenglish
If this option is on each command that splits line insert as many spaces in the
beginning of a line as there are in the beginning of previous line.

If hard tabs mode is used tabs are inserted until remaining number of spaces is
less than tab size. Spaces are added then.

\else
Если эта опция включена, то каждая разбивающая строку команда вставляет столько
пробелов в начало новой строки, сколько их было в начале предыдущей.

Если используется режим жесткой табуляции то до тех пор, пока число остающихся
пробелов не станет меньше рамера табуляции, вставляются символы табуляции.
\fi
\end{popup}
%---------------------------------------------------------------------

\begin{popup}
\ifenglish
\caption{BackSpace joins lines}
\else
\caption{BackSpace соединяет строки}
\fi
\public{CFGEDIT\_FEATURES\_BSJOINS}

\ifenglish
{\bf BackSpace} key pressed at the first column deletes previous line break
and joins current line with previous one if this option is turned on, does nothing
otherwise.

\else
Клавиша {\bf BackSpace}, будучи нажатой когда курсор находится на первом столбце,
удаляет предыдущий разделитель строк и присоединяет текущую строку к предыдущей,
если эта опция включена, и ничего не делает в противном случае.
\fi
\end{popup}

\begin{popup}
\ifenglish
\caption{DEL joins lines}
\else
\caption{DEL соединяет строки}
\fi
\public{CFGEDIT\_FEATURES\_DEL\_JOINS}
\ifenglish
{\bf Delete} key pressed beyond the end of line deletes following line break
and joins current line with next one if this option is turned on, does nothing
otherwise.

\else
Клавиша {\bf Delete}, будучи нажатой когда курсор находится вне конца строки,
удаляет следующий разделитель строк и присоединяет следующую строку к текущей,
если эта опция включена, и ничего не делает в противном случае.
\fi
\end{popup}

\begin{popup}
\ifenglish
\caption{CR in insert mode splits line}
\else
\caption{CR в режиме вставки разбивает строки}
\fi
\public{CFGEDIT\_FEATURES\_CRSPLITS}
\ifenglish
When this option is turned on {\bf Enter} key pressed in insert mode splits
line in two. When it is turned off it just moves cursor to the beginning of next line
and you must use Split Line ({\bf Ctrl+S}) command to split line.

\else
Когда эта опция включена, клавиша {\bf Enter}, нажатая в режиме вставки, разбивает
строку на две. Когда она выключена, то курсор только передвигается на начало
следующей строки, и чтобы разбить строку, нужно использовать команду Split Line.
({\bf Ctrl+S})
\fi
\end{popup}

\ifenglish
{\bf Enter} never splits line when in overwrite mode.

\else
{\bf Enter} никогда не разбивает строку в режиме перезаписи.
\fi

\begin{popup}
\ifenglish
\caption{TAB in overwrite mode only moves cursor}
\else
\caption{TAB в режиме перезаписи лишь передвигает курсор}
\fi
\public{CFGEDIT\_FEATURES\_TABMOVES}

\ifenglish
Usually (that is, when this option is turned on) pressing {\bf Tab} key in overwrite mode
actually changes next character by Tab or next several characters by spaces depending
on tabulation mode. Turning this option off causes it just to move to next tab-stop
without modifying text.

Tab in insert mode always inserts something (either tab or spaces).

\else
Обычно, (то есть, когда эта опция включена) нажатие клавиши {\bf Enter} в режиме
перезаписи изменяет следующий символ на Tab, или следующие несколько символов на
пробелы, в зависимости от режима табуляции. Выключение этой опции приводит лишь к
передвижению курсора на следующую позицию табуляции без изменения текста.

В режиме вставки Tab всегда что-либо вставляет (табуляцию или пробелы).
\fi
\end{popup}

\begin{popup}
\caption{}
\public{CFGEDIT\_FEATURES\_INSERT}
\ifenglish
You flip editor from Insert to Overwrite mode and back by pressing {\bf Insert} key.
Insert mode is maintained separately for each editor window. This option configures
initial state of Insert mode of newly created editor.
\else
Вы переключаете редактор из режима Вставки в режим Перезаписи и обратно нажатием
клавиши {\bf Insert}. Режим вставки поддерживается отдельно для каждого окна
редактирования. Эта опция настраивает исходное состояние режима вставки вновь
созданного редактора.
\fi
\end{popup}
%---------------------------------------------------------------------

\begin{popup}
\ifenglish
\caption{Hard Tabs}
\else
\caption{Символы жесткой табуляции}
\fi
\public{CFGEDIT\_SHOW\_TABS}
\ifenglish
Hard tabs are displayed as dashed lines when this option is on.

\else
Когда эта опция включена, символы жесткой табуляции отображаются как пунктирная
линия.
\fi
\end{popup}

\begin{popup}
\ifenglish
\caption{Left Margin}
\else
\caption{Левое поле}
\fi
\public{CFGEDIT\_SHOW\_LEFTMARGIN}

\ifenglish
Left edge of edited text is displayed as dotted line when this option is on.
This is useful when bookmarks displaying is turned on and left margin is wide.

\else
Когда эта опция включена, левый край редактируемого текста отображается как
точечный пунктир. Это полезно, когда включено отображение закладок, и левое поле
-- широкое.
\fi
\end{popup}

\begin{popup}
\ifenglish
\caption{Right Margin}
\else
\caption{Правое поле}
\fi
\public{CFGEDIT\_SHOW\_RIGHTMARGIN}

\ifenglish
Right edge of edited text is displayed as dotted line when this option is on.
This margin is introduced just for user's convenience. Editor allows lines of arbitrary
length but it is sometimes convenient to see if lines get too big (for example if
you wish to print text or see it in text mode).

The position of right margin is configurable.

\else
Когда эта опция включена, правый край редактируемого текста отображается как
точечный пунктир. Это поле введено только для удобства пользователя. Редактор
допускает строки произвольной длины, но иногда бывает удобно проверить, не стали
ли строки слишком длинными (например если вы хотите распечатать текст, или
просмотреть его в текстовом режиме).

Положение правого поля может настраиваться.
\fi
\end{popup}

\begin{popup}
\ifenglish
\caption{Right Margin position}
\else
\caption{Положение правого поля}
\fi
\public{CFGEDIT\_SHOW\_RIGHTMARGINVAL}

\ifenglish
If right margin display is enabled you may set up position of right margin.
The default value is 80 characters (equals to width of punch cards in medieval times).

\else
Если включено отображение правого поля, вы можете выбрать положение правого поля.
Значение по умолчанию -- 80 символов (ширина средневековой перфокарты).
\fi
\end{popup}

\begin{popup}
\ifenglish
\caption{Bookmarks}
\else
\caption{Пометки}
\fi
\public{CFGEDIT\_SHOW\_BOOKMARKS}

\ifenglish
Reserve some space to the left of edited text for displaying bookmarks set
in the text. Bookmarks are displayed as circles with numbers inside.
If there are more than one bookmarks set in a line the one with biggest
number is displayed.

\else
Оставьте немного места слева от редактируемого текста для отображения закладок,
установленных в тексте. Закладки отображаются как кружки с номерами внутри.
Если в строке установлено более одной закладки, то отображается закладка с
большим номером.
\fi
\end{popup}

%---------------------------------------------------------------------
\begin{popup}
\ifenglish
\caption{Before the block}
\else
\caption{Перед блоком}
\fi
\public{CFGEDIT\_BLOCK\_PASTE\_BEFORE}

\ifenglish
After {\bf Paste} operation cursor is placed in the beginning of pasted block.
\else
После операции вклеивания курсор располагается в начале вклеенного блока.
\fi
\end{popup}

\begin{popup}

\ifenglish
\caption{After the block}
\else
\caption{После блока}
\fi
\public{CFGEDIT\_BLOCK\_PASTE\_AFTER}

\ifenglish
After {\bf Paste} operation cursor is placed in the end of pasted block.
\else
После операции вклеивания курсор располагается в конце вклеенного блока.
\fi
\end{popup}

\begin{popup}

\ifenglish
\caption{Block is selected}
\else
\caption{Блок отмечен}
\fi
\public{CFGEDIT\_BLOCK\_PASTE\_SELECT}
\ifenglish
After {\bf Paste} operation pasted text is selected
\else
После операции вклеивания вклеенный текст помечается.
\fi
\end{popup}

\begin{popup}
\ifenglish
\caption{Inserts before the block}
\else
\caption{Вставка перед блоком}
\fi
\public{CFGEDIT\_BLOCK\_PASTE\_INSERTS\_BEFORE}

\ifenglish
If you {\bf Paste} text when block is selected new text will be inserted before
selected block.
\else
Если вы {\bf вклеиваете} текст, когда блок отмечен, то новый текст будет вставлен
перед отмеченным блоком.
\fi
\end{popup}

\begin{popup}
\ifenglish
\caption{Inserts after the block}
\else
\caption{Вставка после блока}
\fi
\public{CFGEDIT\_BLOCK\_PASTE\_INSERTS\_AFTER}

\ifenglish
If you {\bf Paste} text when block is selected new text will be inserted after
selected block.
\else
Если вы {\bf вклеиваете} текст когда блок отмечен, то новый текст будет вставлен
после отмеченного блока.
\fi

\end{popup}

\begin{popup}

\ifenglish
\caption{Replaces the block}
\else
\caption{Замена блока}
\fi
\public{CFGEDIT\_BLOCK\_PASTE\_REPLACES}
\ifenglish
If you {\bf Paste} text when block is selected new text will replace the
selected block.
\else
Если вы {\bf вклеиваете} текст, когда блок отмечен, то новый текст заменит
отмеченный блок.
\fi
\end{popup}

\begin{popup}
\ifenglish
\caption{Inserts before the block}
\else
\caption{Вставка перед блоком}
\fi
\public{CFGEDIT\_BLOCK\_TYPE\_BEFORE}
\ifenglish
Pressing any character when there is a block selected places the character
before the block and unselects the block.
\else
Нажатие любого символа, когда блок отмечен, располагает символ перед блоком
и отменяет отметку блока.
\fi
\end{popup}

\begin{popup}
\ifenglish
\caption{Inserts after the block}
\else
\caption{Вставка после блока}
\fi
\public{CFGEDIT\_BLOCK\_TYPE\_AFTER}
\ifenglish
Pressing any character when there is a block selected places the character
after the block and unselects the block.
\else
Нажатие любого символа, когда блок отмечен, располагает символ после блока
и отменяет отметку блока.
\fi
\end{popup}

\begin{popup}
\ifenglish
\caption{Replaces the block}
\else
\caption{Замена блока}
\fi
\public{CFGEDIT\_BLOCK\_TYPE\_REPLACE}
\ifenglish
Pressing any character when there is a block selected
replaces entire block with this character
\else
Нажатие любого символа, когда блок отмечен, заменяет блок
этим символом.
\fi
\end{popup}

%---------------------------------------------------------------------

\begin{popup}
\caption{OEM Conversion}
\public{CFGEDIT\_ENCODING\_OEMCONVERT}
\ifenglish
When this mode is turned on the editor contents is considered to be in OEM encoding.
Text you type is converted to OEM in this mode. Text is converted to Windows
encoding when displaying on the screen unless the font used is OEM font.

This option also affects Clipboard operations. For example, if you copy text
to clipboard in OEM mode, then change mode to Windows and paste text back it is
translated into Windows mode.
\else
Когда этот режим включен, считается, что содержимое редактора задано в кодировке
OEM. Вводимый вами текст переводится в кодировку OEM. При отображении на экране
текст переводится в Windows кодировку, если используется не OEM шрифт.

Эта опция также оказывает действие на операции с буфером обмена. Например, если
вы копируете текст в буфер обмена в OEM режиме, затем меняете режим на Windows,
и вклеиваете текст, то он переводится в режим Windows.
\fi
\end{popup}

\begin{popup}
\caption{CR}
\public{CFGEDIT\_ENCODING\_CR}
\ifenglish
Files are written to disk with {\bf CR (0x0D)} characters between lines.
\else
Файлы записываются на диск с символом {\bf CR (0x0D)} между строк.
\fi
\end{popup}

\begin{popup}
\caption{LF (UNIX)}
\public{CFGEDIT\_ENCODING\_LF}
\ifenglish
Files are written to disk with {\bf LF (0x0A)} characters between lines.
This is the standard mode for UNIX systems.
\else
Файлы записываются на диск с символом {\bf LF (0x0A)} между строк.
Это стандартный режим для UNIX систем.
\fi
\end{popup}

\begin{popup}
\caption{CR LF (DOS)}
\public{CFGEDIT\_ENCODING\_CRLF}
\ifenglish
Files are written to disk with {\bf CR LF (0x0D 0x0A)} characters between lines.
This is the standard mode for MS DOS and Windows systems.
\else
Файлы записываются на диск с символами {\bf LF (0x0D 0x0A)} между строк.
Это стандартный режим для MS DOS и Windows систем.
\fi
\end{popup}

\begin{popup}
\ifenglish
\caption{Autodetect EOL style}
\else
\caption{Автообнаружение стиля EOL}
\fi
\public{CFGEDIT\_ENCODING\_AUTODETECT}
\ifenglish
When file is loaded number of different line separators ({\bf CR}s, {\bf LF}s
and {\bf CRLF}s) is counted
and the most frequently used line separator is remembered. When this file is lately
written to disk this separator is placed between lines. This mode is useful
when you work with files on several systems at the same time.

Manually configured line separator is still used for new files.

\else
Когда файл загружается, то число различных разделителей строк ({\bf CR}, {\bf LF}
и {\bf CRLF}) подсчитывается, и наиболее часто используемый разделитель
запоминается. Когда затем этот файл записывается на диск, между строк помещается
этот разделитель. Режим полезен, когда вы одновременно работаете с несколькими
файлами различных систем.

Настроенный вручную разделитель используется для новых файлов.
\fi
\end{popup}

\begin{popup}
\ifenglish
\caption{Consider \^Z as EOF}
\else
\caption{Считать \^Z за EOF}
\fi
\public{CFGEDIT\_ENCODING\_READEOF}
\ifenglish
When reading file consider {\bf Ctrl+Z (0x1A)} character as end of file.
This option is useful on DOS machines because many of DOS text editors write
0x1A at the end of file.

\else
При чтении считать символ {\bf Ctrl+Z (0x1A)} концом файла.
Эта опция полезна на DOS машинах, так как многие текстовые редакторы DOS пишут
0x1A в конце файла.
\fi
\end{popup}

\begin{popup}
\ifenglish
\caption{Write \^Z at EOF}
\else
\caption{Писать \^Z как EOF}
\fi
\public{CFGEDIT\_ENCODING\_WRITEEOF}
\ifenglish
Write {\bf Ctrl+Z (0x1A)} character at end of file. Today this style is obsolete.
\else
Записывать символ {\bf Ctrl+Z (0x1A)} в конце файла. На сегодняшний день такой стиль
вышел из употребления.
\fi
\end{popup}

%---------------------------------------------------------------------
\begin{popup}
\ifenglish
\caption{Autosave}
\else
\caption{Автосохранение}
\fi
\public{CFGEDIT\_FILES\_AUTOSAVE}

\ifenglish
If this option is turned on the system never asks if changed file must be saved to disk.
It will silently write it to disk instead. This is done when external tools are run, project
is closed or at exit of IDE.

\else
Если эта опция включена, система никогда не спрашивает, нужно ли сохранять измененный файл.
Вместо этого она сама запишет его на диск. Это происходит, когда запускаются внешние
утилиты, закрывается проект или завершается работа IDE.
\fi
\end{popup}

\begin{popup}
\ifenglish
\caption{Create BAK files}
\else
\caption{Создание BAK файлов}
\fi
\public{CFGEDIT\_FILES\_BAK}

\ifenglish
If this option is turned on, when file is written to disk its old version is saved
as backup copy file. If the file system used is capable to store long file names
the {\bf .bak} suffix is added to file name resulting in names such as {\bf test.mod.bak}.
On other file systems (the only one the author knows is Nowell Netware) suffix {\bf .bak}
replaces existing suffix resulting in names such as {\bf test.bak}.

\else
Если эта опция включена, то, когда файл записывается на диск, его предыдущая версия
сохраняется как резервная копия. Если используемая файловая система поддерживает
длинные имена файлов, то к имени файла добавляется {\bf .bak}, и получаются имена вроде
{\bf test.mod.bak}. В других файловых системах {\bf .bak} заменяет существующее
расширение, и получаются такие имена как {\bf test.bak}.
\fi
\end{popup}
