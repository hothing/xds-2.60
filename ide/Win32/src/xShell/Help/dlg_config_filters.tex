\ifenglish
\section{Configure message filters dialog}
\else
\section{Диалог настройки фильтров сообщений}
\fi
\public{FILTERS\_DIALOG}
\nominitoc

\ifenglish
This dialog allows you to work with \ref[message filters]{MessageFilters}.
In the list box you see list of all message filters known to the system.
You may do following actions in this dialog:
\else                                  
Этот диалог позволяет работать с \ref[фильтрами сообщений]{MessageFilters}.
В списке вы видите все возможные фильтры сообщений, известные системе.
В этом диалоге вы можете производить следующие действия:
\fi
\begin{itemize}
\item 
 \ifenglish
 create new message filter by pressing {\bf New} pushbutton.
\ref[New filter]{NEWFILTER\_DIALOG} dialog is displayed where you will be prompted
 for new filter's name and file in which it is to
be stored, and then \ref[edit filter]{EDITFILTER\_DIALOG} dialog is displayed.
\else
 создать новый фильтр сообщения путем нажатия кнопки {\bf New}. 
 отобразится диалог \ref[New filter]{NEWFILTER\_DIALOG}, где будет запрошено 
 имя нового фильтра и имя файла, в котором он будет сохранен,
 а затем будет отображен диалог \ref[edit filter]{EDITFILTER\_DIALOG}.
 \fi
\item 
 \ifenglish
 edit existing message filter by selecting it in the list and pressing
the {\bf Edit} pushbutton. \ref[Edit filter]{EDITFILTER\_DIALOG} dialog
will be displayed.
\else
 редактировать существующий фильтр сообщений путем выбора его в списке и 
 нажатия кнопки {\bf Edit}. Будет выведен диалог \ref[Edit filter]{EDITFILTER\_DIALOG}.
 \fi
\item 
 \ifenglish
 remove message filter and delete the file it is stored in by pressing the
{\bf Remove} button.
\else
 удалить фильтр сообщения и стереть файл, в котором он хранится с помощью 
 нажатия кнопки {\bf Remove}.
 \fi
\end{itemize}

%---------------------------------------------------------------------

\begin{popup}
\ifenglish
\caption{Message filters available}
\else
\caption{Доступные фильтры сообщений}
\fi
\public{FILTERS\_LIST}

\ifenglish
This list shows all message filters known to the system. You may add new
message filter to the list, remove or edit selected message filter.
\else
Этот список отображает все фильтры сообщений, известные системе. Вы можете 
добавить новый фильтр сообщений в список, удалить или отредактировать 
выбранный фильтр сообщений.
\fi
\end{popup}

\begin{popup}
\caption{New}
\public{FILTERS\_NEW}

\ifenglish
Create new message filter
\else
Создать новый фильтр сообщений. 
\fi
\end{popup}

\begin{popup}
\caption{Edit}
\public{FILTERS\_EDIT}

\ifenglish
Edit currently selected message filter. {\bf Edit message filter} dialog is
displayed.
\else
Редактировать выбранный фильтр сообщений. Выводится диалог {\bf Edit message filter}.
{\bf Edit message filter}
\fi
\end{popup}

\begin{popup}
\caption{Remove}
\public{FILTERS\_REMOVE}

\ifenglish
Remove currently selected message filter and delete file in which it
is stored.
\else
Удалить выбранный фильтр сообщений и файл в котором он хранится.
\fi
\end{popup}

\begin{popup}
\caption{Close}
\public{FILTERS\_OK}

\ifenglish
Close the message filter configuration dialog.
\else
Закрыть диалог настройки фильтра сообщений. 
\fi
\end{popup}

%===========================================================================

\ifenglish
\subsection{New filter dialog}
\else
\subsection{Диалог New filter}
\fi
\public{NEWFILTER\_DIALOG}

\ifenglish
To create new \ref[message filter]{MessageFilters}:
\else
Чтобы создать новый \ref[фильтр сообщений]{MessageFilters}:
\fi
\begin{enumerate}
\item \ifenglish
      type new message filter name in {\bf Filter name} entry field
      \else
      введите имя нового фильтра сообщений в поле ввода {\bf Filter name}
      \fi
\item \ifenglish
      file name will appear in the {\bf File name} entry field. If you disagree
with this suggestion type name you like or select a file using standard File
dialog by pressing {\bf Browse} pushbutton
      \else
      В поле ввода {\bf File name} появится имя файла. Если предложенное имя вас не 
устроит, введите нужное вам имя или выберите его, используя стандартный файловый 
диалог путем нажатия кнопки {\bf Browse}.
      \fi
\item \ifenglish
      message filters often look similar. If you wish your filter be based
on some existing message filter select its name in the {\bf Based on} list.
New filter will initially be the copy of selected filter. To create blank
filter, select {\bf None} in the list box.
      \else
  фильтры сообщений часто выглядят схожими. Если вы хотите, чтобы ваш фильтр 
основывался на каком-либо существующем фильтре сообщений, выберите его имя в 
списке {\bf Based on}. Новый фильтр сначала будет копией выбранного. Чтобы создать
пустой фильтр, выберите пункт {\bf None} списка.
      \fi
\item \ifenglish
      press the {\bf OK} pushbutton.
     \else
      нажмите кнопку {\bf OK}.
     \fi
\end{enumerate}

\ifenglish
The \ref[Edit filter]{EDITFILTER\_DIALOG} dialog will appear allowing you to
define new filter.
\else
Появится диалог \ref[Edit filter]{EDITFILTER\_DIALOG}, позволяющий определить 
новый фильтр.                       
\fi
%---------------------------------------------------------------------
\begin{popup}
\ifenglish
\caption{Filter name}
\else
\caption{Имя фильтра}
\fi
\public{NEWFILTER\_NAME}

\ifenglish
Type name for new filter here.
\else
Наберите имя нового фильтра.
\fi
\end{popup}

\begin{popup}
\ifenglish
\caption{File name}
\else
\caption{Имя файла}
\fi
\public{NEWFILTER\_FILE}

\ifenglish
Type a name for new filter. By default the name is the same as the name of
a filter with the suffix {\bf .fil}. If directory is not specified the filter
is created in XDS home directory.
\else
Введите имя нового фильтра. По умолчанию имя задается тем же, что и у фильтра с 
дополнением {\bf .fil}. Если директория не задана, то фильтр создается в коренной
директории XDS.
\fi
\end{popup}

\begin{popup}
\caption{Browse}
\public{NEWFILTER\_BROWSE}

\ifenglish
Select file name for new filter using standard File dialog.
\else
Выберите имя файла для нового фильтра, используя стандартный файловый диалог. 
\fi
\end{popup}

\begin{popup}
\caption{Based On}
\public{NEWFILTER\_LIST}

\ifenglish
Select the filter you wish your new filter to be based on from this list.
This filter will be copied to new filter. Select {\bf None} if you wish to start
from blank filter.
\else
Выберите из списка фильтр, на основе которого вы хотите создать новый.
Этот фильтр будет скопирован в новый.  Выберите {\bf None}, если хотите начать 
с пустого фильтра.
\fi
\end{popup}

\begin{popup}
\caption{OK}
\public{NEWFILTER\_OK}

\ifenglish
Display edit filter dialog where you can actually define the filter.
\else
Отобразить диалог редактирования фильтра, где вы можете определить фильтр.
\fi
\end{popup}

\begin{popup}
\caption{Cancel}
\public{NEWFILTER\_CANCEL}

\ifenglish
Cancel creating new filter
\else
Отменить создание нового фильтра.
\fi
\end{popup}

%===========================================================================

\ifenglish
\subsection{Edit filter dialog}
\else
\subsection{диалог Edit filter}
\fi
\public{EDITFILTER\_DIALOG}

\ifenglish
This dialog allows you to edit the \ref[message filter]{MessageFilters}.
Message filter is an ordered set of message filter
\ref[regular expressions]{MessageFiltersRegExp} with a
\ref[message class]{MessageClasses} assigned to each one.

All the expressions are collected in a list here. The selected one is
duplicated in the {\bf Filter expression} entry field. You can modify it
and press {\bf Modify} button to store its value to the list. Set message class
for current expression line by selecting it in the {\bf Message class} selection
box.

You can remove current line by pressing {\bf Remove} button or add new line
by the {\bf Add} button. The new line is copied from currently selected line
and is inserted in current place.

You may change order of lines in the filter by moving them up and down using {\bf Up}
and {\bf Down} buttons.
\else
Этот диалог позволяет вам редактировать \ref[фильтры сообщений]{MessageFilters}.
Фильтр сообщений -- это упорядоченное множество
\ref[регулярных выражений]{MessageFiltersRegExp} с привязанными к каждому
\ref[классами сообщений]{MessageClasses}.

Все выражения собраны здесь в список. Выбранное выражение дублируется в поле ввода 
{\bf Filter expression}. Вы можете изменить его и нажать {\bf Modify}, чтобы 
занести его значение в список. Установите класс сообщений для текущего выражения 
путем выбора его в списке {\bf Message class}.

Вы можете удалить текущую строку нажатием кнопки {\bf Remove} или добавить новую
с помощью кнопки {\bf Add}. Новая строка копируется из выбранной в данный момент 
строки и вставляется на текущую позицию.

Вы можете менять порядок строк в фильтре путем передвижения их вверх и вниз
с помощью кнопок {\bf Up} и {\bf Down}.
\fi
%---------------------------------------------------------------------


\begin{popup}
\caption{OK}
\public{EDITFILTER\_OK}
\ifenglish
Accept the changes made in the message filter, write message filter file
and close the dialog.
\else
Подтвердить изменения, произведенные в фильтре сообщений, сохранить файл фильтра
сообщений и закрыть диалог.  
\fi
\end{popup}

\begin{popup}
\caption{Cancel}
\public{EDITFILTER\_CANCEL}

\ifenglish
Close the dialog without saving the message filter.
\else
Закрыть диалог без сохранения фильтра сообщений.
\fi
\end{popup}

\begin{popup}
\ifenglish
\caption{Filter expression}
\else
\caption{Выражение фильтра}
\fi
\public{EDITFILTER\_TEXT}

\ifenglish
Current regular expression is displayed here. It is the copy of the
expression selected in a list box. If you wish to edit the line, modify
this field and press the {\bf Modify} button.

Press the arrow button to the right to insert regular expression patterns
for current line, column etc.

\else
Здесь отображается текущее регулярное выражение. Это -- копия выражения,
выбранного в списке. Если вы хотите отредактировать строку, внесите изменения 
в это поле и нажмите кнопку {\bf Modify}.

Нажмите кнопку со стрелкой справа, чтобы вставить шаблонные значения 
регулярных выражений для текущей строки, столбца и т. д.
\fi
\end{popup}


\begin{popup}
\ifenglish
\caption{Expressions list}
\else
\caption{Список выражений}
\fi
\public{EDITFILTER\_LIST}

\ifenglish
All expressions from current filter are collected here. To modify expression,
select it in the list, edit the {\bf Filter expression} field and press {\bf Modify}
button. To remove expression select it and press {\bf Remove} button. To add new
expression at current point press {\bf Add} button. To change message class for selected
expression select new value in the {\bf Message class} selection box.

Note that expressions in message filters are ordered, and the order is significant.
Use {\bf Up} and {\bf Down} buttons to move selected line up and down.

\else
Здесь собраны все выражения текущего фильтра. Чтобы изменить выражение, выберите 
его в списке, отредактируйте поле {\bf Filter expression} и нажмите кнопку {\bf Modify}.
Чтобы удалить выражение, выберите его и нажмите кнопку {\bf Remove}. Чтобы добавить
новое выражение на текущей позиции, нажмите кнопку {\bf Add}. Чтобы изменить 
класс сообщения для выражения, выберите новое значение в списке {\bf Message class}.

Заметьте, что выражения в фильтре сообщений упорядочены, и порядок имеет значение. 
Используйте кнопки {\bf Up} и {\bf Down} чтобы передвигать выбранные строки вверх и 
вниз.
\fi
\end{popup}

\begin{popup}
\caption{Add}
\public{EDITFILTER\_ADD}

\ifenglish
Add new filter expression. Expression is initially equal to the selected
expression. Modify it and press {\bf Modify} button. Choose message class for new
expression in the {\bf Message class} selection box.

\else
Добавьте новое выражение в фильтр. Выражение сначала становится равным 
выбранному выражению. Измените его и нажмите кнопку {\bf Modify}. Выберите класс
для нового выражения в списке {\bf Message class}.
\fi
\end{popup}

\begin{popup}
\caption{Modify}
\public{EDITFILTER\_MODIFY}

\ifenglish
Set selected expression in the list equal to the value of the
{\bf Filter expression} entry field.

\else
Установить выбранное выражение равным значению поля ввода. 
{\bf Filter expression}
\fi
\end{popup}

\begin{popup}
\caption{Remove}
\public{EDITFILTER\_REMOVE}

\ifenglish
Remove selected filter expression.
\else
Удалить выбранное выражение фильтра. 
\fi
\end{popup}

\begin{popup}
\caption{Up}
\public{EDITFILTER\_UP}

\ifenglish
Move selected expression up by swapping it with previous one.
\else
Передвинуть выбранное выражение путем перемены мест с предыдущим.
\fi
\end{popup}

\begin{popup}
\caption{Down}
\public{EDITFILTER\_DOWN}

\ifenglish
Move selected expression down by swapping it with next one.
\else
Передвинуть выбранное выражение путем перемены мест со следующим.
\fi
\end{popup}

\begin{popup}
\ifenglish
\caption{Message class}
\else
\caption{Класс сообщений}
\fi
\public{EDITFILTER\_CLASS}

\ifenglish
Set {\bf message class} for currently selected filter expression. Following choices
are available:
\else
Установите {\bf класс сообщения} для выбранного в данный момент выражения. Доступны
следующие варианты:
\fi
\begin{itemize}
\item \ifenglish
      Text
      \else
      Текст
      \fi
\item \ifenglish
      Notice
      \else
      Замечание
      \fi
\item \ifenglish
      Warning
      \else
      Предупреждение
      \fi
\item \ifenglish
      Error
      \else
      Ошибка
      \fi
\item \ifenglish
      Severe Error
      \else
      Грубая ошибка 
      \fi
\end{itemize}

\ifenglish
See help topic for message classes for details.
\else
Для более подробных сведений смотрите раздел подсказки, посвященный 
классам сообщений.
\fi
\end{popup}

\begin{popup}
\ifenglish
\caption{Pop-up menu}
\else
\caption{Всплывающее меню}
\fi
\public{EDITFILTER\_MENU}

\ifenglish
Some character combination have special meaning in filter expressions.
For example, {\tt"\&File"} means name of a file to which the message refers.
To insert such combination in the text press this button and select it
from pop-up menu.
\else
Некоторые сочетания символов имеют особенное значение в выражениях фильтра.
Например, {\tt"\&File"} означает имя файла, на который ссылается сообщение.
Чтобы вставить такое сочетание в текст, нажмите эту кнопку и выберите его
во всплывающем меню.
\fi
\end{popup}

%===========================================================================
