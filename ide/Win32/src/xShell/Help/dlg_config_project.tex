%-----------------------------------------------------------------------
\ifenglish
\section{New Project Creation Options dialog}
\else
\section{диалог New Project Creation Options }
\fi
\public{CFGPROJ\_DIALOG}
\nominitoc

\ifenglish
Use this dialog box to specify default values of new project creation
options:

\else
Используйте этот диалог, чтобы указывать значения опций нового проекта, 
принимаемые по умолчанию:
\fi
\begin{itemize}
\item \ifenglish
      subdirectories to create in the project directory
      \else
      поддиректории, которые надлежит создать в директории проекта
      \fi
\item \ifenglish
      project file template
      \else
      шаблон файла проекта
      \fi
\item \ifenglish
      redirection file template
      \else
      шаблон файла путей поиска
      \fi
\item \ifenglish
      main module name
      \else
      имя главного модуля
      \fi
\end{itemize}


{\bf Directories to create}
\ifenglish
\ref[Create New Project]{NEWPROJ\_DIALOG} dialog allows defining of set of
subdirectories to create in project directory. Here you can enter values that
will be suggested by default. Initial value of this string is

\verb'SRC;OBJ;SYM'

so that directories are created for source, symbol and object files.
\else
Диалог \ref[Create New Project]{NEWPROJ\_DIALOG} позволяет указывать набор 
поддиректорий, которые будут созданы в директории проекта. Здесь вы можете 
ввести значения, которые будут предложены по умолчанию. Исходное значение этой 
строки 

\verb'SRC;OBJ;SYM'

так что создаются директории для исходных символьных и объектных файлов.
\fi

{\bf Default template}

\ifenglish
Here you can change default name of \ref{ProjectTemplate} file to be suggested
by default in Create New Project dialog.

\ref{variables} may be used in this field.

Default value of this field is

\verb'$(xdsdir)\$(xdsname).tpr'

\else
Здесь вы можете менять имя файла \ref{ProjectTemplate}, которое будет предлагаться
по умолчанию диалогом Create New Project.

В этом поле могут быть использованы \ref[переменные]{variables}.

Значение по умолчанию этого поля 

\verb'$(xdsdir)\$(xdsname).tpr'
\fi

{\bf Default RED file}
\ifenglish
Don't be afraid - no file will be painted red by default.

Here you can change default name of \ref{RedirectionTemplate} file to be suggested
by default in Create New Project dialog.

\ref{variables} may be used in this field.

Default value of this field is

\verb'$(xdsdir)\$(xdsname).trd'

\else
Не опасайтесь, что какой-либо файл будет по умолчанию окрашен в красное.

Здесь вы можете менять значение файла \ref{RedirectionTemplate}, которое будет
предлагаться по умолчанию диалогом Create New Project.

В этом поле могут быть использованы \ref[переменные]{variables}.

Значение по умолчанию этого поля 

\verb'$(xdsdir)\$(xdsname).trd'
\fi

{\bf Default main module}
\ifenglish
Type a name of a main module which you want to be used by default.
A file with this name will be created and reference to this name
({\bf !module} project directive) will be added to the project.

No file is created and no reference is added if main module name is left blank.

\ref{variables} may be used in this field. This allows the name to
depend on project name. Default value is

\verb'$(projname).mod'

\else
Введите имя главного модуля, которое вы хотите использовать по умолчанию.  
Будет создан файл с таким именем, и ссылка на это имя (проектная директива 
{\bf !module}) будет добавлена в проект.

Если имя главного модуля оставлено пустым, то файл и ссылка на него не создаются.

В этом поле могут быть использованы \ref[переменные]{variables}. Это позволяет 
имени модуля зависеть от имени проекта. Значение по умолчанию --

\verb'$(projname).mod'
\fi
%-----------------------------------------------------------------------
\begin{popup}
\caption{Directories to create}
\public{CFGPROJ\_DIRECTORIES}
\ifenglish
Type a semicolon-separated list of subdirectories which you would like
to be created in a new project directory.
\else
Наберите список поддиректорий, которые вы хотите создать в директории нового
проекта, разделяя их точкой с запятой.
\fi
\end{popup}

\begin{popup}
\caption{Default template}
\public{CFGPROJ\_TEMPLATE}

\ifenglish
New project creation dialog contains entry field with project template name.
It is initialized by the string you type here or file name you select by pressing
{\bf Browse} button.
\else
Диалог создания нового проекта содержит поле ввода названия шаблона проекта.
Оно инициализируется строкой, которую вы здесь вводите или именем файла, которое
вы выбираете с помощью кнопки {\bf Browse}.
\fi
\end{popup}

\begin{popup}
\caption{Browse}
\public{CFGPROJ\_BROWSETEMPLATE}

\ifenglish
New project creation dialog contains entry field with project template name.
It is initialized by the file name you select by pressing this button or type
in entry field.
\else
Диалог создания нового проекта содержит поле ввода названия шаблона проекта.
Оно инициализируется именем файла, которое вы выбираете с помощью нажатия этой 
кнопки, или набираете в поле ввода.
\fi
\end{popup}

\begin{popup}
\ifenglish
\caption{Default RED file}
\else
\caption{Файл путей поиска, используемый по умолчанию}
\fi
\public{CFGPROJ\_REDFILE}

\ifenglish
New project creation dialog contains entry field with redirection file template name.
It is initialized by the string you type here or file name you select by pressing
{\bf Browse} button.

\else
Диалог создания нового проекта содержит поле ввода имени файла шаблона путей поиска.
Оно инициализируется строкой, которую вы здесь вводите, или файлом, который вы выбираете
с помощью кнопки {\bf Browse}.
\fi
\end{popup}

\begin{popup}
\caption{Browse}
\public{CFGPROJ\_BROWSEREDFILE}

\ifenglish
New project creation dialog contains entry field with redirection file template name.
It is initialized by the file name you select by pressing this button or type
in entry field.
\else
Диалог создания нового проекта содержит поле ввода имени файла шаблона путей поиска.
Оно инициалируется именем файла, которое вы выбирате с помощью нажатия этой 
кнопки, или набираете в поле ввода.
\fi
\end{popup}

\begin{popup}
\caption{Default main module}
\public{CFGPROJ\_DEFMAINMOD}

\ifenglish
Type a name of a main module which you want to be used by default.
Substitution variables may be used here.

\else
Введите имя главного модуля, которое вы хотите использовать по умолчанию. 
Здесь можно использовать переменные замены.
\fi
\end{popup}

\begin{popup}
\caption{OK}
\public{CFGPROJ\_OK}

\ifenglish
Select the {\bf OK} pushbutton to accept the changes you have made
to the new project creation options.

\else
Нажмите кнопку {\bf OK}, чтобы подтвердить изменения, произведенные вами
в опциях создания нового проекта. 
\fi
\end{popup}

\begin{popup}
\caption{Cancel}
\public{CFGPROJ\_CANCEL}

\ifenglish
Select the {\bf Cancel} pushbutton to dismiss the dialog without
making any changes to the new project creation options.

\else
Нажмите кнопку {\bf Cancel}, чтобы закрыть диалог без внесения изменений в 
опции соэдания нового проекта.
\fi
\end{popup}
