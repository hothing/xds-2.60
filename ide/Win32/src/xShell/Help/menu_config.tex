\section{Configure Menu}
\public{M\_CONFIGURE}
\index{Configure Menu}
\nominitoc

\ifenglish
The {\bf Configure} menu contains commands that you use to
configure the IDE.
The following commands appear in the {\bf Configure} menu:
\else
Меню {\bf Configure} содержит команды, используемые для настройки IDE.
В меню {\bf Configure} появляются следующие команды: 
\fi
\begin{description}
\item \ref[General]{IDM\_CONFIGUREGENERAL}
\ifenglish
  - Configure options that fit in neither of below
  \else
  - Настроечные опции, которые не входят ни в один из нижеперечисленных разделов
  \fi
\item \ref[Tools]{IDM\_CONFIGUREWINDOWS}
\ifenglish
  - Configures windows appearance and behavior
  \else
  - Настраивает внешний вид и поведение окон
  \fi
\item \ref[Windows]{IDM\_CONFIGURETOOLS}
\ifenglish
  - Lets you set up external tools
  \else
  - Позволяет устанавливать внешние утилиты
  \fi
\item \ref[Project Creation]{IDM\_CONFIGUREPROJECT}
\ifenglish
  - Tunes new project creation
  \else
  - Настройка создания нового проекта
  \fi
\item \ref[Editor]{IDM\_CONFIGUREEDITOR}
\ifenglish
  - Configures various editor options
  \else
  - Настройка различных опций редактора
  \fi
\item \ref[Colors and Fonts]{IDM\_COLOR}
\ifenglish
  - Sets colors and fonts used in IDE windows
  \else
  - Устанавливает цвета и шрифты, используемые в окнах IDE
  \fi
\item \ref[Printing]{IDM\_CONFIGUREPRINT}
\ifenglish
  - Sets up printing parameters
  \else
  - Устанавливает параметры печати
  \fi
\item \ref[Message Filters]{IDM\_CONFIGUREFILTERS}
\ifenglish
  - Add, remove and edit \ref[message filters]{MessageFilters}
  \else
  - Добавляет, удаляет и редактирует \ref[фильтры сообщений]{MessageFilters}
  \fi
\item \ref[Languages]{IDM\_CONFIGURE\_LANGUAGES}
\ifenglish
  - Configure support for \ref[programming languages]{Languages}
  \else
  - Настраиваает поддержку \ref[языков программирования]{Languages}
  \fi
\end{description}

\subsection{Configure: General}
\public{IDM\_CONFIGUREGENERAL}
\ifenglish
{\bf General command} invokes the \ref[dialog]{CFGGENERAL\_DIALOG} in which you
can modify general system options.
\else
Пункт меню {\bf General command} вызывает 
\ref[диалог]{CFGGENERAL\_DIALOG}, в котором вы можете изменять общие 
опции системы.
\fi

\ifenglish
Three options are currently available:
\else
В настоящий момент доступны три опции: 
\fi
\begin{description}
\item \ifenglish
      - whether the IDE must show logo picture at startup
      \else
      - показывать ли IDE при запуске логотип
      \fi
\item \ifenglish
      - whether the IDE must remember and load last loaded project
      \else
      - должна ли IDE запоминать и загружать последний загруженный проект
      \fi
\item \ifenglish
      - whether the toolbar is old style (3D) or new fashion (flat)
      \else
      - стиль панели инструментов, трехмерный или плоский
      \fi
\end{description}

\subsection{Configure: Windows}
\public{IDM\_CONFIGUREWINDOWS}
\ifenglish
XDS IDE uses its own version of {\bf MDI} (Multiple Document Interface).
This version provides {\bf docked} windows (windows that are never overlapped by others)
and {\bf free} windows (windows that appear on the desktop rather than inside
IDE frame). It also allows window numbering that is more convenient than
in classic MDI and provides {\bf winbar} to make navigation among your windows
easier.

The {\bf Windows} displays the \ref[dialog]{CFGWIN\_DIALOG} that
lets you choose placing and appearance of winbar,
window numbering schema, Window menu style and some other options.
\else
XDS IDE использует свою собственную версию {\bf MDI} (Multiple Document Interface). 
Эта версия дает возможность использования неперекрывающихся окон 
(которые никогда не закрываются другими) и свободных окон 
(которые появляются скорее на рабочем столе чем внутри IDE оболочки).
Она также предоставляет более удобный, чем в классическом MDI способ 
нумерации окон, и позволяет легче управляться с окнами.

Пункт меню {\bf Windows} отображает диалог \ref[dialog]{CFGWIN\_DIALOG}, который
позволяет вам выбирать местоположение и внешний вид WinBar,
схему нумерации окон, стиль оконного меню и некоторые другие опции. 
\fi

\subsection{Configure: Tools}
\public{IDM\_CONFIGURETOOLS}

\ifenglish
Use the {\bf Tools} command to configure external tools
which appear in the {\bf Tools} menu. Each tool has menu string, command line
and (optionally) keyboard shortcut.

The command displays \ref[Configure Tools]{CFGTOOLS\_DIALOG} dialog.
\else
Используйте команду {\bf Tools}, чтобы настраивать внешние утилиты,
которые появляются в меню {\bf Tools}. Для каждой утилиты имеется строка меню,
командная строка и (необязательно) комбинация клавиш клавиатуры. 

Команда выводит диалог \ref[Configure Tools]{CFGTOOLS\_DIALOG}.
\fi
\subsection{Configure: Project creation}
\public{IDM\_CONFIGUREPROJECT}

\ifenglish
Use the {\bf Project creation} command to fine tune new
project creation process: specify directories to create,
project template, redirection file template, etc.

The command displays \ref[New Project Creation Options]{CFGPROJ\_DIALOG} dialog.
\else
Используйте команду меню {\bf Project creation} для тонкой
настройки процесса создания проекта определите директории, которые надлежит
создать, файл шаблонов проекта, файл шаблонов путей поиска и т. д.

Команда выводит диалог \ref[New Project Creation Options]{CFGPROJ\_DIALOG}.
\fi
\subsection{Configure: Editor}
\public{IDM\_CONFIGUREEDITOR}

\ifenglish
It is well known that initially convenient system is better than the system
that allows to tune everything and that may become convenient after hours of tuning.
Nevertheless, the editor used in the IDE (called {\bf PoorEdit}) provides lots
of tunable parameters. You can specify tabulation size and method, fine tune
subtle properties of {\bf BS}, {\bf Delete} and {\bf Enter} keys, set file encoding
and and-of-line handling, cursor position after {\bf Paste} operation and many other
things.

All those settings are available via {\bf Editor} command that displays
\ref[Editor Configuration dialog]{CFGEDIT\_DIALOG}.
\else
Общеизвестно, что изначально удобная система лучше системы, которая позволяет все
настраивать и которая может стать удобной лишь после долгих часов, 
потраченных на настройку. Тем не менее, используемый в IDE редактор (называемый 
{\bf PoorEdit}) предоставляет множество настраиваемых параметров.
Вы можете определить ширину и способ табуляции, плавно настроить хитроумные
свойства клавиш {\bf BS}, {\bf Delete} и {\bf Enter}, установить кодировку файла
и способ обработки конца строки, положение курсора после операции {\bf Paste} 
и многое другое.

Все эти опции доступны посредством команды {\bf Editor}, которая отображает диалог
\ref[Editor Configuration dialog]{CFGEDIT\_DIALOG}.
\fi
\subsection{Configure: Colors and fonts}
\public{IDM\_COLOR}

\ifenglish
Use the {\bf Colors and fonts} command to specify colors
and fonts used in IDE windows.

You can specify foreground and background colors and fonts for editor windows,
file lists and message windows. For editor windows you can also specify color
of selection and color of highlighted text. If you have set of
\ref[programming languages]{Languages} configured you can also configure colors
for all syntax elements recognized in the languages. Each language is configured
separately.

You can also set the background color
of the IDE. 
\else
Используйте команду {\bf Colors and fonts}, чтобы определить цвета и шрифты,
задействованные в окнах IDE.

Вы можете определять цвет фона и рисующий цвет, шрифты для окон редактора
списков файлов и окон сообщений. Для окон редактора вы можете также определить 
цвета отмеченного и выделенного текста. Если у вас настроен набор  
\ref[языков программирования]{Languages}, то вы также можете настроить цвета 
для всех элементов синтаксиса, распознаваемых в языке. Каждый язык настраивается 
отдельно.

Вы также можете установить цвет фона IDE.
\fi
\subsection{Configure: Printing}
\public{IDM\_CONFIGUREPRINT}

\ifenglish
Use the {\bf Printing} command to specify how you want your files printed.
You can specify how you wish to handle long lines, what information to put
in the page header and whether to print line numbers. You can also choose
printer font and page margins.
\else
Используйте команду {\bf Printing}, чтобы определить, как вы хотите 
распечатывать ваши файлы. Вы также можете указать, как обрабатывать длинные 
строки, какие сведения помещать в заголовок страницы, и нужно ли печатать 
номера страниц. Вы также можете выбрать шрифт принтера и размеры полей.
\fi
\subsection{Configure: Message Filters}
\public{IDM\_CONFIGUREFILTERS}

\ifenglish
Use the {\bf Message Filters} command to display a dialog box where you can
work with \ref[Message Filters]{MessageFilters}. You can create new filters,
edit or remove existing filters. You can assign created message filters to tools
using \ref[Configure]{M\_CONFIGURE}:\ref[Tools]{IDM\_CONFIGURETOOLS} menu.
\else
Используйте команду {\bf Message Filters}, чтобы отбразить диалоговое окно,
где вы можете работать с \ref[фильтрами сообщений]{MessageFilters}. Вы также 
можете создавать новые фильтры, редактировать или удалять существующие.
Вы можете привязать созданные фильтры сообщений к утилитам, используя меню
\ref[Configure]{M\_CONFIGURE}:\ref[Tools]{IDM\_CONFIGURETOOLS}. 
\fi
\subsection{Configure: Languages}
\public{IDM\_CONFIGURE\_LANGUAGES}

\ifenglish
Use the {\bf Languages} command to display a dialog box where you can define
set of \ref[programming languages]{Languages} the IDE recognizes. You can
add, remove languages and configure \ref[language drivers]{LangDriver}.
\else
Используйте команду {\bf Languages}, чтобы отобразить диалог, где вы можете 
определить набор распознаваемых IDE \ref[языков программирования]{Languages}.
Вы можете добавлять, удалять языки и настраивать \ref[драйверы языков]{LangDriver}.
\fi