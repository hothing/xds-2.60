\ifenglish
\section{Message Filters}
\else
\section{Фильтры сообщений}
\fi
\public{MessageFilters}

\ifenglish
When IDE calls external tools, it must receive error and
warning messages with their position information (file name, line and column)
to display messages in a user-friendly form and be able to navigate among
them.

XDS IDE has built-in interface for tools to transmit error messages. This
interface is based on writing of special character sequences in tool's
standard output. This interface provides tools with enhanced capabilities such
as displaying the progress indicator. Most tools that are shipped with XDS
use this interface. Programmers writing external tools for XDS are encouraged
to use this interface that is available through {\bf xShell} module.

\else
Когда IDE вызывает внешние утилиты, она должна получать сообщения об ошибках
и предупреждениях и информацию об их расположении (имя файла, строку и столбец),
чтобы отображать сообщения в дружественной пользователю форме и быть в 
состоянии переключаться между ними.

В XDS IDE есть встроенный интерфейс позволяющий утилитам передавать сообщения об
ошибках. Этот интерфейс основан на записи последовательностей специальных 
символов в стандартный вывод утилиты. Этот интерфейс обеспечивает утилиты 
расширенными возможностями, такими как отображение индикаторов хода работы.
Большинство утилит, поставляемых с XDS, используют этот интерфейс. Программисты,
пишушие для XDS внешние утилиты, могут использовать этот интерфейс с помощью
модуля {\bf xShell}.	  
\fi
\ifenglish
However, if the tool you intend to use comes from third-party vendor the
only way to get error messages is to parse its output looking for lines
of specific format. This format depends on the tool used. For example MASM's
output looks like this:
\else
Однако, если утилита которую вы намереваетесь использовать происходит из 
третьей фирмы, то единственный способ получить сообщения об ошибках -- это
проанализировать ее вывод, отыскивая строки специального формата. Этот
формат зависит от используемой утилиты. К примеру, вывод MASM может выглядеть
так:  
\fi

\begin{verbatim}test.asm(10): error A2009: Symbol not defined: jopa \end{verbatim}

\ifenglish
while TopSpeed's output looks like this:
\else
В то же время, вывод TopSpeed выглядит так:
\fi
\verb'(TEST.C 5,3) Error: Variable "x" not declared'

\ifenglish
It happens that different tools share same message format.
For example, {\bf MASM} format was once very popular.
So instead of defining the format for each tool, XDS uses
{\bf message filters schemes} also called just {\bf message filters}.

Each filter is stored in the separate file. By default when you create a filter
the IDE assigns to it a file with the same name as the filter's and with
extension "{\bf.fil}", residing in the directory XDS INI file was read from.

Each filter consists of several patterns. Each pattern consists of restricted form
or \ref[regular expression]{MessageFiltersRegExp} and the \ref[message class]{MessageClasses}.
When external tool is called,
each line of its standard output is matched against all the patterns in a filter,
If the line matches some pattern it is considered a message of a class assigned to
the pattern and is transferred to the message window.

Note that a line can match to more than one pattern. Since the patterns are
tried sequentially it is recommended to place most general patterns at the
end of pattern list.
\else
Случается, что у разных утилит оказывается один и тот же формат сообщений.
Например, когда-то был очень популярен формат {\bf MASM}.
Так что, вместо того чтобы определять формат для каждой утилиты,
XDS использует {\bf системы фильтров сообщений}, называемые также просто
{\bf фильтрами сообщений}.

Каждый фильтр хранится в отдельном файле. По умолчанию, когда вы создаете  
фильтр, IDE приписывает его файлу с тем же именем, что и у фильтра, и с 
расширением "{\bf.fil}", располагая его в той директории, откуда был прочитан 
файл XDS INI.

Каждый фильтр состоит из нескольких моделей. Каждая модель состоит из специальной
формы или \ref[регулярного выражения]{MessageFiltersRegExp} и \ref[класса сообщений]{MessageClasses}.
Когда вызывается внешняя утилита, каждая строка ее стандартного вывода сопоставляется
со всеми моделями в фильтре. Если строка удовлетворяет некоторой модели, то она 
считается сообщением приписанного модели класса и выводится в окно сообщений.

Заметьте, что строка может удовлетворять более чем одной модели. Так так модели
перебираются последовательно, рекомендуется помещать наиболее общие модели в 
конце списка.
\fi
\ifenglish
\subsection{Message Filters regular expressions}
\else
\subsection{регулярные выражения фильтров сообщений}
\fi
\public{MessageFiltersRegExp}

\ifenglish
The regular expressions used in message filters are the character strings
where characters are matched literally to those of a line being tested
except for following character sequences that have special meaning:
\else
Регулярные выражения, используемые в фильтрах сообщений, -- это символьные 
строки, где символы в точности совпадают с содержимым проверяемой строки,
за исключением следующих последовательностей, которые имеют особый смысл:
\fi
{\bf ?} \ifenglish
        -- any character
        \else
        -- любой символ
        \fi
{\bf *} \ifenglish
        -- any character sequence (possibly null)
        \else
        -- любая последовательность символов (возможно, пустая)
        \fi
{\bf \&file} 
           \ifenglish
           -- any non-null character sequence specifying file name
           \else
            -- любая непустая последовательность символов, задающая имя файла
           \fi
{\bf \&line} \ifenglish
             -- sequence of digits specifying line number
             \else
             -- последовательность цифр, задающая номер строки
             \fi
{\bf \&col} \ifenglish
            -- sequence of digits specifying column number
            \else
            -- последовательность цифр, задающая номер столбца
            \fi
{\bf \&line0} 
            \ifenglish  
            -- sequence of digits specifying zero-based line number
            \else
             -- последовательность цифр, задающая номер строки (нумерация 
                 начинается с нуля)
            \fi
{\bf \&col0} \ifenglish
             -- sequence of digits specifying zero-based column number
             \else
              -- последовательность цифр, задающая номер столбца  (нумерация
                начинается с нуля)
             \fi
{\bf \&num} 
            \ifenglish
            -- sequence of digits
            \else
            -- последовательность цифр
            \fi
{\bf \&body} \ifenglish
             -- any character sequence specifying message body
             \else
             -- любая последовательность символов, задающая тело сообщения 
             \fi
{\bf \bs{}?} \ifenglish
             -- literal "?"
             \else
             -- литера "?"
             \fi
{\bf \bs{}*} \ifenglish
             -- literal "*"
             \else
             -- литера "?"
             \fi
{\bf \bs\&} \ifenglish
            -- literal "\bs\&"
            \else
            -- литера "\bs\&
            \fi
{\bf \bs\bs} \ifenglish
             -- literal "\bs"
             \else
             -- литера "\bs"
             \fi
{\bf Examples:}

\ifenglish
One of TopSpeed message patterns looks like this:
\else
Так выглядит одна из моделей сообщений TopSpeed:
\fi
\verb'(&file &line,&column) Error:&body'

\ifenglish
MASM message pattern looks like this:
\else
Так выглядит модель сообщений MASM:
\fi
\verb'&file(&line): error*: &body'

\ifenglish
\subsection{Message Classes}
\else
\subsection{Классы сообщений}
\fi
\public{MessageClasses}

\ifenglish
Currently there are five message classes defined in XDS:

{\bf Text} -- message that doesn't contain any position information, such as
"Now compiling file ququ.mod". This message is inserted into message window
but nothing happens if you click on it.

{\bf Notice} -- message that contains position information but isn't considered
neither warning nor error. For example, XDS History utility (his) uses this
message to show run-time stack trace.

{\bf Warning} -- a warning. Compile or make with warnings is considered successful
but it is recommended to look at warnings because they often indicate some
logical errors in a program. It is usually possible to turn off warnings.

{\bf Error} -- a syntax error detected; compile or make is considered unsuccessful
if there are any

{\bf Severe Error} -- some fatal thing happened such as absence of vital components
of program or tool itself, or exceeding of error limit etc. Usually there's only
one severe error, and it is the last one.
\else
На настоящий момент в XDS определены пять классов сообщений:
{\bf Text} -- сообщение, которое не содержит местоуказания, вроде
"Now compiling file ququ.mod". Это сообщение помещается в окно сообщений,
но когда вы щелкаете по нему, ничего не происходит. 

{\bf Notice} -- сообщение, которое содержит местоуказание, но не связано ни с  
предупреждением, ни с ошибкой. Например, утилита XDS History (his) использует
это сообщение, чтобы отображать во время запуска содержимое стека.

{\bf Warning} -- предупреждение. Компиляция или сборка с предупреждениями считается
успешной, но на них рекомендуется взглянуть, потому что они часто указывают
на логические ошибки в программе. Обычно есть возможность обойтись без предупреждений.

{\bf Error} -- обнаружена синтаксическая ошибка; если такие есть, то компиляция или 
сборка считаются неудачными.

{\bf Severe Error} -- случилось нечто фатальное, такое, как отсутствие жизненно
важных компонент программы или самой утилиты, или превышение лимита ошибок и т.д.
Обычно имеется лишь одна такая ошибка, она же -- последняя.
\fi