\ifenglish
\section{Configure Colors and Fonts}
\else
\section{Настройка цветов и шрифтов}
\fi
\public{COLOR\_DIALOG}
\nominitoc

\ifenglish
Use this dialog to select fonts and colors used in IDE windows.

The dialog contains following pages:
\else
Используйте этот диалог, чтобы выбирать шрифты и цвета, используемые в 
окнах IDE.

Диалог содержит следующие страницы:
\fi

\ref[Workspace]{COLOR\_WORKSPACE\_DIALOG} 
\ifenglish
-- set main window background color
\else
-- установка цвета фона главного окна
\fi
\ref[Windows]{COLOR\_WINDOWS\_DIALOG} 
\ifenglish
-- set default text and background color
   for all windows
\else
-- установка цвета текста и фона, принимаемых по умолчанию для всех окон
\fi

\ref[Editor]{COLOR\_EDIT\_DIALOG} 
 \ifenglish
 -- set text and background colors for
        normal, selected and  highlighted text for different syntax elements
        in different programming languages in the editor windows
 \else
-- установка цвета текста и фона для обычного, отмеченного и выcвеченного текста
   в окнах редактирования, содержащего различные синтаксические элементы 
   различных языков программирования 
\fi

\ref[Messages]{COLOR\_MESSAGES\_DIALOG} 
\ifenglish
-- set colors for {\bf Messages} window
\else                                                         
-- установка цветов для окна {\bf Messages}
\fi
\ref[File list]{COLOR\_FLIST\_DIALOG} 
\ifenglish
-- set colors for {\bf File list} window
        and search results windows.
\else
-- Установка цветов для окна {\bf File list} и окон результатов поиска.
\fi
\ifenglish
Fonts are selected using standard font selection dialog. Colors are selected
using \ref[Color Boxes]{ColorBoxes}.

Colors for editor window are assigned based on \ref{Languages}. The set of
languages may be configured, and the editor knows to which language each text
file belongs. It calls \ref[language driver]{LangDriver} to parse the text
and paints each syntax element in its own color. This dialog allows setting
different colors for all syntax elements and all languages.

Setting all the colors would be incredibly difficult if reasonable and
convinient \ref[Auto color assignment]{AutoColor} scheme was not implemented.
\else
Шрифты выбираются с помощью стандартного диалога выбора шрифта. Цвета выбираются
с помощью \ref[таблиц выбора цвета]{ColorBoxes}.

Выбор цветов для окон редактирования основывается на списке \ref[языков]{Languages}
Набор языков может быть перенастроен, и редактору известно, какому языку 
принадлежит каждый текстовый файл. Он вызывает \ref[драйвер языка]{LangDriver}
для анализа синтаксиса текста, и раскрашивает каждый синтаксический элемент в 
его собственный цвет. Этот диалог позволяет устанавливать различные цвета для
всех синтаксических элементов всех языков.

Установка всех цветов была бы невероятно затруднительной без удобной схемы
\ref[автоматического присваивания цветов]{AutoColor}
\fi
\ifenglish
Sample windows are provided to preview changes you make. There is sample
message window, sample file list window and one sample window for each language
you have configured and one for default language. You may work with
those windows as you will normally work with IDE child windows. You may even
type text in sample editor windows.

Selecting the page in the dialog as well as selecting the language automatically
brings corresponding sample window to the top. Opposite is also true: activating
sample window by the mouse automatically activates corresponding page in the
dialog and the language (if editor window was activated). Moving cursor within
editor sample window also selects syntax element which is under the cursor.
\else
Существует возможность предварительного просмотра произведенных вами изменений на
примере шаблонных окон. Для каждого настроенного вами языка и языка по умолчанию
будет создано по одному окну списка файлов и шаблонному окну. Вы можете работать 
с этими окнами так же, как вы обычно работаете с дочерними окнами IDE, и даже 
набирать текст в шаблонных окнах редактирования.

Выбор страницы диалога, как и выбор языка, автоматически выводит соответствующее
шаблонное окно. Обратное также верно, активация мышью шаблонного окна 
автоматически активирует соответствующую страницу диалога и язык (если было
активировано окно редактирования). Перемещение курсора внутри шаблонного окна
редактирования показывает также, на каком синтаксичексом элементе находится 
курсор.
\fi
%===========================================================================
\ifenglish
\subsection{Workspace}
\else
\subsection{Рабочая среда}
\fi
\public{COLOR\_WORKSPACE\_DIALOG}

\ifenglish
This page contains one \ref[Color Box]{ColorBoxes} that allows setting color
for application workspace (the part of main IDE window that is not obscured
by child windows).

Setting the color to {\bf Auto} sets the color to the default system color.
Default system color for the workspace is the color that is 
set for {\bf Application Background} item in the {\bf Appearance} tab of
{\bf Display Properties} dialog from the Control Panel.

Both solid and dithered colors are allowed for the Workspace.

Changes made for this color are immediately reflected in sample
window in the right part of the dialog.

\else
Эта страница содержит \ref[таблицу выбора цвета]{ColorBoxes}, которая позволяет 
устанавливать цвет рабочей среды (не скрытой дочерними окнами части главного 
окна IDE).

Установка цвета с помощью {\bf Auto} задает систему принимаемых по умолчанию 
цветов. Цвет рабочей среды, принимаемый по умолчанию, -- это цвет, установленный
в опции {\bf Application Background} закладки {\bf Appearance} диалога 
{\bf Display Properties} панели управления.

Для рабочей среды можно использовать как цельные, так и смешанные цвета.

Изменения, произведенные с этим цветом, немедленно отображаются в шаблонном 
окне в правой части диалога.
\fi
%===========================================================================
\ifenglish
\subsection{Windows}
\else
\subsection{Окна}
\fi
\public{COLOR\_WINDOWS\_DIALOG}

\ifenglish
This page contains two \ref[Color Box]{ColorBoxes} that allow setting
colors for all windows in the IDE. Any particular kind of window can override
these settings.

See \ref{AutoColor} to know how these settings affect resulting colors and
what happens if they are set to {\bf Auto}.

Only solid colors are allowed here.

Changes made for these colors are immediately reflected in sample
windows in the right part of the dialog.

\else
Эта страница содержит две \ref[таблицы выбора цвета]{ColorBoxes}, которые 
позволяют устанавливать цвета всех окон IDE. Установки для любого особенного 
типа окна могут подавить эти изменения.

Посмотрите сведения по \ref{AutoColor} чтобы понять, как эти установки влияют 
на получающиеся в результате цвета, и что происходит, если они заданы как {\bf Auto}.

Здесь доступны только цельные цвета.

Изменения, произведенные с этими цветами, немедленно отображаются в шаблонных 
окнах в правой части диалога.
\fi
%===========================================================================
\ifenglish
\subsection{Editor}
\else
\subsection{Редактор}
\fi
\public{COLOR\_EDIT\_DIALOG}

\ifenglish
This page allows setting colors for editor windows. Setting colors is based
on \ref[programming languages]{Languages}. You select language in the {\bf Language}
box, then you select syntax element you wish to set colors to in {\bf Syntax element}
box. After that you can select six colors for this element:
\else
Эта страница позволяет устанавливать цвета для окон редактирования. Установка
цветов основывается на \ref[языках программирования]{Languages}. Вы выбираете
язык в списке {\bf Language}, затем выбираете элемент синтаксиса, для которого 
хотите задать цвет в списке {\bf Syntax element}. После этого вы можете выбрать
для этого элемента шесть цветов:
\fi
\begin{itemize}
\item \ifenglish
      {\bf Background} color
     \else
      Цвет {\bf фона}
      \fi
\item \ifenglish
      {\bf Text} color
     \else
      Цвет {\bf текста}
      \fi
\item \ifenglish
      {\bf Selection} color (background color of text in editor block)
    \else
      Цвет {\bf блока отметки} (цвет фона в блоке редактирования)
     \fi
\item \ifenglish
      {\bf Selected text} color (foreground color of text in editor block)
      \else
      Цвет {\bf отмеченного текста} (цвет текста в блоке редактирования)
      \fi
\item \ifenglish
     {\bf Highlight} color (background color of text in a line that corresponds
to current error or warning message)
      \else
      Цвет {\bf подсветки} (цвет фона текста в строке, соответсвующей текущему
сообщению об ошибке или предупреждении) 
      \fi
\item \ifenglish
      {\bf Highlighted text} color (foreground color of text in a line that corresponds
to current error or warning message)
      \else
      Цвет {\bf подсвеченного текста} (цвет текста в строке, соответсвующей текущему
сообщению об ошибке или предупреждении) 
      \fi
\end{itemize}
\ifenglish
Each of these colors may be configured in each syntax element of each language
separately. To minimize configuration necessary to achieve good results use
the \ref{AutoColor}.

Only solid colors are allowed here.

Note that instead of selecting language and syntax element in selection box you
may activate corresponding sample window and select corresponding syntax element
in the sample text.

Sample texts for various languages are supplied by language drivers. If you
don't have language driver for some language default sample text will be supplied
by system. Of course, this language will have only one syntax element --
{\bf Generic Text}.

You may also set font for editor window by pressing {\bf Set Font} button.
There is only one font you may set for all the editors. This version of editor
doesn't support different fonts in different parts of a window or even in
different windows.

\else
Каждый из этих цветов может быть настроен отдельно для каждого синтаксического 
элемента каждого языка. Чтобы минимизировать объем настройки, используйте 
\ref{AutoColor}.

Здесь доступны только цельные цвета.

Заметьте, что, вместо выбора языка и синтаксического элемента в списке, вы можете
активировать соответствующее шаблонное окно и выбрать соответсвующий 
синтаксический элемент в тексте-образце.

Тексты-образцы различных языков предоставляются драйверами этих языков.
Если у вас нет драйвера для какого-то из языков, то система сама составит текст-
-образец. Разумеется в этом тексте будет только один синтаксичечкий элемент -- 
{\bf Generic Text}.

Вы также можете установить шрифт для окна редактирования путем нажатия кнопки
{\bf Set Font}. Для всех редакторов можно установить только один шрифт. Данная
версия редактора не поддерживает использование различных шрифтов в различных
участках окна, или даже в разных окнах.
\fi
%===========================================================================
\ifenglish
\subsection{Messages}
\else
\subsection{Сообщения}
\fi
\public{COLOR\_MESSAGES\_DIALOG}

\ifenglish
This page contains two \ref[Color Box]{ColorBoxes} that allow setting
colors for message window in the IDE. These settings override default windows
colors defined in {\bf Windows} page of this dialog.

See \ref{AutoColor} to know details of color assignment and overriding.

Only solid colors are allowed here.

The font for message window can also be configured here.
The default is {\bf System} font.

Changes made for these colors and fonts are immediately reflected in sample
message window in the right part of the dialog.

\else
Эта страница содержит две \ref[таблицы цветов]{ColorBoxes}, которые позволяют 
устанавливать цвета для окон сообщений в IDE. Эти установки подавляют установки
оконных цветов по умолчанию, содержащиеся в странице {\bf Windows} этого диалога.

Посмотрите сведения по \ref{AutoColor}, чтобы ознакомиться с заданием
и подавлением цветов подробнее.
 
Здесь доступны только цельные цвета.

Здесь также можно настроить шрифт окна сообщений. По умолчанию
задается шрифт {\bf System}.

Изменения, производимые с этими цветами и шрифтами немедленно отображаются в 
шаблонном окне в правой части диалога.
\fi
%===========================================================================
\ifenglish
\subsection{File list}
\else
\subsection{Список файлов}
\fi
\public{COLOR\_FLIST\_DIALOG}

\ifenglish
This page contains two \ref[Color Box]{ColorBoxes} that allow setting
colors for file list window in the IDE. These settings override default windows
colors defined in {\bf Windows} page of this dialog.

See \ref{AutoColor} to know details of color assignment and overriding.

Only solid colors are allowed here. The colors defined here apply also for
search results windows.

The font for file list window can also be configured here.
The default is {\bf System} font.

Changes made for these colors and fonts are immediately reflected in sample
file list window in the right part of the dialog.

\else
Эта страница содержит две \ref[таблицы выбора цвета]{ColorBoxes}, которые 
позволяют устанавливать цвета для окон списков файлов IDE. Эти установки 
подавляют установки оконных цветов по умолчанию, 
содержащиеся на странице {\bf Windows} этого диалога.

Посмотрите сведения по \ref{AutoColor} чтобы ознакомиться с заданием
и подавлением цветов подробнее.
 
Здесь доступны только цельные цвета.

Здесь также можно настроить шрифт окна списка файлов. По умолчанию
задается шрифт {\bf System}.                                 

Изменения, производимые с этими цветами и шрифтами, немедленно отображаются в 
шаблонном окне в правой части диалога.
\fi
%===========================================================================
\ifenglish
\subsection{Color Boxes}
\else
\subsection{Таблицы выбора цвета}
\fi
\public{ColorBoxes}

\ifenglish
Color box control is the drop-down list filled with colors. It contains 16 basic
colors such as green, blue, magenta etc. It also contains {\bf Custom} element.
Selecting this element invokes the standard Color selection dialog. After the
color is selected the control will show current selection as "Custom" and it
will be filled with selected color.

The control also contains {\bf Auto} element. Selecting this element assigns
default color to the element whose color is selected. {\bf Auto} is the
default value for all color boxes.

Different elements have different default values but all have some reasonable
defaults. If nothing is selected at all, those default values originate from
Control Panel.

The {\bf Auto} item in the color box has color that would be assigned if
this item is selected.

Note that only solid colors are available for all settings with the only
exception of the Workspace.

\else
Контроль "таблица цветов" -- это ниспадающий список, содержащий цвета Он содержит
16 основных цветов, таких как зеленый, синий, магента и т. д. Он также содержит 
элемент {\bf Custom}. Выбор этого элемента вызывает стандартный диалог выбора
цвета. После выбора цвета контроль отобразит текущую позицию цвета как "Custom"
и заполнит ее выбранным цветом.

Контроль содержит также элемент {\bf Auto}. Выбор этого элемента присваивает
цвет по умолчанию выбранной позиции цвета. {\bf Auto} -- это значение по 
умолчанию для всех списков цветов.

Различные элементы обладают различными значениями, принятыми по умолчанию, но
все эти значения обоснованы. Если цвет не выбран, то эти значения по умолчанию 
соответствуют установкам, заданным на панели управления.

Пункт {\bf Auto} списка цветов имеет цвет, который будет задан при выборе этого 
пункта.

Заметьте, что только цельные цвета доступны для всех установок, за исключением 
рабочей среды.
\fi
%---------------------------------------------------------------------------
\ifenglish
\subsection{Auto Color Assignment}
\else
\subsection{Автоматическое задание цвета}
\fi
\public{AutoColor}

\ifenglish
The following rules are used to determine colors for the elements in the IDE
for which colors are not set explicitly:

\else
Следующие правила используются чтобы определять цвета элементов IDE, для которых 
цвета явно не заданы:
\fi
\ifenglish
{\bf The Workspace} color, if set to {\bf Auto}, is the color that is 
set for {\bf Application Background} item in the {\bf Appearance} tab.

\else
Цвет {\bf The Workspace}, установленный в {\bf Auto} есть цвет установленный 
в пункте {\bf Application Background} закладки {\bf Appearance}.
\fi
\ifenglish
{\bf The Messages Window} color, if set to {\bf Auto}, is the color that
is set for all windows in {\bf Windows} page. This applies to both
foreground and background colors.

{\bf The File Lists Window} color, if set to {\bf Auto}, is the color that
is set for all windows in {\bf Windows} page. This applies to both
foreground and background colors.

{\bf The color of some syntax element of some language}, if set to {\bf Auto},
is the color set for the {\bf Generic Text} element for this language. This applies
to background, text, selection and selected text colors.

\else
Цвет {\bf The Messages Window}, установленный в {\bf Auto}, есть цвет, 
установленный для всех окон в странице {\bf Windows}. Это касается как цвета 
фона, так и цвета текста.

Цвет {\bf The File Lists Window}, установленный в {\bf Auto}, есть цвет, 
установленный для всех окон в странице {\bf Windows}. Это касается как цвета 
фона, так и цвета текста.

{\bf Цвет некоторого синтаксического элемента некоторого языка}, установленный 
в {\bf Auto}, есть цвет, установленный для элемента {\bf Generic Text} этого 
языка. Это касается цветов фона, текста, отметки и отмеченного текста.
\fi
\ifenglish
{\bf The highlight color} of some syntax element of some language,
if set to {\bf Auto} is calculated by following rule. It is set to
highlight color of {\bf Generic Text} element of the current language if it is
not set to {\bf Auto}. Otherwise it is set to {\bf Default} language highlight
color if it is not set to {\bf Auto}. Otherwise the color is set to 
the enlightened color of selection for the same
element of the same language. Enlightening is achieved by adding 128 to
color's red, green and blue component with the only exception:
white color, when enlightened, is set to light gray (192, 192, 192).

\else
{\bf Цвет высвечивания} некоторого синтаксического элемента некоторого языка, 
установленный в {\bf Auto}, вычисляется по следующему правилу. Он совпадает с 
цветом элемента {\bf Generic Text} текущего языка, если тот не установлен в 
{\bf Auto}. Иначе он совпадает с цветом высвечивания языка {\bf Default}, если
тот не установлен в {\bf Auto}. Иначе цвет принимает значение, равное 
осветленному тексту отметки этого же элемента этого же языка. Осветление 
достигается путем увеличения красной, зеленой и синей компонент цвета на 128,
за одним исключением -- осветленный белый становится светло-серым (192, 192, 192).
\fi
\ifenglish
{\bf The highlighted text color} of some syntax element of some language,
if set to {\bf Auto} is calculated by following rule. It is set to
highlighted text color of {\bf Generic Text} element of the current language if it is
not set to {\bf Auto}. Otherwise it is set to {\bf Default} language highlighted text
color if it is not set to {\bf Auto}. Otherwise the color is set to
the normal text color of the same
element of the same language.

{\bf The Generic Text element of some language} color, if set to {\bf Auto},
is the color set for the {\bf Default} language.  This applies
to background, text, selection, selected text, highlight and highlighted text
colors.

\else
{\bfЦвет высвеченного текста} некоторого синтаксического элемента некоторого
языка, установленный в {\bf Auto}, вычисляется по следующему правилу. Он совпадает
с цветом высвеченного текста элемента {\bf Generic Text} текущего языка, если тот
не установлен в {\bf Auto}. Иначе он совпадает с цветом высвеченного текста языка
{\bf Default}, если тот не установлен в {\bf Auto}. Иначе цвет принимает значение
цвета обычного текста этого же элемента этого же языка.

{\bf Элемент Generic Text некоторого языка}, установленный в {\bf Auto},
есть цвет, установленный для языка {\bf Default}. Это касается цветов фона,
текста, отметки, отмеченного текста, высветки и высвеченного текста.
\fi
\ifenglish
{\bf The Default Language} color, if set to {\bf Auto},
is the color that is set for all windows in {\bf Windows} page.
This applies to normal text and background colors.

{\bf The Default Language selection and selection text colors},
if set to {\bf Auto}, are the colors
set for {\bf Selected Items} item in the {\bf Appearance} tab of
{\bf Display Properties} dialog from the Control Panel.

{\bf The default windows colors}, if set to {\bf Auto}, are the colors
set for {\bf Window} item in the {\bf Appearance} tab of
{\bf Display Properties} dialog from the Control Panel.

\else
Цвет языка {\bf Default}, установленный в {\bf Auto}, есть цвет, установленный
для всех окон в странице {\bf Windows}. Это касается обычных цветов текста и
фона.

{\bf Цвета отметки и отмеченного текста языка Default}, установленные в {\bf Auto}
есть цвета, установленные в пункте {\bf Selected Items} закладки {\bf Appearance}
диалога {\bf Display Properties} панели управления.

{\bf Цвета окон, задаваемые по умолчанию}, установленные в {\bf Auto}, есть цвета,
установленные в пункте {\bf Window} закладки {\bf Appearance}
диалога {\bf Display Properties} панели управления.
\fi
%===========================================================================

\begin{popup}
\caption{Apply}
\public{COLOR\_APPLY}

\ifenglish
Select the {\bf Apply} pushbutton to apply the changes you've made to
font and color settings to IDE windows without dismissing the dialog.

\else
Нажмите кнопку {\bf Apply}, чтобы применить измененные вами установки
для окон IDE, не закрывая диалог.
\fi
\end{popup}

\begin{popup}
\caption{OK}
\public{COLOR\_OK}

\ifenglish
Select the {\bf OK} pushbutton to confirm the changes you've made to
color and font settings.

\else
Нажмите кнопку {\bf OK}, чтобы подтвердить произведенные вами изменения
установок цвета и шрифта.
\fi
\end{popup}

\begin{popup}
\subsection{Cancel}
\public{COLOR\_CANCEL}

\ifenglish
Select the {\bf Cancel} pushbutton if you decided not to change
fonts and colors.

\else
Нажмите кнопку {\bf Cancel}, если решили не производить изменений установок
цвета и шрифта.
\fi
\end{popup}

\begin{popup}
\ifenglish
\caption{Sample screen}
\else
\caption{Область шаблонных окон}
\fi
\public{COLOR\_SAMPLE}

\ifenglish
This area contains sample windows of three types - editor, messages, and
project files list. You may change their position, size, and z-order in
a regular way - using the mouse or the system menus. The system menus also
contain commands which may be used to change the colors and font used
in every window type.

\else
Эта область содержит шаблонные окна трех типов - редактирования, сообщений и
списка файлов проекта. Вы можете изменять их положение и размер
стандартным образом - используя мышь и системные меню. Системные меню содержат
также команды, которые можно использовать, чтобы менять цвета и шрифт,
используемые в каждом типе окна.
\fi
\end{popup}

%---------------------------------------------------------------------------
\begin{popup}
\ifenglish
\caption{Workspace color}
\else
\caption{Цвет рабочей среды}
\fi
\public{COLOR\_WORKSPACE\_BACK}

\ifenglish
Select the color for the workspace, that is, the color of a space within
the IDE frame that is free from MDI windows.

Selecting {\bf Auto} here sets the color to the default system color (the color
set for {\bf Application Background} item in the {\bf Appearance} tab of
{\bf Display Properties} dialog from the Control Panel)

\else
Выберите цвет для рабочей среды, то есть, цвет пространства IDE, не занятого
MDI окнами.

Здесь выбор {\bf Auto} устанавливает выбираемый системой по умолчанию цвет.
(цвет, установленный в пункте {\bf Application Background} закладки
{\bf Appearance} диалога {\bf Display Properties} контрольной панели).
\fi
\end{popup}

%---------------------------------------------------------------------------

\begin{popup}
\ifenglish
\caption{Background}
\else
\caption{Фон}
\fi
\public{COLOR\_WINDOWS\_BACK}

\ifenglish
Select the color for background of all windows in the IDE.
You may override this color for particular types of windows (messages window,
file list or editors for different languages).

Selecting {\bf Auto} here sets the color to the default system color
(the background color set for {\bf Window} item in the {\bf Appearance} page of
{\bf Display Properties} dialog from the Control Panel)

\else
Выберите цвет фона всех окон IDE.
Вы можете подавить этот выбор в случае отдельных типов окон
(окон сообщений списков файлов или редакторов для различных языков).

Здесь выбор {\bf Auto} устанавливает выбираемый системой по умолчанию цвет
(цвет, установленный в пункте {\bf Application Background} закладки
{\bf Appearance} диалога {\bf Display Properties} контрольной панели).
\fi
\end{popup}

\begin{popup}
\ifenglish
\caption{Text}
\else
\caption{Текст}
\fi
\public{COLOR\_WINDOWS\_FORE}

\ifenglish
Select the color for text of all windows in the IDE.
You may override this color for particular types of windows (messages window,
file list or editors for different languages).

Selecting {\bf Auto} here sets the color to the default system color
(the text color set for {\bf Window} item in the {\bf Appearance} page of
{\bf Display Properties} dialog from the Control Panel)

\else
Выберите цвет текста для всех окон IDE.
Вы можете подавить этот выбор в случае отдельных типов окон
(окон сообщений списков файлов или редакторов для различных языков).

Здесь выбор {\bf Auto} устанавливает выбираемый системой по умолчанию цвет.
(цвет, установленный в пункте {\bf Application Background} закладки
{\bf Appearance} диалога {\bf Display Properties} контрольной панели).
\fi
\end{popup}
%---------------------------------------------------------------------------

\begin{popup}
\ifenglish
\caption{Background}
\else
\caption{Фон}
\fi
\public{COLOR\_MESSAGES\_BACK}

\ifenglish
Select the color for background of messages window.

Selecting {\bf Auto} here sets the color to the default window color
defined in {\bf Background} selection box of the {\bf Windows} page of this
dialog.

\else
Выберите цвет фона окна сообщений.

Здесь выбор {\bf Auto} устанавливает выбираемый по умолчанию цвет,
определенный в списке {\bf Background} страницы {\bf Windows} этого диалога.
\fi
\end{popup}

\begin{popup}
\ifenglish
\caption{Text}
\else
\caption{Текст}
\fi
\public{COLOR\_MESSAGES\_FORE}

\ifenglish
Select the color for text of messages window.

Selecting {\bf Auto} here sets the color to the default window text color
defined in {\bf Text} selection box of the {\bf Windows} page of this
dialog.

\else
Выберите цвет текста для окна сообщений.

Здесь выбор {\bf Auto} устанавливает выбираемый по умолчанию цвет,
определенный в списке {\bf Background} страницы {\bf Windows} этого диалога.
\fi
\end{popup}

\begin{popup}
\ifenglish
\caption{Set Font}
\else
\caption{Установить шрифт}
\fi
\public{COLOR\_MESSAGES\_FONT}
\ifenglish
Set the font for messages window. The default is the {\bf System} font.
\else
Установите шрифт окна собщений. По умолчанию устанавливается шрифт {\bf System}.
\fi
\end{popup}

%---------------------------------------------------------------------------

\begin{popup}
\ifenglish
\caption{Background}
\else
\caption{Фон}
\fi
\public{COLOR\_FLIST\_BACK}
\ifenglish
Select the color for background of file list window.

Selecting {\bf Auto} here sets the color to the default window color
defined in {\bf Background} selection box of the {\bf Windows} page of this
dialog.

\else
Выберите цвет фона для окна списка файлов.

Здесь выбор {\bf Auto} устанавливает выбираемый по умолчанию цвет,
определенный в списке {\bf Background} страницы {\bf Windows} этого диалога.
\fi
\end{popup}

\begin{popup}
\ifenglish
\caption{Text}
\else
\caption{Текст}
\fi
\public{COLOR\_FLIST\_FORE}
\ifenglish
Select the color for text of file list window.

Selecting {\bf Auto} here sets the color to the default window text color
defined in {\bf Text} selection box of the {\bf Windows} page of this
dialog.
\else

Выберите цвет текста для окна списка файлов.

Здесь выбор {\bf Auto} устанавливает выбираемый по умолчанию цвет текста,
определенный в списке {\bf Text} страницы {\bf Windows} этого диалога.
\fi
\end{popup}

\begin{popup}
\ifenglish
\caption{Set Font}
\else
\caption{Установить шрифт}
\fi
\public{COLOR\_FLIST\_FONT}

\ifenglish
Set the font for file list window. The default is the {\bf System} font.

\else
Установите шрифт для окна списка файлов. По умолчанию устанавливается шрифт
{\bf System}.
\fi
\end{popup}

%---------------------------------------------------------------------------

\begin{popup}
\ifenglish
\caption{Language}
\else
\caption{Язык}
\fi
\public{COLOR\_EDIT\_LANG}

\ifenglish
Select the language for which you wish to select colors.
List of languages is configured by {\bf Configure Languages} dialog.

Selecting {\bf Default} language allows setting of default colors values.
These values will be applied for all windows for which the language is unknown.
Besides, these values will be used as default values for languages elements
that you left unspecified.

\else
Выберите язык, для которого хотите определить цвета.
Список языков настраивается с помощью диалога {\bf Configure Languages}.

Выбор языка {\bf Default} позволяет устаналивать значения цветов, выбираемые
по умолчанию. Эти значения будут применены ко всем окнам, язык которых
неизвестен. Кроме того, эти значения будут применяться как значения,
используемые по умолчанию, для всех элементов языков, которые вы не определите
сами.
\fi
\end{popup}

\begin{popup}
\ifenglish
\caption{Syntax elements}
\else
\caption{Синтаксические элементы}
\fi
\public{COLOR\_EDIT\_ELEM}

\ifenglish
Once you have selected a {\bf Language}, you may choose {\bf Syntax element}
you wish to define colors to. The list of allowed elements depends on the
languages. If {\bf Default} is selected in a language selection box this
list is empty. Otherwise it is filled by values supplied by {\bf Language Driver}.
Typically the elements correspond to classes of tokens in which text is
parsed, such as identifiers, keywords, comments, etc.

The first element is always {\bf "Generic text"}. It corresponds to the text
that doesn't belong to any known syntax element. The formatting you define
for this element becomes default for all other syntax elements and applies
to them unless you explicitly override it.

\else
Определив {\bf Язык} вы можете выбрать {\bf Синтаксический элемент} для,
которого вы хотите определить цвета. Список доступных элементов зависит от
языка. Если в списке выбора языка выбран язык {\bf Default}, то этот список пуст.
Иначе он заполнен значениями, поставляемыми {\bf Драйвером языка}. Обычно элементы
соответствуют классам лексем, на которые разбивается язык, таким как
идентификаторы, ключевые слова, комментарии и т. д.

Первый элемент -- это всегда {\bf "Generic text"}. Он соответствует тексту,
который не принадлежит ни одному известному синтаксическому элементу.
Формат, который вы определите для этого элемента, станет принимаемым по
умолчанию для всех остальных синтаксических элементов, и применяется к ним,
если вы явно его для них не перезададите.
\fi
\end{popup}

\begin{popup}
\ifenglish
\caption{Background}
\else
\caption{Фон}
\fi
\public{COLOR\_EDIT\_BACK}

\ifenglish
Select background color for currently selected syntax element of currently selected
language.

\else
Выберите цвет фона для выбранного в настоящий момент синтаксического элемента
текущего языка.
\fi
\ifenglish
Selecting {\bf Auto} value here sets the default value for the color. The default
value is:

\else
Здесь выбор {\bf Auto} устанавливает принимаемое по умолчанию значение цвета
Значение принимаемое по умолчанию это:
\fi
\begin{itemize}
\item 
 
 \ifenglish
 the background color for {\bf Generic text} element for current language
 if it is set  to value other than {\bf Auto}, otherwise
 \else
 цвет фона элемента {\bf Generic text} текущего языка, если он задан
 отличным от {\bf Auto} значением, иначе
 \fi
\item 

 \ifenglish
 the background color for {\bf Default} language
 if it is set  to value other than {\bf Auto}, otherwise
 \else
 цвет фона для языка {\bf Default}, если он задан отличным от {\bf Auto}
 значением, иначе
 \fi
\item 
 \ifenglish
 the background color specified in {\bf Windows} page of a dialog
 if it is set  to value other than {\bf Auto}, otherwise
 \else
 цвет фона, определенный страницей {\bf Windows} диалога, если он
 задан отличным от {\bf Auto} значением, иначе
 \fi
\item 
 \ifenglish
 the system default window color defined in Control Panel
 \else
 принимаемый системой по умолчанию цвет, определенный в панели упраления
 \fi
\end{itemize}
\end{popup}

\begin{popup}
\ifenglish
\caption{Text}
\else
\caption{Текст}
\fi
\public{COLOR\_EDIT\_FORE}

\ifenglish
Select text color for currently selected syntax element of currently selected
language.
\else
Выберите цвет текста для выбранного в настоящий момент синтаксического элемента
текущего языка.
\fi

\ifenglish
Selecting {\bf Auto} value here sets the default value for the color. The default
value is:
\else
Здесь выбор {\bf Auto} устанавливает принимаемое по умолчанию значение цвета.
Значение принимаемое по умолчанию это:
\fi
\begin{itemize}
\item 
 \ifenglish
 the text color for {\bf Generic text} element for current language
 if it is set  to value other than {\bf Auto}, otherwise
 \else
 цвет текста элемента {\bf Generic text} текущего языка, если он задан
 отличным от {\bf Auto} значением, иначе
 \fi
\item 
 \ifenglish
 the text color for {\bf Default} language
 if it is set  to value other than {\bf Auto}, otherwise
 \else
  цвет фона для языка {\bf Default}, если он задан отличным от {\bf Auto}
 значением, иначе
 \fi
\item 
 \ifenglish
 the text color specified in {\bf Windows} page of a dialog
 if it is set  to value other than {\bf Auto}, otherwise
 \else
 цвет фона, определенный страницей {\bf Windows} диалога, если он
 задан отличным от {\bf Auto} значением, иначе
 \fi
\item 
 \ifenglish
 the system default window text color defined in Control Panel
 \else
 принимаемый системой по умолчанию цвет, определенный в панели упраления
 \fi
\end{itemize}
\end{popup}

\begin{popup}
\ifenglish
\caption{Selection}
\else
\caption{Отметка}
\fi
\public{COLOR\_EDIT\_SELBACK}

\ifenglish
Set background color for currently selected syntax element of currently selected
language when it appears within the editor block.
\else
Установите цвет фона отмеченого в данный момент синтаксического элемента
текущего языка, когда он появится в области редактирования.
\fi
\ifenglish
Selecting {\bf Auto} value here sets the default value for the color. The default
value is:

\else
Здесь выбор {\bf Auto} устанавливает принимаемое по умолчанию значение цвета.
Значение, принимаемое по умолчанию это: 
\fi

\begin{itemize}
\item 
 \ifenglish
 the selection color for {\bf Generic text} element for current language
 if it is set  to value other than {\bf Auto}, otherwise
 \else
 цвет отметки элемента {\bf Generic text} текущего языка, если он задан
  отличным от {\bf Auto} значением, иначе
 \fi
\item 
 \ifenglish
 the selection color for {\bf Default} language
 if it is set  to value other than {\bf Auto}, otherwise
 \else
 цвет отметки для языка {\bf Default}, если он задан отличным от {\bf Auto}
 значением, иначе
 \fi
\item 
 \ifenglish
the system default {\bf Selected Items} background color defined
in Control Panel
  \else
  принимаемый системой по умолчанию цвет  {\bf Selected Items},
  определенный в панели упраления
  \fi
\end{itemize}
\end{popup}

\begin{popup}
\ifenglish
\caption{Selected text}
\else
\caption{Отмеченный текст}
\fi
\public{COLOR\_EDIT\_SELFORE}

\ifenglish
Set text color for currently selected syntax element of currently selected
language when it appears within the editor block.

Selecting {\bf Auto} value here sets the default value for the color. The default
value is:

\begin{itemize}
\item the selected text color for {\bf Generic text} element for current language
 if it is set  to value other than {\bf Auto}, otherwise
\item the selected text color for {\bf Default} language
 if it is set  to value other than {\bf Auto}, otherwise
\item the system default {\bf Selected Items} text color defined
in Control Panel
\end{itemize}

\else
Установите цвет текста выбранного в данный момент синтаксического элемента
текущего языка, когда он появится в области редактирования.

Здесь выбор {\bf Auto} устанавливает принимаемое по умолчанию значение цвета.
Значение принимаемое по умолчанию это:

\begin{itemize}
\item
 цвет отмеченного текста элемента {\bf Generic text} текущего языка, если он задан
 отличным от {\bf Auto} значением, иначе
\item
 цвет отмеченного текста для языка {\bf Default}, если он задан отличным от {\bf Auto}
 значением, иначе
\item
  принимаемый системой по умолчанию цвет  {\bf Selected Items},
  определенный в панели упраления

\end{itemize}
\fi
\end{popup}

\begin{popup}
\ifenglish
\caption{Highlight}
\else
\caption{Подсветка}
\fi
\public{COLOR\_EDIT\_HIGHBACK}

\ifenglish
Set background color for currently selected syntax element of currently selected
language when it appears within the editor highlighted line.
A line with current error is highlighted when the error is selected in the
messages window.

\else
Установите цвет фона для выбранного в настоящий момент синтаксического элемента,
когда он появляется внутри строки подсветки редактора.
Строка, содержащая текущую ошибку, высвечивается, когда ошибка выбрана в окне
сообщений.
\fi
\ifenglish
Selecting {\bf Auto} value here sets the default value for the color. The default
value is:

\else
Здесь выбор {\bf Auto} устанавливает принимаемое по умолчанию значение цвета.
Значение принимаемое по умолчанию это:
\fi
\begin{itemize}
\item 
 \ifenglish
 the highlight color for {\bf Generic text} element for current language
 if it is set  to value other than {\bf Auto}, otherwise
 \else
 Цвет высвечивания для элемента {\bf Generic text} текущего языка, если он
 задан отличным от {\bf Auto} значением, иначе
 \fi
\item 
 \ifenglish
 the highlight color for {\bf Default} language
 if it is set  to value other than {\bf Auto}, otherwise
 \else
 Цвет высвечивания языка {\bf Default}, если он
 задан отличным от {\bf Auto} значением, иначе
 \fi
\item 
 \ifenglish
 the enlightened color of selection set for current syntax element for
current language.
 \else
 осветленный цвет отметки, установленный для текущего синтаксического
 элемента текущего языка.
  \fi
 \ifenglish
The color is enlightened by adding some values for its red,
green and blue components.
 \else
 Осветление цвета происходит путем увеличивания на определенное значение
 его красной, зеленой и синей компонент.
 \fi
\end{itemize}
\end{popup}

\begin{popup}
\ifenglish
\caption{Hihlighted text}
\else
\caption{Подсвеченный текст}
\fi
\public{COLOR\_EDIT\_HIGHFORE}

\ifenglish
Set text color for currently selected syntax element of currently selected
language when it appears within the editor highlighted line.
A line with current error is highlighted when the error is selected in the
messages window.

\else
Установите цвет текста для выбранного в настоящий момент синтаксического элемента,
когда он появляется внутри строки подсветки редактора.
Строка, содержащая текущую ошибку, высвечивается, когда ошибка выбрана в окне
сообщений.
\fi
\ifenglish
Selecting {\bf Auto} value here sets the default value for the color. The default
value is:

\else
Здесь выбор {\bf Auto} устанавливает принимаемое по умолчанию значение цвета
Значение принимаемое по умолчанию это:
\fi
\begin{itemize}
\item 
 \ifenglish
 the highlight text color for {\bf Generic text} element for current language
 if it is set  to value other than {\bf Auto}, otherwise
\else
  Цвет высвеченного текста для элемента {\bf Generic text} текущего языка, если он
 задан отличным от {\bf Auto} значением, иначе
 \fi
\item 
 \ifenglish
 the highlight text color for {\bf Default} language
 if it is set  to value other than {\bf Auto}, otherwise
 \else
  Цвет высвеченного текста языка {\bf Default}, если он
  задан отличным от {\bf Auto} значением, иначе
 \fi
\item 
 \ifenglish
 the normal text color for current syntax element for current language.
 \else
 обычный цвет текста текущего синтаксического элемента текущего языка.
 \fi
\end{itemize}
\end{popup}

\begin{popup}
\ifenglish
\caption{Set Font}
\else
\caption{Установить шрифт}
\fi
\public{COLOR\_EDIT\_FONT}
\ifenglish
Select font for all edit windows using standard font selection dialog.
Only fixed-pitch fonts are available. The default is the system fixed font
(usually {\bf FixedSys}).

In current version of the editor font can only be selected for all windows
at once.

\else
Выберите шрифт для всех окон редактирования, используя стандартный диалог
выбора шрифта. Доступны только моноширинные шрифты. По умолчанию устанавливается 
системный шрифт (обычно {\bf FixedSys}).

В настоящей версии редактора шрифт может быть выбран только для всех окон сразу.
\fi
\end{popup}

%---------------------------------------------------------------------------

