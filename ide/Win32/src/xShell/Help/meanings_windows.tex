\section{WinBar}
\public{WinBar}

\ifenglish
A {\bf WinBar} is a bar of buttons usually situated at the bottom of IDE frame
window that looks similar to Windows 95 TaskBar.
Each button corresponds to opened window. Pressing a button activates corresponding
window.
\else
{\bf WinBar} -- это поле кнопок, обычно расположенное внизу окна IDE оболочки,
и похожее на панель задач Windows 95.
Каждая кнопка соответствует открытому окну. Нажатие кнопки активирует соответствующее 
окно.
\fi
\ifenglish
Buttons are ordered by window numbers.

When more windows are opened the buttons get smaller and smaller. However,
there is lower limit on their size. When the limit is reached the buttons are arranged 
in two and more rows.

Pressing right mouse button on the WinBar button invokes System menu of
corresponding window.

Each button may contain window number, file name and an icon. You may configure
the appearance of buttons in \ref[WinBar]{CFGWIN\_WINBAR\_DIALOG} page of
\ref[Configure Windows System]{CFGWIN\_DIALOG} dialog that appears via
\ref[Configure]{M\_CONFIGURE}:\ref[Windows]{IDM\_CONFIGUREWINDOWS} menu.

There you may also configure whether WinBar should be at the top. at the bottom
or nowhere at all.
\else
Кнопки упорядочены в соответствие с номерами окон.

Когда открываются, новые окна кнопки становятся все меньше и меньше. Тем не менее, 
существует нижнее ограничение на их размер. Когда оно достигается, кнопки 
перестраиваются в два и более рядов.

Нажатие правой кнопки мыши на WinBar вызывает появление системного меню в 
соответствующем окне.

Каждая кнопка может содержать номер окна, имя файла или иконку. Вы можете
настраивать внешний вид кнопок в страничке \ref[WinBar]{CFGWIN\_WINBAR\_DIALOG}
диалога \ref[Configure Windows System]{CFGWIN\_DIALOG}, что появляется через
меню \ref[Configure]{M\_CONFIGURE}:\ref[Windows]{IDM\_CONFIGUREWINDOWS}.

Там вы также можете указать, где должен находиться WinBar - вверху, внизу или
вообще нигде.
\fi