\nominitoc
\ifenglish
Use the {\bf Main Menu} to open pull-down menus which contain various 
Environment commands. Following submenus are available:
\else
Используйте {\bf Main Menu}, чтобы разворачивать ниспадающие меню, которые
содержат различные команды среды. Доступны следующие подменю: 
\fi

\ref[File]{M\_FILE} -
 \ifenglish
 File operations (open, save, close, etc.)
 \else
 Действия с файлами (открыть, сохранить, закрыть и т. д.)
 \fi

\ref[Edit]{M\_EDIT} - 
 \ifenglish
 editor commands (find/replace, clipboard etc.)
 \else
 Команды редактора (найти/заменить, буфер обмена и т. д.)
 \fi

\ref[Project]{M\_PROJECT} - 
 \ifenglish
 opening, closing and editing of projects
 \else
 Открытие, закрытие и редактирование проектов
 \fi

\ref[Tools]{M\_TOOLS} - 
 \ifenglish
 external tools (predefined and custom)
 \else
 Внешние утилиты (стандартные и разработанные пользователем)
 \fi

\ref[Debug]{M\_DEBUG} - 
 \ifenglish
 running user program on itself and under debugger
 \else
 Запуск программы пользователя - самой по себе и в отладчике
 \fi

\ref[Messages]{M\_MESSAGES} - 
 \ifenglish
 messages window control
 \else
 Контроль окна сообщений
 \fi

\ref[Configure]{M\_CONFIGURE} - 
 \ifenglish
 configuring various properties of IDE
 \else
 Настройка различных свойств IDE
 \fi

\ref[Window]{M\_WINDOWS} - 
 \ifenglish
 arranging of MDI windows
 \else
 Размещение MDI-окон
 \fi

\ref[Help]{M\_HELP} - 
 \ifenglish
 displaying various kinds of help
 \else
 Отображение различного рода подсказок
 \fi

\section{File Menu}
\public{M\_FILE}
\index{File Menu}
\nominitoc

\ifenglish
The {\bf File} menu contains commands that you use to create,
open, and save source files. It also contains a list of files
that were most recently opened or saved using the {\bf Save As} command.
The following commands appear in the {\bf File} menu:
\else
Меню {\bf File} содержит команды, которые вы используете, чтобы создавать,
открывать и сохранять исходные файлы. Оно также содержит список наиболее часто 
открывавшихся или сохраненных командой {\bf Save As} файлов.
В меню {\bf File} содержатся следующие команды:
\fi

\begin{description}
\item \ref[New]{IDM\_NEW} - 
  \ifenglish
    Opens new empty window
  \else 
    Открывает новое пустое окно
  \fi
\item \ref[Open]{IDM\_OPEN} -
  \ifenglish
    Opens a new window and load a file into it
  \else
    Открывает новое окно и загружает в него файл
  \fi
\item \ref[Save]{IDM\_SAVE} -
  \ifenglish
    Saves any changes made to the current file
  \else
    Сохраняет любые произведенные в текущем файле изменения
  \fi
\item \ref[Save As]{IDM\_SAVEAS} -
  \ifenglish
    Saves the current file using a new name
  \else
    Сохраняет текущий файл под новым именем
  \fi
\item \ref[Save All]{IDM\_SAVEALL} - 
  \ifenglish
    Saves all changed files
  \else
    Сохраняет все измененные файлы
  \fi
\item \ref[Share]{IDM\_SHARE} -
  \ifenglish
    Opens another window with same file
  \else
    Открывает новое окно с тем же самым файлом
  \fi
\item \ref[Print]{IDM\_PRINT} -
  \ifenglish
    Prints the current file or selection
  \else
    Распечатывает содержимое текущего файла или отмеченный текст
  \fi
\item \ref[Close]{IDM\_CLOSE} -
  \ifenglish
    Closes the current window
  \else
    Закрывает текущее окно
  \fi
\item \ref[Exit]{IDM\_EXIT} -
  \ifenglish
    Exits the IDE
  \else
    Закрывает IDE 
  \fi
\item \ref[More Files]{IDM\_MOREFILES} -
  \ifenglish
    Show long history of opened files
  \else
    Отображает полный архив сведений по открывавшимся файлам
  \fi
\end{description}

\subsection{File: New}
\public{IDM\_NEW}

\ifenglish
Use the {\bf New} command to create a new empty window.

The words "New file" will appear in its title bar,
You'll have to specify a name for the new file when you save it.
\else
Используйте команду {\bf New}, чтобы создать новое пустое окно

В заголовке нового окна появятся слова "New file" .
При сохранении нового файла вы должны будете указать его имя.
\fi

\subsection{File: Open}
\public{IDM\_OPEN}

\ifenglish
Use the {\bf Open} command to load an existing file into
a new window.

A dialog will appear, providing different methods of
loading files.

Keyboard shortcut: {\bf F3}
\else
Используйте команду {\bf Open}, чтобы загрузить существующий файл в новое 
окно.

Появится диалог, обеспечивающий различные способы загрузки файлов.

Комбинация клавиш: {\bf F3} 
\fi

\subsection{File; Save}
\public{IDM\_SAVE}

\ifenglish
Use the {\bf Save} command to store the current file to disk.
After the file is saved, the text remains in the window
so that you can continue editing it. However, undo buffer is cleared
so you can't undo the changes made before file was saved.

Keyboard shortcut: {\bf F2}

{\bf Note:} If you are editing a new file, select the
{\bf Save} or {\bf Save as} choices to display the
{\bf Save as} dialog so that you can name the file you are editing.
A file must have a name to be saved.
\else
Используйте команду {\bf Save}, чтобы сохранить содержимое текущего файла на 
диск. После того, как файл сохранен, текст остается в окне, так что вы можете 
продолжить его изменять. Тем не менее, буфер отмены очищается, так что вы не 
можете отменить изменения, произведенные до того, как файл был сохранен.

{\bf Замечание: } если вы редактируете новый файл, выберите {\bf Save} или 
{\bf Save as}, чтобы отобразить {\bf Save as}-диалог, позволяющий вам дать имя
редактируемому файлу. Для того, чтобы быть сохраненным, файл дожен иметь имя. 
\fi

\subsection{File: Save As}
\public{IDM\_SAVEAS}

\ifenglish
Use the {\bf Save As} command to name and save a new file
or to save a file under a different name, or to a different location.

A dialog box will appear, prompting you for a new filename. You can
also select different drive/directory.

The new name will appear in the title bar of current window. If there are
several windows with the file being saved, all of them change title bar.
\else
Используйте команду {\bf Save As}, чтобы назвать и сохранить новый файл,
или сохранить файл под другим именем или в другом месте.

Появится диалог, запрашивающий имя файла. Вы также можете указать носитель
и директорию.

Новое имя будет отображено в заголовке текущего окна. Если сохраняемый файл 
содержат несколько окон, все они изменят заголовок.
\fi

\subsection{File: Save All}
\public{IDM\_SAVEALL}

\ifenglish
Use the {\bf Save All} command to save all changed files.
\else
Используйте команду {\bf Save All}, чтобы сохранить все измененные файлы.
\fi

\subsection{File: Share}
\public{IDM\_SHARE}

\ifenglish
Use the {\bf Share} command to open another window with current file.
Both windows are kept consistent; all changes made in one window
are immediately visible in other window.

You may open as many copies of a file as you wish.

{\bf Note:} You will be prompted if you wish to save file to disk only when
the last window with a file is closed.
\else
Используйте команду {\bf Share}, чтобы открыть другое окно, содержащее текущий 
файл. Оба окна согласуются, все изменения, произведенные в одном окне, сразу же
отображаются в другом.  

Вы можете открывать сколько угодно копий одного и того же файла.

{\bf Замечание:} Вас запросят, хотите ли вы сохранить файл на диск, только тогда,
когда будет закрыто последнее окно, содержащее файл.  
\fi

\subsection{File: Print}
\public{IDM\_PRINT}

\ifenglish
Use the {\bf Print} command to print the current file or
selection.
\else
Используйте команду {\bf Print}, чтобы распечатать содержимое текущего файла
или отмеченный текст.
\fi

\subsection{File: Close}
\public{IDM\_CLOSE}

\ifenglish
Use the {\bf Close} command to close the current window.
You will be prompted to save changes, if any.
\else
Используйте команду {\bf Close}, чтобы закрыть текущее окно.
Вас запросят, хотите ли вы сохранить изменения, если они были произведены.
\fi

\subsection{File: Exit}
\public{IDM\_EXIT}

\ifenglish
Use the {\bf Exit} command to close all edit windows
and terminate the IDE. You will be prompted to save changes,
if any.
\else
Используйте команду {\bf Exit}, чтобы закрыть все окна ввода и завершить IDE. 
Вас запросят, хотите ли вы сохранить изменения, если они были произведены.
\fi

\subsection{File: More Files}
\public{IDM\_MOREFILES}

\ifenglish
Only 5 last opened files appear at the bottom of \ref[File]{M\_FILE} menu.
The IDE actually remembers last 100 opened files. To display \ref[full list]{MOREFILES\_DIALOG}
of them use the {\bf More Files} command.
\else
Только 5 последних открывавшихся файлов появятся в нижней части меню 
\ref[File]{M\_FILE}. IDE на самом деле запоминает 100 последних открывавшихся
файлов.
Чтобы отобразить их полный список, используйте команду 
\ref[full list]{MOREFILES\_DIALOG}.
\fi
\section{Edit menu}
\public{M\_EDIT}
\nominitoc

\ifenglish
The {\bf Edit} menu contains commands that you use to undo
changes you've made to file, to work with Clipboard
and to find and replace text.
The following commands appear in the
{\bf Edit} menu:
\else
Меню {\bf Edit} содержит команды, которые вы используете, чтобы отменять 
произведенные вами в файле изменения, работать с буфером обмена
и искать и заменять текст. 
Меню {\bf Edit} содержит следующие команды:
\fi
\begin{description}
\item \ref[Undo]{IDM\_UNDO} -
  \ifenglish
  Undoes the last edit action
  \else
  Отменяет последнее изменение 
  \fi
\item \ref[Cut]{IDM\_CUT} -
  \ifenglish
  Cuts selected text to the clipboard
  \else
  Перемещает отмеченный текст в буфер обмена
  \fi
\item \ref[Copy]{IDM\_COPY} -
  \ifenglish
  Copies selected text to the clipboard
  \else
  Копирует отмеченный текст в буфер обмена
  \fi
\item \ref[Add]{IDM\_ADD} -
  \ifenglish
  Adds selected text to the clipboard
  \else
  Добавляет отмеченный текст в буфер обмена
  \fi
\item \ref[Exchange]{IDM\_EXCHANGE} -
 \ifenglish
  Exchanges selected text with clipboard contents
  \else
  Меняет отмеченный текст на содержимое буфера обмена
 \fi
\item \ref[Paste]{IDM\_PASTE} -
 \ifenglish
  Inserts contents of the clipboard at the current position
  \else
  Вставляет содержимое буфера обмена на текущую позицию
  \fi
\item \ref[Find]{IDM\_FIND} -
  \ifenglish
  Searches for a string
  \else
  Ищет строку
  \fi
\item \ref[Find Next]{IDM\_FINDNEXT} -
 \ifenglish
  Searches for the next occurrence of last searched string
  \else
  Ищет следующее вхождение последней искомой строки
  \fi
\item \ref[Find Previous]{IDM\_FINDPREV} -
 \ifenglish
  Searches for the previous occurrence of last searched string
  \else
  Ищет предыдущее вхождение последней искомой строки
 \fi
\item \ref[Replace]{IDM\_REPLACE} -
 \ifenglish
 Searches for a string, replacing it with another one
 \else
  Ищет строку, заменяя ее на другую
 \fi
\item \ref[Replace Next]{IDM\_REPLACENEXT} -
 \ifenglish
  Repeats previous replace action
 \else
 Повторяет последнюю операцию замены
 \fi
\item \ref[Find in Files]{IDM\_FINDFILES}
 \ifenglish
  Searches for the string in multiple files
 \else
  Ищет строку во множестве файлов
 \fi
\item \ref[Go to line]{IDM\_GOTOLINE}
 \ifenglish
  Moves cursor to the line with specified number
 \else
  Перемещает курсор на строку с указанным номером
 \fi
\end{description}

\subsection{Edit: Undo}
\public{IDM\_UNDO}

\ifenglish
Use the {\bf Undo} command to undo the last change you've made to
the current file.
\else
Используйте команду {\bf Undo}, чтобы отменить последнее изменение, 
произведенное вами в текущем файле
\fi
\ifenglish
For this to work, for your changes the editor saves information necessary to
revert them into so-called undo buffer.
This buffer has limited size; when the buffer is full
you are prompted to abandon current change. If you refuse, the editor will
automatically drop earliest records from undo buffer in order to make room for
new records. This leads to inability to undo all the actions.

Casual editing operation such as typing of new text don't use much room in
undo buffer and so are unlikely to fill it. So you never get into mentioned
situation unless you frequently delete large blocks of text.

The undo buffer is cleared when you save file to disk, so you can't
undo changes done before saving.

Keyboard shortcuts: {\bf Alt+Backspace}, {\bf Ctrl+Z}
\else
Чтобы это срабатывало, редактор сохраняет необходимую информацию о ваших
изменениях в так называемом буфере отмены.
Этот буфер имеет ограниченный размер, когда он заполнится, вас попросят
отказаться от текущего изменения. Если вы откажетесь, редактор автоматически
выбросит более ранние сведения из буфера для того, чтобы создать место для
новых записей. Это приводит к невозможности отменить все изменения.

Нерегулярные операции редактирования, такие, как ввод нового текста, не 
занимают много места в буфере отмены, и потому вряд ли его переполнят.
Таким образом, вы никогда ни попадете в упомянутую ситуацию, если не 
удаляете часто большие участки текста.

Когда вы сохраняете файл на диск, буфер отмены очищается, поэтому вы не
можете отменить произведенные до записи изменения.
\fi
\subsection{Edit: Cut}
\public{IDM\_CUT}

\ifenglish
Use the {\bf Cut} command to delete selected text from the
file and move it to the Clipboard. You can use the \ref[Paste]{IDM\_PASTE}
command later to insert the Clipboard contents at another position
or even into another file. You can also insert the text to other applications.

Keyboard shortcuts: {\bf Shift+Delete}, {\bf Ctrl+X}
\else
Используйте команду {\bf Cut}, чтобы удалять отмеченный текст из файла и 
помещать его в буфер обмена. Затем вы можете использовать команду
\ref[Paste]{IDM\_PASTE}, чтобы вставить содержимое буфера обмена на
другую позицию или даже в другой файл. Вы также можете вставить текст
в другие приложения.         

Комбинации клавиш: {\bf Shift+Delete}, {\bf Ctrl+X}
\fi
\subsection{Edit: Copy}
\public{IDM\_COPY}

\ifenglish
Use the {\bf Copy} command to duplicate selected text. A copy of
selected text is moved to the Clipboard. You can use the \ref[Paste]{IDM\_PASTE}
command later to insert the Clipboard contents at another position
or even into another file.  You can also insert the text to other applications.

Keyboard shortcuts: {\bf Ctrl+Insert}, {\bf Ctrl+C}
\else
Используйте команду {\bf Copy}, чтобы дублировать отмеченный текст.
Копия отмеченного текста помещается в буфер обмена. Затем вы можете 
использовать команду \ref[Paste]{IDM\_PASTE}, чтобы вставить содержимое 
буфера обмена на другую позицию или даже в другой файл. Вы также можете 
вставить текст в другие приложения.         

Комбинации клавиш: {\bf Ctrl+Insert}, {\bf Ctrl+C}
\fi
\subsection{Edit: Add}
\public{IDM\_ADD}

\ifenglish
Use the {\bf Add} command to combine several portions of text together.
This command appends selected text to the Clipboard contents.
If the Clipboard is empty or doesn't contain text, the command works
exactly as \ref[Copy]{IDM\_COPY} command.
You can use the {\bf Paste}
command later to insert the resulting text into desired place.
\else                                    
Чтобы сочетать несколько участков текста, используйте команду {\bf Add}.
Эта команда добавляет отмеченный текст к содержимому буфера обмена.
Если буфер обмена пуст или не содержит текста, команда действует в точности
как и команда \ref[Copy]{IDM\_COPY}.
Затем вы можете использовать команду {\bf Paste},
чтобы вставить полученный текст на нужное место.
\fi
\ifenglish
Keyboard shortcuts: {\bf Ctrl+Shift+Insert}, {\bf Ctrl+A}

Be careful when pressing {\bf Ctrl+Shift+Insert}: if you release {\bf Shift} key
the command will be recognized as \ref[Copy]{IDM\_COPY} command and contents
of the Clipboard will be lost.
\else
Комбинации клавиш: {\bf Ctrl+Shift+Insert}, {\bf Ctrl+A}

Будьте внимательны, когда нажимаете {\bf Ctrl+Shift+Insert}, если вы отпустите 
клавишу {\bf Shift}, команда будет воспринята как \ref[Copy]{IDM\_COPY}, и 
содержимое буфера обмена будет утеряно.
\fi
\subsection{Edit: Paste}
\public{IDM\_PASTE}

\ifenglish
Use the {\bf Paste} command to insert the text from the Clipboard
at the cursor position in the current file.

Text appears in the Clipboard as a result of \ref[Cut]{IDM\_CUT}, \ref[Copy]{IDM\_COPY},
\ref[Add]{IDM\_ADD}, \ref[Exchange]{IDM\_EXCHANGE} operations, or it can be copied
from another application.

Keyboard shortcuts: {\bf Shift+Insert}, {\bf Ctrl+V}
\else
Используйте команду {\bf Paste}, чтобы вставлять текст из буфера обмена
на позицию курсора в текущем файле.

Появляющийся в буфере обмена текст есть результат операций \ref[Cut]{IDM\_CUT}, 
\ref[Copy]{IDM\_COPY}, \ref[Add]{IDM\_ADD}, \ref[Exchange]{IDM\_EXCHANGE} или
может быть пренесен из других приложений.

Комбинации клавиш: {\bf Shift+Insert}, {\bf Ctrl+V}
\fi
\subsection{Edit: Exchange}
\public{IDM\_EXCHANGE}

\ifenglish
Use the {\bf Exchange} command to exchange contents of the Clipboard
with the current selection.

The command is only available when Clipboard contain text.

This command is useful for exchanging two strings in a file. You select one
string, \ref[Cut]{IDM\_CUT} it to the Clipboard, then select another string,
{\bf Exchange} it with the Clipboard, then return to a place where first string
once was, and \ref[Paste]{IDM\_PASTE} the Clipboard.

Keyboard shortcuts: {\bf Ctrl+Shift+Delete}, {\bf Ctrl+E}

Be careful when pressing {\bf Ctrl+Shift+Delete}: if you release {\bf Ctrl} key
the command will be recognized as \ref[Cut]{IDM\_CUT} command and contents
of the Clipboard will be lost.
\else
Используйте команду {\bf Exchange}, чтобы обменять содержимое буфера обмена и 
отмеченный в данный момент текст.

Команда доступна только в случае, если буфер обмена содержит текст.

Эта команда полезна при обмене двух строк в файле. Вы отмечаете одну строку
командой \ref[Cut]{IDM\_CUT}, перемещаете ее в буфер обмена, затем отмечаете
другую строку командой {\bf Exchange}, меняете ее на содержимое буфера, затем
возвращаетесь на место, где прежде находилась первая строка, и используете
команду \ref[Paste]{IDM\_PASTE}.

Комбинации клавиш: {\bf Ctrl+Shift+Delete}, {\bf Ctrl+E}.

Будьте внимательны, когда нажимаете {\bf Ctrl+Shift+Delete}, если вы отпустите 
клавишу {\bf Ctrl}, команда будет воспринята как \ref[Cut]{IDM\_CUT}, и 
содержимое буфера обмена будет утеряно.
\fi
\subsection{Edit: Find}
\public{IDM\_FIND}

\ifenglish
Use the {\bf Find} command to locate occurrences of a string in
the current file. A dialog box will appear, allowing you to
specify the string to find and various options such as case sensitiveness,
search direction and others.

Keyboard shortcut: {\bf Ctrl+F}
\else
Используйте команду {\bf Find}, чтобы обнаружить вхождения строки в текущем 
файле. Будет выведен диалог, позволяющий вам указать искомую строку и различные
опции, такие как чувствительность к регистру, направление поиска и другие.

Комбинация клавиш: {\bf Ctrl+F}
\fi
\subsection{Edit: Find Next}
\public{IDM\_FINDNEXT}

\ifenglish
Use the {\bf Find Next} command to find the next occurrence
of the current search string.

Note that term "next" here is relative to search direction: if last search was
backwards this command will find previous occurrence of a text.

Keyboard shortcut: {\bf Ctrl+L}
\else
Чтобы найти следующее вхождение текущей искомой строки, используйте команду
{\bf Find Next}.

Заметьте, что термин "следующее" относится здесь к направлению поиска: если
последний поиск велся в обратном направлении: эта команда найдет предыдущее 
вхождение текста.

Комбинация клавиш: {\bf Ctrl+L}
\fi
\subsection{Edit: Find Previous}
\public{IDM\_FINDPREV} {\bf Ctrl+L}

\ifenglish
Use the {\bf Find Previous} command to find the previous occurrence
of the current search string.

Note that term "previous" here is relative to search direction: if last search was
backwards this command will find next occurrence of a text.

Keyboard shortcut: {\bf Ctrl+Shift+L}
\else
Для того, чтобы найти предыдущее вхождение текущей искомой строки, 
используйте команду {\bf Find Previous}. 

Заметьте, что термин "предыдущее" относится здесь к направлению поиска, если
последний поиск велся в обратном направлении, эта команда найдет следующее
вхождение текста.

Комбинация клавиш: {\bf Ctrl+Shift+L}
\fi
\subsection{Edit: Replace}
\public{IDM\_REPLACE}

\ifenglish
Use the {\bf Replace} command to locate occurrences of a string in
the current file, replacing it with another string. 
A dialog box will appear, allowing you to
specify the search string, the replacement string, and various options such as
operation scope, case-sensetiveness and others.

Keyboard shortcut: {\bf Ctrl+H}
\else
Используйте команду {\bf Replace}, чтобы обнаруживать вхождения строки
в текущем файле, заменяя их на другую строку.
Будет выведен диалог, позволяющий указать искомую строку, строку замены и 
различные опции, такие как рамки операции, чувствительность к регистру и другие.

Комбинация клавиш: {\bf Ctrl+H}
\fi
\subsection{Edit: Replace Next}
\public{IDM\_REPLACENEXT}

\ifenglish
Use the {\bf Replace Next} command to repeat the previous replace
action.
\else
Используйте команду {\bf Replace Next}, чтобы повторить предыдущую замену.
\fi
\subsection{Edit: Find in Files}
\public{IDM\_FINDFILES}

\ifenglish
Use the {\bf Find in Files} command to search a string in several files.
You can search for files in open windows, in project files or in files in
specified directory on disk.
A dialog box will appear, allowing you to
specify the search string, and various options such as search scope and
case-sensetiveness.

Keyboard shortcut: {\bf Ctrl+Shift+F}

\else
Используйте команду {\bf Find in Files}, чтобы производить поиск в нескольких
файлах. Вы можете производить поиск в файлах открытых окон, файлах проекта
или файлах в указанной директории или диске.
Будет выведен диалог, позволяющий указать искомую строку и различные опции,
вроде рамок поиска и чувствительности к регистру.

Комбинация клавиш: {\bf Ctrl+Shift+F}
\fi
\subsection{Edit: Go to Line}
\public{IDM\_GOTOLINE}

\ifenglish
Use the {\bf Go to Line} command to quickly move the cursor
to the line with a specific number.

Keyboard shortcut: {\bf Ctrl+G}
\else
Чтобы быстро перемещать курсор на строку с указанным номером, 
используйте команду {\bf Go to Line}.

Комбинация клавиш: {\bf Ctrl+G}
\fi
\section{Project Menu}
\public{M\_PROJECT}
\nominitoc

\ifenglish
The {\bf Project} menu contains commands that you use to create,
open, close, and edit a project file. It also contains a list 
of the recently opened project files to let you quickly switch
to another project. The following commands appear
in the {\bf Project} menu:
\else
Меню {\bf Project} содержит команды, которые вы используете, чтобы создавать,
открывать, закрывать и редактировать файлы проекта. Оно также содержит список
недавно открывавшихся файлов, позволяя бысто переключаться на другой проект.
В меню {\bf Project} содержатся следующие команды:
\fi
\begin{description}
\item \ref[New]{IDM\_PROJNEW}
\ifenglish
  - Creates a new project file
  \else
  - Создает новый файл проекта
 \fi
\item \ref[Open]{IDM\_PROJOPEN}
\ifenglish
  - Opens an existent project file
  \else
  - Открывает существующий файл проекта
  \fi
\item \ref[Close]{IDM\_PROJCLOSE}
\ifenglish
  - Closes the current project file
  \else
  - Закрывает текущий файл проекта
  \fi
\item \ref[Modify]{IDM\_PROJOPTIONS}
\ifenglish
  - Lets you modify the current project in a visual project editor
  \else
  - Позволяет изменять текущий проект с помощью средств графического 
    интерфейса
  \fi
\item \ref[Edit]{IDM\_PROJEDIT}
\ifenglish
  - Lets you edit the current project file manually
  \else
  - Позволяет вручную редактировать текущий файл проекта
  \fi
\item \ref[File list]{IDM\_FILELIST}
\ifenglish
  - Displays a list of the current project source files
  \else
  - Отображает список исходных файлов текущего проекта  
  \fi
\item \ref[Style sheets]{M\_STYLES}
\ifenglish
  - Activates the {\bf Style sheets} submenu
  \else
  - Активирует подменю {\bf Style sheets}
  \fi
\end{description}

\subsection{Project: New}
\public{IDM\_PROJNEW}

\ifenglish
Use the {\bf New} command to create a new project. A dialog
box will appear, letting you specify a project file name,
location, template, main module and some options.
\else
Используйте команду {\bf New}, чтобы создавать новый проект. Будет
выведен диалог, позволяющий указать имя файла проекта, его место на диске,
шаблон, главный модуль и некоторые опции.
\fi
\subsection{Project: Open}
\public{IDM\_PROJOPEN}

\ifenglish
Use the {\bf Open} command to open an existent project file
by means of a standard file dialog.

The IDE keeps track of opened windows in each project separately.
So opening new project will close all your windows and open windows
that were opened last time you worked with that project.

The system also considers the directory of current project its current directory.
That affects invocation of tools.
\else
Используйте команду {\bf Open}, чтобы открывать существующие файлы проекта 
с помощью стандартного файлового диалога.

IDE отдельно для каждого проекта хранит данные об открытых окнах.
Таким образом, новый проект закроет все ваши окна и откроет окна, которые 
были открыты в последний раз, когда вы работали с этим проектом.

Система также считает директорию текущего проекта текущей директорией.
Это влияет на вызов утилит.
\fi
\ifenglish
{\bf Note:} The IDE maintains a list of the most recently 
opened project files, which is shown at the bottom of the
\ref[Project menu]{M\_PROJECT}. You may use it to quickly load a project.
\else
Замечание: IDE заводит список недавно открывавшихся файлов, который отображается
в нижней части меню \ref[Project menu]{M\_PROJECT}. Вы можете использовать
его для быстрой загрузки проекта.
\fi
\subsection{Project: Close}
\public{IDM\_PROJCLOSE}

\ifenglish
Use the {\bf Close} command to close the current project 
and all windows without terminating the IDE.
\else
Используйте команду {\bf Close}, чтобы закрыть текущий проект и все
окна, не завершая работы IDE.
\fi
\subsection{Project: Modify}
\public{IDM\_PROJOPTIONS}

\ifenglish
Use the {\bf Modify} command to edit the current project file 
in a visual project editor. In this editor all project options
are combined into meaningful hierarchical groups and presented in
human-readable rather than compiler-readable form.

The visual editor can't modify projects with complicated syntax,
for instance, those with conditional clauses. In is recommended
to \ref[Edit]{IDM\_PROJEDIT} such project files in usual text editor.
\else
Используйте команду {\bf Modify}, чтобы редактировать текущий файл проекта,
используя средства графического интерфейса. В этом редакторе все опции 
проекта собраны в значащие иерархические группы и представлены в виде,
предназначенном для прочтения скорее человеком, чем компилятором.

Графический интерфейс не позволяет изменять проекты со сложным 
синтаксисом, содержащие, например, условные операторы. Такие проекты 
рекомендуется редактировать (\ref[Edit]{IDM\_PROJEDIT}) в обычном 
текстовом редакторе.
\fi
\subsection{Project: Edit}
\public{IDM\_PROJEDIT}

\ifenglish
Use the {\bf Edit} command to load the current project file 
into a regular edit window.

If you \ref[Modify]{IDM\_PROJOPTIONS} the project visually
while it is opened in regular edit window the
new changes will immediately appear in that window.
\else
Используйте команду {\bf Edit}, чтобы загружать текущий файл проекта в
обычное окно редактирования.

Если вы будете изменять (\ref[Modify]{IDM\_PROJOPTIONS}) проект, 
используя средства графического интерфейса, в то время как он открыт 
в обычном окне редактирования, то новые изменения будут отображаться 
в этом окне немедленно.
\fi
\subsection{Project: File list}
\public{IDM\_FILELIST}

\ifenglish
Use the {\bf File list} command to display a list of source
files belonging to the current project.

This list is initially empty and is
updated during each make because in languages like Modula-2 and Oberon-2 you
can only tell what files belong to the project after you build the project.

The list is written to disk and kept from session to session but it is considered
unreliable until first project make and is marked as "Not updated" because
source files could be modified outside the IDE.

It is possible to gather project file list quickly by scanning source files
for IMPORT directives without compiling them. This function is available via
the \ref[Tools]{M\_TOOLS}:Update file list option.

If you opened file list window earlier it will be opened in the same position and will have same size.

It is possible to specify that new file list window must be docked.

Keyboard shortcut: {\bf Alt+"="}
\else
Используйте команду {\bf File list}, чтобы отображать список исходных файлов,
принадлежащих текущему проекту.

Список изначально пуст, и обновляется во время каждой компоновки, так как в 
таких языках как Modula-2 и Oberon-2 вы можете определить, какие файлы 
принадлежат проекту, только после его сборки.

Список хранится на диске в промежутках между сеансами работы, но считается
недостоверным до первой компоновки и помечается как "Not updated", потому что
исходные файлы могли быть изменены вне IDE.

Существует возможность быстрого составления списка файлов проекта путем
поиска в исходных файлах директив IMPORT без их компиляции. Эта функция 
доступна посредством \ref[Tools]{M\_TOOLS}: Опция обновления списка файлов.

Если вы открывали окно списка файлов ранее, оно будет открыто на том же месте 
и будет иметь тот же размер.

Можно указать, чтобы окно списка файлов было неперекрывающимся.

Комбинация клавиш: {\bf Alt+"="}
\fi
\subsection{Project: Style sheets}
\public{M\_STYLES}
\nominitoc

\ifenglish
The {\bf Style sheets} command displays a submenu,
which contains the following commands for work with \ref[Style sheets]{StyleSheets}.
\else
Команда {\bf Style sheets} отображает подменю,
содержащее следующие команды для работы со \ref[стилями]{StyleSheets}.
\fi
\begin{description}
\item \ref[New Style]{IDM\_NEWSTYLE}
\ifenglish
  - Creates a new style file
  \else
  - Создает новый файл стиля
  \fi
\item \ref[Save project as style]{IDM\_PROJSAVESTYLE}
\ifenglish
  - Saves the current project options and equations as style
  \else
  - Сохраняет опции и параметры текущего проекта
  \fi
\item \ref[Apply style]{IDM\_APPLYSTYLE}
\ifenglish
  - Applies a style to the current project
  \else
  - Применяет стиль к текущему проекту
  \fi
\item \ref[Edit style]{IDM\_EDITSTYLE}
\ifenglish
  - Invokes the style editor
  \else
  - Вызывает редактор стиля
  \fi
\item \ref[Delete style]{IDM\_DELETESTYLE}
\ifenglish
  - Deletes a style file
  \else
  - Удаляет файл стиля
  \fi
\end{description}

\ifenglish
\subsubsection{Style sheets}
\else
\subsubsection{Стили}
\fi
\public{StyleSheets}
\ifenglish
It is often necessary to change several project options at once into some
well-known state. For example to make a project in debug mode you usually
turn on generation of debug information, turn on all run-time checks and turn
off optimizations, while for release mode you need opposite settings to all
those options. It takes long time to set all the options manually. Besides it
is possible to forget some options.

This is what the {\bf Style sheet} is for.
The style sheet is some subset of project options saved to separate file.
Applying the style sheet means copying project options from style sheet to
project file while retaining those options in a project that aren't present in
the style sheet.

You can create new empty style sheet. Use \ref[New Style]{IDM\_NEWSTYLE}
menu command.

You can create style sheet based on existing project file. Use
\ref[Save Project as Style]{IDM\_PROJSAVESTYLE}
menu command.
\else
Часто бывает необходимо изменять несколько опций сразу, задавая им вполне 
определенные значения. Например, чтобы скомпоновать проект в режиме отладки,
обычно включаются генерация отладочной информации и прогоночные проверки, 
а все оптимизации отключаются, в то время как для компоновки в режиме рабочей
версии требуются противоположные установки этих же самых опций. Установка опций
вручную требует значительного времени. Кроме того, некоторые опции можно опустить.

Вот для чего нужен {\bf стиль}.
Стиль - это некоторое подмножество опций проекта, сохраненное в отдельном файле.
Применение стиля означает копирование опций проекта из стиля в файл проекта
с сохранением установок незадействованных в стиле опций.

Вы можете создать новый пустой стиль. Используйте команду меню 
\ref[New Style]{IDM\_NEWSTYLE}.

Вы можете создать новый стиль, основываясь на существующем файле проекта 
Используйте команду меню 
\ref[Save Project as Style]{IDM\_PROJSAVESTYLE}.
\fi
\ifenglish
Once created, the style sheet can be edited in a visual editor similar to
the one used for editing projects. Use
\ref[Edit Style]{IDM\_EDITSTYLE} menu command.

Style sheets are usually kept in XDS system directory under names with
{\bf .Sty} extensions.

There are two pre-defined style sheets, {\bf Debug} and {\bf Release} with
debug and release options set.
\else
Будучи созданным, стиль может редактироваться посредством 
графического интерфейса точно также, как и проект.
Используйте команду \ref[Edit Style]{IDM\_EDITSTYLE}

Стили обычно хранятся в системной директории XDS в файлах с расширением
{\bf .Sty}.
\fi

\subsubsection{Project: Style sheets: New Style}
\public{IDM\_NEWSTYLE}

\ifenglish
Use the {\bf New style} command to create a new empty \ref[Style sheet]{StyleSheets}.
A dialog box will appear, prompting you for a 
style name, comment and file location.
\else
Используйте команду {\bf New style}, чтобы создать новый пустой
\ref[стиль]{StyleSheets}.
Появится диалог, запрашивающий имя стиля, комментарий и место файла на диске.
\fi
\subsubsection{Project: Style sheets: Save Project as style}
\public{IDM\_PROJSAVESTYLE}

\ifenglish
Use the {\bf Save Project as style} command to store the 
current project options and equations in a new or existent
\ref[Style sheet]{StyleSheets} file.
A dialog box will appear, prompting you for a 
style name, comment and file location.
\else
Используйте команду {\bf Save Project as style}, чтобы сохранять
опции и параметры текущего проекта в новый или существующий файл
\ref[стиля]{StyleSheets}.
Появится диалог, запрашивающий имя стиля, комментарий и место файла на диске.
\fi
\subsubsection{Project: Style sheets: Apply Style}
\public{IDM\_APPLYSTYLE}

\ifenglish
Use the {\bf Apply Style} command to apply a specific
\ref[Style sheet]{StyleSheets} 
to the current project, i.e. to add style options and equations 
to it. If a style option or equation already exists in the
project, it is replaced by style sheet value. Options and equations that are
not present in the style sheet are not affected.

A dialog box will appear, allowing you to select a style to apply.
\else
Используйте команду {\bf Apply Style}, чтобы применить заданный
\ref[стиль]{StyleSheets} к текущему проекту, иначе говоря, чтобы добавить
к нему опции и параметры стиля. Если опция или параметр стиля уже существуют в
проекте, то они задаются значением, указанным в проекте. Опции и параметры проекта,
не участвующие в стиле, этой командой не затрагиваются.

Будет выведен диалог, позволяющий вам выбрать стиль для применения.
\fi
\subsubsection{Project: Style sheets: Edit Style}
\public{IDM\_EDITSTYLE}

\ifenglish
Use the {\bf Edit Style} command to edit a \ref[Style sheet]{StyleSheets} using a
style visual editor (similar to project editor).
\else
Используйте команду {\bf Edit Style}, чтобы редактировать 
\ref[стиль]{StyleSheets} с помощью средств графического интерфейса.

\fi
\subsubsection{Project: Style sheets: Delete Style}
\public{IDM\_DELETESTYLE}

\ifenglish
Use the {\bf Delete style} command to permanently delete
a \ref[Style sheet]{StyleSheets} file.
\else
Используйте команду {\bf Delete style}, чтобы окончательно удалить файл
\ref[стиля]{StyleSheets}.
\fi
\section{Tools Menu}
\public{M\_TOOLS}
\index{Tools Menu}
\nominitoc

\ifenglish
The {\bf Tools} menu contains commands that you use to invoke
external tools, such as an XDS compiler.  
The following commands appear in the {\bf Tools} menu by default:
\else
Меню {\bf Tools} содержит команды, позволяющие вызывать внешние утилиты, 
такие, как XDS компилятор.
В меню {\bf Tools} по умолчанию содержатся следующие команды:
\fi
\ref[Compile]{IDM\_COMPILE}
\ifenglish
  - Compiles the file in the current edit window
 \else
  - Компилирует файл в текущем окне редактирования
 \fi
\ref[Make]{IDM\_MAKE}
\ifenglish
  - Makes the current project
  \else
  - Компонует текущий проект
  \fi
\ref[Build all]{IDM\_BUILDALL}
\ifenglish
  - Makes the current project, rebuilding all files
  \else
  - Компонует текущий проект, перестраивая все файлы
  \fi
\ref[Update file list]{IDM\_UPDATE}
\ifenglish
  - Updates source files list without compilation
  \else
  - Обновляет список исходных файлов без компиляции
  \fi
\subsection{Tools: Compile}
\public{IDM\_COMPILE}

\ifenglish
Use the {\bf Compile} command to compile a module in the
current window.

Keyboard shortcut: {\bf F9}
\else
Используйте команду {\bf Compile}, чтобы компилировать модуль в 
текущем окне.

Комбинация клавиш: {\bf F9}
\fi
\subsection{Tools: Make}
\public{IDM\_MAKE}

\ifenglish
Use the {\bf Make} command to compile all changed modules
in the current project and link it.

Keyboard shortcut: {\bf Shift+F9}
\else
Используйте команду {\bf Make}, чтобы компилировать все измененные модули
текущего проекта и связать их.

Комбинация клавиш: {\bf Shift+F9} 
\fi
\subsection{Tools: Build all}
\public{IDM\_BUILDALL}

\ifenglish
Use the {\bf Build all} command to rebuild the current project
completely.
\else
Используйте команду {\bf Build all}, чтобы полностью перестроить текущий 
проект.
\fi
\subsection{Tools: Update file list}
\public{IDM\_UPDATE}

\ifenglish
Use the {\bf Update file list} command to update the source
files list without performing any compile or link actions.
\else
Используйте команду {\bf Update file list}, чтобы обновлять 
список исходных файлов без компиляции и связывания.
\fi
\section{Debug Menu}
\public{M\_DEBUG}
\index{Debug Menu}
\nominitoc

\ifenglish
The {\bf Debug} menu contains commands that you use to run and debug
your program.
The following commands appear in the {\bf Debug} menu:
\else
Меню {\bf Debug} содержит команды, которые вы используете, чтобы запускать
и отлаживать вашу программу.
\fi
\begin{description}
\item \ref[Run]{IDM\_RUN}
 \ifenglish
  - Makes and runs your program
  \else
  - Компонует и запускает вашу программу
  \fi
\item \ref[Stop]{IDM\_STOP}
  \ifenglish
  - Stops a running program
  \else
  - Останавливает запущенную программу
  \fi
\item \ref[Run Debugger]{IDM\_DEBUG}
 \ifenglish
  - Invokes an external debugger with your program
 \else
  - Вызывает внешний отладчик для вашей программы
 \fi
\item \ref[View Program Output]{IDM\_VIEWUSERS}
 \ifenglish
  - Switch to the main window of running or finished program
 \else
  - Переключается в основное окно работающей или завершившейся программы
 \fi
\item \ref[Options]{IDM\_RUNOPTIONS}
  \ifenglish
  - Lets you specify various run options
  \else
  - Позволяет задавать различные опции запуска
  \fi
\end{description}

\subsection{Debug: Run}
\public{IDM\_RUN}

\ifenglish
Use the {\bf Run} command to make current project and execute your program.
Program is not run if project make produces errors.

You can specify program name, its command line and starting directory
in the dialog box displayed via \ref[Options]{IDM\_RUNOPTIONS} menu command. 

Keyboard shortcut: {\bf Ctrl+F9}
\else
Используйте команду {\bf Run}, чтобы собрать текущий проект и выполнить вашу 
программу. Программа не запускается, если компоновка проекта обнаружила ошибки.

Вы можете определить имя программы, ее командную строку и начальную директорию
в диалоге, отображаемом командой меню \ref[Options]{IDM\_RUNOPTIONS}.
                             
Комбинация клавиш: {\bf Ctrl+F9}.
\fi
\subsection{Debug: Stop}
\public{IDM\_STOP}

\ifenglish
Use the {\bf Stop} command to stop your running program. The program is stopped
immediately without any chance to prevent termination or execute any finalizing
code.
\else
Используйте команду {\bf Stop}, чтобы остановить запущенную программу. Программа
останавливается немедленно без возможности предотвратить остановку или выполнить
завершающую часть кода.
\fi
\subsection{Debug: Run Debugger}
\public{IDM\_DEBUG}

\ifenglish
Use the {\bf Run Debugger} command to invoke an external debugger,
loading your program into it. By default this command invokes XDS debugger;
you can assign any other debugger to this command. For example, in XDS-C
it might be reasonable to use debugger that comes with the C/C++ compiler used.
\else
Используйте команду {\bf Run Debugger}, чтобы вызвать внешний отладчик, загружая
в него вашу программу. По умолчанию эта команда вызывает XDS отладчик. 
Вы можете привязать к этой команде друной отладчик. Например, в XDS-C можно
использовать отладчик, поставлемый с используемым компилятором C/C++.
\fi
\subsection{Debug: View Program Output}
\public{IDM\_VIEWUSERS}

\ifenglish
Use the {\bf View program output} command to display
your program output window.

Keyboard shortcut: {\bf F5}
\else
Используйте команду {\bf View program output}, чтобы отобразить окно вывода 
вашей программы.

Комбинация клавиш: {\bf F5}
\fi
\subsection{Debug: Options}
\public{IDM\_RUNOPTIONS}

\ifenglish
Use the {\bf Options} command to set various run options
via a dialog box.
\else
Используйте команду {\bf Options}, чтобы установить различные опции запуска
с помощью окна диалога.
\fi
\section{Messages Menu}
\public{M\_MESSAGES}
\index{Messages Menu}
\nominitoc
\ifenglish
The {\bf Messages} menu contains commands that you use
to display the tool messages window and to browse through errors
and warnings positions in source files.
The following commands appear in the {\bf Messages} menu:
\else
Меню {\bf Messages} содержит, команды которые вы используете,
чтобы выводить окно сообщений утилиты и просматривать места ошибок и
предупреждений в исходных файлах.
В меню {\bf Messages} содержатся слдующие команды:
\fi
\begin{description}
\item \ref[Show]{IDM\_SHOWMESS}
\ifenglish
  - Displays the {\bf Messages} window
  \else
  - Выводит окно {\bf Messages}
  \fi
\item \ref[Hide]{IDM\_HIDEMESS}
\ifenglish
  - Hides the {\bf Messages} window
  \else
  - Сворачивает окно {\bf Messages}
  \fi
\item \ref[Previous]{IDM\_PREVMESS}
\ifenglish
  - Moves cursor to the previous error position
  \else
  - Перемещает курсор к предыдущей ошибке
  \fi
\item \ref[Next]{IDM\_NEXTMESS}
\ifenglish
  - Moves cursor to the next error position
  \else
  - Перемещает курсор к следующей ошибке
  \fi
\end{description}

\subsection{Messages: Show}
\public{IDM\_SHOWMESS}

\ifenglish
Use the {\bf Show} command to display the {\bf Messages}
window, which contains error, warning and notification
messages from the last invoked tool.

If you opened the message window earlier, it will be opened in the
same position and will have the same size.

It is possible to specify that new message window must be docked.

Keyboard shortcut: {\bf Alt+"0"}
\else
Используйте команду {\bf Show}, чтобы отображать окно {\bf Messages},
содержащее ошибку, предупреждение или сообщения
последней вызывавшейся утилиты.

Если вы открывали окно сообщений ранее, оно будет открыто на том же месте
и будет иметь тот же размер.

Существует возможность создать новое неперекрывающееся окно сообщений.

Комбинация клавиш: {\bf Alt+"0"}.
\fi
\subsection{Messages: Hide}
\public{IDM\_HIDEMESS}

\ifenglish
Use the {\bf Hide} command to hide the {\bf Messages}
window.
\else
Чтобы свернуть окно {\bf Messages}, используйте команду {\bf Hide}.
\fi
\subsection{Messages: Previous}
\public{IDM\_PREVMESS}

\ifenglish
Use the {\bf Previous} command to move cursor to the position
of the previous error or warning.

Keyboard shortcut: {\bf F7}
\else
Используйте команду {\bf Previous}, чтобы переместить курсор к предыдущей
ошибке или предупреждению.

Комбинация клавиш: {\bf F7}
\fi
\subsection{Messages: Next}
\public{IDM\_NEXTMESS}

\ifenglish
Use the {\bf Next} command to move cursor to the position
of the next error or warning.

Keyboard shortcut: {\bf F8}
\else
Используйте команду {\bf Next}, чтобы переместить курсор к следующей ошибке
или предупреждению.

Комбинация клавиш: {\bf F8}
\fi
\section{Configure Menu}
\public{M\_CONFIGURE}
\index{Configure Menu}
\nominitoc

\ifenglish
The {\bf Configure} menu contains commands that you use to
configure the IDE.
The following commands appear in the {\bf Configure} menu:
\else
Меню {\bf Configure} содержит команды, используемые для настройки IDE.
В меню {\bf Configure} появляются следующие команды: 
\fi
\begin{description}
\item \ref[General]{IDM\_CONFIGUREGENERAL}
\ifenglish
  - Configure options that fit in neither of below
  \else
  - Настроечные опции, которые не входят ни в один из нижеперечисленных разделов
  \fi
\item \ref[Tools]{IDM\_CONFIGUREWINDOWS}
\ifenglish
  - Configures windows appearance and behavior
  \else
  - Настраивает внешний вид и поведение окон
  \fi
\item \ref[Windows]{IDM\_CONFIGURETOOLS}
\ifenglish
  - Lets you set up external tools
  \else
  - Позволяет устанавливать внешние утилиты
  \fi
\item \ref[Project Creation]{IDM\_CONFIGUREPROJECT}
\ifenglish
  - Tunes new project creation
  \else
  - Настройка создания нового проекта
  \fi
\item \ref[Editor]{IDM\_CONFIGUREEDITOR}
\ifenglish
  - Configures various editor options
  \else
  - Настройка различных опций редактора
  \fi
\item \ref[Colors and Fonts]{IDM\_COLOR}
\ifenglish
  - Sets colors and fonts used in IDE windows
  \else
  - Устанавливает цвета и шрифты, используемые в окнах IDE
  \fi
\item \ref[Printing]{IDM\_CONFIGUREPRINT}
\ifenglish
  - Sets up printing parameters
  \else
  - Устанавливает параметры печати
  \fi
\item \ref[Message Filters]{IDM\_CONFIGUREFILTERS}
\ifenglish
  - Add, remove and edit \ref[message filters]{MessageFilters}
  \else
  - Добавляет, удаляет и редактирует \ref[фильтры сообщений]{MessageFilters}
  \fi
\item \ref[Languages]{IDM\_CONFIGURE\_LANGUAGES}
\ifenglish
  - Configure support for \ref[programming languages]{Languages}
  \else
  - Настраиваает поддержку \ref[языков программирования]{Languages}
  \fi
\end{description}

\subsection{Configure: General}
\public{IDM\_CONFIGUREGENERAL}
\ifenglish
{\bf General command} invokes the \ref[dialog]{CFGGENERAL\_DIALOG} in which you
can modify general system options.
\else
Пункт меню {\bf General command} вызывает 
\ref[диалог]{CFGGENERAL\_DIALOG}, в котором вы можете изменять общие 
опции системы.
\fi

\ifenglish
Three options are currently available:
\else
В настоящий момент доступны три опции: 
\fi
\begin{description}
\item \ifenglish
      - whether the IDE must show logo picture at startup
      \else
      - показывать ли IDE при запуске логотип
      \fi
\item \ifenglish
      - whether the IDE must remember and load last loaded project
      \else
      - должна ли IDE запоминать и загружать последний загруженный проект
      \fi
\item \ifenglish
      - whether the toolbar is old style (3D) or new fashion (flat)
      \else
      - стиль панели инструментов, трехмерный или плоский
      \fi
\end{description}

\subsection{Configure: Windows}
\public{IDM\_CONFIGUREWINDOWS}
\ifenglish
XDS IDE uses its own version of {\bf MDI} (Multiple Document Interface).
This version provides {\bf docked} windows (windows that are never overlapped by others)
and {\bf free} windows (windows that appear on the desktop rather than inside
IDE frame). It also allows window numbering that is more convenient than
in classic MDI and provides {\bf winbar} to make navigation among your windows
easier.

The {\bf Windows} displays the \ref[dialog]{CFGWIN\_DIALOG} that
lets you choose placing and appearance of winbar,
window numbering schema, Window menu style and some other options.
\else
XDS IDE использует свою собственную версию {\bf MDI} (Multiple Document Interface). 
Эта версия дает возможность использования неперекрывающихся окон 
(которые никогда не закрываются другими) и свободных окон 
(которые появляются скорее на рабочем столе чем внутри IDE оболочки).
Она также предоставляет более удобный, чем в классическом MDI способ 
нумерации окон, и позволяет легче управляться с окнами.

Пункт меню {\bf Windows} отображает диалог \ref[dialog]{CFGWIN\_DIALOG}, который
позволяет вам выбирать местоположение и внешний вид WinBar,
схему нумерации окон, стиль оконного меню и некоторые другие опции. 
\fi

\subsection{Configure: Tools}
\public{IDM\_CONFIGURETOOLS}

\ifenglish
Use the {\bf Tools} command to configure external tools
which appear in the {\bf Tools} menu. Each tool has menu string, command line
and (optionally) keyboard shortcut.

The command displays \ref[Configure Tools]{CFGTOOLS\_DIALOG} dialog.
\else
Используйте команду {\bf Tools}, чтобы настраивать внешние утилиты,
которые появляются в меню {\bf Tools}. Для каждой утилиты имеется строка меню,
командная строка и (необязательно) комбинация клавиш клавиатуры. 

Команда выводит диалог \ref[Configure Tools]{CFGTOOLS\_DIALOG}.
\fi
\subsection{Configure: Project creation}
\public{IDM\_CONFIGUREPROJECT}

\ifenglish
Use the {\bf Project creation} command to fine tune new
project creation process: specify directories to create,
project template, redirection file template, etc.

The command displays \ref[New Project Creation Options]{CFGPROJ\_DIALOG} dialog.
\else
Используйте команду меню {\bf Project creation} для тонкой
настройки процесса создания проекта определите директории, которые надлежит
создать, файл шаблонов проекта, файл шаблонов путей поиска и т. д.

Команда выводит диалог \ref[New Project Creation Options]{CFGPROJ\_DIALOG}.
\fi
\subsection{Configure: Editor}
\public{IDM\_CONFIGUREEDITOR}

\ifenglish
It is well known that initially convenient system is better than the system
that allows to tune everything and that may become convenient after hours of tuning.
Nevertheless, the editor used in the IDE (called {\bf PoorEdit}) provides lots
of tunable parameters. You can specify tabulation size and method, fine tune
subtle properties of {\bf BS}, {\bf Delete} and {\bf Enter} keys, set file encoding
and and-of-line handling, cursor position after {\bf Paste} operation and many other
things.

All those settings are available via {\bf Editor} command that displays
\ref[Editor Configuration dialog]{CFGEDIT\_DIALOG}.
\else
Общеизвестно, что изначально удобная система лучше системы, которая позволяет все
настраивать и которая может стать удобной лишь после долгих часов, 
потраченных на настройку. Тем не менее, используемый в IDE редактор (называемый 
{\bf PoorEdit}) предоставляет множество настраиваемых параметров.
Вы можете определить ширину и способ табуляции, плавно настроить хитроумные
свойства клавиш {\bf BS}, {\bf Delete} и {\bf Enter}, установить кодировку файла
и способ обработки конца строки, положение курсора после операции {\bf Paste} 
и многое другое.

Все эти опции доступны посредством команды {\bf Editor}, которая отображает диалог
\ref[Editor Configuration dialog]{CFGEDIT\_DIALOG}.
\fi
\subsection{Configure: Colors and fonts}
\public{IDM\_COLOR}

\ifenglish
Use the {\bf Colors and fonts} command to specify colors
and fonts used in IDE windows.

You can specify foreground and background colors and fonts for editor windows,
file lists and message windows. For editor windows you can also specify color
of selection and color of highlighted text. If you have set of
\ref[programming languages]{Languages} configured you can also configure colors
for all syntax elements recognized in the languages. Each language is configured
separately.

You can also set the background color
of the IDE. 
\else
Используйте команду {\bf Colors and fonts}, чтобы определить цвета и шрифты,
задействованные в окнах IDE.

Вы можете определять цвет фона и рисующий цвет, шрифты для окон редактора
списков файлов и окон сообщений. Для окон редактора вы можете также определить 
цвета отмеченного и выделенного текста. Если у вас настроен набор  
\ref[языков программирования]{Languages}, то вы также можете настроить цвета 
для всех элементов синтаксиса, распознаваемых в языке. Каждый язык настраивается 
отдельно.

Вы также можете установить цвет фона IDE.
\fi
\subsection{Configure: Printing}
\public{IDM\_CONFIGUREPRINT}

\ifenglish
Use the {\bf Printing} command to specify how you want your files printed.
You can specify how you wish to handle long lines, what information to put
in the page header and whether to print line numbers. You can also choose
printer font and page margins.
\else
Используйте команду {\bf Printing}, чтобы определить, как вы хотите 
распечатывать ваши файлы. Вы также можете указать, как обрабатывать длинные 
строки, какие сведения помещать в заголовок страницы, и нужно ли печатать 
номера страниц. Вы также можете выбрать шрифт принтера и размеры полей.
\fi
\subsection{Configure: Message Filters}
\public{IDM\_CONFIGUREFILTERS}

\ifenglish
Use the {\bf Message Filters} command to display a dialog box where you can
work with \ref[Message Filters]{MessageFilters}. You can create new filters,
edit or remove existing filters. You can assign created message filters to tools
using \ref[Configure]{M\_CONFIGURE}:\ref[Tools]{IDM\_CONFIGURETOOLS} menu.
\else
Используйте команду {\bf Message Filters}, чтобы отбразить диалоговое окно,
где вы можете работать с \ref[фильтрами сообщений]{MessageFilters}. Вы также 
можете создавать новые фильтры, редактировать или удалять существующие.
Вы можете привязать созданные фильтры сообщений к утилитам, используя меню
\ref[Configure]{M\_CONFIGURE}:\ref[Tools]{IDM\_CONFIGURETOOLS}. 
\fi
\subsection{Configure: Languages}
\public{IDM\_CONFIGURE\_LANGUAGES}

\ifenglish
Use the {\bf Languages} command to display a dialog box where you can define
set of \ref[programming languages]{Languages} the IDE recognizes. You can
add, remove languages and configure \ref[language drivers]{LangDriver}.
\else
Используйте команду {\bf Languages}, чтобы отобразить диалог, где вы можете 
определить набор распознаваемых IDE \ref[языков программирования]{Languages}.
Вы можете добавлять, удалять языки и настраивать \ref[драйверы языков]{LangDriver}.
\fi
\section{Window Menu}
\public{M\_WINDOWS}
\index{Window Menu}
\nominitoc

\ifenglish
The {\bf Window} menu contains commands that you use to arrange
open IDE windows. It also contains a list of all open windows,
allowing you to switch to a specific window.
The following commands appear in the {\bf Window} menu:
\else
Меню {\bf Window} содержит команды, которые вы используете для выстраивания
открытых окон IDE. Оно также содержит список всех открытых окон, позволяя
переключаться на одно из них.
Меню {\bf Window} содержит следующие команды:
\fi
\begin{description}
\item \ref[Cascade]{IDM\_CASCADE}
\ifenglish
  - Makes an overlapping windows layout
  \else
  - Создает систему перекрывающихся окон
  \fi
\item \ref[Tile Vertically]{IDM\_TILEVERT}
\ifenglish
  - Makes a non-overlapping layout of "vertical" windows
  \else
  - Создает систему окон, неперекрывающихся по-вертикали
  \fi
\item \ref[Tile Horizontally]{IDM\_TILEHORIZ}
\ifenglish
  - Makes a non-overlapping layout of "horizontal" windows
  \else
  - Создает систему окон, неперекрывающихся по-горизонтали
  \fi
\item \ref[Arrange Icons]{IDM\_ARRANGE}
\ifenglish
  - Arranges icons of minimized windows into rows
  \else
  - Перестраивает иконки минимизированных окон в ряды
  \fi
\item \ref[Close All Windows]{IDM\_CLOSEALL}
\ifenglish
  - Close all opened windows
  \else
  - Закрывает все открытые окна
  \fi
\item \ref[Free]{IDM\_FREE}
\ifenglish
  - Make current window free
  \else
  - Делает текущее окно свободным
  \fi
\item \ref[Dock]{IDM\_DOCK}
\ifenglish
  - Dock current window
  \else
  - Осуществляет привязку текущего окна
  \fi
\item \ref[Undock]{IDM\_UNDOCK}
\ifenglish
  - Undock previously docked window
  \else
  - Отменяет привязку окна
 \fi
\item \ref[Windows List]{IDM\_MOREWINDOWS}
\ifenglish
  - Display full list of all MDI windows
  \else
  - Отображает полный список всех окон MDI 
 \fi
\end{description}

\subsection{Window: Cascade}
\public{IDM\_CASCADE}

\ifenglish
Use the {\bf Cascade} command to arrange all open windows
in an overlapping manner, making their titles visible.
\else
Используйте команду {\bf Cascade} чтобы расположить все открытые окна 
каскадом, оставляя видимым их заголовок.
\fi
\subsection{Windows: Tile Vertically}
\public{IDM\_TILEVERT}

\ifenglish
Use the {\bf Tile Vertically} command to arrange all open windows
in a non-overlapping manner, so that the whole windows are
visible, when the number of lines visible at the same time 
is more important than the number of columns.
\else
Используйте команду {\bf Tile Vertically} чтобы перестроить все открытые
окна так, чтобы, по возможности, они были видимы полностью, или чтобы количество
видимых строк преобладало над количеством видимых столбцов.
\fi
\subsection{Window: Tile Horizontally}
\public{IDM\_TILEHORIZ}

\ifenglish
Use the {\bf Tile Horizontally} command to arrange all open windows
in a non-overlapping manner, so that the whole windows are
visible, when the number of columns visible at the same time 
is more important than the number of lines.
\else
Используйте команду {\bf Tile Vertically} чтобы перестроить все открытые
окна так, чтобы по возможности они были видимы полностью, или чтобы количество
видимых столбцов преобладало над количеством видимых строк.
\fi
\subsection{Window: Arrange Icons}
\public{IDM\_ARRANGE}

\ifenglish
Use the {\bf Arrange Icons} command to arrange icons of minimized 
windows into rows.
\else
Используйте команду {\bf Arrange Icons}, чтобы перестроить иконки 
минимизированных окон в ряды.
\fi
\subsection{Window: Close All Windows}
\public{IDM\_CLOSEALL}

\ifenglish
Use the {\bf Close All Windows} command to close all opened windows.
If some of them contain changed files you will be asked to save the files
on disk.
\else
Используйте команду {\bf Close All Windows}, чтобы закрыть все открытые окна.
Если одно из них содержит измененный файл, вас запросят, хотите ли вы сохранить
изменения на диск.
\fi
\subsection{Window: Free}
\public{IDM\_FREE}

\ifenglish
In classic {\bf MDI} interface all the windows are children of main (frame)
windows. In MDI version used in XDS IDE you can {\bf Free} window. After that
the window is not clipped by IDE frame window but exists on computer desktop
as an application window. It can be switched to or from using {\bf Alt+Tab} keys
and is accessible using Windows 95 TaskBar. It does not disappear from XDS IDE
WinBar. To return the window to the child state use {\bf Return to Frame} command
of its System menu.
\else
В классическом интерфейсе MDI все окна являются дочерними по отношению к 
главному окну. В версии MDI, использованной в XDS IDE, вы можете {\bf освободить}
окно. После этого окно не принадлежит каркасному окну оболочки а существует 
на рабочем столе, как окно приложения. К нему можно обращаться с помощью 
комбинации клавиш {\bf Alt+Tab}. Чтобы вернуть окно в дочернее состояние,
используйте команду {\bf Return to Frame} Системного меню.	
\fi
\subsection{Window: Dock}
\public{IDM\_DOCK}

\ifenglish
This command {\bf Docks} current window. Docked windows are arranged along the
sides of frame window without overlapping. All non-docked windows occupy area
free of docked windows. They are clipped by this area and therefore never overlap
docked windows. Maximized non-docked window occupy all this area.

Docked windows are useful for information you want displayed permanently such
as project files list and messages window.

Once the window is docked, you can undock it using \ref[Undock]{IDM\_UNDOCK}
command of \ref[Window]{M\_WINDOWS} menu.
\else
Эта команда делает текущее окно {\bf неперекрывающимся}. Неперекрывающиеся окна
выстраиваются вдоль сторон окна оболочки, не закрывая друг друга. Все остальные
окна занимают свободное от неперекрывающихся окон пространство. Они ограничены
этим пространством, и, таким образом, никогда не оказываются поверх 
неперекрывающихся окон. Если максимизировать такое окно, то оно займет все это 
пространство. 

Как только окно сделано неперекрывающимся, вы можете отменить привязку,
используя команду \ref[Undock]{IDM\_UNDOCK} меню \ref[Window]{M\_WINDOWS}.
\fi
\subsection{Window: Undock}
\public{IDM\_UNDOCK}

\ifenglish
This command returns previously \ref[docked]{IDM\_DOCK} window to normal MDI
child state.
\else
Эта команда возвращает сделанное некогда \ref[неперекрывающимся]{IDM\_DOCK} окно
в нормальное дочернее MDI- состояние.
\fi
\subsection{Window: Windows List}
\public{IDM\_MOREWINDOWS}

\ifenglish
In classic MDI full windows list can be displayed only if there is no room in Window
menu and is called {\bf More Windows}. In XDS version you can display it any time
using {\bf Windows List} menu.

You can dock, undock, save or close the window right from Windows List dialog.
\else
В классическом MDI полный список окон может быть отображен, только если 
в меню Window недостаточно места, и называется он {\bf More Windows}. В варианте
XDS вы можете вывести его в любой момент с помощью меню {\bf Windows List}.

Вы можете сделать окно неперекрывающимся, отменить привязку, сохранить или 
закрыть окно прямо из диалога Windows List.
\fi
\section{Help Menu}
\public{M\_HELP}
\index{Help Menu}
\nominitoc

\ifenglish
The {\bf Help} menu contains commands that you use to display
the on-line help.
The following commands appear in the {\bf Help} menu:
\else
Меню {\bf Help} содержит команды, которые вы используете чтобы отображать
текущую подсказку.

В меню {\bf Help} содержатся следующие команды:
\fi
\begin{description}
\item \ref [Contents]{IDM\_HELP}
\ifenglish
  - Displays the XDS on-line help contents
  \else
  - Отображает содержание текущей подсказки XDS
  \fi
\item \ref [Keyboard]{IDM\_HELPKEYBOARD}
\ifenglish
  - Displays help on editor keyboard commands
  \else
  - Отображает подсказку по командам клавиатуры редактора
  \fi
\item \ref [Current Word]{IDM\_HELPKEYWORD}
\ifenglish
  - Displays help on the word under the cursor position
  \else
  - Отображает подсказку по слову, на котором находится курсор
  \fi
\item \ref [How to Use Help]{IDM\_HELPONHELP}
\ifenglish
  - Displays general information about the help system
  \else
  - Отображает общую информацию о справочной системе
  \fi
\item \ref [About]{IDM\_HELPABOUT}
\ifenglish
  - Displays XDS version and system information
  \else
  - Отображает версию XDS и системную информацию
  \fi
\end{description}

\subsection{Help: Contents}
\public{IDM\_HELP}

\ifenglish
Use the {\bf Contents} command to display the XDS on-line
help contents
\else
Используйте команду {\bf Contents}, чтобы отобразить содержание
подсказки.
\fi
\subsection{Help: Keyboard}
\public{IDM\_HELPKEYBOARD}

\ifenglish
Use the {\bf Keyboard} command to display \ref[help]{HELP\_KEYBOARD} on editor commands
\else
Используйте команду {\bf Keyboard}, чтобы отобразить \ref[подсказку]{HELP\_KEYBOARD}
к редактору.
\fi
\subsection{Help: Current Word}
\public{IDM\_HELPKEYWORD}

\ifenglish
Use the {\bf Current word} command to display the on-line
help for the word under the cursor.

Keyboard shortcut: {\bf F1}
\else
Чтобы вывести подсказку к слову, на котором находится курсор, используйте
команду {\bf Current word}.

Комбинация клавиш: {\bf F1}
\fi
\subsection{Help: How to Use Help}
\public{IDM\_HELPONHELP}

\ifenglish
Use the {\bf How to use help} command to display general
information about the on-line help system.
\else
Используйте команду {\bf How to use help}, чтобы отобразить общую
информацию по справочной системе.
\fi
\subsection{Help: About}
\public{IDM\_HELPABOUT}

\ifenglish
Use the {\bf About} command to display the XDS version,
copyright, and system information in a dialog box.
\else
Используйте команду {\bf About}, чтобы отобразить версию XDS,
патентную информацию и системную информацию в диалоговом окне.
\fi
