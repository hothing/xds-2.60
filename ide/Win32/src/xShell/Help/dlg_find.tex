\ifenglish
\section{Find Text dialog}
 \else
\section{диалог Find Text}
\fi
\public{SEARCH\_DIALOG}
\nominitoc

\ifenglish
This dialog is invoked by \ref[Edit]{M\_EDIT}:\ref[Find]{IDM\_FIND} menu command
or by pressing {\bf Ctrl+F} key.

To search the current file for a certain string:
 \else
Этот диалог вызывается командой меню \ref[Edit]{M\_EDIT}:\ref[Find]{IDM\_FIND}
или путем нажатия клавиши {\bf Ctrl+F}.

Чтобы найти в текущем файле определенную строку:
\fi
\begin{enumerate}
\item 
      \ifenglish
      Type it in the {\bf Text to find} field or select it from search history
      by pressing down arrow button next to the entry field.
       \else
      Наберите ее в поле {\bf Text to find} или выберите из архива поиска путем
      нажатия на кнопку со стрелкой рядом с полем ввода.
      \fi
\item 
      \ifenglish
      Select in {\bf Find text as} group which occurrences of the text you are interested in.
      Following options are available:
       \else
      В группе {\bf Find text as} определите, какие вхождения вашей строки вас интересуют
      \fi
      
      {\bf Any text} 
      \ifenglish
      -- any occurrence is acceptable
      \else
      -- любое вхождение приемлемо
      \fi
      {\bf Word prefix} 
      \ifenglish
      -- word break must precede the text
      \else
      -- перед текстом должно находиться начало слова
      \fi
      {\bf Word suffix} 
      \ifenglish
      -- word break must follow the text
      \else
      -- после текста должен следовать конец слова
      \fi
      {\bf Whole word}
     \ifenglish 
      -- text must be surrounded by word breaks
      \else
      -- текст должен быть ограничен промежутками между словами
      \fi
\item 
      \ifenglish
      Select the {\bf Case sensitive} checkbox if you want only
      case-matching occurrences of a string to be found
      \else
      Установите опцию {\bf Case sensitive}, если хотите чтобы отыскивались
      вхождения вашей строки с точностью до регистра
      \fi
\item 
      \ifenglish
      Choose the {\bf Search Range} and direction. You may search either {\bf From cursor forward}
      or {\bf From cursor backwards} or {\bf In whole text}.
      \else
      Выберите участок и направление поиска. Вы можете призводить поиск 
      {\bf впереди позиции курсора}  или {\bf позади позиции курсора} или
      {\bf во всем тексте}.
      \fi
\item 
      \ifenglish
      Select the {\bf OK} pushbutton
      \else
      Нажмите кнопку {\bf OK}
      \fi
\end{enumerate}

\ifenglish
You may select {\bf insert current word} checkbox to allow automatic selection
of current word in the editor as a search pattern.

If the text is found, it will be selected as a block.
After the first search you may use \ref[Edit]{M\_EDIT}:\ref[Find Next]{IDM\_FINDNEXT}
menu command or press {\bf Ctrl+L}.

There is also shortcut for searching for current word. You may press
{\bf Ctrl+Up} or {\bf Ctrl+Down} to search for current word either up or down.
\else
Вы можете установить опцию {\bf insert current word}, чтобы позволить 
автоматический выбор текущего слова в редакторе в качестве шаблона поиска.

Если текст найден, он будет отмечен как блок.
После первого поиска вы можете использовать команду меню \ref[Edit]{M\_EDIT}:\ref[Find Next]{IDM\_FINDNEXT}
или нажать {\bf Ctrl+L}.

Существует также сокращенный вариант команды поиска. Вы можете нажать
{\bf Ctrl+Up} или {\bf Ctrl+Down}, чтобы прозвести поиск слова вверх или вниз.
\fi
 %-----------------------------------------------------------------------
\begin{popup}
\caption{Insert current word}
\public{SEARCH\_INSERTCURRENT}

\ifenglish
When this checkbox is selected the search pattern field is automatically
initialized with current word from the editor window.
\else
Когда эта опция установлена, поле шаблона поиска автоматически инициализируется
текущим словом в окне редактирования.
\fi
\end{popup}

\begin{popup}
\caption{Text to find}
\public{SEARCH\_TEXT}

\ifenglish
Type the string you want to find - the {\em search string} in this
entry field. All the strings you type are kept in history list. You may select
one of previously searched patterns by pressing the down arrow button to the
right of this field.
\else
Наберите строку, которую хотите отыскать - пункт {\em search string} в поле ввода.
Все строки, которые вы вводите, сохраняются в архивном списке. Вы можете выбрать
один из изпользованных прежде шаблонов посредством нажатия клавиши со стрелкой
справа от этого поля.
\fi
\end{popup}

\begin{popup}
\caption{Case sensitive}
\public{SEARCH\_CASE}

\ifenglish
Select the {\bf Case sensitive} checkbox if you want to find only
case-matching occurrences of a string. For instance, if you search for
the string {\tt "Win32"}, the string {\tt "WIN32"} will be skipped.
\else
Установите опцию {\bf Case sensitive}, если хотите отыскать участки текста, 
содержащие вашу строку с точностью до регистра. Например, если вы ищете 
строку {\tt "Win32"}, строка {\tt "WIN32"} будет отброшена.
\fi
\end{popup}

\begin{popup}
\caption{Any text}
\public{SEARCH\_ANYTEXT}

\ifenglish
This option allows searching for all occurrences of a string, either
as separate words, or within other words. For example, pattern {\tt "ear"} will
be found in text {\tt "Search"}.
\else
Эта опция позволяет отыскивать все вхождения, где строка встречается
как отдельное слово, или как часть других. К примеру, шаблон {\tt "ear"} 
будет обнаружен в тексте {\tt "Search"}.
\fi
\end{popup}

\begin{popup}
\caption{Word prefix}
\public{SEARCH\_PREFIX}

\ifenglish
Allows the text to be found in the beginning but not in the middle of words
(that means, requires word break to precede text occurrences). For example,
pattern {\tt "Wind"} will be found in word {\tt"Windows"}.
\else
Позволяет отыскивать текст в начале, но не в середине слов (что означает
необходимость наличия промежутка между словами перед обнаруженным текстом).
Например, шаблон {\tt "Wind"} будет найден в слове {\tt"Windows"}. 
\fi
\end{popup}

\begin{popup}
\caption{Word suffix}
\public{SEARCH\_SUFFIX}

\ifenglish
Allows the text to be found in the end but not in the middle of words
(that means, requires word break to follow text occurrences). For example,
pattern {\tt "fix"} will be found in word {\tt"suffix"}.
\else
Позволяет отыскать текст в конце, но не в середине слов (что означает
требование к промежутку между словами находиться сразу за обнаруженным 
текстом). Например, шаблон {\tt "fix"} будет найден в {\tt"suffix"}.
\fi
\end{popup}

\begin{popup}
\caption{Whole word}
\public{SEARCH\_WHOLEWORD}


\ifenglish
Select the {\bf Whole words} check box if you want occurrences of a search
string within other words to be skipped during search.
So neither pattern {\tt "hole"} nor {\tt "ole"} nor letter {\tt "h"}
will be found in word {\tt "Whole"}.
\else
Установите опцию {\bf Whole words}, если хотите, чтобы поиск внутри 
других слов не велся. Так, ни шаблон {\tt "hole"}, ни {\tt "ole"}, ни буква 
{\tt "h"} не будут найдены в слове {\tt "Whole"}.
\fi
\end{popup}

\begin{popup}
\caption{From cursor forward}
\public{SEARCH\_FORWARD}

\ifenglish
Select this radiobutton if you want search to be performed
starting from the cursor position to the end of file.
\else
Включите этот переключатель, если хотите, чтобы поиск производился 
от текущей позиции курсора до конца файла. 
\fi
\end{popup}

\begin{popup}
\caption{From cursor backwards}
\public{SEARCH\_BACK}

\ifenglish
Select this radiobutton if you want search to be performed
starting from the cursor position to the beginning of the file.
The last occurrence (the closest to the cursor) will be found.
\else
Включите эту кнопку, если хотите, чтобы поиск производился, начиная от текущей
позиции курсора до конца файла. Будет обнаружено последнее (ближайшее к 
курсору) вхождение искомой строки.
\fi
\end{popup}

\begin{popup}
\caption{In whole text}
\public{SEARCH\_WHOLETEXT}

\ifenglish
Select this radiobutton if you want search to be started
from the top of file.
\else
Включите эту кнопку, если хотите, чтобы поиск производился от начала файла.
\fi
\end{popup}

\begin{popup}
\caption{OK}
\public{SEARCH\_OK}

\ifenglish
Select the {\bf OK} pushbutton to start search.
\else
Нажмите кнопку {\bf OK}, чтобы начать поиск.
\fi
\end{popup}

\begin{popup}
\caption{Cancel}
\public{SEARCH\_CANCEL}

\ifenglish
Select the {\bf Cancel} pushbutton to dismiss the dialog not performing a
search.
\else
Нажмите кнопку {\bf Cancel}, чтобы отказаться от поиска. 
\fi
\end{popup}

%=======================================================================

\ifenglish
\section{Find and Replace text dialog}
\else
\section{диалог Find and Replace text}
\fi
\public{REPLACE\_DIALOG}

\ifenglish
This dialog is invoked by \ref[Edit]{M\_EDIT}:\ref[Replace]{IDM\_REPLACE} menu command
or by pressing {\bf Ctrl+H} key.

To search the current file for a certain string and replace it
with another:
\else
Этот диалог вызывается командой меню \ref[Edit]{M\_EDIT}:\ref[Replace]{IDM\_REPLACE}
или нажатием клавиши {\bf Ctrl+H}.

Чтобы отыскать в текущем файле некоторую строку и заменить ее на другую:
\fi
\begin{enumerate}
\item 
      \ifenglish
      Type a {\em search string} in the {\bf Text to find} field
      \else
      Введите искомую строку в поле {\bf Text to find}
      \fi
\item 
      \ifenglish
      Type a {\em replacement string} in the {\bf Text to replace with} field
      \else
      Введите строку замещения в поле {\bf Text to replace with}
      \fi
\item  
      \ifenglish
      Select the {\bf Case sensitive} checkbox if you want only
      case-matching occurrences of a string to be replaced
      \else
      Установите опцию {\bf Case sensitive}, если вы хотите, чтобы 
      заменились только вхождения, содержащие искомую строку с точностью до 
      регистра
      \fi
\item 
      \ifenglish
      Select the {\bf Whole words} checkbox if you want a string not to
      be replaced if it occurs as a substring in a word
      \else
      Установите опцию {\bf Whole words}, если хотите чтобы ваша строка 
      не заменялась, в случае если она часть некоторого слова.
      \fi
\item 
      \ifenglish
      Select the {\bf Replace all} checkbox if you want all occurrences of
      a search string in the search range to be replaced automatically
      \else
      Установите опцию {\bf Replace all}, если хотите, чтобы замена
      всех вхождений, содержащих искомую строку, произошла автоматически.
      \fi
\item 
      \ifenglish
      Select the {\bf Prompt user} checkbox if you want to manually
      confirm each replace
      \else
      Установите опцию {\bf Prompt user}, если хотите подтвердить каждую замену 
      сами.
      \fi
\item 
       \ifenglish
       Choose the search range by selecting one of the {\bf From cursor},
      {\bf In whole text}, or {\bf In selection} pushbuttons
       \else
      Выберите участок поиска, нажав кнопки {\bf From cursor} 
      или {\bf In whole text} или {\bf In selection}.
        \fi
\item 
      \ifenglish
      Select the {\bf OK} pushbutton
      \else
      Нажмите кнопку {\bf OK}
      \fi
\end{enumerate}

\ifenglish
You may select {\bf insert current word} checkbox to allow automatic selection
of current word in the editor as a search pattern.

If you don't specify {\bf Replace all}, replace will stop after first replacement.
You may continue replacement from this point by selecting \ref[Replace Next]{IDM\_REPLACENEXT}
from the \ref[Edit]{M\_EDIT} menu or by pressing {\bf Ctrl+R}.
\else
Вы можете установить опцию {\bf insert current word}, чтобы позволить 
автоматический выбор текущего словав редакторе в качестве шаблона поиска.

Если вы не зададите {\bf Replace all}, то произойдет только одно замещение. 
Вы можете продолжить замещение с этого момента, выбрав \ref[Replace Next]{IDM\_REPLACENEXT} 
из меню \ref[Edit]{M\_EDIT} или нажав {\bf Ctrl+R}.
\fi
%-----------------------------------------------------------------------
\begin{popup}
\caption{Insert current word}
\public{REPLACE\_INSERTCURRENT}

\ifenglish
When this checkbox is selected the search string field is automatically
initialized with current word from the editor window.
\else
Когда установлена эта опция, поле строки поиска автоматически инициализируется
текущим словом в окне редактирования.
\fi
\end{popup}

\begin{popup}
\caption{Text to find}
\public{REPLACE\_TEXT}

\ifenglish
Type the string you want to find -- the {\em search string} in this
entry field.
\else
Введите строку, которую вы хотите найти в этом поле ввода.
\fi
\end{popup}

\begin{popup}
\caption{Text to replace with}
\public{REPLACE\_BY\_TEXT}

\ifenglish
Type the string with which you want a search string to be replaced --
the {\em replacement string} in this entry field. Leave this field empty
to delete occurrences of a search string.
\else
В этом поле ввода наберите строку, на которую вы хотите заменить строку 
поиска -- {\em строку замещения}. Оставьте это поле пустым, если хотите
удалить вхождения строки поиска.
\fi
\end{popup}

\begin{popup}
\caption{Case sensitive}
\public{REPLACE\_CASE}

\ifenglish
Select the {\bf Case sensitive} checkbox if you want to find only
case-matching occurrences of a string. For instance, if you search for
the string {\tt "Win32"}, the string {\tt "WIN32"} will be skipped.
\else
Установите опцию {\bf Case sensitive}, если хотите отыскать вхождения строки 
с точностью до регистра. Например, если вы ищете строку {\tt "Win32"}, 
строка {\tt "WIN32"} будет отброшена.
\fi
\end{popup}

\begin{popup}
\caption{Whole words}
\public{REPLACE\_WORDS}

\ifenglish
Select the {\bf Whole words} check box if you want occurrences of a search
string within other words to be skipped during search. For instance,
{\tt "place"} will not be found in {\tt "replace"}.
\else
Установите опцию {\bf Whole words}, если хотите чтобы поиск внутри слов не велся Например
{\tt "place"} не будет найдена в {\tt "replace"}.
\fi
\end{popup}

\begin{popup}
\caption{Replace all}
\public{REPLACE\_ALL}

\ifenglish
Select the {\bf Replace all} checkbox if you want all occurrences of a search
string within a search range to be replaced. If this checkbox isn't selected,
the replacement stops after the first occurrence of a text.
\else
Установите опцию {\bf Replace all}, если хотите, чтобы все вхождения строки поиска
подверглись замещению. Если эта опция не установлена, произойдет только 
одно замещение.
\fi
\end{popup}

\begin{popup}
\caption{Prompt user}
\public{REPLACE\_PROMPT}

\ifenglish
Select the {\bf Prompt user} checkbox if you want to manually confirm each
replace. For each occurrence of a text you will be asked if you wish to 
replace it, to skip it or to cancel the operation.
\else
Установите опцию {\bf Prompt user}, если хотите подтверждать каждое замещение 
вручную. Для каждого вхождения строки поиска вы должны будете подтвердить
замену, пропустить ее либо отменить операцию.
\fi
\end{popup}

\begin{popup}
\caption{From cursor}
\public{REPLACE\_FROMCURSOR}

\ifenglish
Select the {\bf From cursor} radiobutton if you want replacement to be performed
starting from the cursor position to the bottom of file.
\else
Включите кнопку {\bf From cursor}, если хотите чтобы замещение происходило 
начиная от текущей позиции курсора до конца файла.
\fi
\end{popup}

\begin{popup}
\caption{In whole text}
\public{REPLACE\_WHOLETEXT}

\ifenglish
Select the {\bf In whole text} radiobutton if you want to replace the text in whole
file.
\else
Включите кнопку {\bf In whole text}, если хотите заменить текст во всем файле.
\fi
\end{popup}

\begin{popup}
\caption{In selection}
\public{REPLACE\_SELECTION}

\ifenglish
Select the {\bf In selection} radiobutton if you want search to be performed
within the current selection. This option is disabled if there is no selection.
\else
Включите кнопку {\bf In selection}, если хотите чтобы поиск происходил в текущем 
выделенном блоке. Эта опция не доступна, если выделения нет.
\fi
\end{popup}

\begin{popup}
\caption{OK}
\public{REPLACE\_OK}

\ifenglish
Select the {\bf OK} pushbutton to start replace.
\else
Нажмите кнопку {\bf OK}, чтобы начать замещение.
\fi
\end{popup}

\begin{popup}
\caption{Cancel}
\public{REPLACE\_CANCEL}

\ifenglish
Select the {\bf Cancel} pushbutton if you decided not to replace any text.
\else
Нажмите кнопку {\bf Cancel}, если вы решили не производить замещение.
\fi
\end{popup}

%=======================================================================

\ifenglish
\section{Replace prompt dialog}
\else
\section{диалог Replace prompt}
\fi
\public{REPLACE\_PROMPT\_DIALOG}

\ifenglish
This dialog is displayed when a replace action is about to be performed
and the {\bf Prompt user} mode is enabled. Select:
\else
Этот диалог появляется перед произведением замены в случае, если установлена
опция. Нажмите: 
\fi
\begin{description}
\item \ifenglish
      [{\bf Yes}] to confirm replace of the selected search string
      \else
      [{\bf Yes}], чтобы подтвердить замену        
      \fi
\item \ifenglish
     [{\bf No}] to skip replace
     \else
     [{\bf No}], чтобы пропустить замену
     \fi
\item \ifenglish
      [{\bf Yes to All}] to replace the remaining occurrences of the
   search string without prompting
     \else
     [{\bf Yes to All}], чтобы заменить все оставшиеся вхождения
     строки поиска без запроса
     \fi
\item \ifenglish
      [{\bf Cancel}] to terminate the replace action
     \else
     [{\bf Cancel}], чтобы завершить процесс замены
     \fi
\end{description}

%-----------------------------------------------------------------------

\begin{popup}
\caption{Search text}
\public{REPLACE\_PROMPT\_TEXT1}
\ifenglish
This is the string you search.
\else
Это строка, которую вы ищете.
\fi
\end{popup}

\begin{popup}
\caption{Replace text}
\public{REPLACE\_PROMPT\_TEXT2}

\ifenglish
This is the string you replace the search string with.
\else
Это строка, на которую вы заменяете строку поиска.
\fi
\end{popup}

\begin{popup}
\caption{Yes}
\public{REPLACE\_PROMPT\_OK}
\ifenglish
Confirm replacement of this occurrence.
\else
Подтвердить замещение данного вхождения.
\fi
\end{popup}

\begin{popup}
\caption{No}
\public{REPLACE\_PROMPT\_NO}

\ifenglish
Skip this occurrence.
\else
Пропустить это вхождение.
\fi
\end{popup}

\begin{popup}
\caption{Yes to All}
\public{REPLACE\_PROMPT\_YESALL}

\ifenglish
Replace all remaining occurrences without asking.
\else
Заменить все оставшиеся вхождения без запроса.
\fi
\end{popup}

\begin{popup}
\caption{Cancel}
\public{REPLACE\_PROMPT\_Cancel}
\ifenglish
Cancel replacement. Don't replace anything more.
\else
Отменить замену. Ничего больше не замещать.
\fi
\end{popup}

%=======================================================================
\section{Search in files}
\public{MSEARCH\_DIALOG}
\nominitoc

\ifenglish
This dialog is invoked by \ref[Edit]{M\_EDIT}:\ref[Find in Files]{IDM\_FINDFILES} menu command
or by pressing {\bf Ctrl+Shift+F} key.
\else

Этот диалог появляется при исполнении команды меню \ref[Edit]{M\_EDIT}:\ref[Find in Files]{IDM\_FINDFILES} 
или при нажатии клавиши {\bf Ctrl+Shift+F}.
\fi
\ifenglish
To search the group of files for a certain string:
\else
Чтобы найти определенную строку в группе файлов:
\fi
\begin{enumerate}
\item 
      \ifenglish
      Type it in the {\bf Text to find} field or select it from search history
      by pressing down arrow button next to the entry field.
      You may also leave this field blank if you wish just build list of files
      with names that match given pattern.
      \else
      Наберите ее в поле {\bf Text to find}, или выберите в архиве поиска посредством 
      нажатия кнопки со стрелкой рядом с полем ввода. 
      Вы также можете оставить это поле пустым, если вы лишь хотите получить
      список файлов, имена которых соответствуют данному шаблону.
      \fi
\item 
      \ifenglish
      Specify where you wish to find. Following options are available:
      \else
      Определите, где вы хотите производить поиск. Возможны следующие варианты:
      \fi
      \begin{itemize}
      \item \ifenglish
            {\bf Open windows}
            \else
            {\bf открытые окна}
            \fi
      \item \ifenglish
            {\bf Project files}
            \else
            {\bf файлы проекта}
            \fi
     \item 
            \ifenglish
            {\bf Disk files}.
            \else
            {\bf файлы на диске}
            \fi
    \end{itemize}

       \ifenglish
      In the last case you should also specify the directory to find in
      and, if you wish, turn on searching subdirectories.
      \else
      В последнем случае вы должны также указать директорию поиска и, 
      если пожелаете, включить поиск в поддиректориях.
      \fi
\item \ifenglish
      Specify file name mask if you wish to find in some subset of files.
      File name mask is a regular expression in the same format as one in
      redirection file.
      \else
      Укажите шаблон имени файла, если хотите производить поиск в некотором
      подмножестве файлов.
      \fi
\item \ifenglish
      Select in {\bf Find text as} group which occurrences of the text you are interested in.
      Following options are available:
      \else
      В меню {\bf Find text as} выберите, какие места, содержащие ваш текст, вас 
      интересуют.
      Возможны следующие варианты:
      \fi
      \begin{itemize}
      \item {\bf Any text} 
      \ifenglish 
      -- any occurrence is acceptable
      \else
      -- любое вхождение приемлемо
      \fi
      \item {\bf Word prefix} 
       \ifenglish
       -- word break must precede the text
       \else
       -- перед текстом должен быть промежуток между словами
       \fi
      \item {\bf Word suffix} 
       \ifenglish
       -- word break must follow the text
       \else
       -- за текстом следует промежуток между словами
       \fi
      \item {\bf Whole word} 
       \ifenglish
       -- text must be surrounded by word breaks
       \else
       -- текст должен быть окружен промежутками между словами
       \fi
      \end{itemize}
\item 
      \ifenglish
      Select the {\bf Case sensitive} checkbox if you want only
      case-matching occurrences of a string to be found
      \else
      Установите опцию {\bf Case sensitive}, если хотите чтобы поиск велся 
      с точностью до регистра
      \fi
\item 
      \ifenglish
      Select the {\bf OK} pushbutton
      \else
      Нажмите кнопку {\bf OK}
      \fi
\end{enumerate}

\ifenglish
When you press OK, the dialog appears with progress information and {\bf Cancel}
button. You may cancel the operation by pressing it.

After the search is done you get list of files in which the pattern is found.
Several such lists may be opened at the same time. Select the {\bf Open new window}
checkbox if you wish each new result to be opened in separate window. Otherwise
existing window, if any, will be opened and filled with new file list.

Selecting file from the list opens it in the IDE and finds first occurrence of
a search string in the file. Press {\bf Ctrl+L} (Search Next) to see other occurrences.

Note that if some files are opened in the IDE and modified, search will always
work on those modified versions, even when the range is specified as Directory.
\else
Когда вы нажимаете OK, появляется диалог с информацией о ходе поиска и кнопкой
{\bf Cancel}. Вы можете прекратить поиск, если нажмете на нее.

После того, как поиск завершен, вы располагаете списком файлов, где встретился 
ваш шаблон. Одновременно может быть открыто несколько таких списков. Установите 
опцию {\bf Open new window}, если хотите, чтобы каждый новый результат отображался 
в отдельном окне. Иначе список файлов попадет в уже существующее окно. 

Выбор файла из списка открывает его в IDE и находит первое вхождение
строки поиска. Чтобы перейти к следующему вхождению, нажмите {\bf Ctrl+L}.

Заметьте, что если несколько файлов открыты и изменены в IDE, то поиск будет
производиться в этих измененных вариантах, даже если область поиска определена 
как Директория.
\fi
%-----------------------------------------------------------------------

\begin{popup}
\caption{OK}
\public{MSEARCH\_OK}

\ifenglish
Start search operation. Dialog box will appear with {\bf Cancel} button
that allows you to cancel the operation.
\else
Начать процедуру поиска. Появится диалоговое окно с кнопкой {\bf Cancel},
что позволяет вам прекратить процедуру.
\fi
\end{popup}

\begin{popup}
\caption{Cancel}
\public{MSEARCH\_CANCEL}

\ifenglish
Press {\bf Cancel} button if you decided not to perform file search.

\else
Нажмите кнопку {\bf Cancel}, если решили не производить поиск.
\fi
\end{popup}

\begin{popup}
\caption{Any text}
\public{MSEARCH\_ANYTEXT}

\ifenglish
This option allows searching for all occurrences of a string, either
as separate words, or within other words. For example, pattern {\tt "ear"} will
be found in text {\tt "Search"}.

\else
Эта опция позволяет отыскивать все вхождения, где строка содержится как отдельное
слово или как часть других слов. Например, шаблон {\tt "ear"} будет
найден в тексте {\tt "Search"}.
\fi
\end{popup}

\begin{popup}
\caption{Word prefix}
\public{MSEARCH\_PREFIX}

\ifenglish
Allows the text to be found in the beginning but not in the middle of words
(that means, requires word break to precede text occurrences). For example,
pattern {\tt "Wind"} will be found in word {\tt"Windows"}.
\else
Позволяет отыскивать текст в начале, а не в середине слов (что означает, что
перед местом, где встретился текст должен быть промежуток между словами).
Например, шаблон {\tt "Wind"} будет найден в слове {\tt"Windows"}.
\fi
\end{popup}

\begin{popup}
\caption{Word suffix}
\public{MSEARCH\_SUFFIX}

\ifenglish
Allows the text to be found in the end but not in the middle of words
(that means, requires word break to follow text occurrences). For example,
pattern {\tt "fix"} will be found in word {\tt"suffix"}.
\else
Позволяет отыскать текст в конце, но не в середине слов (что означает
требование к промежутку между словами находиться сразу за обнаруженным 
текстом). Например, шаблон {\tt "fix"} будет найден в {\tt"suffix"}.
\fi
\end{popup}

\begin{popup}
\caption{Whole word}
\public{MSEARCH\_WHOLEWORD}

\ifenglish
Select the {\bf Whole words} check box if you want occurrences of a search
string within other words to be skipped during search.
So neither pattern {\tt "hole"} nor {\tt "ole"} nor letter {\tt "h"}
will be found in word {\tt "Whole"}.
\else
Установите опцию {\bf Whole words}, если хотите, чтобы вхождения строки поиска в 
середине слов не рассматривались. Так ни {\tt "hole"}, ни {\tt "ole"}, ни буква 
 {\tt "h"} не будут найдены в слове {\tt "Whole"}.
\fi
\end{popup}

\begin{popup}
\caption{Insert current word}
\public{MSEARCH\_INSERTCURRENT}

\ifenglish
When this checkbox is selected the search string field is automatically
initialized with current word from the editor window.
\else
Когда эта опция установлена, строка поиска автоматически инициализируется 
текущим словом в окне редактирования.
\fi
\end{popup}

\begin{popup}
\caption{Text to find}
\public{MSEARCH\_TEXT}

\ifenglish
Type the string you want to find - the {\em search string} in this
entry field. All the strings you type are kept in history list. You may select
one of previously searched patterns by pressing the down arrow button to the
right of this field.

Leaving this field blank builds list of files in given search area that match
specifies file name mask.
\else
Наберите строку, которую хотите отыскать в этом поле ввода. Все строки, 
которые вы вводите, хранятся в архивном списке. Вы можете выбрать один из ранее 
использованных шаблонов путем нажатия кпопки со стрелкой справа от этого поля.

Если оставить это поле пустым, то будет построен список файлов, имена которых
отвечают указанному шаблону имени файла.
\fi
\end{popup}

\begin{popup}
\caption{Project files}
\public{MSEARCH\_PROJFILES}

\ifenglish
Instructs the IDE to search in all project files. The project files list
is built during project make.  Only files whose names
matches the specified file mask are searched.
\else
Заставляет IDE производить поиск во всех проектных файлах. Список проектных 
файлов строится во время компоновки проекта. Поиск прозводится только в 
файлах, имена которых отвечают указанному шаблону имени файла.
\fi
\end{popup}

\begin{popup}
\caption{Open windows}
\public{MSEARCH\_OPEN}

\ifenglish
Instructs IDE to search in all open windows whose names match the specified
file name mask.
\else
Заставляет IDE производить поиск во всех открытых окнах, названия которых
отвечают указанному шаблону имени файла.
\fi
\end{popup}

\begin{popup}
\caption{Directory}
\public{MSEARCH\_DISK}

\ifenglish
Instructs IDE to search in disk files in given directory. Only files whose names
matches the specified file mask are searched.
\else
Заставляет IDE производить поиск в файлах в данной директории. Поиск будет
произведен только в файлах, имена которых отвечают заданному шаблону.
\fi
\end{popup}

\begin{popup}
\caption{Case sensitive}
\public{MSEARCH\_CASE}

\ifenglish
Select the {\bf Case sensitive} checkbox if you want to find only
case-matching occurrences of a string. For instance, if you search for
the string {\tt "Win32"}, the string {\tt "WIN32"} will be skipped.
\else
Установите опцию {\bf Case sensitive}, если хотите отыскать участки текста, 
содержащие вашу строку с точностью до регистра. Например, если вы ищете строку
{\tt "Win32"}, строка {\tt "WIN32"} будет отброшена.
\fi
\end{popup}

\begin{popup}
\caption{Open new window}
\public{MSEARCH\_NEWWINDOW}

\ifenglish
If you already have one search result window and this checkbox is checked
a new window will be created after the search. Otherwise current window will
be used.
\else
Если у вас уже имеется одно окно с результатами поиска, и эта опция установлена,
то после очередного поиска будет создано новое окно. Иначе будет использовано
текущее окно.
\fi
\end{popup}

\begin{popup}
\caption{Directory}
\public{MSEARCH\_DIR}

\ifenglish
If {\bf Directory} radio button is selected you set directory to find files in here.
You may also check {\bf and subdirectories} check box to search in subdirectories of this
directory.
\else
Если включена кнопка {\bf Directory}, то здесь вы указываете директорию, в которой 
будет производиться поиск файлов. Вы также можете установить опцию, чтобы поиск 
велся также и в поддиректориях данной директории.
\fi
\end{popup}

\begin{popup}
\caption{and subdirectories}
\public{MSEARCH\_SUBDIR}

\ifenglish
Specify that you wish to search in subdirectories as well as in specified
directory. The search will happen in whole directory subtree, without any
depth limit.
\else
Указывает, что вы хотите производить поиск и в поддиректориях данной директории.
Поиск произойдет во всем поддереве директорий без ограничения глубины.
\fi
\end{popup}

\begin{popup}
\caption{File name mask}
\public{MSEARCH\_MASK}

\ifenglish
Specify in which files you wish to search the pattern. Any regular expression
is allowed here. Leaving this field blank invokes searching in all files. Be careful!
\else
Укажите, в каких файлах вы хотите отыскать ваш шаблон. Здесь возможно любое
регулярное выражение. Если оставить это поле пустым, то поиск будет произведен 
во всех файлах. Будьте осторожны.
\fi
\end{popup}

\begin{popup}
\caption{Browse}
\public{MSEARCH\_BROWSE}

\ifenglish
Select directory where you wish to search using standard directory selection
dialog.
\else
Выберите директорию, где вы хотите произвести поиск, используя стандартный диалог
выбора директории.
\fi
\end{popup}

