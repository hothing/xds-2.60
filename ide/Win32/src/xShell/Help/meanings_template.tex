\ifenglish
\section{Project template}
\else
\section{Шаблон проекта}
\fi
\public{ProjectTemplate}

\ifenglish
When new project file is created it is not initially empty. It contains
reference to main module. It may also contain some predefined values
for project options. This is accomplished by creating new project file
on basis of {\bf project file template}.

Project file template is a text file, usually with {\bf .tpr} extension.
During project creation this file is copied into target directory. As the
file is copied all \ref{variables} found in it are expanded (replaced by their
values).
\else
Новый файл проекта создается непустым. Он содержит ссылку на главный модуль.
Также он может содержать некоторые стандартные значения для опций проекта.
Это получается в результате создания нового файла проекта на основе
{\bf файла шаблона проекта}.

Файл шаблона проекта -- это текстовый файл, обычно с расширением {\bf .tpr}.
Во время создания проекта этот файл копируется в целевую директорию. 
Как только файл скопирован, все обнаруженные в нем \ref[переменные]{variables}  
заменяются на соответствующие значения.
\fi
\ifenglish
Variables that refer to the project name are allowed,
despite the fact the project does not exist yet.

Note that {\bf !module} directive is not copied from the template. 
If project file resulting from a template contains {\bf !module} directive, the file
name in this directive is replaced by file name provided in
\ref[Create New Project dialog]{NEWPROJ\_DIALOG}. Otherwise this directive with this
file is appended at the end of the project, unless start file name in a dialog is left blank.
\else
Допускаются переменные, ссылающиеся на имя проекта, несмотря на то,
что проект еще не существует.

Заметьте, что директива {\bf !module} не копируется из шаблона. 
Если получающийся в результате использования шаблона файл проекта содержит
директиву {\bf !module}, то имя файла в этой директиве заменяется на имя файла
из \ref[диалога Create New Project]{NEWPROJ\_DIALOG}. В противном случае эта
директива с этим файлом заносится в конец проекта, если только имя стартового 
файла в диалоге не оставлено пустым. 
\fi
\ifenglish
Initially project template file name is equal to
\else
Сперва имя файла шаблона проекта полагается равным 
\fi
\verb'$(xdsdir)\$(xdsname).tpr'

\ifenglish
This allows to have different project templates for different compiler executables
(for example, one template for native compiler, another one for converter to C).

Project file template to use can be configured in
\ref[Create New Project]{NEWPROJ\_DIALOG} dialog.
Default value for template name can be changed in
\ref[New Project Creation Options]{CFGPROJ\_DIALOG}
dialog that is invoked by
\ref[Configure]{M\_CONFIGURE}:\ref[Project Creation]{IDM\_CONFIGUREPROJECT} menu command.
\else
Это позволяет иметь разные шаблоны проектов для разных исполняемых файлов компилятора.

Шаблон файла проекта может быть настроен в диалоге
\ref[Create New Project]{NEWPROJ\_DIALOG}.
Значение по умолчанию для шаблона может быть изменено в диалоге 
\ref[New Project Creation Options]{CFGPROJ\_DIALOG},
который вызывается с помощью команды меню
\ref[Configure]{M\_CONFIGURE}:\ref[Project Creation]{IDM\_CONFIGUREPROJECT}.
\fi
%---------------------------------------------------------------------------
\ifenglish
\section{Redirection file template}
\else
\section{Шаблон файла путей поиска}
\fi
\public{RedirectionTemplate}

\ifenglish
When new project is created a {\bf Redirection file} is usually created in
project directory (this can be turned off in project creation dialog).

This file contains common file name patterns and lists of directories to find
them. Some patterns and directories must be present in the file to keep system
working -- for example, directories for run-time library components must be provided.
That's why redirection file is constructed on basis of
{\bf Redirection file template}.
\else
Когда создается новый проект, в директории проекта обычно создается 
{\bf файл путей поиска} (это можно отключить в диалоге создания проекта).

Этот файл содержит общие шаблоны имен файлов и списки директорий для их поиска.
Некоторые шаблоны и директории должны присутствовать в файле чтобы поддерживать
работу системы -- например, должны указываться директории для компонент библиотеки
рабочих программ. Вот почему файл путей поиска создается на основе 
{\bf шаблона файла путей поиска}.
\fi
\ifenglish
Redirection file template is a text file, usually with {\bf .trd} extension.
During project creation this file is copied into target directory. As the
file is copied all \ref{variables} found in it are expanded (replaced by their
values).

Variables that refer to the project name are allowed,
despite the fact the project does not exist yet. This allows using
file patterns and paths that depend on project.
\else
Шаблон файла путей поиска -- это текстовый файл, обычно с расширением {\bf .trd}.
Во время создания проекта этот файл копируется в целевую директорию. Как только
файл скопирован, все обнаруженные в нем \ref[переменные]{variables} заменяются
на соответствующие значения.

Допускаются переменные, ссылающиеся на имя проекта, несмотря на то,
что проект еще не существует. Это дает возможность использовать шаблоны 
файлов и путей в зависимости от проекта. 
\fi
\ifenglish
Initially redirection file name is defined as
\else
Сперва имя файла путей поиска определяется как
\fi
\verb'$(xdsdir)\$(xdsname).trd'

\ifenglish
The name of redirection file created is
\else
Имя создаваемого файла путей поиска -- 
 \fi
\verb'$(projdir)\$(xdsname).red'

\ifenglish
So both redirection file template and redirection file itself depend on
XDS compiler executable name, so multiple redirection files may exist
for multiple compilers.

Redirection file template to use can be configured in
\ref[Create New Project dialog]{NEWPROJ\_DIALOG}.
Default value for template name can be changed in
\ref[New Project Creation Options]{CFGPROJ\_DIALOG} dialog
that is invoked by
\ref[Configure]{M\_CONFIGURE}:\ref[Project Creation]{IDM\_CONFIGUREPROJECT} menu command.
\else
Таким образом, как шаблон файла путей поиска, так и сам файл путей поиска 
зависят от имени исполняемого файла XDS-компилятора, так что для многих 
компиляторов могут существовать многие файлы путей поиска.

Шаблон файла путей поиска может быть настроен в диалоге
\ref[Create New Project]{NEWPROJ\_DIALOG}.
Значение по умолчанию для шаблона может быть изменено в диалоге 
\ref[New Project Creation Options]{CFGPROJ\_DIALOG},
который вызывается с помощью команды меню
\ref[Configure]{M\_CONFIGURE}:\ref[Project Creation]{IDM\_CONFIGUREPROJECT}.
\fi     