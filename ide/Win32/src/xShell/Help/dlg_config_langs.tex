\ifenglish
\section{Configure languages}
\else
\section{Настройка языков}
\fi
\public{CFGLANG\_DIALOG}

\ifenglish
This dialog allows to define set of \ref[programming languages]{Languages}
that the system recognize.

All known languages are collected in the list. For each language you may define
\ref[language driver]{LangDriver} -- the DLL responsible for actions that depend
on a language such as syntax highlighting. Type the name of language driver
in the {\bf Language driver} edit field or select it in a standard
file dialog by pressing {\bf Browse} button.
\else
Этот диалог позволяет определять множество распознаваемых системой
\ref[языков программирования]{Languages}.

В списке собраны все известные языки. Для каждого языка вы можете определить   
\ref[драйвер языка]{LangDriver} -- DLL, ответственную за действия, которые 
зависят от языка, как, например, подсветка синтаксиса. Наберите название драйвера 
языка в поле редактирования {\bf Language driver} или выберите его в стандартном 
файловом диалоге (кнопка {\bf Browse}).
\fi
\ifenglish
You must specify set of file extensions for files that are recognized as
belonging to this language. Type it in {\bf Extensions} entry field. Extensions
are typed without {\bf "."} character and are separated by semicolons.

You may add new or remove existing language by pressing {\bf Add} or {\bf Remove}
button. You will be prompted for a name of new language.

You may configure a language by pressing {\bf Configure} button. If the driver
provides configuration dialog it will appear, otherwise the message will appear
that describes the situation. So pressing this button lets you know if the driver
is found and loaded.

Configuration is kept separately for each language, not just for each driver.
\else
Вы можете указать набор расширений файлов, которые будут считаться 
принадлежащими этому языку. Введите это в поле ввода {\bf Extensions}. 
Расширения вводятся без символа {\bf "."} и разделяются точкой с запятой.

Вы можете добавить новый или удалить существующий язык нажатием кнопки {\bf Add}
или {\bf Remove}. Будет выведен запрос о названии нового языка.

Вы можете настраивать язык, используя кнопку {\bf Configure}. Если позволяет
драйвер, то будет выведен конфигурационный диалог, иначе появится описывающее
ситуацию сообщение. Таким образом, нажатие этой кнопки дает знать, найден и 
загружен ли драйвер.
 
Настройки для каждого языка хранятся отдельно (не для каждого драйвера).
\fi
%---------------------------------------------------------------------

\begin{popup}
\caption{OK}
\public{CFGLANG\_OK}

\ifenglish
Save the changes made in the languages configuration and close the dialog.
\else
Сохранить изменения, произведенные в настройках языков, и закрыть диалог.
\fi
\end{popup}

\begin{popup}
\caption{Cancel}
\public{CFGLANG\_CANCEL}

\ifenglish
Close the dialog without saving changes made in languages configuration.
\else
Закрыть диалог без сохранения изменений в настройках языков.
\fi
\end{popup}

\begin{popup}
\ifenglish
\caption{Languages available}
\else
\caption{Доступные языки}
\fi
\public{CFGLANG\_LIST}

\ifenglish
List of all known programming languages
\else
Список всех известных языков программирования
\fi
\end{popup}

\begin{popup}
\ifenglish
\caption{Language driver}
\else
\caption{Драйвер языка}
\fi
\public{CFGLANG\_DRIVER}
\public{CFGLANG\_DRIVERTITLE}

\ifenglish
Language driver for currently selected language. The driver is a DLL that
supplies functions responsible for syntax highlighting. It is possible to
have language without language driver but syntax highlighting will be
inaccessible for it.

\else
Драйвер языка для выбранного языка. Драйвер -- это DLL, которая поставляет 
функции, ответственные за подсветку синтаксиса. Существует возможность 
использовать язык без драйвера, но тогда для него будет невозможна подсветка
синтаксиса.
\fi
\end{popup}

\begin{popup}
\caption{Add}
\public{CFGLANG\_ADD}

\ifenglish
Add new language. You will be asked for new language name. New language has neither
file extensions set for it nor language driver.
\else
Добавить новый язык. Вас запросят об имени нового языка. Новый язык не содержит ни
набора расширений, ни драйвера.
\fi
\end{popup}

\begin{popup}
\caption{Remove}
\public{CFGLANG\_REMOVE}

\ifenglish
Remove selected language from language list.
\else
Удалить выбранный язык из списка.
\fi
\end{popup}

\begin{popup}
\ifenglish
\caption{Extensions}
\else
\caption{Расширения}
\fi
\public{CFGLANG\_EXTENSIONS}
\public{CFGLANG\_EXTENSIONS_TITLE}

\ifenglish
Type here semicolon-separated list of extensions of files that must
be recognized as belonging for current language. Don't use {\bf "."}
character. For example, this line for Modula-2 looks like this"
\verb'def;mod'

\else
Введите здесь список расширений файлов, которые будут считаться принадлежащими
языку, разделяя их точкой с запятой. Не используйте символ {\bf "."}.
К примеру, строчка для Modula-2 выглядит так \verb'def;mod'
\fi
                     
\end{popup}

\begin{popup}
\caption{Browse}
\public{CFGLANG\_BROWSE}

\ifenglish
Select a language driver DLL using standard File dialog.
\else
Выберите DLL языкового драйвера с помощью стандартного файлового диалога.
\fi
\end{popup}

\begin{popup}
\caption{Configure}
\public{CFGLANG\_CONFIGURE}

\ifenglish
Configure current language driver. A driver may provide a dialog for configuration.
This command invokes this dialog. If driver can't be loaded or if it doesn't
provide a dialog the message will appear with description of situation.
Even if you don't need any configuration you can press this button to see if language
driver can be found, loaded and initialized.

The results of the configuration dialog are stored in the main XDS configuration
file under the name of a language. Separate configuration is kept for each language,
not just for each driver.
\else
Настроить текущий драйвер языка. Драйвер может предоставлять диалог для настройки.
Эта команда выводит этот диалог. Если драйвер не может быть загружен, или если он
не предоставляет диалог, то появится сообщение с описанием ситуации.
Даже если вам и не нужна настройка, вы можете нажать эту кнопку, чтобы проверить,
может ли драйвер быть найден, загружен и инициализирован. 

Результаты работы конфигурационного диалога сохраняются в главном конфигурационном 
файле XDS под именем языка. Настройки для каждого языка хранятся отдельно (не для 
каждого драйвера).
\fi
\end{popup}
