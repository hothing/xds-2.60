\ifenglish
\section {Configure Printing dialog}
\else
\section {диалог Configure Printing}
\fi
\public{CFGPRINT\_DIALOG}
\nominitoc

\ifenglish
This dialog allows you to configure how your files are printed.

You can select paper size, source and orientation and set up page margins
using {\bf Page setup} button.

You can select printer font using {\bf Choose font} button.

You can also select decorations of printed page. You can configure pages to
have headers which can optionally include {\bf Page numbers} and {\bf File name}
by selecting these checkboxes. It is possible to select either short or full
file names printed by {\bf File path} checkbox.

You can turn on printing line numbers with each line by {\bf Line numbers}
checkbox.

\else
Этот диалог предоставляет средства настройки печати файлов.  

Вы можете выбрать размер листа, устройство подачи бумаги и ориентацию печати 
и установить поля с помощью кнопки {\bf Page setup}.

Вы можете выбрать шрифт принтера, используя кнопку {\bf Choose font}.

Вы также можете выбрать элементы отделки печатного листа. Установив 
соответствующие контроли, вы можете сделать так, чтобы листы печатались с 
заголовком, который может включать {\bf номер листа} ({\bf Page numbers}) 
и {\bf имя файла} ({\bf File name}). Можно выдавать как краткое, так и полное 
имя распечатываемого файла (выбор с помощью контроля {\bf File path}).

Вы можете включить печать номеров строк в каждой строке с помощью контроля.
\fi
{\bf Line numbers}

\ifenglish
Files often contain lines that are too long to fin into printer page. Most
systems simply truncate these lines. XDS IDE has three modes of handling long
lines. You can choose the mode in {\bf Long lines handling} group.
Long lines can be truncated by selecting {\bf Truncate} button.
They can be continued on the next line by selecting {\bf Continue on next line}
button. In this case one source line can occupy more than one printer lines.
Lines are split at the page boundary. No word wrapping is provided.
At last, lines can be continued on next page by selecting {\bf Continue on next page}
button. In this case additional page will be printed with trailers of long lines.
As many additional pages as necessary will be printed. All pages can be then glued
together to make one wide page.

Actual printing is invoked by \ref[File]{M\_FILE}:\ref[Print]{IDM\_PRINT} menu item.

\else
Файлы часто содержат слишком длинные по сравнению с размерами печатного листа 
строки. Большинство систем просто обрывает такие строки. В XDS IDE есть три 
режима обработки длинных строк. Вы можете выбрать один из них в группе 
{\bf Long lines handling}. Длинные строки могут быть обрезаны с помощью кнопки
{\bf Truncate}. Они могут быть продолжены на следующей строке с помощью кнопки
{\bf Continue on next line}. Строки разбиваются на границе листа. Это происходит
без заворачивания слов. Наконец, строки могут быть продолжены на следующем листе
с помощью кнопки  {\bf Continue on next page}. В этом случае будет напечатан 
дополнительный лист, содержащий обрывки длинных строк. Дополнительных листов
будет напечатано столько, сколько необходимо. Все листы могут затем быть склеены,
образуя один широкий лист.

Процесс печати начинается после выбора пункта меню
\ref[File]{M\_FILE}:\ref[Print]{IDM\_PRINT}.
\fi
%--------------------------------------------------------------------------
\begin{popup}
\caption{OK}
\public{CFGPRINT\_OK}
\ifenglish
Accept changes in printing settings and close dialog.
\else
Подтвердить изменения установок печати и закрыть диалог.
\fi
\end{popup}

\begin{popup}
\caption{Cancel}
\public{CFGPRINT\_CANCEL}
\ifenglish
Discard changes in printing settings and close dialog.
\else
Отказаться от изменений установок печати и закрыть диалог.
\fi
\end{popup}

\begin{popup}
\caption{Truncate}
\public{CFGPRINT\_TRUNCATE}
\ifenglish
When this option is selected,
if some lines are too long to fit in the printer page they are  truncated
\else
Когда установлена эта опция,
длинные строки обрезаются, чтобы поместиться на печатный лист.
\fi
\end{popup}

\begin{popup}
\caption{Continue on next line}
\public{CFGPRINT\_NEXTLINE}
\ifenglish
When this option is selected,
if some lines are too long to fit in the printer page they are continued on
next line so one source line can occupy several printer lines.
\else
Когда установлена эта опция,
тогда, если некоторые строки слишком длинные, чтобы поместиться на печатный лист, 
то они будут продолжены на следующей строке, так что одна строка исходного файла
может занять несколько строк печатного листа.
\fi
\end{popup}

\begin{popup}
\caption{Continue on next page}
\public{CFGPRINT\_NEXTPAGE}
\ifenglish
When this option is selected,
if some lines are too long to fit in the printer page new page with line trailers
is printed so that it's possible to combine several pages into one wide page.
Note that the number  of pages is determined by the longest line in a page,
so this method is potentially paper-consuming.

For this method to work properly, turn off duplex printing in your printer
configuration dialog.
\else
Когда установлена эта опция,
тогда, если некоторые строки слишком длинные, чтобы поместиться на печатный лист,
то печатается новый лист, содержащий обрывки длинных строк, так что становится 
возможным совместить несколько листов в один широкий лист. Заметьте, что 
количество листов определяется самой длинной строкой листа, а поэтому этот способ
потенциально чрезмерно бумагоемкий.

Чтобы этот способ действовал надлежащим образом, выключите двустороннюю печать
в конфигурационном диалоге вашего принтера. 
\fi
\end{popup}

\begin{popup}
\caption{Page setup}
\public{CFGPRINT\_PAGE}
\ifenglish
Invokes standard Page Setup dialog where you can select paper size, source and
orientation and set up page margins.
\else
Вызывает стандартный диалог Page Setup, где вы можете выбрать размер листа, устройство
подачи бумаги и ориентацию, и установить поля. 
\fi
\end{popup}

\begin{popup}
\caption{Choose font}
\public{CFGPRINT\_FONT}
\ifenglish
Invokes standard dialog to choose font to use for printing. Only fixed-pitch
fonts are available.
\else
Вызывает стандартный диалог выбора шрифта для печати. Доступны только
моноширинные шрифты.
\fi
\end{popup}

\begin{popup}
\caption{File name}
\public{CFGPRINT\_FILENAME}
\ifenglish
Turns on printing of file name in the page header.
\else
Включает печать имени файла в заголовке листа.
\fi
\end{popup}

\begin{popup}
\caption{File path}
\public{CFGPRINT\_FILEPATH}
\ifenglish
If file names in page headers are turned on, this option turns on printing full
file paths.
\else
Если включена печать имени файла в заголовке листа, эта опция включает печать
полного пути файла.
\fi
\end{popup}

\begin{popup}
\caption{Page numbers}
\public{CFGPRINT\_PAGENUM}
\ifenglish
Turns on printing of page numbers in page headers.
\else
Включает печать номера листа в заголовке листа.
\fi
\end{popup}

\begin{popup}
\caption{Line numbers}
\public{CFGPRINT\_LINENUM}
\ifenglish
Turns on printing of line numbers at the left of each line.
\else
Включает печать номера строки слева от каждой строки.
\fi
\end{popup}
