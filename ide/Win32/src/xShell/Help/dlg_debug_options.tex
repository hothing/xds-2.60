\ifenglish
\section{Run options dialog}
 \else
\section{диалог Run options}
\fi
\public{CFGRUN\_DIALOG}
\nominitoc

\ifenglish
Use this dialog to set target program execution options:
 \else
В этом диалоге устанавливаются опции запуска целевой программы:  
\fi
\begin{itemize}
\item 
\ifenglish 
Executable file name  \else   Имя исполняемого файла \fi
\item 
\ifenglish
Startup directory    \else    Директория запуска   \fi
\item 
\ifenglish
Command line arguments  \else Аргументы командной строки \fi
\item 
\ifenglish
Window size        \else      Размер окна  \fi
\item 
\ifenglish
Console creation mode  \else  Режим создания консоля \fi
\end{itemize}

\ifenglish
All this settings are kept separately for each project in project configuration
file. One set of settings is also kept for no-project configuration.
 \else
Все эти установки содержатся в отдельности для каждого проекта в файле настроек
проекта. Один набор установок содержится также для непроектных настроек.
\fi
\ifenglish
Following controlling elements are available:
 \else
Доступны следующие управляющие элементы:
\fi

{\bf EXE file} 
\ifenglish
-- name of executable file to be run or debugged.
 \else
-- имя исполняемого файла, который надлежит запустить или отлаживать
\fi
\ifenglish
\ref{variables} may be used in this field. Default value of this field is
 \else
В этом поле могут быть использованы \ref[переменные]{variables}. 
Значение по умолчанию для этого поля
\fi
\verb'$(projdir)\$(projname).exe'

\ifenglish
for project files,
 \else
для проектных файлов
\fi
\verb'$(filedir)\$(filename).exe'

\ifenglish
for no-project configurations.
 \else
для непроектных настроек.
\fi
\ifenglish
The \verb'$(exefile)' \ref[substitution variable]{variables}
is set equal to this field value. It may be used in command line of tools
that work with the executable file, for example, in that of the debugger.
 \else 
Этим значением инициализируется \ref[переменная замены]{variables} \verb'$(exefile)'.
Она может быть использована в командной строке утилит, которые работают с 
исполняемым файлом, как, например, в отладчике.
\fi
\ifenglish
{\bf Start Directory} -- the directory to be set as current when the program
is run. \ref{variables} may be used in this field. Default value of this field is
\verb'$(projdir)' for project files,
\verb'$(filedir)' for no-project configurations.
 \else
-- директория, которая станет текущей во время работы программы.
   В этом поле могут быть использованы \ref[переменные]{variables}. 
   Значение по умолчанию для этого поля
   \verb'$(projdir)' для проектных файлов,
   \verb'$(filedir)' для непроектных файлов.
\fi
\ifenglish
{\bf Arguments} -- optional command-line arguments to be passed to the program.
The \verb'$(arguments)' \ref[substitution variable]{variables} is set equal to this field
value.
 \else
--  аргументы командной строки, которые будут переданы программе  
    этому значению приравниваетя переменная \verb'$(arguments)'.
 \fi
\ifenglish
{\bf Run Mode} -- this group regulates how the main window should be opened.
For GUI applications you may set main window to be either {\bf Normal} or
{\bf Maximized}. For console application you may either configure the console
to close automatically as program finishes or remain opened. In latter case
output of new run session will be appended to old output.
 \else
-- этот набор настроек управляет тем, как будет открыто главное окно.
 Для GUI приложений можно сделать главное окно как нормальным, так и  
 максимизированным.
 Для консольных приложений можно также настроить консоль так, чтобы он 
 автоматически закрывался после окончания работы программы, или оставался  
 открытым. В последнем случае вывод последующих запусков будет дополнять 
 старый вывод.
\fi
%---------------------------------------------------------------------
\begin{popup}
\caption{Configuration file}
\public{CFGRUN\_TITLE}

\ifenglish
This label shows the configuration file in which run options are stored.
Separate configuration file is kept for each project.
 \else
Эта метка отбражает файл настроек, в котором хранятся опции запуска.
Для каждого проекта заводится отдельный файл настроек.
\fi
\end{popup}

\begin{popup}
\caption{EXE file}
\public{CFGRUN\_EXE}

\ifenglish
Type a name of an executable file to be run or debugged.
You may also press the {\bf Browse} pushbutton to choose it using
a standard file dialog.

Substitution variables may be used in this field.
 \else
Введите имя исполняемого файла, который надлежит запустить или отладить.

Вы также можете нажать кнопку {\bf Browse}, чтобы выбрать его с помощью
стандартного файлового диалога.

Здесь могут использоваться переменные замены.
\fi
\ifenglish
The \verb'$(exefile)' substitution variable is set equal to this field value.
\else
Этому значению сопоставляется переменная замены \verb'$(exefile)'.
\fi
\end{popup}

\begin{popup}
\caption{Browse}
\public{CFGRUN\_BROWSEEXE}

\ifenglish
Select the program to run using standard file dialog.
 \else
Выберите программу для запуска, пользуясь стандартным файловым диалогом.
\fi
\ifenglish
The \verb'$(exefile)' substitution variable is set equal to this field value.
 \else
Этому полю ставится в соответствие переменная замены \verb'$(exefile)'.
\fi
\end{popup}
\begin{popup}
\caption{Startup directory}
\public{CFGRUN\_STARTDIR}

\ifenglish
Type a full name of a directory which you want to be made current when
your program is invoked.

Substitution variables may be used in this field.
 \else
Введите полное имя директории, которую вы хотите сделать текущей на время
работы вашей программы.

В этом поле можно использовать переменные замены.
\fi
\ifenglish
The \verb'$(rundir)' substitution variable is set equal to this field value.
 \else
Этому полю ставится в соответствие переменная замены \verb'$(rundir)'. 
\fi
\end{popup}

\begin{popup}
\caption{Browse}
\public{CFGRUN\_STARTBROWSE}

\ifenglish
Select startup directory for your program using standard directory
selection dialog.
 \else
Выберите директорию запуска для вашей программы, используя стандартный 
диалог выбора директории.
\fi
\end{popup}

\begin{popup}
\caption{Argument}
\public{CFGRUN\_ARGS}

\ifenglish
Type a string which you would like to be passed to your program
as a command-line argument.
 \else
Введите строку, которую вы хотите передать вашей программе в качестве
командной.
\fi
\ifenglish
The \verb'$(arguments)' substitution variable is set equal to this field value.
 \else
Этому полю ставится в соответствие переменная замены \verb'$(arguments)'.
 \fi
\end{popup}


\begin{popup}
\caption{Normal}
\public{CFGRUN\_NORMAL}

\ifenglish
Select this radio button if you
want your program to be run in a normal size window.
 \else
Включите эту кнопку, если хотите, чтобы ваша программа выполнялась в окне 
обычного размера.
\fi
\ifenglish
This only applies to GUI applications that do not set their own window size.
 \else
Это применимо только к GUI приложениям, которые не устанавливают размер своего 
окна сами.
\fi
\end{popup}

\begin{popup}
\caption{Maximized}
\public{CFGRUN\_MAXIMIZE}

\ifenglish
Select this radio button if you
want your program to be run in a maximized window.

This only applies to GUI applications that do not set their own window size.
 \else
Включите эту кнопку, если хотите, чтобы ваша программа выполнялась в 
максимизированном окне.

Это применимо только к GUI приложениям, которые не устанавливают размер своего 
окна сами.
\fi
\end{popup}

\begin{popup}
\caption{Maximized}
\public{CFGRUN\_SAMECONSOLE}

\ifenglish
Normally, when a console application finishes, the console is closed.
Select this checkbox if you want to prevent the console
from closing.

Note that the console is not cleared before program run, so program output
immediately follows output from previous run sessions.
 \else
Обычно, когда завершается консольное приложение, консоль закрывается.
Включите эту опцию, если не хотите, чтобы консоль закрылся.

Заметьте, что когда консоль не очищен перед запуском, вывод порграммы
будет отображен сразу за выводом предыдущих запусков.
\fi
\end{popup}

\begin{popup}
\caption{OK}
\public{CFGRUN\_OK}

\ifenglish
Accept the changes you have made to the run options.
 \else
Подтверждение изменений, которые вы произвели в опциях запуска.
 \fi
\end{popup}

\begin{popup}
\caption{Cancel}
\public{CFGRUN\_CANCEL}

\ifenglish
Dismiss the dialog without saving any changes.
 \else
Закрытие диалога без сохранения изменений.
\fi
\end{popup}
