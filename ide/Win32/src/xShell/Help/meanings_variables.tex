\ifenglish
\section{Substitution variables}
\else
\section{Переменные замены}
\fi
\label{variables}
\public{xdsmain}
\public{xdsname}
\public{xdsdir}
\public{file}
\public{filedir}
\public{filename}
\public{filewholename}
\public{fileext}
\public{project}
\public{projdir}
\public{projname}
\public{projwholename}
\public{projext}
\public{projini}
\public{homedir}
\public{startdir}
\public{inifile}
\public{arguments}
\public{exefile}
\public{rundir}

\ifenglish
The Environment maintains a number of {\em substitution variables}
containing string values, for instance, full name of the
currently open project file, the directory in which it resides, the
currently used XDS compiler executable, etc. When configuring
tools or creating a project, you may use these variables in some of
the entry fields to have their values to be substituted:
\else
Оболочка поддерживает некоторое количество {\em переменных замены},
содержащих строковые значения, например, полное имя открытого на данный 
момент файла проекта, директорию в котором он размещается, используемый
на данный момент исполняемый файл XDS-компилятора, и т.д. При настройке
утилит или создании проекта вы можете использовать эти переменные в 
некоторых полях ввода для того, чтобы их значения заменялись: 
\fi
\verb'    $(variable-name)'

\ifenglish
The full list of all variables follows:
\else
Ниже приведен полный список всех переменных:
\fi
\begin{description}
\item[{\tt \$(xdsmain)}] 
 \ifenglish
 XDS compiler executable
 \else
 исполняемый файл XDS компилятора
 \fi
\item[{\tt \$(xdsname)}] 
 \ifenglish
 its name without directory and extension
 \else
 его имя без директории и расширения
 \fi
\item[{\tt \$(xdsdir)}] 
 \ifenglish
 its directory
 \else
 его директория
 \fi
\item[{\tt \$(file)}] 
 \ifenglish
 full file name in current edit window
 \else
 полное имя файла в текущем окне редактирования
 \fi
\item[{\tt \$(filedir)}] 
 \ifenglish
 current file directory
 \else
 директория текущего файла
 \fi
\item[{\tt \$(filename)}] 
 \ifenglish
 current file name without path and extension
 \else
 имя текущего файла без директории и расширения
 \fi
\item[{\tt \$(filewholename)}] 
 \ifenglish
 current file name with extension but without path
 \else
 имя текущего файла с расширением, но без пути 
 \fi
\item[{\tt \$(fileext)}] 
 \ifenglish
 current file extension
 \else
 расширение теущего файла
 \fi
\item[{\tt \$(project)}] 
 \ifenglish
 full project file name
 \else
 полное имя файла проекта
 \fi
\item[{\tt \$(projdir)}] 
 \ifenglish
 project directory
 \else
 директория проекта
 \fi
\item[{\tt \$(projname)}] 
 \ifenglish
 project name without path and extension
 \else
 имя проекта без пути и расширения
 \fi
\item[{\tt \$(projwholename)}] 
 \ifenglish
 project name with extension but without path
 \else
 имя проекта с расширением, но без пути
 \fi
\item[{\tt \$(projext)}] 
 \ifenglish
 project file extension
 \else
 расширение файла проекта
 \fi
\item[{\tt \$(projini)}] 
 \ifenglish
 full name of ini file corresponding to project file
 \else
 полное имя ini-файла, соответствующего файлу проекта
 \fi
\item[{\tt \$(homedir)}] 
 \ifenglish
 directory where XDS Environment executable resides
 \else
 директория, где размещается рабочая среда XDS
 \fi
\item[{\tt \$(startdir)}] 
 \ifenglish
 the current directory on the Environment startup
 \else
 текущая директория запуска рабочей среды
 \fi
\item[{\tt \$(inifile)}] 
 \ifenglish
 full name of the Environment ini file
 \else
 полное имя ini-файла рабочей среды
 \fi
\item[{\tt \$(arguments)}] 
 \ifenglish
 command line arguments for user program
  \else
 аргументы командной строки пользовательской программы 
 \fi
\item[{\tt \$(exefile)}] 
 \ifenglish
 full name of the user program executable file
 \else
 полное имя исполняемого файла пользовательской программы
 \fi
\item[{\tt \$(rundir)}] 
 \ifenglish
 directory to be set current when running user program
 \else
 директория, которую надлежит сделать текущей на время работы пользовательской
 программы  
 \fi
\end{description}
