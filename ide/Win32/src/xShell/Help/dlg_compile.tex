\ifenglish
\section{Tool pop-up window}
\else
\section{Всплывающее окно утилиты}
\fi
\public{COMPILE\_DIALOG}

\ifenglish
This window is activated when you invoke an external tool which is
configured to be run in the pop-up window. A tool may display here
its warning and error messages, as well as progress indicators.
Tool output is also shown in this window, so after termination of a tool
you may review it before closing this window.
\else
Это окно появляется, когда вы вызываете внешнюю утилиту, которая работает во 
всплывающем окне. Утилита может отобразить здесь свои сообщения об ошибках и 
предупреждениях, равно как и индикаторы хода работы.
Вывод утилиты также показывается в этом окне, так что после завершения работы 
утилиты вы можете просмотреть его перед закрытием этого окна. 
\fi

\ifenglish
To interrupt a running tool, select the {\bf Stop} pushbutton.
\else
Чтобы прервать работу утилиты, нажмите кнопку {\bf Stop}.
\fi
\ifenglish
After a tool has finished his job, you may use the {\bf Copy program output}
and {\bf Save program output} choices from this dialog system menu to,
respectively, copy the tool output to the Clipboard or save it in a file.
\else
После того как утилита завершила свою работу, вы можете использовать
{\bf Copy program output} и {\bf Save program output} - пункты системного меню 
этого диалога, чтобы соответственно скопировать вывод в Clipboard или сохранить 
в файле.
\fi
\begin{popup}
\ifenglish
\caption{Messages}
\else
\caption{Сообщения}
\fi
\public{COMPILE\_OK}

\ifenglish
Press this button to see error and warning messages generated by tool.
This button becomes default button is there are errors, otherwise
{\bf Close} button becomes default.
\else
Нажмите эту кнопку, чтобы увидеть сгенерированные утилитой сообщения об ошибках 
и предупреждениях. Если обнаружены ошибки, эта кнопка срабатывает по умолчанию,
иначе по умолчанию задействуется кнопка {\bf Close}.
\fi

\end{popup}


\begin{popup}
\caption{Stop}
\public{COMPILE\_STOP}

\ifenglish
Press this button to terminate running tool immediately.
\else
Нажмите эту кнопку, если хотите немедленно прервать работу утилиты.
\fi
\end{popup}


\begin{popup}
\caption{Close}
\public{COMPILE\_CANCEL}

\ifenglish
Dismiss the pop-up window without activating Messages window.
This button becomes default if there are no error messages (but warnings
are possible). Otherwise {\bf Messages} button becomes the default one.
\else
Закрывает всплывающее окно, не активируя окно сообщений.
Эта кнопка задействуется по умолчанию, если нет сообщений об ошибках.
(но, возможно, имеются сообщения о предупреждениях).
Иначе по умолчанию задействуется кнопка {\bf Messages}.
\fi
\end{popup}

\begin{popup}
\caption{Errors}
\public{COMPILE\_ERRORS}
\ifenglish
This label displays number of errors detected by the tool.
\else
Эта метка отображает количество ошибок, обнаруженных утилитой.
\fi
\end{popup}

\begin{popup}
\caption{Warnings}
\public{COMPILE\_WARNINGS}
\ifenglish
This label displays number of warnings detected by the tool.
\else
Эта метка отображает количество предупреждений, обнаруженных утилитой.
\fi
\end{popup}



\begin{popup}
\ifenglish
\caption{Caption}
\else
\caption{Заголовок}
\fi
\public{COMPILE\_CAPTION}

\ifenglish
While the tool is running its caption is displayed here. The caption is configured
in tools configuration dialog.
\else
Во время работы утилиты здесь отображается ее заголовок. Заголовок можно 
оформить в конфигурационном диалоге утилиты.
\fi
\ifenglish
After the tool finishes this label displays the result: {\bf "Success"},
{\bf "Success with warnings"}, {\bf "Errors detected"} and other conditions
may be displayed here.
\else
После окончания работы утилиты здесь отображается результат : {\bf "Success"},
{\bf "Success with warnings"}, {\bf "Errors detected"}, и другие.
\fi
\end{popup}

\begin{popup}
\ifenglish
\caption{Progress comment}
\else
\caption{Комментарий хода работы}
\fi
\public{COMPILE\_PROGRESS\_COMMENT}

\ifenglish
This label displays whatever progress information the tool wishes to display.
\else
Эта метка отображает всю информацию о ходе работы, выдаваемую утилитой.
\fi
\end{popup}


\begin{popup}
\ifenglish
\caption{Compile comment}
\else
\caption{Комментарий компиляции}
\fi
\public{COMPILE\_COMMENT}
\ifenglish
This label displays information on what the tool is doing now.
\else
Эта метка показывает информацию о том, чем утилита занимается в данный момент.
\fi
\end{popup}

\begin{popup}
\ifenglish
\caption{Progress indicator}
\else
\caption{Индикатор хода работы}
\fi
\public{COMPILE\_PROGRESS\_INDICATOR}
\ifenglish
This progress indicator displays how close the tool is to the finish of the work.
\else
Этот идикатор показывает, насколько близка утилита к окончанию работы.
\fi
\end{popup}

\begin{popup}
\ifenglish
\caption{Save program output}
\else
\caption{Сохранение вывода программы}
\fi
\public{COMPILE\_SAVE\_OUTPUT}

\ifenglish
Save the finished tool output
in a file. A standard file dialog will be displayed, allowing you to
specify name and location of this file.
\else
Сохраняет вывод закончившей работу утилиты в файл. Будет выведен стандартный
файловый диалог, что позволит вам указать имя и место этого файла.
\fi

\end{popup}

\begin{popup}
\ifenglish
\caption{Program output}
\else
\caption{Вывод программы}
\fi
\public{COMPILE\_LISTBOX}

\ifenglish
You can see the output of a tool here.
\else
Здесь можно видеть вывод утилиты.
\fi
\end{popup}

\begin{popup}
\ifenglish
\caption{Copy program output}
\else
\caption{Копирование вывода программы}
\fi
\public{COMPILE\_COPY\_OUTPUT}

\ifenglish
Copy the finished tool output to the Clipboard.
\else
Копирует вывод закончившей работу утилиты в Clipboard.
\fi
\end{popup}

\begin{popup}
\ifenglish
\caption{Change font}
\else
\caption{Смена шрифта}
\fi
\public{COMPILE\_CHOOSE\_FONT}

\ifenglish
Change the font of tool output window.
\else
Меняет шрифт окна вывода утилиты.
\fi
\end{popup}

