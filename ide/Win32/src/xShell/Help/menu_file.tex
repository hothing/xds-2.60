\section{File Menu}
\public{M\_FILE}
\index{File Menu}
\nominitoc

\ifenglish
The {\bf File} menu contains commands that you use to create,
open, and save source files. It also contains a list of files
that were most recently opened or saved using the {\bf Save As} command.
The following commands appear in the {\bf File} menu:
\else
Меню {\bf File} содержит команды, которые вы используете, чтобы создавать,
открывать и сохранять исходные файлы. Оно также содержит список наиболее часто 
открывавшихся или сохраненных командой {\bf Save As} файлов.
В меню {\bf File} содержатся следующие команды:
\fi

\begin{description}
\item \ref[New]{IDM\_NEW} - 
  \ifenglish
    Opens new empty window
  \else 
    Открывает новое пустое окно
  \fi
\item \ref[Open]{IDM\_OPEN} -
  \ifenglish
    Opens a new window and load a file into it
  \else
    Открывает новое окно и загружает в него файл
  \fi
\item \ref[Save]{IDM\_SAVE} -
  \ifenglish
    Saves any changes made to the current file
  \else
    Сохраняет любые произведенные в текущем файле изменения
  \fi
\item \ref[Save As]{IDM\_SAVEAS} -
  \ifenglish
    Saves the current file using a new name
  \else
    Сохраняет текущий файл под новым именем
  \fi
\item \ref[Save All]{IDM\_SAVEALL} - 
  \ifenglish
    Saves all changed files
  \else
    Сохраняет все измененные файлы
  \fi
\item \ref[Share]{IDM\_SHARE} -
  \ifenglish
    Opens another window with same file
  \else
    Открывает новое окно с тем же самым файлом
  \fi
\item \ref[Print]{IDM\_PRINT} -
  \ifenglish
    Prints the current file or selection
  \else
    Распечатывает содержимое текущего файла или отмеченный текст
  \fi
\item \ref[Close]{IDM\_CLOSE} -
  \ifenglish
    Closes the current window
  \else
    Закрывает текущее окно
  \fi
\item \ref[Exit]{IDM\_EXIT} -
  \ifenglish
    Exits the IDE
  \else
    Закрывает IDE 
  \fi
\item \ref[More Files]{IDM\_MOREFILES} -
  \ifenglish
    Show long history of opened files
  \else
    Отображает полный архив сведений по открывавшимся файлам
  \fi
\end{description}

\subsection{File: New}
\public{IDM\_NEW}

\ifenglish
Use the {\bf New} command to create a new empty window.

The words "New file" will appear in its title bar,
You'll have to specify a name for the new file when you save it.
\else
Используйте команду {\bf New}, чтобы создать новое пустое окно

В заголовке нового окна появятся слова "New file" .
При сохранении нового файла вы должны будете указать его имя.
\fi

\subsection{File: Open}
\public{IDM\_OPEN}

\ifenglish
Use the {\bf Open} command to load an existing file into
a new window.

A dialog will appear, providing different methods of
loading files.

Keyboard shortcut: {\bf F3}
\else
Используйте команду {\bf Open}, чтобы загрузить существующий файл в новое 
окно.

Появится диалог, обеспечивающий различные способы загрузки файлов.

Комбинация клавиш: {\bf F3} 
\fi

\subsection{File; Save}
\public{IDM\_SAVE}

\ifenglish
Use the {\bf Save} command to store the current file to disk.
After the file is saved, the text remains in the window
so that you can continue editing it. However, undo buffer is cleared
so you can't undo the changes made before file was saved.

Keyboard shortcut: {\bf F2}

{\bf Note:} If you are editing a new file, select the
{\bf Save} or {\bf Save as} choices to display the
{\bf Save as} dialog so that you can name the file you are editing.
A file must have a name to be saved.
\else
Используйте команду {\bf Save}, чтобы сохранить содержимое текущего файла на 
диск. После того, как файл сохранен, текст остается в окне, так что вы можете 
продолжить его изменять. Тем не менее, буфер отмены очищается, так что вы не 
можете отменить изменения, произведенные до того, как файл был сохранен.

{\bf Замечание: } если вы редактируете новый файл, выберите {\bf Save} или 
{\bf Save as}, чтобы отобразить {\bf Save as}-диалог, позволяющий вам дать имя
редактируемому файлу. Для того, чтобы быть сохраненным, файл дожен иметь имя. 
\fi

\subsection{File: Save As}
\public{IDM\_SAVEAS}

\ifenglish
Use the {\bf Save As} command to name and save a new file
or to save a file under a different name, or to a different location.

A dialog box will appear, prompting you for a new filename. You can
also select different drive/directory.

The new name will appear in the title bar of current window. If there are
several windows with the file being saved, all of them change title bar.
\else
Используйте команду {\bf Save As}, чтобы назвать и сохранить новый файл,
или сохранить файл под другим именем или в другом месте.

Появится диалог, запрашивающий имя файла. Вы также можете указать носитель
и директорию.

Новое имя будет отображено в заголовке текущего окна. Если сохраняемый файл 
содержат несколько окон, все они изменят заголовок.
\fi

\subsection{File: Save All}
\public{IDM\_SAVEALL}

\ifenglish
Use the {\bf Save All} command to save all changed files.
\else
Используйте команду {\bf Save All}, чтобы сохранить все измененные файлы.
\fi

\subsection{File: Share}
\public{IDM\_SHARE}

\ifenglish
Use the {\bf Share} command to open another window with current file.
Both windows are kept consistent; all changes made in one window
are immediately visible in other window.

You may open as many copies of a file as you wish.

{\bf Note:} You will be prompted if you wish to save file to disk only when
the last window with a file is closed.
\else
Используйте команду {\bf Share}, чтобы открыть другое окно, содержащее текущий 
файл. Оба окна согласуются, все изменения, произведенные в одном окне, сразу же
отображаются в другом.  

Вы можете открывать сколько угодно копий одного и того же файла.

{\bf Замечание:} Вас запросят, хотите ли вы сохранить файл на диск, только тогда,
когда будет закрыто последнее окно, содержащее файл.  
\fi

\subsection{File: Print}
\public{IDM\_PRINT}

\ifenglish
Use the {\bf Print} command to print the current file or
selection.
\else
Используйте команду {\bf Print}, чтобы распечатать содержимое текущего файла
или отмеченный текст.
\fi

\subsection{File: Close}
\public{IDM\_CLOSE}

\ifenglish
Use the {\bf Close} command to close the current window.
You will be prompted to save changes, if any.
\else
Используйте команду {\bf Close}, чтобы закрыть текущее окно.
Вас запросят, хотите ли вы сохранить изменения, если они были произведены.
\fi

\subsection{File: Exit}
\public{IDM\_EXIT}

\ifenglish
Use the {\bf Exit} command to close all edit windows
and terminate the IDE. You will be prompted to save changes,
if any.
\else
Используйте команду {\bf Exit}, чтобы закрыть все окна ввода и завершить IDE. 
Вас запросят, хотите ли вы сохранить изменения, если они были произведены.
\fi

\subsection{File: More Files}
\public{IDM\_MOREFILES}

\ifenglish
Only 5 last opened files appear at the bottom of \ref[File]{M\_FILE} menu.
The IDE actually remembers last 100 opened files. To display \ref[full list]{MOREFILES\_DIALOG}
of them use the {\bf More Files} command.
\else
Только 5 последних открывавшихся файлов появятся в нижней части меню 
\ref[File]{M\_FILE}. IDE на самом деле запоминает 100 последних открывавшихся
файлов.
Чтобы отобразить их полный список, используйте команду 
\ref[full list]{MOREFILES\_DIALOG}.
\fi