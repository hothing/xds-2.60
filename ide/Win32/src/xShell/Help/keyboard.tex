\public{HELP\_KEYBOARD}
\index{Keyboard}
\nominitoc

\ifenglish
The editor keyboard commands can be split into following groups:
\else
Команды редактора, воспринимаемые с клавиатуры, можно разбить на следующие группы:
\fi
\begin{description}
\item \ifenglish
      \ref[Cursor movement]{CursorMovement}
      \else
      \ref[Передвижение курсора]{CursorMovement}
      \fi
\item \ifenglish
      \ref[Block selection]{BlockSelection}
      \else
      \ref[Отметка блока]{BlockSelection}
      \fi
\item \ifenglish
      \ref[Typing, editing and deleting text]{Typing}
      \else
      \ref[Ввод, редактрирование и удаление текста]{Typing}
      \fi
\item \ifenglish
      \ref[Clipboard operations]{ClipboardOperations}
      \else
      \ref[Операции с буфером обмена]{ClipboardOperations}
      \fi
\item \ifenglish
      \ref[Search and replace]{SearchReplace}
      \else
      \ref[Поиск и замена]{SearchReplace}
      \fi
\item \ifenglish
      \ref[Miscellaneous]{Miscellaneous}
      \else
      \ref[Разное]{Miscellaneous}
      \fi
\item \ifenglish
      \ref[Macros]{Macros} (new in 2.6)
      \else
      \ref[Макрокоманды]{Macros} (новое в версии 2.6)
      \fi
\end{description}

\ifenglish
\section{Cursor movement}
\else
\section{Передвижение курсора}
\fi
\public{CursorMovement}

\ifenglish
Following commands move cursor:
\else
Курсор передвигается с помощью следующих команд:
\fi

{\bf Left} 
\ifenglish 
- one character left
\else
- на символ влево
\fi

{\bf Right} 
\ifenglish
- one character right
\else
- на символ вправо
\fi

{\bf Up} 
\ifenglish
- one line up
\else
- на строку вверх
\fi

{\bf Down} 
\ifenglish
- one line down
\else
- на строку вниз
\fi

{\bf Page Up} 
\ifenglish
- one page up
\else
- на страницу вверх
\fi

{\bf Page Down} 
\ifenglish
- one page down
\else
- на страницу вниз
\fi

{\bf Ctrl+Right} 
\ifenglish
- to the beginning of the next word
\else
- к началу следующего слова
\fi

{\bf Ctrl+Left} 
\ifenglish
- to the beginning of the previous word
\else
- к началу предыдущего слова
\fi

{\bf Home} 
\ifenglish
- to the beginning of current line
\else
- к началу текущей строки
\fi

{\bf End} 
\ifenglish
- to the end of current line
\else
- к концу текущей строки
\fi

{\bf Ctrl+N}, {\bf Shift+Enter} 
\ifenglish
- to the beginning of the next line
\else
- к началу следующей строки
\fi

{\bf Shift+Tab} 
\ifenglish
- to previous tab stop
\else
- к предыдущей позиции табуляции
\fi

{\bf Ctrl+Page Up} 
\ifenglish
- to the beginning of current page
\else
- к началу текущей страницы
\fi

{\bf Ctrl+Page Down} 
\ifenglish
- to the end of current page
\else
- к концу текущей страницы
\fi

{\bf Ctrl+Home} 
\ifenglish
- to the beginning of file
\else
- к началу файла
\fi

{\bf Ctrl+End} 
\ifenglish
- to the end of file
\else
- к концу файла
\fi

\ifenglish
\section{Block selection}
\else
\section{Отметка блока}
\fi
\public{BlockSelection}

\ifenglish
A block is determined by two points. All listed commands move one of the points
to new position, thus expanding or shrinking the block depending on movement
direction and on the choice of point.

Following commands move block margin:
\else
Блок определяется двумя точками. Все перечисленные команды передвигают одну из
точек к новому положению, тем самым расширяя или сужая блок, в зависимости от
направления передвижения и выбора точки.

Следующие команды управляют передвижением ограничителя блока:
\fi

{\bf Shift+Left} 
\ifenglish
- one character left
\else
- на символ влево
\fi

{\bf Shift+Right} 
\ifenglish
- one character right
\else
- на символ вправо
\fi

{\bf Shift+Up} 
\ifenglish
- one line up
\else
- на строку вверх
\fi

{\bf Shift+Down} 
\ifenglish
- one line down
\else
- на строку вниз
\fi

{\bf Shift+Page Up} 
\ifenglish
- one page up
\else
- на страницу вверх
\fi

{\bf Shift+Page Down} 
\ifenglish
- one page down
\else
- на страницу вниз
\fi

{\bf Ctrl+Shift+Right} 
\ifenglish
- to the beginning of the next word
\else
- к началу следующего слова
\fi

{\bf Ctrl+Shift+Left} 
\ifenglish
- to the beginning of the previous word
\else
- к началу предыдущего слова
\fi

{\bf Shift+Home} 
\ifenglish
- to the beginning of current line
\else
- к началу текущей строки
\fi

{\bf Shift+End} 
\ifenglish
- to the end of current line
\else
- к концу текущей строки
\fi

{\bf Ctrl+Shift+Page Up} 
\ifenglish
- to the beginning of current page
\else
- к началу текущей страницы
\fi

{\bf Ctrl+Shift+Page Down} 
\ifenglish
- to the end of current page
\else
- к концу текущей страницы
\fi

{\bf Ctrl+Shift+Home} 
\ifenglish
- to the beginning of file
\else
- к началу файла
\fi

{\bf Ctrl+Shift+End} 
\ifenglish
- to the end of file
\else
- к концу файла
\fi

\ifenglish
\section{Typing, editing and deleting text}
\else
\section{Набор, редактирование и удаление текста}
\fi
\public{Typing}
\nominitoc

{\bf Insert} 
\ifenglish
- toggle insert and overwrite mode
\else
- переключить режимы вставки и наложения
\fi

{\bf Delete} 
\ifenglish
- delete character to the right from cursor; see \ref[Note]{DeleteNote}
\else
- удалить символ справа от курсора; смотрите также \ref[замечание]{DeleteNote}
\fi

{\bf BackSpace} 
\ifenglish
- delete character to the left from cursor; see \ref[Note]{BSNote}
\else
- удалить символ слева от курсора; смотрите также \ref[замечание]{BSNote}
\fi

{\bf Tab} 
\ifenglish
- insert Tab character or expand tab with spaces; see \ref[Note]{Tabulation}
\else
- вставить символ табуляции или расширить табуляцию пробелами; смотрите также \ref[замечание]{Tabulation}
\fi

{\bf Ctrl+Delete} 
\ifenglish
- \ref[delete end of current word]{CtrlDelNote}
\else
- \ref[удалить конец текущего слова]{CtrlDelNote}
\fi

{\bf Ctrl+BackSpace} 
\ifenglish
- \ref[delete beginning of current word]{CtrlBSNote}
\else
- \ref[удалить начало текущего слова]{CtrlBSNote}
\fi

{\bf Ctrl+I} 
\ifenglish
- insert a line before the current line
\else
- вставить строку перед текущей строкой
\fi

{\bf Ctrl+S} 
\ifenglish
- split a line in two at the cursor position
\else
- в месте, где расположен курсор, разбить строку на две  
\fi

{\bf Ctrl+J} 
\ifenglish
- join current line with the next line
\else
- соединить текущую строку со следующей
\fi

{\bf Enter} 
\ifenglish
- split a line or go to the next line; see \ref[Note]{CRNote}
\else
- разбить строку или перейти к следующей; смотрите также \ref[замечание]{CRNote}
\fi

{\bf Ctrl+Y} 
\ifenglish
- delete current line
\else
- удалить текущую строку 
\fi

\ifenglish
\subsection{{\bf Delete} key}
\else
\subsection{клавиша {\bf Delete}}
\fi
\public{DeleteNote}

\ifenglish
If the {\bf Delete} key is pressed when the cursor is at the end of line,
the editor usually joins the line with the next one, unless the cursor is at the last line.
If the cursor is to the right from the end of line, the next line is appended
at cursor position retaining all the spaces.

This behavior can be disabled using \ref[Configure: Editor]{IDM\_CONFIGUREEDITOR}
menu. In this case the editor would ignore {\bf Delete} command when cursor is
at the end of line or to the right of it.
\else
Если клавиша {\bf Delete} нажата, когда курсор находится на конце строки,
редактор обычно присоединяет к текущей строке следующую, за исключением случая 
когда текущая строка - последняя. Если курсор расположен справа от конца строки,
следующая строка добавляется к текущей с сохранением всех пробелов.

Это может быть исправлено с помощью меню \ref[Configure: Editor]{IDM\_CONFIGUREEDITOR}.
В этом случае редактор проигнорирует команду {\bf Delete}, когда курсор 
находится в конце строки или справа от него.
\fi

\ifenglish
\subsection{{\bf BackSpace} key}
\else
\subsection{клавиша {\bf BackSpace}}
\fi
\public{BSNote}

\ifenglish
If the {\bf BackSpace} key is pressed when the cursor is at the beginning of line,
the editor usually joins the line with the previous one, unless the cursor is at
the first line.

This behavior can be disabled using \ref[Configure: Editor]{IDM\_CONFIGUREEDITOR}
menu. In this case the editor would ignore {\bf BackSpace} command when cursor is
at the beginning of line.
\else
Если клавиша {\bf BackSpace} нажата, когда курсор находится в начале строки,
редактор обычно присоединяет текущую строку к предыдущей, за исключением случая,
когда текущая строка - первая.
\fi

\ifenglish
\subsection{Delete end of current word}
\else
\subsection{Удалить конец текущего слова}
\fi
\public{CtrlDelNote}

\ifenglish
If the cursor is not in a white space, the {\bf Ctrl+Delete} commands
deletes everything from the cursor position
to the end of current word. Otherwise, it deletes everything
from the cursor position to the beginning of next word.
\else
Если курсор находится не на пробеле, команда {\bf Ctrl+Delete}
удаляет все, начиная от текущего положения курсора и кончая концом 
текущего слова. Иначе команда удаляет все, начиная от текущего положения курсора
и кончая началом следующего слова.
\fi

\ifenglish
\subsection{Delete beginning of current word}
\else
\subsection{Удалить начало текущего слова}
\fi
\public{CtrlBSNote}

\ifenglish
If the character to the left from the cursor is not in a white space,
the {\bf Ctrl+BackSpace} command
deletes everything from the beginning of the current word to that character.
Otherwise, it deletes everything from the end of previous word to mentioned
character.
\else
Если символ справа от курсора -- не пробел, команда {\bf Ctrl+BackSpace} 
удаляет все от начала текущего слова до этого символа.
Иначе она удаляет все от конца предыдущего слова до упомянутого символа.
\fi

\ifenglish
\subsection{{\bf Enter} key}
\else
\subsection{клавиша {\bf Enter}}
\fi
\public{CRNote}

\ifenglish
In overwrite mode the {\bf Enter} key works exactly like {\bf Ctrl+N} key.
The cursor moves to the beginning of the next line. A blank line is appended
if the cursor was at the last line.
 
In insert mode the {\bf Enter} key works like {\bf Ctrl+S} key, that is, it
splits a line in two at cursor position. It can be disabled in 
\ref[Configure: Editor]{IDM\_CONFIGUREEDITOR} menu, after which this key
works like in overwrite mode.
\else
В режиме наложения клавиша {\bf Enter} действует так же, как и {\bf Ctrl+N}.
Курсор передвигается к началу следующей строки, а если курсор находился на
последней строке, то добавляется пустая.

В режиме вставки клавиша {\bf Enter} действует так же, как и  {\bf Ctrl+S},
то есть разбивает строку на две в месте, где находился курсор. Это может быть
отключено в меню \ref[Configure: Editor]{IDM\_CONFIGUREEDITOR}, после чего
эта клавиша действует так же, как и в режиме наложения.
\fi

\ifenglish
\section{Clipboard operations}
\else
\section{Операции с буфером обмена}
\fi
\public{ClipboardOperations}

{\bf Ctrl+Insert}, {\bf Ctrl+C} 
\ifenglish
- \ref[Copy]{IDM\_COPY}
\else
- \ref[Копировать]{IDM\_COPY}
\fi

{\bf Shift+Delete}, {\bf Ctrl+X} 
\ifenglish
- \ref[Cut]{IDM\_CUT}
\else
- \ref[Вырезать]{IDM\_CUT}
\fi

{\bf Ctrl+Insert},  {\bf Ctrl+V} 
 \ifenglish
- \ref[Paste]{IDM\_PASTE}
\else
- \ref[Вклеить]{IDM\_PASTE}
\fi

{\bf Ctrl+Shift+Delete}, {\bf Ctrl+E} 
\ifenglish
- \ref[Exchange]{IDM\_EXCHANGE}
\else
- \ref[Обменять]{IDM\_EXCHANGE}
\fi

{\bf Ctrl+Shift+Insert}, {\bf Ctrl+A} 
\ifenglish
- \ref[Add]{IDM\_ADD}
\else
- \ref[Добавить]{IDM\_ADD}
\fi

\ifenglish
\section{Search and Replace}
\else
\section{Поиск и замена}
\fi
\public{SearchReplace}

{\bf Ctrl+F} 
\ifenglish
	     - \ref[Find]{IDM\_FIND} text
             \else
             - \ref[Найти]{IDM\_FIND} текст
             \fi

{\bf Ctrl+H} \ifenglish
             - Find text and \ref[Replace]{IDM\_REPLACE} it with another text
             \else
             - Найти текст и \ref[Заменить]{IDM\_REPLACE} его другим текстом
\fi

{\bf Ctrl+L} \ifenglish
             - \ref[Find next]{IDM\_FINDNEXT}
             \else
             - \ref[Найти следующее]{IDM\_FINDNEXT}
 \fi

{\bf Ctrl+Shift+L} \ifenglish
                   - \ref[Find previous]{IDM\_FINDPREV}
                   \else 
                   - \ref[Найти предыдущее]{IDM\_FINDPREV}
\fi

{\bf Ctrl+R} \ifenglish
             - \ref[Replace next]{IDM\_REPLACENEXT}
             \else
             - \ref[Заменить следущее]{IDM\_REPLACENEXT}
\fi

{\bf Ctrl+Shift+F} \ifenglish
                  - \ref[Find in files]{IDM\_FINDFILES}
                  \else
                  - \ref[Найти в файлах]{IDM\_FINDFILES}
\fi
\ifenglish
\section{Miscellaneous Editing Commands}
\else
\section{Разные команды редактирования}
 \fi
\public{Miscellaneous}

{\bf Alt+BackSpace}, {\bf Ctrl+Z} \ifenglish
     				  - \ref[Undo]{IDM\_UNDO} last action
     				  \else
                                  - \ref[Отменить]{IDM\_UNDO} последнее изменение 
\fi

{\bf Ctrl+Shift+1}...{\bf Ctrl+Shift+9} 
\ifenglish
- set a bookmark with given number at the current cursor position
\else
- установить закладку с заданным номером в месте, где находится курсор
\fi

{\bf Ctrl+1}...{\bf Ctrl+9} 
\ifenglish
- move cursor to the bookmark with given number
\else
- переместить курсор к закладке с заданным номером
\fi

{\bf Alt+Right} 
\ifenglish
- Indent current line or block right by one character
\else
- сделать отступ в текущей строке или блоке на один символ вправо
\fi

{\bf Alt+Left}  
\ifenglish
- Indent current line or block left by one character
\else
- сделать отступ в текущей строке или блоке на один символ влево
\fi

{\bf Alt+Shift+Right} 
\ifenglish
- Indent current line or block right by tab size
\else
- сделать отступ в текущей строке или блоке вправо на расстояние 
 табуляции
\fi

{\bf Alt+Shift+Left} 
\ifenglish
- Indent current line or block left by tab size
\else
- сделать отступ в текущей строке или блоке влево на расстояние 
 табуляции       
\fi

{\bf Ctrl+U} 
\ifenglish
- Turn current word or block to uppercase
\else
- Повысить регистр текущего слова или блока
\fi

{\bf Ctrl+Shift+U} 
\ifenglish
- Change case of the current word or block to opposite
\else
- Изменить регистр текущего слова или блока на противоположный
\fi

{\bf F1} 
\ifenglish
- Display \ref[help]{IDM\_HELPKEYWORD} on current word
\else
- Отобразить \ref[подсказку]{IDM\_HELPKEYWORD} по текущему слову
\fi

{\bf Ctrl+G} 
\ifenglish
- \ref[Go to line]{IDM\_GOTOLINE} with given number
\else
- \ref[Перейти на строку]{IDM\_GOTOLINE} с заданным номером
\fi

\ifenglish
\section{Macros}
\else
\section{Макрокоманды}
 \fi
\public{Macros}

\ifenglish
{\bf New in 2.6:}
\else
{\bf Новое в версии 2.6:}
\fi

\ifenglish
You may define your own macro commands and assign them to keyboard events.
To do that, you need to create a {\it macro defiition file} in the \verb'BIN\'
subdirectory of your XDS installation. 
You may also create language-specific sets of macros.
\else
Вы можете определить свои собственные макрокоманды и назначить им клавиатурные события.
Для этого нужно создать {\it файл с определениями макрокоманд} в папке \verb'XDS\BIN'.
Вы также можете создать наборы макрокоманд, специфичные для данного языка 
программирования.
\fi

\ifenglish
\subsection{Macro definitions}
\else
\subsection{Определения макрокоманд}
\fi

\ifenglish
A macro is defined as

\verb'    '{\it Event}\verb'='{\it Actions}

{\it Event} is a keyboard event that shall invoke the macro.
It consists of a {\it key code} and optional \verb'ALT', \verb'CONTROL', and \verb'SHIFT' 
modifiers, listed in any order and separated by commas.
A key code may be a literal printable character in quotes (single or double),
or one of the Win32 API virtual key constant names without the \verb'VK_' prefix,
such as \verb'F1', \verb'LEFT', etc.
(You may look those constants up in files 
\verb'Windows.def' or \verb'WinUser.def' located in the \verb'DEF\Win32\' 
subdirectory of your XDS installation.)

Examples:
\else
Каждый макрос описывается как 

\verb'    '{\it Событие}\verb'='{\it Действия}

{\it Событие} --- описание клавиатурного события, по которому запускается 
данная макрокоманда. Оно состоит из кода клавиши и необязательных модификаторов
\verb'ALT', \verb'CONTROL', и \verb'SHIFT', перечисленных в произвольном порядке 
через запятую. 
В качестве кода клавиши можно указать печатный символ в кавычках, или 
константу Win32 API, обозначающую виртуальную клавишу,
без префикса ``\verb'VK_''': \verb'F1', \verb'LEFT' и т.д. 
Полный набор этих констант можно найти в файле \verb'Windows.def' или
\verb'WinUser.def' в подпапке \verb'DEF\Win32\'.

Примеры:
\fi

\begin{verbatim}
  CONTROL,"[",SHIFT= ...
  CONTROL,PRIOR= ...
  CONTROL,HOME= ...
\end{verbatim}

\ifenglish
{\it Actions} is a comma-separated list of actions triggered by {\it Event},
Each action is either a literal string of characters that must be "typed"
into the current source file, or a \ref[editor command]{EditorCommands}.
Long lines may be split using backslash ``\verb'\'''.
\else
{\it Действия} --- последовательность действий, которые оболочка должна выполнить
при получении {\it События}, перечисленная через запятую. 
Каждое действие --- это или строчка в кавычках, 
которая просто ``впечатывается'' в текущий файл, 
или \ref[команда редактора]{EditorCommands}. 
При необходимости можно разбить {\it Действия} на несколько строк, 
при этом все строки, кроме последней, должны заканчиваться символом ``\verb'\'''.

Примечание: кавычки везде имеются в виду Модуловские, то есть открывающая 
или одинарная или двойная, закрывающая --- такая же, 
а то, что между ними --- содержимое.

Примеры:
\fi

\verb|CONTROL,'W'="WHILE  DO",CR,"END;",UP,LINEEND,LEFTWORD,LEFT| \\
{\small
\ifenglish
  creates an empty WHILE-loop and moves the cursor to the condition placeholder
\else
  делает болванку while-цикла и ставит каретку в позицию его условия
\fi
} % \small

\verb|CONTROL,'1'=BLKLEFTWORD,COPY,LEFT,"DEFINITION MODULE ",\|\\
\verb|LINEEND,CR,"END ",PASTE,".",UP,LINEEND,\|\\
\verb|"; (* COPYRIGHT (c) Excelsior LLC *)",CR,CR,CR,UP|\\
{\small
\ifenglish
  creates a definition module template with copyright notice 
  (usage: create a new file, type the desired module name, and press {\bf Ctrl-1}.)
\else
  делает болванку DEFINITION-модуля c уведомлением об авторском праве.
\fi
} % \small

\ifenglish
\subsection{Editor commands}
\else
\subsection{Команды редактора}
\fi
\public{EditorCommands}

\ifenglish
Cursor movement
\else
Перемещения каретки:
\fi

RIGHT, LEFT, UP, DOWN, PGUP, PGDOWN, LINESTART, LINEEND, PAGESTART, PAGEEND, RIGHTWORD, LEFTWORD

\ifenglish
Cursor movement with selection (same commands with \verb'BLK' prefix)
\else
Перемещения каретки с выделением текста (те же слова, что и предыдущий раздел, но с префиксом BLK):
\fi

BLKRIGHT, BLKLEFT, BLKUP, BLKDOWN, BLKPGUP, BLKPGDOWN, BLKLINESTART, BLKLINEEND, BLKPAGESTART, BLKPAGEEND, BLKRIGHTWORD, BLKLEFTWORD

\ifenglish
Miscellaneous:
\else
Разнообразные действия:
\fi

DEL, BS, TAB, BACKTAB, DELWORD, DELWORDLEFT, FLIPINSERT, NEXTLINE, INSERTLINE, CR, DELLINE, SPLITLINE, JOINLINES, UNDO, UPPERCASE, FLIPCASE

\ifenglish
Block indenting:
\else
Сдвижка блоков:
\fi

INDENTLEFT, INDENTLEFT1, INDENTRIGHT, INDENTRIGHT1

\ifenglish
Private clipboard operations (each editor window has a private clipboard, only accessible from macros)
\else
Взаимодействие с приватным буфером обмена, имеющимся у каждого окна редактирования специально для работы с макросами
\fi

COPY, PASTE, CUT, EXCHANGE, ADD

\ifenglish
System clipboard operations
\else
Взаимодействие с системным буфером обмена
\fi

SYSCOPY, SYSPASTE, SYSCUT, SYSEXCHANGE, SYSADD

\ifenglish
Bookmarks
\else
Закладки
\fi

SETBOOKMARK1, SETBOOKMARK2..SETBOOKMARK9 \\
GOTOBOOKMARK1, GOTOBOOKMARK2..GOTOBOOKMARK9

\ifenglish
\subsubsection{Example macro definition file for Modula-2}
\else
\subsubsection{Пример файла описаний макросов для Модулы-2}
\fi

\begin{verbatim}
    CONTROL,'I'="INTEGER"
    CONTROL,'I',SHIFT="IF  THEN",CR,"END;",UP,LINEEND,LEFTWORD,LEFT
    CONTROL,'C'="CHAR"
    CONTROL,'C',SHIFT="CARDINAL"
    CONTROL,'W'="WHILE  DO",CR,"END;",UP,LINEEND,LEFTWORD,LEFT
    CONTROL,'R'="REPEAT",CR,"UNTIL ;",LEFT
    CONTROL,'P',SHIFT=BLKLEFTWORD,COPY,LEFT,"PROCEDURE ",LINEEND,CR,"BEGIN",CR,"END ",PASTE,";",UP,UP,LINEEND,"();",LEFT,LEFT
    CONTROL,'['="ARRAY ["
    CONTROL,']'="] OF "
    CONTROL,'V'="VAR"
    CONTROL,'S'="ARRAY OF CHAR"
    CONTROL,'X'="ARRAY OF"
    CONTROL,'1'=BLKLEFTWORD,COPY,LEFT,"DEFINITION MODULE "    ,LINEEND,CR,\
      "END ",PASTE,".",UP,   LINEEND,"; (* COPYRIGHT (c) Excelsior *)",CR,CR,CR,UP
    CONTROL,'2'=BLKLEFTWORD,COPY,LEFT,"IMPLEMENTATION MODULE ",LINEEND,CR,\
      "BEGIN",CR,"END ",PASTE,".",UP,UP,LINEEND,"; (* COPYRIGHT (c) Excelsior *)",CR,CR,CR,UP
    CONTROL,'3'=BLKLEFTWORD,COPY,LEFT,"MODULE "               ,LINEEND,CR,\
      "BEGIN",CR,"END ",PASTE,".",UP,UP,LINEEND,"; (* COPYRIGHT (c) Excelsior *)",CR,CR,CR,UP
\end{verbatim}


\ifenglish
\subsection{Prefix mode}
\else
\subsection{Префиксный режим}
\fi

\ifenglish
In this mode, macros are invoked by a combination of two keyboard events:
the macro toggle key ({\bf F1}) and {\it Event}. 
To enable this mode, insert the following line at the top of your 
macro definition file:
\else
Режим cделан по воспомнинаниям о Кроносе. 
Макрокоманды в этом режиме запускается не одним событием клавиатуры, 
а последовательностью из двух --- нажатие {\bf F1} и собственно {\it Cобытие}. 

Чтобы его этот режим, в соответствующий файл с описаниями макросов 
надо вставить строчку
\fi

\verb'    toggleon'

\ifenglish
Then a macro defined as
\else
После чего макрос, описанный как
\fi

\verb'    CONTROL,"C"="CHAR"'

\ifenglish
will not fire each time you press {\bf Ctrl-C}, but only if you press
{\bf F1} first.
\else
будет запускаться не по нажатию {\bf Ctrl-C}, а по последовательности нажатий
({\bf F1}, {\bf Ctrl-C}).
\fi

\ifenglish
{\bf Note:} As you might have guessed, in this mode you need to press 
{\bf F1} twice to display online help.
\else
{\bf Примечание:} В префиксном режиме подсказка вызывается 
двойным нажатием {\bf F1}.
\fi

\ifenglish
\subsubsection{Example macro definition file for Modula-2 (prefix mode)}
\else
\subsubsection{Пример файла описаний макросов для Модулы-2 (префиксный режим)}
\fi
\begin{verbatim}
    toggleon
    'I'="INTEGER"
    'I',SHIFT="IF  THEN",CR,"END;",UP,LINEEND,LEFTWORD,LEFT
    'C'="CHAR"
    'C',SHIFT="CARDINAL"
    'W'="WHILE  DO",CR,"END;",UP,LINEEND,LEFTWORD,LEFT
    'R'="REPEAT",CR,"UNTIL ;",LEFT
    'P',SHIFT=BLKLEFTWORD,COPY,LEFT,"PROCEDURE ",LINEEND,CR,"BEGIN",CR,"END ",PASTE,";",UP,UP,LINEEND,"();",LEFT,LEFT
    '['="ARRAY ["
    ']'="] OF "
    'V'="VAR"
    'S'="ARRAY OF CHAR"
    'X'="ARRAY OF"
    '1'=BLKLEFTWORD,COPY,LEFT,"DEFINITION MODULE "    ,LINEEND,CR,\
      "END ",PASTE,".",UP,   LINEEND,"; (* COPYRIGHT (c) Excelsior *)",CR,CR,CR,UP
    '2'=BLKLEFTWORD,COPY,LEFT,"IMPLEMENTATION MODULE ",LINEEND,CR,\
      "BEGIN",CR,"END ",PASTE,".",UP,UP,LINEEND,"; (* COPYRIGHT (c) Excelsior *)",CR,CR,CR,UP
    '3'=BLKLEFTWORD,COPY,LEFT,"MODULE "               ,LINEEND,CR,\
      "BEGIN",CR,"END ",PASTE,".",UP,UP,LINEEND,"; (* COPYRIGHT (c) Excelsior *)",CR,CR,CR,UP
\end{verbatim}


\ifenglish
\subsection{Registering macro definition files}
\else
\subsection{Регистрация наборов макрокоманд}
\fi
\public{MacroSets}

\ifenglish
You may define macro sets specific to programming languages
and the IDE will select one for each editor window, provided it
will have recognized the language. Otherwise, it will use the default set,
which is, in turn, loaded from the file \verb'BIN\xds.mdf'
by default. To change the name of the default macro definition file,
add the following line to the \verb'[macro]' section in \verb'BIN\XDS.ini':
\else
Оболочка выбирает набор макросов для каждого окна редактора 
в соответствии с \ref[языком программирования]{Languages}, 
который в нем используется. Если язык определить не удалось или
для него не задан отдельный набор макросов, то для этого окна 
используется умолчательный. Опять же по умолчанию этот набор 
берется из файла \verb'BIN\xds.mdf' в папке, куда установлена система XDS.
Чтобы заменить это имя на, скажем, ``\verb'mymacro.file''', нужно в 
файле \verb'BIN\XDS.ini' завести секцию \verb'[macro]' и в ней написать:
\fi

\verb'    default=mymacro.file'

\ifenglish
For the IDE to load a different macro set for Modula-2 source files
from file \verb'BIN\m2o2.mdf', add the following line to the same section:
\else  
А чтобы для файлов с исходными текстами на Модуле-2 использовался 
набор макрокоманд из файла "m2o2.mdf", в той же секции ini-файла нужно написать:
\fi

\verb'    Modula-2=m2o2.mdf'

\ifenglish
You may add a macro set for any of the languages listed in 
the \verb'[languages]' section of the same ini-file.
\else  
Для других известных оболочке языков все делается аналогично.
(имена этих языков перечислены в секции \verb'[languages]' того же ini-файла).
\fi

\ifenglish
Example:
\else  
Пример:
\fi

\begin{verbatim}
       .  .  .
    [languages]
    lang1=Modula-2
    lang2=Oberon-2
    lang3=Java
       .  .  .
    [macro]
    default=m2o2.mdf
    Java=java.xdsmacro
       .  .  .
\end{verbatim}


