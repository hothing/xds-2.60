\ifenglish
\section{Programming Languages}
\else
\section{Языки программирования}
\fi
\public{Languages}
\nominitoc

\ifenglish
The XDS IDE allows you to configure several {\bf programming languages} and specify
which files belong to each language. This allows the editor to behave differently
depending on the language in which the file is written.

{\bf New in 2.6:} 

You may define a set of \ref[keyboard macros]{Macros} specific for a language 
as described in \ref{MacroSets}.

The only other feature the languages affect is text coloring.
Each language has its own set of syntax elements such as
identifiers, numbers, etc. You may define how each element looks in the
editor window.
\else
Оболочка XDS позволяет настраивать несколько {\bf языков программирования} и указывать, 
какие файлы относятся к каждому языку. Это дает редактору возможность по-разному
вести себя, в зависимости от того, на каком языке пишется данный файл.

{\bf Новое в версии 2.6:}

Вы можете определить набор \ref[макрокоманд]{Macros}, специфичный для конкретного
языка, как описано в \ref{MacroSets}.

Помимо этого, единственный атрибут, на который имеет влияние язык, 
это окраска текста. У каждого языка есть свой собственный набор синтаксических
элементов, таких как идентификаторы, числа и т.д. Вы можете определить, как 
будет выглядеть каждый из элементов в окне редактирования. 
\fi
\ifenglish
The set of programming languages is not built in the IDE. You may specify
as many languages as you wish, provided that files in different languages
have names with different suffixes.

You must assign a suffix or several suffixes to the language. After this you
can configure properties that depend on a language, for example, you can define
the color of a text in all windows with the text in this language. To use
syntax highlighting you must provide a \ref[language driver]{LangDriver} --
external component that possesses knowledge of the language and structure of
texts in it.

One driver can handle more than one language.
\else
Набор языков программирования не встроен в IDE. Вы можете задать столько 
языков, сколько вам угодно, лишь бы относящиеся к ним файлы имели разные 
расширения.

Вы должны приписать языку одно или несколько расширений. После этого вы можете 
настроить зависящие от языка свойства, например, вы можете определить цвет текста
во всех окнах для этого языка. Чтобы воспользоваться подсветкой синтаксиса вы 
должны располагать \ref[драйвером языка]{LangDriver} -- внешней составляющей,
обладающей знаниями о языке и структуре его текста.
\fi
\ifenglish
\subsection{Language Drivers}
\else
\subsection{Драйверы языков}
\fi
\public{LangDriver}

\ifenglish
Language driver is external dynamic link library (DLL) that implements
specific interface and provides functionality necessary for working with
programming languages.

The language driver provides:
\else
Драйвер языка -- это внешняя динамически подключаемая библиотека (DLL),
которая предоставляет специфический интерфейс и необходимую для работы с 
языками программирования функциональность.

Драйвер языка предоставляет: 
\fi
\begin{itemize}
\item \ifenglish
      the set of syntax elements it can recognize in a program such as constants,
      identifiers and keywords
      \else
      набор синтаксических элементов, которые он может распознавать в программе, 
      таких как константы, идентификаторы и ключевые слова.
      \fi
\item \ifenglish
      the routine to parse given line of text into the sequence of syntax elements
      \else
      порядок разбора данной строки текста на последовательность синтаксических 
      элементов
      \fi
\item \ifenglish
      the routine to be called when some text fragment has changed to prepare
      some data for parsing (for example, Modula-2 driver calculates comment nesting
      in this routine)
      \else
      процедуру, подготавливающую текст к анализу после того, как некоторый его
      участок был изменен (например, драйвер Modula-2 высчитывает с помощью этой
      процедуры границы комментариев).  
      \fi
\item \ifenglish
      the sample text in a language for color selection dialog
      \else
      шаблонный текст языка для диалога выбора цвета
      \fi
\item \ifenglish
      the configure dialog (for example, Modula-2 driver allows to configure
      name of file with keywords and to turn on and off syntax extensions)
      \else
      диалог настройки (например, драйвер Modula-2 позволяет указывать имя 
      файла, содержащего ключевые слова, и переключать расширения синтаксиса)
      \fi
\item \ifenglish
      the size of driver environment block
      \else
      размер блока рабочей среды драйвера
      \fi
\end{itemize}

\ifenglish
All the components are optional. For example, driver may have no configure
dialog or no sample text.

Currently the system has only one language driver for Modula-2 and Oberon-2 but
the interface of language driver is specified so it is possible to develop
custom drivers. The interface of a driver is not shipped with XDS but is available upon request.

One driver can be used for several programming languages. This is convenient if
languages are similar (such as Modula-2 and Oberon-2 and as C and C++).
The driver is loaded independently for each language and is configured independently.
This allows to have different settings for different programming languages, for
example, to use different keyword files.

All the driver DLLs are loaded at IDE startup so you can't change DLL on disk
while IDE is running and this DLL is configured for any language. DLL is unloaded if
you configure another DLL or remove the language from the IDE.

\else
Все компоненты необязательны. Например, у драйвера может не быть диалога 
настройки или шаблонного текста.

На настоящий момент в системе есть только один драйвер для Modula-2 и Oberon-2,
но интерфейс языкового драйвера устроен так, что его можно перенастраивать,
получая другие драйверы. Интерфейс драйвера не поставляется с XDS, но его 
можно потребовать.
 
Один драйвер может быть использован для нескольких языков программирования.
Это удобно в случае, когда языки похожи (как Modula-2 и Oberon-2 или C и C++).
Драйвер для каждого языка загружается и настраивается независимо. Это дает 
возможность иметь разные установки для разных язаков программирования, например,
использовать разные файлы ключевых слов. 

Все DLL драйвера загружаются при запуске IDE, так что нельзя изменять DLL на 
диске во время работы IDE, и эта DLL настраивается для любого языка. DLL 
выгружается, если вы настраиваете другую DLL или удаляете язык из IDE.
\fi
