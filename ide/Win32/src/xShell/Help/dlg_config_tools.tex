
\ifenglish
\section{Configure tools dialog}
\else
\section{диалог Configure tools}
\fi
\public{CFGTOOLS\_DIALOG}
\nominitoc

\ifenglish
This dialog is provided for configuration of external \ref[{\em Tools}]{Tools}.
You may add and remove tools here and set their properties such as invocation
conditions, menu strings, hot keys etc.

Following configurable options are provided:

\else
Этот диалог позволяет настраивать внешние \ref[{\em утилиты}]{Tools}.
Здесь вы можете добавлять и удалять утилиты, и устанавливать их свойства, такие 
как условия вызова, строки меню, активизирующие клавиши и т. д.

Предусмотрены следующие настраиваемые опции:
\fi

{\bf Compiler executable}

\ifenglish
Four standard tools ({\bf Compile}, {\bf Make}, {\bf Build} and {\bf MakeList})
and four versions of standard tools for working without project usually are
based on common executable file with different options. To make configuration easier
this field is provided. Once entered here, the main XDS executable file name may
be referenced as many times as you wish.

You may type the name of an XDS compiler executable file in the entry field,
or select the {\bf Browse} pushbutton to display a standard file dialog.

\ref[variables]{variables} may be used in this field.

\verb'$(xdsmain}', \verb'$(xdsname)', and \verb'$(xdsdir)'
\ref[substitution variables]{variables} receive their values from this field.
They may be used in command lines of other tools.

Separate definition of main XDS executable allows quick change of compiler
used, for example, change from using native-code compiler to the C converter.

\else
Четыре стандартных утилиты ({\bf Compile}, {\bf Make}, {\bf Build} и {\bf MakeList}),
и четыре версии стандартных утилит для работы без проекта основываются на одном
исполняемом файле с применением различных опций. Это поле делает настройку проще.
На однажды введенный здесь главный исполняемый файл XDS можно ссылаться сколько
угодно раз.

Вы можете набрать в этом поле ввода имя исполняемого файла XDS компилятора,
или нажать кнопку {\bf Browse}, чтобы вывести стандартный файловый диалог.

В этом поле могут быть использованы \ref[переменные]{variables}.

\ref[переменные подстановки]{variables}
\verb'$(xdsmain}', \verb'$(xdsname)', and \verb'$(xdsdir)'
получают в этом поле свои значения. Они могут быть использованы в командных 
строках других утилит.

Отдельное определение главного исполняемого модуля XDS позволяет быстро 
изменять используемый компилятор, например, переходить с компилятора, порождающего
объектный код, на транслятор в язык C.
\fi

{\bf Select a tool to configure}

\ifenglish
This list contains names of all currently available tools. Select
the tool you want to configure.

\else
Этот список содержит имена всех доступных на данный момент утилит. Выберите
утилиту, которую вы хотите настроить. 
 \fi

{\bf Add/edit}
\ifenglish
This pushbutton to add a new tool to the list or
to edit the currently selected tool's comment and category. The
\ref[Add New or Edit Existing tool]{ADDTOOL\_DIALOG} dialog will appear.
\else
Используйте эту кнопку, чтобы добавлять новую утилиту в список, или менять 
комментарий к выбранной в данный момент утилите и ее категорию.
Появится диалог \ref[Add New or Edit Existing tool]{ADDTOOL\_DIALOG}.
\fi

{\bf Remove}
\ifenglish
This pushbutton removes the currently selected
tool from the list.

{\bf Note:} predefined tools can not be removed.

\else
Эта кнопка удаляет выбранную утилиту из списка.

{\bf Заметьте:} стандартные утилиты не могут быть удалены.
\fi

{\bf Category}
\ifenglish
Displays current tool's category. The category can be edited by pressing
{\bf Add/edit} button. What category is and how it affects tools behavior see
\ref[here]{Tools}

\else
Отображает категорию текущей утилиты. Категорию можно изменять с помощью 
нажатия кнопки {\bf Add/edit}. Чтобы узнать, что такое категория, и как она 
влияет на утилиту, смотрите \ref[сюда]{Tools}.
\fi

{\bf Command line}
\ifenglish
Type in this field an external  command you want to be invoked when
you activate the currently selected tool. You may select the
{\bf Browse} pushbutton to locate an executable file via a
standard file dialog. For a predefined tool, selecting the
{\bf Default} pushbutton sets the default command line.

\ref{variables} may be used in this field.
\else
Наберите в этом поле внешнюю команду, которую вы хотите вызвать, когда вы 
запускаете выбранную утилиту. Вы можете нажать кнопку {\bf Browse}, чтобы выбрать
исполняемый файл посредством стандартного файлового диалога. В случае стандартной
утилиты нажатие кнопки {\bf Default} устанавливает задаваемую по умолчанию
командную строку.

В этом поле могут быть использованы \ref[переменные]{variables}.
\fi

{\bf Menu item}
\ifenglish
If you want the currently selected tool to appear in the {\bf Tools}
menu when the tool is enabled, select the checkbox and type a menu item
text in the entry field.

\ref{variables} may be used in this field.

{\bf Note:} The rules that regulate how tools appear in main menu see
\ref[here]{Tools}.

\else
Если вы хотите, чтобы выбранная утилита появилась в меню {\bf Tools},
когда утилита задействована, установите этот контроль и наберите текст пункта 
меню в этом поле ввода.

В этом поле могут быть использованы \ref[переменные]{variables}.

{\bf Замечание:} Сведения о правилах, управляющих появлением утилиты в главном меню
содержатся \ref[здесь]{Tools}.
\fi

{\bf Inactive menu item}
\ifenglish
If you want the currently selected tool to appear in the {\bf Tools}
menu when the tool is disabled, select the checkbox and type a menu item
text in the entry field.

\ref{variables} may be used in this field.

{\bf Note:} See previous {\bf Note}.
\else
Если вы хотите, чтобы выбранная утилита появилась в меню {\bf Tools},
когда утилита не задействована, установите этот контроль и наберите текст 
пункта меню в этом поле ввода.

В этом поле могут быть использованы \ref[переменные]{variables}.

{\bf Замечание:} Смотрите предыдущее замечание.
\fi

{\bf Run when project open}
\ifenglish
Select this checkbox if you want the currently selected tool to be
available only when a project is open in the Environment.
\else
Установите этот контроль, если хотите чтобы выбранная в данный момент утилита 
была доступна, только если в рабочей среде открыт проект. 
\fi

{\bf Run on files with suffixes}
\ifenglish
Select this checkbox if you want the currently selected tool to be
available only if the current edit window contains a file which name
ends in a suffix belonging to a certain list. Type the list of suffixes
(separated with semicolons) in the entry field.

Suffix {\bf "*"} means any file is suitable.
\else
Установите этот контроль, если хотите, чтобы выбранная в данный момент утилита 
была доступна, только если текущее окно редактирования содержит файл, имя 
которого оканчивается расширением, приписываемым определенному списку. Наберите
в этом поле ввода список расширений (разделяемых точкой с запятой). 

Расширение {\bf "*"} означает, что подходит любой файл
\fi

{\bf Start in directory}
\ifenglish
Defines what directory is current when the tool is run. Usually this is set to
{\bf \$(projdir)} for tools that work on projects and to
{\bf \$(filedir)} for
other tools. You may select the directory by pressing {\bf Browse} button.
For predefined tools pressing the {\bf Default} button sets the directory to the
default state.
\else
Определяет, какая директория становится текущей, когда запускается утилита. Обычно 
это равно {\bf \$(projdir)} для утилит, которые работают с проектами, и 
{\bf \$(filedir)} для других утилит. 
Вы можете выбрать директорию с помощью нажатия кнопки {\bf Browse}.
Для стандартных утилит нажатие кнопки {\bf Default} инициализирует директорию
значением, принимаемым по умолчанию. 
\fi

{\bf Advanced}
\ifenglish
Select the {\bf Advanced} pushbutton to display a \ref[dialog]{ADVANCED\_TOOL\_DIALOG}
that allows setup of advanced properties of
the currently selected tool, that include:
\else
Нажмите кнопку {\bf Advanced}, чтобы вывести \ref[диалог]{ADVANCED\_TOOL\_DIALOG},
который позволяет устанавливать дополнительные свойства 
выбранной в данный момент утилиты, которые включают:
\fi
\begin{itemize}
\item \ifenglish
      how to run tool (in its own window or in pop-up)
     \else
      как запускать утилиту (в своем окне или во всплывающем)
      \fi
\item \ifenglish
      if in pop-up, close it automatically or wait for keypress
      \else  
      если во всплывающем, то закрывать его автоматически, или ждать нажатия 
      клавиши 
      \fi
\item \ifenglish
      caption of the window
      \else
      заголовок окна
      \fi
\item \ifenglish
      hot key
      \else
      активизирующая клавиша
      \fi
\item \ifenglish
      \ref[Message Filter]{MessageFilters}
       \else
      \ref[фильтр сообщений]{MessageFilters}
      \fi
\item \ifenglish
      checking of program return code
       \else
      проверка возвращаемого программой кода
      \fi
\end{itemize}

%-----------------------------------------------------------------------
\begin{popup}
\ifenglish
\caption{Configuration file}
 \else
\caption{Файл конфигураций}
\fi
\public{CFGTOOLS\_INIFILE}
\ifenglish
This label indicates main XDS configuration file in which tools settings are kept.
 \else
Эта метка показывает главный конфигурационный файл XDS, в котором хранятся установки
утилит.
\fi
\end{popup}

\begin{popup}
\caption{Compiler executable}
\public{CFGTOOLS\_XDSMAIN}

\ifenglish
Type the name of an XDS compiler executable file in the entry field,
or select the {\bf Browse} pushbutton to display a standard file dialog.

Substitution variables may be used in this field.
 \else
Наберите имя исполняемого файла XDS компилятора в этом поле ввода,
или нажмите кнопку {\bf Browse}, чтобы вывести стандартный файловый диалог.

В этои поле могут быть использованы переменные замены.
\fi
\end{popup}

\begin{popup}
\caption{Compiler executable}
\public{CFGTOOLS\_XDSMAIN\_BROWSE}

\ifenglish
Select XDS compiler executable file in a standard file dialog,
or type it manually in the entry field.
 \else
Выберите исполняемый файл XDS компилятора в стандартном файловом диалоге 
или введите его вручную в этом поле ввода.
\fi
\end{popup}

\begin{popup}
\ifenglish
\caption{Select a tool to configure}
 \else
\caption{Выберите утилиту для настройки}
\fi
\public{CFGTOOLS\_LIST}

\ifenglish
This list contains names of all currently available tools. Select
the tool you want to configure.
 \else
Это список содержит имена всех доступных на данный момент утилит. Выберите 
утилиту, которую вы хотите настроить.
\fi
\end{popup}

\begin{popup}
\caption{Category}
\public{CFGTOOLS\_CATEGORY}

\ifenglish
Defines the {\bf category} of current tool. Several tool of the same category
share single menu entry, toolbar button and keyboard shortcut.

{\bf Note:} See general help on tools to know what category is and how it
affect tools behavior.
 \else
Определяет {\bf категорию} текущей утилиты. Несколько утилит одной категории
делят между собой одну строчку меню, кнопку toolbar и комбинацию клавиш 
клавиатуры. 

{\bf Замечание:} Смотрите общую справку по утилитам, чтобы узнать, что такое 
категория, и как она влияет на поведение утилит.
\fi
\end{popup}

\begin{popup}
\caption{Comment}
\public{CFGTOOLS\_COMMENT}

\ifenglish
This label contains the comment of currently selected tool.
Comment is a text that appears in the Statusbar when the user selects this
tool in a menu or places the mouse cursor over this tool's button in the Toolbar.

The comment can be edited using {\bf add/edit} button.
 \else
Эта метка содержит комментарий к выбранной в данный момент утилите. 
Комментарий -- это текст, который появляется в Statusbar когда пользователь 
выбирает эту утилиту в меню, или помещает курсор мыши на кнопку этой утилиты в
Toolbar. 

Коментарий можно менять с помощью кнопки {\bf add/edit}.
\fi
\end{popup}

\begin{popup}
\caption{Command line}
\public{CFGTOOLS\_CMDLINE}

\ifenglish
Type in this field an external  command you want to be invoked when
you activate the currently selected tool. The file name component of
a command line may be located using standard file dialog by pressing the
{\bf Browse} button.

Substitution variables may be used in this field.
 \else
Наберите в этом поле внешнюю команду, которую вы хотите вызвать при запуске
выбранной в данный момент утилиты. Компонент имени файла командной строки
может быть задан с помощью стандартного файлового диалога посредством нажатия 
кнопки {\bf Browse}.

В этом поле могут быть использованы переменные замены.
\fi
\end{popup}

\begin{popup}
\caption{Browse}
\public{CFGTOOLS\_BROWSE}

\ifenglish
Locate an executable file using standard file dialog and store it into
file name component of current tool command line. Only file name is
replaced; parameters remain untouched.
 \else
Выберите исполняемый файл, используя стандартный файловый диалог, и вставьте
его на место компоненты имени файла командной строки текущей утилиты. Заменяется
только имя файла, параметры не затрагиваются. 
\fi
\end{popup}

\begin{popup}
\caption{Default}
\public{CFGTOOLS\_DEFAULT}

\ifenglish
This option is only available for standard tools, for which the IDE knows
command line format. This button sets tool command line to default state.
 \else
Эта опция доступна только для стандартных утилит, формат командной строки 
которых известен IDE. Эта кнопка устанавливает командную строку утилиты в 
положение, принимаемое по умолчанию. 
\fi
\end{popup}

\begin{popup}
\caption{Add/edit}
\public{CFGTOOLS\_ADD}

\ifenglish
Select the {\bf Add/edit} pushbutton to add a new tool to the list or
to edit the currently selected tool comment.
 \else
Нажмите кнопку {\bf Add/edit}, чтобы добавить новую утилиту в список, или 
редактировать комментарий к выбранной в данный момент утилите.
\fi
\end{popup}

\begin{popup}
\caption{Remove}
\public{CFGTOOLS\_REMOVE}

\ifenglish
Select the {\bf Remove} pushbutton to remove the currently selected
tool from the list.

{\bf Note:} predefined tools can not be removed.

 \else
Нажмите кнопку {\bf Remove}, чтобы удалить выбранную утилиту из списка.

{\bf Замечание:} стандартные утилиты не могут быть удалены.
\fi
\end{popup}

\begin{popup}
\caption{Menu item}
\public{CFGTOOLS\_MENU}
\public{CFGTOOLS\_MENUITEM}

\ifenglish
If you want the currently selected tool to appear in the {\bf Tools}
menu when it is enabled, select the checkbox and type a menu item text in the entry field.

Substitution variables may be used in this field.

{\bf Note:} See general help on tools to know how exactly tools menu is 
constructed.
 \else
Если вы хотите, чтобы выбранная в данный момент утилита появилась в меню 
{\bf Tools}, когда она разрешена к употреблению, установите этот контроль, 
и наберите текст пункта меню в поле ввода.

В этом поле могут быть использованы переменные замены.

{\bf Замечание:} Смотрите общую справку по утилитам, чтобы узнать, как устроено 
меню утилит.
\fi
\end{popup}

\begin{popup}
\caption{Inactive menu item}
\public{CFGTOOLS\_INACTIVE\_MENU}
\public{CFGTOOLS\_INACTIVE\_MENUITEM}

\ifenglish
If you want the currently selected tool to appear in the {\bf Tools}
menu when it is disabled, select the checkbox and type a menu item text in the entry field.

Substitution variables may be used in this field.

{\bf Note:} See general help on tools to know how exactly tools menu is 
constructed.
 \else
Если вы хотите, чтобы выбранная в данный момент утилита появилась в меню 
{\bf Tools}, когда она не разрешена к употреблению, установите этот контроль 
и наберите текст пункта меню в поле ввода.

В этом поле могут быть использованы переменные замены.

{\bf Замечание:} Смотрите общую справку по утилитам, чтобы узнать, как устроено 
меню утилит.
\fi
\end{popup}


\begin{popup}
\caption{Run when project open}
\public{CFGTOOLS\_RUNPROJECT}

\ifenglish
Select this checkbox if you want the currently selected tool to be
available only when a project is open in the Environment.
 \else
Установите этот контроль, если хотите, чтобы выбранная в данный момент утилита 
была доступна только когда в рабочей среде открыт проект.
\fi
\end{popup}

\begin{popup}
\caption{Run on files with suffixes}
\public{CFGTOOLS\_RUNFILE}

\ifenglish
Select this checkbox if you want the currently selected tool to be
available only if current window is edit window and name of file in it
ends in a suffix belonging to a certain list. Type the list of suffixes
in the entry field.
 \else
Установите этот контроль, если хотите чтобы выбранная в данный момент утилита
была доступна только если текущее окно -- это окно редактирования, и имя файла,
содержащегося в нем, оканчивается расширением, принадлежащим определенному 
списку. Наберите список расширений в этом поле ввода.
\fi
\end{popup}

\begin{popup}
\caption{Run on files with suffixes}
\public{CFGTOOLS\_SUFFIX}

\ifenglish
When {\bf Run on files with suffixes} box is checked this field defines
list of possible file suffixes. Suffixes are listed without "{\bf .}"
character and separated by a semicolon. Special value "{\bf *}" means that
any suffix is allowed.
 \else
Когда установлен контроль {\bf Run on files with suffixes}, это поле определяет
список возможных расширений файлов. Расширения перечисляются без символа "{\bf .}"
и разделяются точкой с запятой. Особое значение "{\bf *}" говорит, что допущено
любое расширение.
\fi
\end{popup}

\begin{popup}
\caption{Advanced}
\public{CFGTOOLS\_ADVANCED}

\ifenglish
Select the {\bf Advanced} pushbutton to setup advanced properties of
the currently selected tool, such as run mode, hot-key, return code
checking, and message filter.
 \else
Нажмите кнопку {\bf Advanced}, чтобы установить доплнительные свойства 
выбранной утилиты, такие, как режим запуска, активизирующую клавишу, проверку 
возвратного кода и фильтр сообщений.
\fi
\end{popup}

\begin{popup}
\caption{Start in directory}
\public {CFGTOOLS\_DIR}

\ifenglish
Type here the start directory for a tool. Alternatively, you may select it
using the {\bf Browse} button.

Substitution variables may be used here.
 \else
Введите здесь директорию запуска утилиты. Вы также можете выбрать ее, используя 
кнопку {\bf Browse}.

Здесь могут быть использованы переменные замены.
\fi
\end{popup}

\begin{popup}
\caption{Browse}
\public {CFGTOOLS\_DIRBROWSE}

\ifenglish
Select start directory for a tool. You may also write it manually into the entry field.
 \else
Выберите директорию запуска для утилиты. Вы также можете вручную написать ее в 
поле ввода. 
\fi
\end{popup}

\begin{popup}
\caption{Default}
\public{CFGTOOLS\_DEFAULTDIR}

\ifenglish
This option is available for predefined tools. It sets startup directory for a tool
to default value.
 \else
Эта опция доступна для стандартных утилит. Она инициализирует директорию запуска
утилиты значением, принимаемым по умолчанию.
\fi
\end{popup}

\begin{popup}
\caption{OK}
\public{CFGTOOLS\_OK}

\ifenglish
Select the {\bf OK} pushbutton to confirm the changes you've made to the
tools configuration.
 \else
Нажмите кнопку {\bf OK}, чтобы подтвердить произведенные вами в настройках 
утилит изменения.
\fi
\end{popup}

\begin{popup}
\caption{Cancel}
\public{CFGTOOLS\_CANCEL}

\ifenglish
Select the {\bf Cancel} pushbutton if you decided not to make any changes
to the current tools configuration.
 \else
Нажмите кнопку {\bf Cancel}, если решили не производить изменений в настройках
утилит.
\fi
\end{popup}

%-----------------------------------------------------------------------

\ifenglish
\subsection{Add New or Edit Existing Tool dialog}
 \else
\subsection{диалог Add New or Edit Existing Tool}
\fi
\public{ADDTOOL\_DIALOG}
\nominitoc

\ifenglish
This dialog is displayed when you select the {\bf Add/edit} pushbutton
in the \ref[Configure Tools]{CFGTOOLS\_DIALOG} dialog.
To add a new tool, type its name, category and comment in
corresponding entry fields and select the {\bf OK}
pushbutton. To change existing tool's category and comment,
leave its name unchanged.

Tool's {\em comment} is a string that is displayed in the
Statusbar when the user selects corresponding item in menu or moves mouse over
toolbar button.

What is a {\em category} and how it affects tools behavior see
\ref[here]{Tools}.
 \else
Этот диалог отображается, когда вы нажимаете кнопку {\bf Add/edit}
в диалоге \ref[Configure Tools]{CFGTOOLS\_DIALOG}. 
Чтобы добавить новую утилиту, введите ее имя, категорию и комментарий 
в соответствующих полях ввода и нажмите кнопку {\bf OK}.
Чтобы изменить категорию и комментарий к существующей утилите,
оставьте ее имя неизмененным.

{\em Комментарий} утилиты -- это строка, которая отображается в 
Statusbar, когда пользователь выбирает соответствующий пункт меню
или перемещает мышь вдоль toolbar.

Что такое {\em категория}, и как она влияет на поведение утилиты, смотрите 
\ref[здесь]{Tools}.
\fi
%-----------------------------------------------------------------------

\begin{popup}
\caption{Tool name}
\public{ADDTOOL\_NAME}

\ifenglish
Selected tool name. Changing this name will create new tool.
Leaving it unchanged allows changing of current tool properties.
 \else
Имя выбранной утилиты. Изменение этого имени создаст новую утилиту.
Если оставить имя неизмененным, то можно будет менять свойства текущей утилиты.
\fi
\end{popup}

\begin{popup}
\caption{Category}
\public{ADDTOOL\_CATEGORY}

\ifenglish
Type the category of new or existing tool. Read general tools section for
description of tool category.
 \else
Введите категорию новой или существующей утилиты. Прочтите раздел, посвященный
утилитам, чтобы посмотреть описание категории утилиты.
\fi
\end{popup}

\begin{popup}
\caption{Comment}
\public{ADDTOOL\_COMMENT}

\ifenglish
Type tool's comment here. Tool's comment is a string that is displayed in the
Statusbar when the user selects corresponding item in menu or moves mouse over
tool bar button.
 \else
Введите здесь комментарий к утилите. Комментарий к утилите -- это строка, которая
отображается в Statusbar, когда пользователь выбирает соответствующий пункт меню
или передвигает мышь через кнопку toolbar.
\fi
\end{popup}

\begin{popup}
\caption{OK}
\public{ADDTOOL\_OK}

\ifenglish
Select the {\bf OK} pushbutton to add a new tool or confirm editing
of an existent one.
 \else
Нажмите кнопку {\bf OK}, чтобы добавить новую утилиту или редактировать
существующую.
\fi
\end{popup}

\begin{popup}
\caption{Cancel}
\public{ADDTOOL\_CANCEL}

\ifenglish
Select the {\bf Cancel} pushbutton if you decided not to add or modify
a tool.
 \else
Нажмите кнопку {\bf Cancel}, если решили ни добавлять, ни менять утилиту.
\fi
\end{popup}

%-----------------------------------------------------------------------

\ifenglish
\subsection{Advanced tool configuration dialog}
 \else
\subsection{диалог Advanced tool configuration}
\fi
\public{ADVANCED\_TOOL\_DIALOG}
\nominitoc

\ifenglish
Use this dialog to specify advanced tool settings:
 \else
Используйте этот диалог, чтобы определять дополнительные установки утилиты.
\fi
\begin{itemize}
\item \ifenglish
      Run mode
       \else
      Режим запуска
      \fi
\item \ifenglish
      Pop-up window caption
       \else
      Заголовок всплывающего окна
      \fi
\item \ifenglish
      Hot-key
       \else
      Активизирующая клавиша
      \fi
\item \ifenglish
      Return code checking
       \else
      Проверка возвратного кода 
      \fi
\item \ifenglish
      \ref[Message filter]{MessageFilters}
       \else
      \ref[фильтр сообщений]{MessageFilters}
      \fi
\end{itemize}

{\bf Run in pop-up}

\ifenglish
Select this checkbox if you would like the tool output and progress
indicators to be displayed in a pop-up window; clear it if you want
a separate console to be created for the tool, or if a tool provides
its own window.
 \else
Установите этот контроль, если хотите, чтобы вывод и индикаторы 
работы утилиты отображались во всплывающем окне. Очистите его, если
хотите, чтобы для утилиты был создан отдельный консоль, или если утилита
заводит свое собственное окно. 
\fi
{\bf Wait for keypress}

\ifenglish
Select this checkbox if you want the pop-up window
to remain on the screen after termination of the tool.
 \else
Установите этот контроль, если хотите, чтобы всплывающее окно
осталось на экране после завершения работы утилиты.
 \fi
{\bf Caption}

\ifenglish
Type in this field a string which you would like to be displayed in the
pop-up window or the tool console title bar.

\ref{variables} may be used in this field.
 \else
Наберите в этом поле строку, которую вы хотите отобразить в заголовке 
всплывающего окна или консоля утилиты.

В этом поле могут быть использованы \ref[переменные]{variables}.
 \fi
{\bf Assign hot key}

\ifenglish
Press the hot-key combination with which you would like to invoke the
currently selected tool.
 \else
Нажмите активизирующую комбинацию, посредством которой вы хотели бы 
вызывать текущую утилиту.
 \fi
{\bf Messages filter}

\ifenglish
Select \ref[messages filter]{MessageFilters} for current tool from a set of
configured filters. Generally message filters are maintained by a dialog that
follows \ref[Configure]{M\_CONFIGURE}:\ref[Message Filters]{IDM\_CONFIGUREFILTERS}
menu command, but there is a shortcut: you may create new filter by pressing
{\bf New Filter} pushbutton of edit currently assigned filter by {\bf Edit filter}
pushbutton.

Select {\bf None} from the list if you want to remove message filter from a tool.

 \else
Выберите \ref[фильтр сообщений]{MessageFilters} для текущей утилиты из набора
настроенных фильтров. Вообще, фильтры сообщений поддерживаются диалогом, который 
появляется после выбора команды меню 
\ref[Configure]{M\_CONFIGURE}:\ref[Message Filters]{IDM\_CONFIGUREFILTERS},
но существует более короткий путь: вы можете создать новый фильтр посредством 
нажатия кнопки {\bf New Filter}, или редактировать текущий приписанный фильтр
с помощью кнопки {\bf Edit filter}.

Выберите {\bf None}, если хотите удалить фильтр сообщений из утилиты.
  \fi
{\bf Check return code}

\ifenglish
Select this checkbox if the tool success or failure may be determined by
examining it return code.
 \else
Установите этот контроль, если удачный и неблагоприятный результаты работы 
утилиты могут быть распознаны с помощью анализа ее возвратного кода.
  \fi
{\bf Max good value}

\ifenglish
Type here the maximum value of the return code which indicates successful
completion of a tool. In most cases, however, it should be set to zero.
 \else
Наберите здесь максимальное значение возвратного кода, которое сигнализирует
об удачном завершении работы утилиты. В большинстве случаев, однако, его следует
задавать нулем.
 \fi
%-----------------------------------------------------------------------

\begin{popup}
\caption{Run in pop-up}
\public{ADVANCED\_TOOL\_POPUP}

\ifenglish
Select this checkbox if you would like the tool output and progress
indicators to be displayed in a pop-up window; clear it if you want
a separate console to be created for the tool, or if the tool is GUI
application itself and it provides its own window.
 \else
Установите этот контроль, если хотите, чтобы вывод и индикаторы работы 
утилиты отображались во всплывающем окне; очистите его, если хотите, 
чтобы для утилиты был создан отдельный консоль, или если утилита 
есть GUI-приложение и создает себе окно сама. 
 \fi
\end{popup}

\begin{popup}
\caption{Wait for keypress}
\public{ADVANCED\_TOOL\_WAITOK}

\ifenglish
Select this checkbox if you want the pop-up window in which the tool was run
to remain on the screen after termination of the tool.
 \else
Установите этот контроль, если хотите, чтобы всплывающее окно, в котором была 
запущена утилита, оставалось на экране после завершения работы утилиты.
 \fi
\end{popup}

\begin{popup}
\caption{Comment}
\public{ADVANCED\_TOOL\_CAPTION}

\ifenglish
This label shows tool's comment.
 \else
Эта метка отображает комментарий к утилите.
 \fi
\end{popup}

\begin{popup}
\caption{Caption}
\public{ADVANCED\_TOOL\_POPUP\_CAPTION}

\ifenglish
Type in this field a string which you would like to be displayed in the
pop-up window or the tool console title bar.

Substitution variables may be used in this field.
 \else
Наберите в этом поле строку, которую вы хотите отобразить в заголовке
всплывающего окна или консоля утилиты.

В этом поле могут использоваться переменные замены.
 \fi
\end{popup}

\begin{popup}
\caption{Assign hot key}
\public{ADVANCED\_TOOL\_HOTKEY}

\ifenglish
Press the hot-key combination with which you would like to assign to the
currently selected tool. Pressing this combination will invoke the tool without
entering the menu.
 \else
Нажмите активизирующую комбинацию клавиш, которую вы хотите приписать к 
выбранной утилите. Нажатие этой комбинации запустит утилиту. 
 \fi
\end{popup}

\begin{popup}
\caption{Messages filter}
\public {ADVANCED\_TOOL\_FILTERS}

\ifenglish
This combo box displays all available message filter. Choose one to assign it
to current tool. Choose {\bf none} to assign no message filter.
New message filter may be created, if necessary, by pressing {\bf New filter}
button.
  \else
Этот комбинированный список отображает все доступные фильтры сообщений. Выберите
один из них чтобы, приписать к текущей утилите. Выберите {\bf none}, чтобы не
приписывать фильтр сообщений. Если необходимо, может быть создан новый фильтр
сообщений посредством нажатия кнопки {\bf New filter}.
  \fi
\end{popup}

\begin{popup}
\caption{Check return code}
\public{ADVANCED\_TOOL\_CHECKRETURN}

\ifenglish
Select this checkbox if the tool success or failure may be determined by
examining it return code.
 \else
Установите этот контроль, если удачное и неблагоприятное завершения работы
утилиты могут быть определены с помощью ее возвратного кода.
 \fi
\end{popup}

\begin{popup}
\caption{Max good value}
\public{ADVANCED\_TOOL\_MAXGOOD}

\ifenglish
Type here the maximum value of the return code which indicates successful
completion of a tool. Usually success is indicated by zero return code;
in this case you should type zero here. Some crazy tools exist, however,
that indicate "Success" by 0, "Success with warnings" by 1 and "Failure"
by bigger numbers.

 \else
Введите здесь максимальное значение возвратного кода, которое сигнализирует об 
удачном завершении работы утилиты. Обычно об успешном завершении говорит
нулевой возвратный код, в этом случае здесь нужно ввести ноль. Однако, существуют 
некоторые ненормальные утилиты, которые сигнализируют об "Успехе" нулем, об
"Успехе с предупреждениями" единицей, и об "Неудаче" -- большими числами.
 \fi
\end{popup}

\begin{popup}
\caption{Edit filter}
\public{ADVANCED\_TOOL\_EDITFILTER}

\ifenglish
Edit message filter that is assigned to current tool.
 \else
Редактировать фильтр сообщений, который приписан к текущей утилите.
 \fi
\end{popup}

\begin{popup}
\caption{New filter}
\public{ADVANCED\_TOOL\_NEWFILTER}

\ifenglish
Create new message filter and assign it to current tool.
 \else
Создать новый фильтр сообщений и приписать его к текущей утилите.
 \fi
\end{popup}

\begin{popup}
\caption{OK}
\public{ADVANCED\_TOOL\_OK}

\ifenglish
Select the {\bf OK} pushbutton to confirm changes you've made to the
advanced tool configuration options.
 \else
Нажмите кнопку {\bf OK}, чтобы подтвердить изменения, которые вы произвели 
в продвинутых опциях настройки утилиты. 
 \fi
\end{popup}

\begin{popup}
\caption{Cancel}
\public{ADVANCED\_TOOL\_CANCEL}

\ifenglish
Select the {\bf Cancel} pushbutton to dismiss the dialog without
applying any changes you've made.
 \else
Нажмите кнопку {\bf Cancel}, чтобы закрыть диалог, не применяя произведенные
вами изменения.
 \fi
\end{popup}

