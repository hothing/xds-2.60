%-----------------------------------------------------------------------
\ifenglish
\section{Open File dialog}
 \else
\section{диалог Open File}
\fi
\public{OPENFILE\_DIALOG}
\nominitoc

\ifenglish
The {\bf Open File} dialog box allows you to load an existent file
into a new window.

Why special dialog not the standard File Open dialog everyone is used to?

The standard dialog requires the user to specify both file name and its position
(directory). XDS has powerful instrument of finding files, called {\bf Redirection}.
With the Redirection you only need to type file name. File will be searched for
in directories specified by Redirection.
 \else
Диалог {\bf Open File}  позволяет загрузить в окно существующий файл.

Почему не использован стандартный диалог File Open?

Стандартный диалог требует от пользователя указания и имени файла, и его 
местоположения (директории). XDS обладает мощным инструментом поиска файлов 
называемым {\bf пути поиска}. Вам нужно лишь набрать имя файла. Файл будет отыскан
в директориях указанных в путях поиска.
\fi
\ifenglish
In {\bf File Open} dialog you may either:
 \else
В диалоге {\bf File Open} вы также можете:
\fi
\begin{itemize}
\item 
      \ifenglish
      Type in the edit box a file name {\em without}
      path and select the {\bf OK}
      pushbutton to search for a file using the current redirection
       \else
   Набрать в строке редактирования имя файла без пути и нажать
   кнопку {\bf OK}, чтобы начать поиск файла, используя текущий файл 
   путей поиска
    \fi
\item 
      \ifenglish
      Select the {\bf Browse} pushbutton to load a file via a
      standard file dialog
       \else
 Нажмите кнопку {\bf Browse}, чтобы загрузить файл с помощью стандартного 
 файлового диалога.
 \fi
\item 
      \ifenglish
      Select the {\bf File history} pushbutton to pick a file
      from the list of the most recently loaded files
       \else
  Нажмите кнпоку {\bf File history}, чтобы выбрать файл из списка 
  наиболее часто загружаемах файлов.
  \fi
\end{itemize}

%-----------------------------------------------------------------------

\begin{popup}

\ifenglish
\caption{Type a File Name}
 \else
\caption{Введите имя файла}
\fi
\public{OPENFILE\_COMBOBOX}

\ifenglish
Type here the name of a file you want to open and select
the {\bf OK} pushbutton. You may also select the down arrow button
next to this field to pick from a drop-down list one of the strings
you typed here before.

If you type a name without a path (either absolute or relative),
the IDE will search for a file using the current redirection.

If you type a name containing wildcard characters ({\bf *} and
{\bf ?}), a standard file dialog will be displayed with that name
used as a filter.

 \else
Наберите здесь имя файла, который вы хотите открыть и нажмите кнопку {\bf OK}.
Вы также можете нажать кнопку с направленной вниз стрелкой рядом с этим
полем, чтобы выбрать из ниспадающего списка одну из введенных ранее строк.

Если вы введете имя файла без пути (абсолютного или относительного),
IDE будет искать файл пользуясь текущим файлом путей поиска.

Если вы наберете имя, содержащее шаблонные символы {\bf *} и
{\bf ?}), будет выведен стандартный файловый диалог с фильтром в виде 
этого имени.
\fi
\end{popup}

\begin{popup}
\caption{Browse}
\public{OPENFILE\_BROWSE}

\ifenglish
Select this pushbutton to display a standard file open
dialog.
 \else
Нажмите эту кнопку, чтобы вызвать стандартный диалог открытия файла.
\fi
\end{popup}

\begin{popup}
\caption{File History}
\public{OPENFILE\_FILELIST}

\ifenglish
Select the {\bf File History} pushbutton to pick a file from
the list of the most recently loaded files.

The system keeps track of 100 most recently opened files.
First 5 of them are available in a {\bf File} menu.
Others are available through {\bf More Files} dialog
that is displayed after pressing {\bf File History} button here, or by
{\bf File:More Files} menu.

 \else
Нажмите кнопку  {\bf File History}, чтобы выбрать файл из списка 
наиболее часто загружаемых файлов

Система содержит сведения о 100 наиболее часто открываемых файлах.
Первые 5 из них доступны в меню {\bf File}.
Остальные доступны в диалоге {\bf More Files},
который появляется после нажатия кнопки {\bf File History} или в меню
{\bf File:More Files}.
\fi

\end{popup}

\begin{popup}
\caption{OK}
\public{OPENFILE\_OK}

\ifenglish
Select the {\bf OK} pushbutton to open the file with the name you have typed.
Standard file open dialog appears if the name is absent or contains wildcard
characters  ({\bf *} and {\bf ?}).
 \else
Нажмите кнопку {\bf OK}, чтобы открыть файл с введенным вами именем. 
В случае, если имя отсутствует или содержит шаблонные символы ({\bf *} и {\bf ?}), 
появится стандартный диалог открытия файла.
\fi
\end{popup}

\begin{popup}
\caption{Cancel}
\public{OPENFILE\_CANCEL}

\ifenglish
Select the {\bf Cancel} pushbutton if you decided not to open a file.
 \else
Нажмите кнопку {\bf Cancel}, если решили не открывать файл.
\fi
\end{popup}
%-----------------------------------------------------------------------

\ifenglish
\section{More files dialog}
 \else
\section{диалог More files}
\fi
\public{MOREFILES\_DIALOG}
\nominitoc

\ifenglish
This dialog contains a list of the most recently opened files, allowing
to quickly re-open one of them. Maximum 100 files are remembered. First 5 of them
are available in {\bf File} menu.

Files are sorted in reverse order of last opening time.

 \else
Диалог содержит список недавно открытых файлов, позволяя быстро открыть один
из них снова. Запоминается самое большее 100 файлов. Первые 5 из них
доступны в меню.

Файлы отсортированы в обратном по времени последнего открытия порядке.
\fi

\begin{popup}
\ifenglish
\caption{File list}
 \else
\caption{Список файлов}
\fi
\public{MOREFILES\_LBOX}

\ifenglish
This list box contains a list of the most recently opened files.
You may double-click on a file to re-open it and leave the dialog.
 \else
Здесь содержится список недавно открытых файлов .
С помощью двойного щелчка вы можете открыть один из них и выйти 
из диалога.
\fi
\end{popup}

\begin{popup}
\caption{OK}
\public{MOREFILES\_OK}

\ifenglish
Select the {\bf OK} pushbutton to open the currently selected file.
 \else
Нажмите кнопку {\bf OK}, чтобы открыть выбранный в данный момент файл.
\fi
\end{popup}

\begin{popup}
\caption{Cancel}
\public{MOREFILES\_CANCEL}

\ifenglish
Select the {\bf Cancel} pushbutton if you decided not to open a file.
 \else
Нажмите кнопку {\bf Cancel}, если решили не открывать файл.
\fi
\end{popup}
