
\ifenglish
\section{New Style Sheet dialog}
\else
\section{диалог New Style Sheet}
\fi
\public{SAVESTYLE\_DIALOG}

\ifenglish
To save the current project as a style sheet or create new style sheet:
\else
Чтобы сохранить текущий проект как таблицу стилей, или создать новый стиль:
\fi
\begin{enumerate}
\item \ifenglish
      Type a name for the style sheet in the corresponding entry field
      or select it from the list below to overwrite one of the previously
      defined style sheets
      \else
      Введите имя стиля в соответствующем поле ввода или выберите
      его из расположенного ниже списка, чтобы переписать один из ранее
      определенных стилей
      \fi
\item \ifenglish
       Type a style sheet file name or select the {\bf Browse} button
      to bring up a standard file dialog.
      \else
      Введите имя файла стиля или нажмите кнопку {\bf Browse},
      чтобы вызвать стандартный файловый диалог.
      \fi
\item \ifenglish
      Select the {\bf OK} pushbutton
      \else
      Нажмите кнопку {\bf OK}  
      \fi
\end{enumerate}

%---------------------------------------------------------------------

\begin{popup}
\ifenglish
\caption{Type a name for a style sheet}
\else
\caption{Введите имя нового стиля}
\fi
\public{SAVESTYLE\_NAMES}

\ifenglish
Type in this field a name for a new style sheet. This name will be displayed
in a style sheet list. The name you type is automatically copied to
the {\bf File name} field.
\else
Наберите в этом поле имя нового стиля. Это имя будет отображено в 
списке стилей. Введенное вами имя будет автоматически скопировано
в поле {\bf File name}.
\fi
\end{popup}

\begin{popup}
\ifenglish
\caption{Type a file name}
\else
\caption{Введите имя файла}
\fi
\public{SAVESTYLE\_FILE}

\ifenglish
Type a name for a file, in which you want the style sheet to be stored,
in the entry field, or select the {\bf Browse} pushbutton to choose a name
using a standard file dialog.
\else
Наберите имя файла, где вы хотите сохранить стиль, в поле ввода,
или нажмите кнопку {\bf Browse}, чтобы выбрать имя, используя стандарный файловый 
диалог.
\fi
\end{popup}

\begin{popup}
\caption{Browse}
\public{SAVESTYLE\_BROWSE}

\ifenglish
Select name for a file in which you want the style sheet to be stored
using a standard file dialog.

\else
Выберите имя файла, где вы хотите сохранить стиль, используя стандартный 
файловый диалог.
\fi
\end{popup}

\begin{popup}
\caption{OK}
\public{SAVESTYLE\_OK}

\ifenglish
Select the {\bf OK} pushbutton to save a style sheet.
\else
Нажмите кнопку {\bf OK}, чтобы сохранить таблицу стилей.
\fi
\end{popup}

\begin{popup}
\caption{Cancel}
\public{SAVESTYLE\_CANCEL}

\ifenglish
Select the {\bf Cancel} pushbutton if you decided not to make new style sheet.
\else
Нажмите кнопку {\bf Cancel}, если решили не создавать новый стиль.
\fi
\end{popup}

%=====================================================================

\ifenglish
\section{Apply style sheet dialog}
\else
\section{диалог Apply style sheet}
\fi
\public{STYLE\_DIALOG}

\ifenglish
To {\em apply} a style sheet to the current project, select it in the
listbox and select the {\bf OK} pushbutton.

The applying style sheet statements will be added to the current
project or will replace the matching statements, if any.
\else
Для добавления стиля в текущий проект, выберите его в списке 
и нажмите кнопку {\bf OK}.

Установки применяемого стиля будут добавлены в текущий проект,
или заменят уже существующие, если такие есть.
\fi
%---------------------------------------------------------------------

\begin{popup}
\caption{Select style sheet}
\public{STYLE\_LISTBOX}

\ifenglish
This listbox contains names of all style sheets known to the Environment.
Select the style sheet and press {\bf OK} button. Double-clicking in this list
also works.
\else
Этот список содержит имена всех стилей, известных Среде.
Выберите таблицу стилей и нажмите {\bf OK}. Можно также дважды щелкнуть по 
списку.
\fi
\end{popup}

\begin{popup}
\caption{OK}
\public{STYLE\_OK}

\ifenglish
Select the {\bf OK} pushbutton to apply the selected style sheet to
the current project.
\else
Нажмите кнопку {\bf OK}, чтобы применить выбранную таблицу стилей к текущему
проекту.
\fi
\end{popup}

\begin{popup}
\caption{Cancel}
\public{STYLE\_CANCEL}

\ifenglish
Select the {\bf Cancel} pushbutton if you decided not to apply a style
sheet.
\else
Нажмите кнопку {\bf Cancel}, если решили не применять таблицу стилей.
\fi
\end{popup}
%=====================================================================

\ifenglish
\section{Delete style sheet dialog}
\else
\section{диалог Delete style sheet}
\fi
\public{DELETESTYLE\_DIALOG}

\ifenglish
To delete a style sheet, select it in the
listbox and select the {\bf OK} pushbutton.

The style sheet file will also be deleted.
\else
Чтобы удалить таблицу стилей, выберите ее в списке 
и нажмите кнопку {\bf OK}.
\fi
%---------------------------------------------------------------------

\begin{popup}
\caption{OK}
\public{DELETESTYLE\_OK}

\ifenglish
Select the {\bf OK} pushbutton to delete the selected style sheet from
the current project.
\else
Нажмите кнопку {\bf OK}, чтобы удалить выбранный стиль из
текущего проекта.
\fi
\end{popup}

\begin{popup}
\caption{Cancel}
\public{DELETESTYLE\_CANCEL}

\ifenglish
Select the {\bf Cancel} pushbutton if you decided not to delete a style
sheet.
\else
Нажмите кнопку {\bf Cancel}, если решили не удалять таблицу стилей.
\fi
\end{popup}
%=====================================================================

\ifenglish
\section{Edit style sheet dialog}
\else
\section{диалог Edit style sheet}
\fi
\public{EDITSTYLE\_DIALOG}

\ifenglish
To edit a style sheet, select it in the
listbox and select the {\bf OK} pushbutton.

Style sheets are edited in the same manner as projects are. The visual options
editor will appear.
\else
Для редактирования стиля, выберите его в списке
и нажмите кнопку {\bf OK}.

Стили редактируются таким же образом, как и проекты. Появится редактор
визуальных опций.
\fi
%---------------------------------------------------------------------

\begin{popup}
\caption{OK}
\public{EDITSTYLE\_OK}

\ifenglish
Select the {\bf OK} pushbutton to edit the selected style sheet.
\else
Нажмите кнопку {\bf OK}, чтобы редактировать выбранную таблицу стилей.
\fi
\end{popup}

\begin{popup}
\caption{Cancel}
\public{EDITSTYLE\_CANCEL}

\ifenglish
Select the {\bf Cancel} pushbutton if you decided not to edit a style
sheet.
\else
Нажмите кнопку {\bf Cancel}, если решили не редактировать стиль.
\fi
\end{popup}
