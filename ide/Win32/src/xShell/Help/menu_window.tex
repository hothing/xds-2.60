\section{Window Menu}
\public{M\_WINDOWS}
\index{Window Menu}
\nominitoc

\ifenglish
The {\bf Window} menu contains commands that you use to arrange
open IDE windows. It also contains a list of all open windows,
allowing you to switch to a specific window.
The following commands appear in the {\bf Window} menu:
\else
Меню {\bf Window} содержит команды, которые вы используете для выстраивания
открытых окон IDE. Оно также содержит список всех открытых окон, позволяя
переключаться на одно из них.
Меню {\bf Window} содержит следующие команды:
\fi
\begin{description}
\item \ref[Cascade]{IDM\_CASCADE}
\ifenglish
  - Makes an overlapping windows layout
  \else
  - Создает систему перекрывающихся окон
  \fi
\item \ref[Tile Vertically]{IDM\_TILEVERT}
\ifenglish
  - Makes a non-overlapping layout of "vertical" windows
  \else
  - Создает систему окон, неперекрывающихся по-вертикали
  \fi
\item \ref[Tile Horizontally]{IDM\_TILEHORIZ}
\ifenglish
  - Makes a non-overlapping layout of "horizontal" windows
  \else
  - Создает систему окон, неперекрывающихся по-горизонтали
  \fi
\item \ref[Arrange Icons]{IDM\_ARRANGE}
\ifenglish
  - Arranges icons of minimized windows into rows
  \else
  - Перестраивает иконки минимизированных окон в ряды
  \fi
\item \ref[Close All Windows]{IDM\_CLOSEALL}
\ifenglish
  - Close all opened windows
  \else
  - Закрывает все открытые окна
  \fi
\item \ref[Free]{IDM\_FREE}
\ifenglish
  - Make current window free
  \else
  - Делает текущее окно свободным
  \fi
\item \ref[Dock]{IDM\_DOCK}
\ifenglish
  - Dock current window
  \else
  - Осуществляет привязку текущего окна
  \fi
\item \ref[Undock]{IDM\_UNDOCK}
\ifenglish
  - Undock previously docked window
  \else
  - Отменяет привязку окна
 \fi
\item \ref[Windows List]{IDM\_MOREWINDOWS}
\ifenglish
  - Display full list of all MDI windows
  \else
  - Отображает полный список всех окон MDI 
 \fi
\end{description}

\subsection{Window: Cascade}
\public{IDM\_CASCADE}

\ifenglish
Use the {\bf Cascade} command to arrange all open windows
in an overlapping manner, making their titles visible.
\else
Используйте команду {\bf Cascade} чтобы расположить все открытые окна 
каскадом, оставляя видимым их заголовок.
\fi
\subsection{Windows: Tile Vertically}
\public{IDM\_TILEVERT}

\ifenglish
Use the {\bf Tile Vertically} command to arrange all open windows
in a non-overlapping manner, so that the whole windows are
visible, when the number of lines visible at the same time 
is more important than the number of columns.
\else
Используйте команду {\bf Tile Vertically} чтобы перестроить все открытые
окна так, чтобы, по возможности, они были видимы полностью, или чтобы количество
видимых строк преобладало над количеством видимых столбцов.
\fi
\subsection{Window: Tile Horizontally}
\public{IDM\_TILEHORIZ}

\ifenglish
Use the {\bf Tile Horizontally} command to arrange all open windows
in a non-overlapping manner, so that the whole windows are
visible, when the number of columns visible at the same time 
is more important than the number of lines.
\else
Используйте команду {\bf Tile Vertically} чтобы перестроить все открытые
окна так, чтобы по возможности они были видимы полностью, или чтобы количество
видимых столбцов преобладало над количеством видимых строк.
\fi
\subsection{Window: Arrange Icons}
\public{IDM\_ARRANGE}

\ifenglish
Use the {\bf Arrange Icons} command to arrange icons of minimized 
windows into rows.
\else
Используйте команду {\bf Arrange Icons}, чтобы перестроить иконки 
минимизированных окон в ряды.
\fi
\subsection{Window: Close All Windows}
\public{IDM\_CLOSEALL}

\ifenglish
Use the {\bf Close All Windows} command to close all opened windows.
If some of them contain changed files you will be asked to save the files
on disk.
\else
Используйте команду {\bf Close All Windows}, чтобы закрыть все открытые окна.
Если одно из них содержит измененный файл, вас запросят, хотите ли вы сохранить
изменения на диск.
\fi
\subsection{Window: Free}
\public{IDM\_FREE}

\ifenglish
In classic {\bf MDI} interface all the windows are children of main (frame)
windows. In MDI version used in XDS IDE you can {\bf Free} window. After that
the window is not clipped by IDE frame window but exists on computer desktop
as an application window. It can be switched to or from using {\bf Alt+Tab} keys
and is accessible using Windows 95 TaskBar. It does not disappear from XDS IDE
WinBar. To return the window to the child state use {\bf Return to Frame} command
of its System menu.
\else
В классическом интерфейсе MDI все окна являются дочерними по отношению к 
главному окну. В версии MDI, использованной в XDS IDE, вы можете {\bf освободить}
окно. После этого окно не принадлежит каркасному окну оболочки а существует 
на рабочем столе, как окно приложения. К нему можно обращаться с помощью 
комбинации клавиш {\bf Alt+Tab}. Чтобы вернуть окно в дочернее состояние,
используйте команду {\bf Return to Frame} Системного меню.	
\fi
\subsection{Window: Dock}
\public{IDM\_DOCK}

\ifenglish
This command {\bf Docks} current window. Docked windows are arranged along the
sides of frame window without overlapping. All non-docked windows occupy area
free of docked windows. They are clipped by this area and therefore never overlap
docked windows. Maximized non-docked window occupy all this area.

Docked windows are useful for information you want displayed permanently such
as project files list and messages window.

Once the window is docked, you can undock it using \ref[Undock]{IDM\_UNDOCK}
command of \ref[Window]{M\_WINDOWS} menu.
\else
Эта команда делает текущее окно {\bf неперекрывающимся}. Неперекрывающиеся окна
выстраиваются вдоль сторон окна оболочки, не закрывая друг друга. Все остальные
окна занимают свободное от неперекрывающихся окон пространство. Они ограничены
этим пространством, и, таким образом, никогда не оказываются поверх 
неперекрывающихся окон. Если максимизировать такое окно, то оно займет все это 
пространство. 

Как только окно сделано неперекрывающимся, вы можете отменить привязку,
используя команду \ref[Undock]{IDM\_UNDOCK} меню \ref[Window]{M\_WINDOWS}.
\fi
\subsection{Window: Undock}
\public{IDM\_UNDOCK}

\ifenglish
This command returns previously \ref[docked]{IDM\_DOCK} window to normal MDI
child state.
\else
Эта команда возвращает сделанное некогда \ref[неперекрывающимся]{IDM\_DOCK} окно
в нормальное дочернее MDI- состояние.
\fi
\subsection{Window: Windows List}
\public{IDM\_MOREWINDOWS}

\ifenglish
In classic MDI full windows list can be displayed only if there is no room in Window
menu and is called {\bf More Windows}. In XDS version you can display it any time
using {\bf Windows List} menu.

You can dock, undock, save or close the window right from Windows List dialog.
\else
В классическом MDI полный список окон может быть отображен, только если 
в меню Window недостаточно места, и называется он {\bf More Windows}. В варианте
XDS вы можете вывести его в любой момент с помощью меню {\bf Windows List}.

Вы можете сделать окно неперекрывающимся, отменить привязку, сохранить или 
закрыть окно прямо из диалога Windows List.
\fi