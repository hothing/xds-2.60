\ifenglish
\section {Configure Windows System dialog}
 \else
\section {диалог Configure Windows System}
 \fi
\public{CFGWIN\_DIALOG}
\nominitoc

\ifenglish
XDS IDE uses its own version of {\bf MDI} (Multiple Document Interface).
This version provides {\bf docked} windows (windows that are never overlapped by others)
and {\bf free} windows (windows that appear on the desktop rather than inside
IDE frame). It also allows window numbering that is more convenient than
in classic MDI and provides {\bf winbar} to make navigation among your windows
easier.

This dialog allows you to configure various details of windows behavior
and interface elements such as Winbar and Windows menu.

This dialog contains following pages:
 \else
XDS IDE использует свою собственную версию {\bf MDI}. Эта версия дает возможность 
использования неперекрывающихся окон ( которые никогда не закрываются другими) 
и свободных окон (которые появляются скорее на рабочем столе, чем внутри 
IDE оболочки). Она также предоставляет более удобный, чем в классическом MDI 
способ нумерации окон и позволяет легче управляться с окнами.

Этот диалог позволяет настраивать различные детали поведения окон и элементы
интерфейса, такие как Winbar и Windows menu.
\fi 

\ref[Numbering]{CFGWIN\_NUMBERING\_DIALOG} -- 
\ifenglish
configure window numbering scheme
 \else
настройка системы нумерации окон
 \fi

\ref[Windows Menu]{CFGWIN\_MENU\_DIALOG} -- 
\ifenglish
configure Windows Menu appearance
 \else
настройка внешнего вида Windows Menu
  \fi

\ref[Docked Windows]{CFGWIN\_DOCKED\_DIALOG} -- 
\ifenglish
configure which windows are docked by default
 \else
указать, какие окна неперекрывающиеся по умолчанию
 \fi

\ref[WinBar]{CFGWIN\_WINBAR\_DIALOG} -- 
\ifenglish
configure \ref{WinBar}
 \else
настройка \ref{WinBar}
 \fi
%---------------------------------------------------------------------

\ifenglish
\subsection {Numbering tab}
 \else
\subsection {Схема нумерации}
 \fi
\public{CFGWIN\_NUMBERING\_DIALOG}

\ifenglish
Windows in the IDE are numbered, that is, small integral numbers are assigned
to them. The rules of this assignment can be configured. Following options
are available:

 \else
Окна в IDE пронумерованы, то есть, им сопоставлены целые числа. Правила 
нумерации могут настраиваться. Имеются следующие варианты:
\fi
\begin{description}
\item {\bf Classic MDI}
\ifenglish
-- this scheme is identical to standard Microsoft MDI scheme. New window
is appended to the list. When the window is destroyed all the windows
are shifted up in the list. When the window with number greater than 9 is
activated it is swapped with the window with number 9. So active window never
has number greater than 9. A bit strange scheme. It's recommended to use it
together with {\bf Classic "More Windows"} window menu style.
 \else
--Эта система идентична стандартной системе Microsoft MDI. Список пополняется
новым окном. Когда окно уничтожается, все окна поднимаются вверх по списку.
Когда активируется окно с номером большим 9, оно заменяется окном с номером 9.
Таким образом, активные окна не могут иметь номер больше 9. Немного странная
схема. Рекомендуется использовать ее со стилем {\bf Classic "More Windows"}.
\fi

\item {\bf Classic MDI, no window swap}

\ifenglish
-- same as Classic MDI but windows are never swapped. Window with any number
can be active.  It's recommended to use it
together with {\bf Long} or {\bf Multi-column} window menu style.
 \else
-- то же, что и Classic MDI, но окна не меняются местами. Окно с любым номером
может быть активным. Рекомендуется использовать вместе со стилями {\bf Long} или {\bf Multi-column}.
\fi

\item {\bf New window first}

\ifenglish
-- new window appears at the beginning, not at the end of a list. All the other
windows are shifted to the bottom. This style collects most recently loaded
files at the top of menu.
 \else
-- новое окно появляется в начале, а не в конце списка. Все остальные окна
сдвигаются вниз. Этот стиль собирает недавно загруженные файлы в начале меню.
\fi
\item {\bf Active window first}

\ifenglish
-- not only new window appears at the beginning of the list, but activation of any
window also moves it to the top of the list. This style collects most frequently
accessed files at the top of the list.
 \else
-- не только новое окно появляется в начале списка, но и активация любого окна
перемещает его на вершину списка. Этот стиль собирает наиболее часто открываемые
файлы в начале списка.
\fi

\item {\bf Fixed window number}

\ifenglish
-- once the window got a number it retains it during all its existence. Windows are
never swapped or shifted. New windows are created at first free position.
When the window is closed its number becomes free and is visible in the Window
menu as free position. Selecting it causes creation of new edit buffer with this
number.
 \else
-- раз получив номер, окно сохраняет его пожизненно. Окна не передвигаются и не 
меняются местами. Новые окна создаются на первой свободной позиции. Когда окно 
закрывается, его номер становится свободным, и выглядит в Windows menu как
свободная позиция. Ее выбор создает новый буфер редактирования с этим номером.
\fi 

\end{description}

\ifenglish
By default window numbers are drawn at the top-right corner of windows,
next to minimize and maximize buttons. To turn this off, use the
{\bf Display window numbers in captions} checkbox.
 \else
По умолчанию номера окон отображаются в верхнем правом углу, сразу за 
кнопками максимизации и минимизации. Чтобы это отключить, используйте 
контроль {\bf Display window numbers in captions}.
\fi

%---------------------------------------------------------------------

\ifenglish
\subsection {Windows Menu tab}
 \else
\subsection {закладка Windows Menu}
\fi
\public{CFGWIN\_MENU\_DIALOG}

\ifenglish
Following options are available for Windows Menu:
 \else
В Windows menu доступны следующие опции:
\fi
\begin{description}
\item{\bf Classic "More Windows"}

\ifenglish
-- windows menu contains at most nine windows.
If there are more windows opened in the IDE, the "More windows" item
appears at the bottom of Windows menu

 \else
-- оконное меню содержит самое большее 9 окон
Если в IDE открыто больше окон, пункт 
появляется в нижней части меню Windows.
 \fi
\item{\bf Long Menu}

\ifenglish
-- behaves basically as Classic "More Windows" but limit on number of windows
is determined by size of the screen rather than by predefined constant. Note that
only first nine windows are available by keyboard shortcut (key with window number).

 \else
-- ведет себя в основном как и классическое "More Windows", но ограничение на
количество окон определяется скорее размерами экрана, чем заданной константой.
Обратите внимание, что только первые 9 окон доступны с клавиатуры (клавиша с
цифрой, соответствующей номеру окна).
 \fi
\item{\bf Multi-column menu}

\ifenglish
-- when menu items do not fit into screen height they are arranged into several
columns.
 \else
-- когда пункты меню не помещаются в экран по высоте, они перестраиваются в 
несколько колонок.
 \fi
\end{description}

\ifenglish
If "More windows" item is not added to the menu (either because menu mode
 chosen does not provide it or because there're too few windows) you can still
get Windows dialog by selecting "Windows List" item in the Windows menu.
 \else
Если пункт "More windows" не добавлен в меню, (потому что этого не позволил 
режим меню, или из-за того что окон недостаточно) вы все же можете
вызвать диалог Windows путем выбора "Windows List" в меню Windows.
 \fi
%---------------------------------------------------------------------

\ifenglish
\subsection {Docked Windows tab}
 \else
\subsection {закладка Docked Windows}
 \fi
\public{CFGWIN\_DOCKED\_DIALOG}

\ifenglish
When messages window, file list or search result window is opened it occupies
the same position that it has last time it was opened before. When any of
mentioned windows is opened for the first time, it can either be opened as
arbitrary new MDI window, or be opened in {\bf Docked} state, when it is aligned
along one side of a frame window without being overlapped by other windows.

You may select the behavior for each type of windows.

Message window is by default docked to the bottom of a frame window, file list to
the left of it, and search results to the right.

You can also configure appearance of docked window. They can either have normal
window caption with all kinds of buttons and an icon, or have small caption
with only one small button at the top-right corner - the close button.
 \else
Когда открывается окно сообщения, списка файлов или результата поиска, оно 
занимает то же положение, которое оно занимало, будучи открытым последний раз.
Когда любое из упомянутых окон открывается первый раз, оно может быть открыто
как произвольное новое окно MDI, или в неперекрывающемся состоянии, когда оно 
выровнено вдоль стороны окна оболочки, и другие окна его не перекрывают.

Вы можете выбрать поведение любого типа окон.

Окно сообщения по умолчанию привязано к низу окна оболочки слева - список 
файлов и результаты поиска поиска - справа. 
                                          
Можно также настраивать внешний вид неперекрывающихся окон. Они могут иметь 
как обычный заголовок со всевозможными кнопками и иконкой, так и  
маленький заголовок лишь с одной небольшой кнопкой в верхнем правом углу 
- кнопкой закрытия.
\fi
%---------------------------------------------------------------------

\ifenglish
\subsection {WinBar tab}
 \else
\subsection {закладка WinBar}
 \fi
\public{CFGWIN\_WINBAR\_DIALOG}

\ifenglish
This page allows configuration of \ref{WinBar}.

You may place it at the top or at the bottom of the frame window, or you
can remove it at all.

You may also configure what information appears in the WinBar buttons. Each
button may contain an icon, a window number and window title. Each of these
components may be turned on and off. Note that turning all of them off makes
the WinBar very close to useless.
 \else
Эта страница позволяет настраивать \ref{WinBar}.

Вы можете поместить его сверху или снизу на окне оболочки или вовсе убрать.

Вы также можете менять появляющуюся на кнопках WinBar информацию. Каждая кнопка 
может содержать иконку, номер окна или заголовок окна. Каждый из этих компонент
может быть включен и выключен. Заметьте, что их полное отключение делает
WInBar почти бесполезным.
\fi
%---------------------------------------------------------------------

\begin{popup}
\ifenglish
\caption{Classic MDI}
 \else
\caption{Классический MDI}
 \fi
\public{CFGWIN\_NUMBERING\_MDI}

\ifenglish
New window is appended to the list.
When the window is destroyed all the windows are shifted up in the list.
When the window with number greater than 9 is
activated it is swapped with the window with number 9.
 \else
К списку добавляется новое окно.
Когда окно уничтожается, все окна поднимаются вверх по списку.
Когда активируется окно с номером большим 9, оно замещается окном с номером 9.
\fi
\end{popup}

\begin{popup}

\ifenglish
\caption{Classic MDI, no window swap}
 \else
\caption{Классический MDI, без передвижения окон в списке}
\fi
\public{CFGWIN\_NUMBERING\_MDINOFLIP}

\ifenglish
New window is appended to the list.
When the window is destroyed all the windows are shifted up in the list.
Window numbers are never swapped. 
Window with any number can be active.
 \else
К списку добавляется новое окно.
Когда окно уничтожается, все окна поднимаются вверх по списку.
Окна не обмениваются номерами.
Окно с любым номером может быть активным.
\fi
\end{popup}

\begin{popup}

\ifenglish
\caption{New window first}
 \else
\caption{Новое окно -- первое}
\fi
\public{CFGWIN\_NUMBERING\_NEWFIRST}

\ifenglish
New window appears at the beginning, not at the end of a list. All the other
windows are shifted to the bottom. This style collects most recently loaded
files at the top of menu.
 \else
Новое окно появляется в начале, а не в конце списка. Все остальные окна 
сдвигаются к низу. Этот стиль собирает недавно открытые файлы в начале меню.
\fi
\end{popup}

\begin{popup}

\ifenglish
\caption{Active window first}
 \else
\caption{Активное окно -- первое}
\fi
\public{CFGWIN\_NUMBERING\_ACTIVEFIRST}

\ifenglish
Activation of any window moves it to the top of the list.
This style collects most frequently accessed files at the top of the list.
 \else
Активация любого окна перемещает его в начало списка.
Этот стиль собирает наиболее часто открываемые файлы в начале списка.
\fi
\end{popup}

\begin{popup}
\ifenglish
\caption{Fixed window numbers}
 \else
\caption{Фиксированные номера окон}
\fi
\public{CFGWIN\_NUMBERING\_FIXED}

\ifenglish
Once the window got a number it retains it during all its existence. Windows are
never swapped or shifted. New windows are created at first free position.

 \else
-- раз получив номер, окно сохраняет его пожизненно. Окна не передвигаются и не 
меняются местами. Новые окна создаются на первой свободной позиции. 
\fi

\end{popup}

\begin{popup}

\ifenglish
\caption{Display window numbers in captions}
 \else
\caption{Отображать номера окон в заголовках}
\fi
\public{CFGWIN\_NUMBERING\_SHOW}

\ifenglish
By default window numbers are drawn at the top-right corner of windows,
next to minimize and maximize buttons.
This checkbox turns this option on and off.
 \else
По умолчанию номера окон отображаются в верхнем правом углу сразу за 
кнопками максимизации и минимизации. Контроль включает и выключает эту опцию. 
\fi 
\end{popup}

\begin{popup}

\ifenglish
\caption{Classic "More windows"}
 \else
\caption{Классическое "More windows"}
\fi
\public{CFGWIN\_MENU\_MOREWINDOWS}

\ifenglish
Window menu contains at most nine windows.
If there are more windows opened in the IDE, the "More windows" item
appears at the bottom of Windows menu.
 \else
Меню Window содержит не более 9 окон.
Если в IDE есть другие открытые окна, в меню Window появляется пункт 
"More Windows".
\fi
\end{popup}

\begin{popup}
\ifenglish
\caption{Long menu}
 \else
\caption{Длинное меню}
\fi
\public{CFGWIN\_MENU\_LONG}

\ifenglish
Window menu contains as many items as the screen height allows.
If there are more windows opened in the IDE, the "More windows" item
appears at the bottom of Windows menu.
 \else
Оконное меню содержит столько пунктов, сколько позволяет высота экрана.
Если в IDE есть другие открытые окна, в нижней части меню Window появляется
пункт "More Windows".
\fi
\end{popup}

\begin{popup}

\ifenglish
\caption{Multi-column menu}
 \else
\caption{Многоколоночное меню}
\fi
\public{CFGWIN\_MENU\_MULTICOLUMN}

\ifenglish
There is no limit on number of items in Window menu.
When menu items do not fit into screen height they are arranged into several
columns.
 \else
Ограничения на количество пунктов меню Window нет.
Когда пункты меню не помещаются в экран по высоте, они перестраиваются в 
несколько колонок.
\fi
\end{popup}

\begin{popup}
\caption{Messages window}
\public{CFGWIN\_DOCKED\_MESSAGES}

\ifenglish
If this option is turned on, the first time message window is displayed,
it appears docked to the bottom edge of frame window.
 \else
Если эта опция включена, то появившись в первый раз, окно сообщений окажется
привязанным к нижнему краю окна оболочки.
\fi
\end{popup}

\begin{popup}
\caption{File list}
\public{CFGWIN\_DOCKED\_FILELIST}

\ifenglish
If this option is turned on, the first time file list window is displayed,
it appears docked to the left side of frame window.
 \else
Если эта опция включена, то, появившись в первый раз, окно списка файлов 
оказется привязанным к левой стороне окна оболочки. 
\fi
\end{popup}

\begin{popup}
\caption{Search results}
\public{CFGWIN\_DOCKED\_SEARCHRES}

\ifenglish
If this option is turned on, the first time search result window is displayed,
it appears docked to the right side of frame window.
 \else
Если эта опция включена, то появившись в первый раз, окно результата поиска
окажется привязанным к правой стороне окна оболочки.
\fi
\end{popup}

\begin{popup}
\caption{Small captions in docked windows}
\public{CFGWIN\_SMALL\_DOCKED}
\ifenglish
When this option is turned on docked windows have narrow caption with no icon and
only one small button at the right (close button). Turning it off makes docked
windows captions look like normal windows captions.
 \else
Когда эта опция включена, неперекрывающиеся окна имеют узкий заголовок без иконки и 
единственную маленькую кнопку справа (кнопка закрытия). Ее выключение заставляет
заголовок неперекрывающегося окна выглядеть как заголовок обычного окна.
\fi
\end{popup}

\begin{popup}
\caption{At the Top}
\public{CFGWIN\_WINBAR\_TOP}
\ifenglish
Places WinBar at the top of frame window, right under Toolbar.
 \else
Помещает WinBar сверху на окне оболочки под Toolbar.
\fi
\end{popup}

\begin{popup}
\caption{At the Bottom}
\public{CFGWIN\_WINBAR\_BOTTOM}

\ifenglish
Places WinBar at the bottom of frame window, just above the Status Bar.
 \else
Помещает Winbar снизу на окне оболочки над StatusBar.
\fi
\end{popup}

\begin{popup}
\caption{None}
\public{CFGWIN\_WINBAR\_NONE}

\ifenglish
Removes the WinBar.
 \else
Удаляет WinBar.
\fi
\end{popup}

\begin{popup}
\caption{Window icon}
\public{CFGWIN\_WINBAR\_ICON}

\ifenglish
When turned on, WinBar buttons contain small icons (same as corresponding windows).
 \else
Если включено, то кнопки Winbar содержат маленькие иконки (те же, что и 
соответствующие окна).
\fi
\end{popup}

\begin{popup}
\caption{Window number}
\public{CFGWIN\_WINBAR\_NUMBER}

\ifenglish
When turned on, WinBar buttons contain window numbers.
 \else
Если включено, то кнопки Winbar содержат номера окон.
\fi
\end{popup}

\begin{popup}
\caption{Short window title}
\public{CFGWIN\_WINBAR\_TEXT}

\ifenglish
When turned on, WinBar buttons short titles of windows (for edit windows short
title is the file name without a path).
 \else
Если включено, то кнопки Winbar содержат краткие заголовки окон (для окон
редактирования краткое название - имя файла без пути).
\fi
\end{popup}
