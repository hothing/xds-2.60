\section{Debug Menu}
\public{M\_DEBUG}
\index{Debug Menu}
\nominitoc

\ifenglish
The {\bf Debug} menu contains commands that you use to run and debug
your program.
The following commands appear in the {\bf Debug} menu:
\else
Меню {\bf Debug} содержит команды, которые вы используете, чтобы запускать
и отлаживать вашу программу.
\fi
\begin{description}
\item \ref[Run]{IDM\_RUN}
 \ifenglish
  - Makes and runs your program
  \else
  - Компонует и запускает вашу программу
  \fi
\item \ref[Stop]{IDM\_STOP}
  \ifenglish
  - Stops a running program
  \else
  - Останавливает запущенную программу
  \fi
\item \ref[Run Debugger]{IDM\_DEBUG}
 \ifenglish
  - Invokes an external debugger with your program
 \else
  - Вызывает внешний отладчик для вашей программы
 \fi
\item \ref[View Program Output]{IDM\_VIEWUSERS}
 \ifenglish
  - Switch to the main window of running or finished program
 \else
  - Переключается в основное окно работающей или завершившейся программы
 \fi
\item \ref[Options]{IDM\_RUNOPTIONS}
  \ifenglish
  - Lets you specify various run options
  \else
  - Позволяет задавать различные опции запуска
  \fi
\end{description}

\subsection{Debug: Run}
\public{IDM\_RUN}

\ifenglish
Use the {\bf Run} command to make current project and execute your program.
Program is not run if project make produces errors.

You can specify program name, its command line and starting directory
in the dialog box displayed via \ref[Options]{IDM\_RUNOPTIONS} menu command. 

Keyboard shortcut: {\bf Ctrl+F9}
\else
Используйте команду {\bf Run}, чтобы собрать текущий проект и выполнить вашу 
программу. Программа не запускается, если компоновка проекта обнаружила ошибки.

Вы можете определить имя программы, ее командную строку и начальную директорию
в диалоге, отображаемом командой меню \ref[Options]{IDM\_RUNOPTIONS}.
                             
Комбинация клавиш: {\bf Ctrl+F9}.
\fi
\subsection{Debug: Stop}
\public{IDM\_STOP}

\ifenglish
Use the {\bf Stop} command to stop your running program. The program is stopped
immediately without any chance to prevent termination or execute any finalizing
code.
\else
Используйте команду {\bf Stop}, чтобы остановить запущенную программу. Программа
останавливается немедленно без возможности предотвратить остановку или выполнить
завершающую часть кода.
\fi
\subsection{Debug: Run Debugger}
\public{IDM\_DEBUG}

\ifenglish
Use the {\bf Run Debugger} command to invoke an external debugger,
loading your program into it. By default this command invokes XDS debugger;
you can assign any other debugger to this command. For example, in XDS-C
it might be reasonable to use debugger that comes with the C/C++ compiler used.
\else
Используйте команду {\bf Run Debugger}, чтобы вызвать внешний отладчик, загружая
в него вашу программу. По умолчанию эта команда вызывает XDS отладчик. 
Вы можете привязать к этой команде друной отладчик. Например, в XDS-C можно
использовать отладчик, поставлемый с используемым компилятором C/C++.
\fi
\subsection{Debug: View Program Output}
\public{IDM\_VIEWUSERS}

\ifenglish
Use the {\bf View program output} command to display
your program output window.

Keyboard shortcut: {\bf F5}
\else
Используйте команду {\bf View program output}, чтобы отобразить окно вывода 
вашей программы.

Комбинация клавиш: {\bf F5}
\fi
\subsection{Debug: Options}
\public{IDM\_RUNOPTIONS}

\ifenglish
Use the {\bf Options} command to set various run options
via a dialog box.
\else
Используйте команду {\bf Options}, чтобы установить различные опции запуска
с помощью окна диалога.
\fi