\section{Project Menu}
\public{M\_PROJECT}
\nominitoc

\ifenglish
The {\bf Project} menu contains commands that you use to create,
open, close, and edit a project file. It also contains a list 
of the recently opened project files to let you quickly switch
to another project. The following commands appear
in the {\bf Project} menu:
\else
Меню {\bf Project} содержит команды, которые вы используете, чтобы создавать,
открывать, закрывать и редактировать файлы проекта. Оно также содержит список
недавно открывавшихся файлов, позволяя бысто переключаться на другой проект.
В меню {\bf Project} содержатся следующие команды:
\fi
\begin{description}
\item \ref[New]{IDM\_PROJNEW}
\ifenglish
  - Creates a new project file
  \else
  - Создает новый файл проекта
 \fi
\item \ref[Open]{IDM\_PROJOPEN}
\ifenglish
  - Opens an existent project file
  \else
  - Открывает существующий файл проекта
  \fi
\item \ref[Close]{IDM\_PROJCLOSE}
\ifenglish
  - Closes the current project file
  \else
  - Закрывает текущий файл проекта
  \fi
\item \ref[Modify]{IDM\_PROJOPTIONS}
\ifenglish
  - Lets you modify the current project in a visual project editor
  \else
  - Позволяет изменять текущий проект с помощью средств графического 
    интерфейса
  \fi
\item \ref[Edit]{IDM\_PROJEDIT}
\ifenglish
  - Lets you edit the current project file manually
  \else
  - Позволяет вручную редактировать текущий файл проекта
  \fi
\item \ref[File list]{IDM\_FILELIST}
\ifenglish
  - Displays a list of the current project source files
  \else
  - Отображает список исходных файлов текущего проекта  
  \fi
\item \ref[Style sheets]{M\_STYLES}
\ifenglish
  - Activates the {\bf Style sheets} submenu
  \else
  - Активирует подменю {\bf Style sheets}
  \fi
\end{description}

\subsection{Project: New}
\public{IDM\_PROJNEW}

\ifenglish
Use the {\bf New} command to create a new project. A dialog
box will appear, letting you specify a project file name,
location, template, main module and some options.
\else
Используйте команду {\bf New}, чтобы создавать новый проект. Будет
выведен диалог, позволяющий указать имя файла проекта, его место на диске,
шаблон, главный модуль и некоторые опции.
\fi
\subsection{Project: Open}
\public{IDM\_PROJOPEN}

\ifenglish
Use the {\bf Open} command to open an existent project file
by means of a standard file dialog.

The IDE keeps track of opened windows in each project separately.
So opening new project will close all your windows and open windows
that were opened last time you worked with that project.

The system also considers the directory of current project its current directory.
That affects invocation of tools.
\else
Используйте команду {\bf Open}, чтобы открывать существующие файлы проекта 
с помощью стандартного файлового диалога.

IDE отдельно для каждого проекта хранит данные об открытых окнах.
Таким образом, новый проект закроет все ваши окна и откроет окна, которые 
были открыты в последний раз, когда вы работали с этим проектом.

Система также считает директорию текущего проекта текущей директорией.
Это влияет на вызов утилит.
\fi
\ifenglish
{\bf Note:} The IDE maintains a list of the most recently 
opened project files, which is shown at the bottom of the
\ref[Project menu]{M\_PROJECT}. You may use it to quickly load a project.
\else
Замечание: IDE заводит список недавно открывавшихся файлов, который отображается
в нижней части меню \ref[Project menu]{M\_PROJECT}. Вы можете использовать
его для быстрой загрузки проекта.
\fi
\subsection{Project: Close}
\public{IDM\_PROJCLOSE}

\ifenglish
Use the {\bf Close} command to close the current project 
and all windows without terminating the IDE.
\else
Используйте команду {\bf Close}, чтобы закрыть текущий проект и все
окна, не завершая работы IDE.
\fi
\subsection{Project: Modify}
\public{IDM\_PROJOPTIONS}

\ifenglish
Use the {\bf Modify} command to edit the current project file 
in a visual project editor. In this editor all project options
are combined into meaningful hierarchical groups and presented in
human-readable rather than compiler-readable form.

The visual editor can't modify projects with complicated syntax,
for instance, those with conditional clauses. In is recommended
to \ref[Edit]{IDM\_PROJEDIT} such project files in usual text editor.
\else
Используйте команду {\bf Modify}, чтобы редактировать текущий файл проекта,
используя средства графического интерфейса. В этом редакторе все опции 
проекта собраны в значащие иерархические группы и представлены в виде,
предназначенном для прочтения скорее человеком, чем компилятором.

Графический интерфейс не позволяет изменять проекты со сложным 
синтаксисом, содержащие, например, условные операторы. Такие проекты 
рекомендуется редактировать (\ref[Edit]{IDM\_PROJEDIT}) в обычном 
текстовом редакторе.
\fi
\subsection{Project: Edit}
\public{IDM\_PROJEDIT}

\ifenglish
Use the {\bf Edit} command to load the current project file 
into a regular edit window.

If you \ref[Modify]{IDM\_PROJOPTIONS} the project visually
while it is opened in regular edit window the
new changes will immediately appear in that window.
\else
Используйте команду {\bf Edit}, чтобы загружать текущий файл проекта в
обычное окно редактирования.

Если вы будете изменять (\ref[Modify]{IDM\_PROJOPTIONS}) проект, 
используя средства графического интерфейса, в то время как он открыт 
в обычном окне редактирования, то новые изменения будут отображаться 
в этом окне немедленно.
\fi
\subsection{Project: File list}
\public{IDM\_FILELIST}

\ifenglish
Use the {\bf File list} command to display a list of source
files belonging to the current project.

This list is initially empty and is
updated during each make because in languages like Modula-2 and Oberon-2 you
can only tell what files belong to the project after you build the project.

The list is written to disk and kept from session to session but it is considered
unreliable until first project make and is marked as "Not updated" because
source files could be modified outside the IDE.

It is possible to gather project file list quickly by scanning source files
for IMPORT directives without compiling them. This function is available via
the \ref[Tools]{M\_TOOLS}:Update file list option.

If you opened file list window earlier it will be opened in the same position and will have same size.

It is possible to specify that new file list window must be docked.

Keyboard shortcut: {\bf Alt+"="}
\else
Используйте команду {\bf File list}, чтобы отображать список исходных файлов,
принадлежащих текущему проекту.

Список изначально пуст, и обновляется во время каждой компоновки, так как в 
таких языках как Modula-2 и Oberon-2 вы можете определить, какие файлы 
принадлежат проекту, только после его сборки.

Список хранится на диске в промежутках между сеансами работы, но считается
недостоверным до первой компоновки и помечается как "Not updated", потому что
исходные файлы могли быть изменены вне IDE.

Существует возможность быстрого составления списка файлов проекта путем
поиска в исходных файлах директив IMPORT без их компиляции. Эта функция 
доступна посредством \ref[Tools]{M\_TOOLS}: Опция обновления списка файлов.

Если вы открывали окно списка файлов ранее, оно будет открыто на том же месте 
и будет иметь тот же размер.

Можно указать, чтобы окно списка файлов было неперекрывающимся.

Комбинация клавиш: {\bf Alt+"="}
\fi
\subsection{Project: Style sheets}
\public{M\_STYLES}
\nominitoc

\ifenglish
The {\bf Style sheets} command displays a submenu,
which contains the following commands for work with \ref[Style sheets]{StyleSheets}.
\else
Команда {\bf Style sheets} отображает подменю,
содержащее следующие команды для работы со \ref[стилями]{StyleSheets}.
\fi
\begin{description}
\item \ref[New Style]{IDM\_NEWSTYLE}
\ifenglish
  - Creates a new style file
  \else
  - Создает новый файл стиля
  \fi
\item \ref[Save project as style]{IDM\_PROJSAVESTYLE}
\ifenglish
  - Saves the current project options and equations as style
  \else
  - Сохраняет опции и параметры текущего проекта
  \fi
\item \ref[Apply style]{IDM\_APPLYSTYLE}
\ifenglish
  - Applies a style to the current project
  \else
  - Применяет стиль к текущему проекту
  \fi
\item \ref[Edit style]{IDM\_EDITSTYLE}
\ifenglish
  - Invokes the style editor
  \else
  - Вызывает редактор стиля
  \fi
\item \ref[Delete style]{IDM\_DELETESTYLE}
\ifenglish
  - Deletes a style file
  \else
  - Удаляет файл стиля
  \fi
\end{description}

\ifenglish
\subsubsection{Style sheets}
\else
\subsubsection{Стили}
\fi
\public{StyleSheets}
\ifenglish
It is often necessary to change several project options at once into some
well-known state. For example to make a project in debug mode you usually
turn on generation of debug information, turn on all run-time checks and turn
off optimizations, while for release mode you need opposite settings to all
those options. It takes long time to set all the options manually. Besides it
is possible to forget some options.

This is what the {\bf Style sheet} is for.
The style sheet is some subset of project options saved to separate file.
Applying the style sheet means copying project options from style sheet to
project file while retaining those options in a project that aren't present in
the style sheet.

You can create new empty style sheet. Use \ref[New Style]{IDM\_NEWSTYLE}
menu command.

You can create style sheet based on existing project file. Use
\ref[Save Project as Style]{IDM\_PROJSAVESTYLE}
menu command.
\else
Часто бывает необходимо изменять несколько опций сразу, задавая им вполне 
определенные значения. Например, чтобы скомпоновать проект в режиме отладки,
обычно включаются генерация отладочной информации и прогоночные проверки, 
а все оптимизации отключаются, в то время как для компоновки в режиме рабочей
версии требуются противоположные установки этих же самых опций. Установка опций
вручную требует значительного времени. Кроме того, некоторые опции можно опустить.

Вот для чего нужен {\bf стиль}.
Стиль - это некоторое подмножество опций проекта, сохраненное в отдельном файле.
Применение стиля означает копирование опций проекта из стиля в файл проекта
с сохранением установок незадействованных в стиле опций.

Вы можете создать новый пустой стиль. Используйте команду меню 
\ref[New Style]{IDM\_NEWSTYLE}.

Вы можете создать новый стиль, основываясь на существующем файле проекта 
Используйте команду меню 
\ref[Save Project as Style]{IDM\_PROJSAVESTYLE}.
\fi
\ifenglish
Once created, the style sheet can be edited in a visual editor similar to
the one used for editing projects. Use
\ref[Edit Style]{IDM\_EDITSTYLE} menu command.

Style sheets are usually kept in XDS system directory under names with
{\bf .Sty} extensions.

There are two pre-defined style sheets, {\bf Debug} and {\bf Release} with
debug and release options set.
\else
Будучи созданным, стиль может редактироваться посредством 
графического интерфейса точно также, как и проект.
Используйте команду \ref[Edit Style]{IDM\_EDITSTYLE}

Стили обычно хранятся в системной директории XDS в файлах с расширением
{\bf .Sty}.
\fi

\subsubsection{Project: Style sheets: New Style}
\public{IDM\_NEWSTYLE}

\ifenglish
Use the {\bf New style} command to create a new empty \ref[Style sheet]{StyleSheets}.
A dialog box will appear, prompting you for a 
style name, comment and file location.
\else
Используйте команду {\bf New style}, чтобы создать новый пустой
\ref[стиль]{StyleSheets}.
Появится диалог, запрашивающий имя стиля, комментарий и место файла на диске.
\fi
\subsubsection{Project: Style sheets: Save Project as style}
\public{IDM\_PROJSAVESTYLE}

\ifenglish
Use the {\bf Save Project as style} command to store the 
current project options and equations in a new or existent
\ref[Style sheet]{StyleSheets} file.
A dialog box will appear, prompting you for a 
style name, comment and file location.
\else
Используйте команду {\bf Save Project as style}, чтобы сохранять
опции и параметры текущего проекта в новый или существующий файл
\ref[стиля]{StyleSheets}.
Появится диалог, запрашивающий имя стиля, комментарий и место файла на диске.
\fi
\subsubsection{Project: Style sheets: Apply Style}
\public{IDM\_APPLYSTYLE}

\ifenglish
Use the {\bf Apply Style} command to apply a specific
\ref[Style sheet]{StyleSheets} 
to the current project, i.e. to add style options and equations 
to it. If a style option or equation already exists in the
project, it is replaced by style sheet value. Options and equations that are
not present in the style sheet are not affected.

A dialog box will appear, allowing you to select a style to apply.
\else
Используйте команду {\bf Apply Style}, чтобы применить заданный
\ref[стиль]{StyleSheets} к текущему проекту, иначе говоря, чтобы добавить
к нему опции и параметры стиля. Если опция или параметр стиля уже существуют в
проекте, то они задаются значением, указанным в проекте. Опции и параметры проекта,
не участвующие в стиле, этой командой не затрагиваются.

Будет выведен диалог, позволяющий вам выбрать стиль для применения.
\fi
\subsubsection{Project: Style sheets: Edit Style}
\public{IDM\_EDITSTYLE}

\ifenglish
Use the {\bf Edit Style} command to edit a \ref[Style sheet]{StyleSheets} using a
style visual editor (similar to project editor).
\else
Используйте команду {\bf Edit Style}, чтобы редактировать 
\ref[стиль]{StyleSheets} с помощью средств графического интерфейса.

\fi
\subsubsection{Project: Style sheets: Delete Style}
\public{IDM\_DELETESTYLE}

\ifenglish
Use the {\bf Delete style} command to permanently delete
a \ref[Style sheet]{StyleSheets} file.
\else
Используйте команду {\bf Delete style}, чтобы окончательно удалить файл
\ref[стиля]{StyleSheets}.
\fi