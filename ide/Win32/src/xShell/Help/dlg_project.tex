
%-----------------------------------------------------------------------

\ifenglish
\section{Create new project dialog}
\else
\section{диалог Create new project}
\fi
\public{NEWPROJ\_DIALOG}
\nominitoc

\ifenglish
To create a new project:
\else
Чтобы создать новый проект:
\fi
\begin{enumerate}
\item \ifenglish
      Type its full name in the \ref[Project name]{NEWPROJ\_PROJECT} field
      \else
      Введите его полное имя в поле \ref[Project name]{NEWPROJ\_PROJECT} 
      \fi
\item \ifenglish
      Type the name of a project template in the \ref[Project template]{NEWPROJ\_TEMPLATE}
      field
      \else
      Введите имя файла шаблона проекта в поле \ref[Project template]{NEWPROJ\_TEMPLATE}
      \fi
\item 
      \ifenglish
      Type a main module name in the \ref[Main module]{NEWPROJ\_STDMODULE} field
      \else
      Введите название главного модуля в поле \ref[Main module]{NEWPROJ\_STDMODULE} 
      \fi
\item \ifenglish
      Select the \ref[Create directories]{NEWPROJ\_CREATEDIR} checkbox if you want
      subdirectories for source, symbol, object, etc. files to be
      created it the project directory
      \else
      Установите контроль \ref[Create directories]{NEWPROJ\_CREATEDIR}, если хотите, чтобы 
      для исходных объектных и прочих файлов были созданы поддиректории в директории проекта.
      \fi
\item 
      \ifenglish
      Select the \ref[Create RED file]{NEWPROJ\_CREATERED} checkbox if you want
      the redirection file to be created in the project directory
      \else
      Установите контроль \ref[Create RED file]{NEWPROJ\_CREATERED}, если хотите 
      чтобы в директории проекта был создан файл путей поиска 
      \fi
\item 
      \ifenglish
      Select the \ref[OK]{NEWPROJ\_OK} pushbutton
      \else
      Нажмите кнопку \ref[OK]{NEWPROJ\_OK}
      \fi
\end{enumerate}

\ifenglish
Instead of typing a filename, you may select the corresponding
{\bf Browse} pushbutton to bring up a standard file dialog, or the
{\bf Default} pushbutton to fill an entry field with the default value.

\ref{variables} may be used in \ref[Project template]{NEWPROJ\_TEMPLATE}
and \ref[Main module]{NEWPROJ\_STDMODULE} fields.

Select the \ref[Configure]{NEWPROJ\_CONFIGURE} pushbutton to view and/or edit new
project defaults.
\else
Вместо того, чтобы вводить имя файла вручную, вы можете нажать кнопку {\bf Browse},
вызвав стандартный файловый диалог или кнопку {\bf Default}, чтобы заполнить 
поле ввода значением, присваиваемым по умолчанию.

\ref[Переменные]{variables} могут быть использованы в полях \ref[Project template]{NEWPROJ\_TEMPLATE} и 
в \ref[Main module]{NEWPROJ\_STDMODULE}.

Чтобы просмотреть и/или исправить значения, присваиваемые по умолчанию,
нажмите кнопку \ref[Configure]{NEWPROJ\_CONFIGURE}.
\fi
\begin{popup}
\caption{Project name}
\public{NEWPROJ\_PROJECT}
\public{NEWPROJ\_PROJBROWSE}

\ifenglish
Type a full name of a new project file in the entry field or select the
{\bf Browse} pushbutton to choose it using a standard file dialog.
\else
Наберите полное имя проекта в поле ввода или нажмите кнопку {\bf Browse}, 
чтобы выбрать его, используя стандартный файловый диалог.
\fi
\end{popup}

\begin{popup}
\caption{Project template}
\public{NEWPROJ\_TEMPLATE}
\public{NEWPROJ\_TEMPLBROWSE}
\public{NEWPROJ\_STDTEMPL}

\ifenglish
Type a full name of a project template file in the entry field or select the
{\bf Browse} pushbutton to choose it using a standard file dialog.

Project template is the file on which new project is based.
The project file results from project template after substituting \ref{variables} with their
values in the template.

Select the {\bf Default} pushbutton to restore the default value of the
entry field. The default template file is selected in a dialog box that
appears after pressing \ref[Configure]{NEWPROJ\_CONFIGURE} button.

\ref{variables} may be used in this field.
\else
Введите полное имя файла шаблонов проекта в поле ввода или нажмите
кнопку {\bf Browse}, чтобы выбрать его, используя стандартный файловый диалог.

Файл шаблона проекта -- это файл, на котором основывается новый проект.
Файл проекта получается из файла шаблона проекта после замещения \ref[переменных]{variables}
их шаблонными значениями.

Нажмите кнопку {\bf Default}, чтобы восстановить значение поля, принимаемое по 
умолчанию. Используемый по умолчанию шаблонный файл выбирается в диалоговом окне, 
которое появляется после нажатия кнопки \ref[Configure]{NEWPROJ\_CONFIGURE}.

В этом поле могут быть использованы \ref[переменные]{variables}.
\fi
\end{popup}


\begin{popup}
\caption{Main module}
\public{NEWPROJ\_STARTFILE}
\public{NEWPROJ\_FILEBROWSE}
\public{NEWPROJ\_STDMODULE}

\ifenglish
Type a name of a main project module in the entry field or select the
{\bf Browse} pushbutton to choose it using a standard file dialog.
Select the {\bf Default} pushbutton to restore the default value of the
entry field. The default value of this field is selected in a dialog box that
appears after pressing \ref[Configure]{NEWPROJ\_CONFIGURE} button.

\ref{variables} may be used in this field.

For example, if you mainly work in Modula-2, it is reasonable to set this field
equal to

\verb'${projname}.mod'

In this case the system will automatically create main file with the same name as the
project and extension "mod".
\else
Введите имя основного модуля проекта в поле ввода или нажмите кнопку 
{\bf Browse}, чтобы выбрать его, используя стандартный файловый диалог.
Нажмите кнопку {\bf Default}, чтобы восстановить значение поля, принимаемое 
по умолчанию. Значение поля, принимаемое по умолчанию, устанавливается в 
диалоге, который появляется после нажатия кнопки \ref[Configure]{NEWPROJ\_CONFIGURE}.

В этом поле могут быть использованы \ref[переменные]{variables}.

Например, если вы в основном работаете в Modula-2, разумно установить это поле
равным

\verb'${projname}.mod'

В этом случае система автоматически создаст основной файл с тем же именем, 
что и у проекта, и расширением "mod". 
\fi
\end{popup}

\begin{popup}
\caption{Create directories}
\public{NEWPROJ\_CREATEDIR}

\ifenglish
Select this checkbox if you want subdirectories, e.g. {\tt BIN}, {\tt SRC},
and {\tt SYM}, to be created in the project directory.

The set of directories that will be created if this option is on can be
configured in a dialog box that
appears after pressing \ref[Configure]{NEWPROJ\_CONFIGURE} button.
\else
Установите этот контроль, если хотите, чтобы в директории проекта были созданы
поддиректории {\tt BIN}, {\tt SRC}, и {\tt SYM}. 

Набор директорий, который будет создан в случае, если эта опция установлена,
настраивается в диалоге, который появляется после нажатия кнопки \ref[Configure]{NEWPROJ\_CONFIGURE}.
\fi
\end{popup}

\begin{popup}
\caption{Create RED file}
\public{NEWPROJ\_CREATERED}

\ifenglish
Select this checkbox if you want a redirection file to be created in the
project directory.
\else
Установите этот контроль, если хотите, чтобы в директории проекта был создан 
файл путей поиска.
\fi
\end{popup}

\begin{popup}
\caption{Configure}
\public{NEWPROJ\_CONFIGURE}

\ifenglish
Select the {\bf Configure} pushbutton to bring up the \ref{CFGPROJ\_DIALOG}.
\else
Нажмите кнопку {\bf Configure} чтобы вызвать диалог \ref{CFGPROJ\_DIALOG}. 
\fi
\end{popup}

\begin{popup}
\caption{OK}
\public{NEWPROJ\_OK}

\ifenglish
Select the {\bf OK} pushbutton to create a new project using the current
settings.
\else
Нажмите кнопку {\bf OK}, чтобы создать новый проект, используя текущие 
установки.
\fi
\end{popup}

\begin{popup}
\caption{Cancel}
\public{NEWPROJ\_CANCEL}

\ifenglish
Select the {\bf Cancel} pushbutton if you decided not to create a new project.
\else
Нажмите кнопку {\bf Cancel}, если решили не создавать новый проект.
\fi
\end{popup}

%-----------------------------------------------------------------------
\ifenglish
\section{Project editor}
\else
\section{Редактор проекта}
\fi
\nominitoc
\public{OPT\_DIALOG}

\ifenglish
Use the {\bf Project Editor} to view and modify contents of the current
project file or a previously created style sheet in a visual manner.

Select a project file \ref[statements category]{StatementCategories}
in the listbox. For project options, the corresponding controls
will appear to the right:
\else
Используйте {\bf Project Editor}, чтобы естественным образом просматривать 
и изменять содержимое текущего файла проекта или предварительно созданный 
стиль.          

Выберите \ref[категорию установок]{StatementCategories} проектного файла. 
Справа появятся контроли, отвечающие 
опциям проекта:
\fi
\ifenglish
\begin{tabular}{ll}
\bf Control  & \bf Represent \\
\hline
Checkbox     & a boolean option \\
Entry field  & an equation with arbitrary value \\
Combo box    & an equation with a fixed set of values
\end{tabular}
\else
\begin{tabular}{ll}
\bf Контроль           & \bf Представление \\
\hline
Переключатель          & двоичная опция \\
Поле ввода             & параметр произвольного значения \\
Комбинированный список & параметр с фиксированным набором значений
\end{tabular}
\fi
\ifenglish
{\bf Present} checkboxes indicate whether the corresponding equations
are present in the project of style being edited. Absence of a boolean
option is represented by the third (grayed) checkbox state.

When you move mouse pointer over the option control (be it edit box, combo box,
or check box) you can see the name of this option and its default state in special panel
at the bottom-right of the dialog.

For user options, user equations, and module list, a listbox containing
all of them is displayed, along with a set of pushbuttons which may be
used to add, remove, or modify its entries.
\else
Контроли показывают, присутствуют ли соответствующие параметры
в проекте редактируемого стиля. Отсутствие булевой опции обозначается
третьим (серым) состоянием контроля. 

Когда вы двигаете курсор мыши по контролю (полю редактирования, 
комбинированному списку или переключателю), то в верхнем правом углу диалога
показывается название опции и ее значение по умолчанию.

Для опций и параметров пользователя и списка модулей выводится список, 
содержащий их все, вместе с набором кнопок, которые могут быть использованы 
для их добавления, удаления или изменения.
\fi
%-----------------------------------------------------------------------

\ifenglish
\subsection{Statement categories}
\else
\subsection{Категории установок}
\fi
\public{StatementCategories}

\ifenglish
All the project file directives are organized in the set of categories.
The left area of the dialog contains list of categories.
When you select a statement category from the list, the corresponding
controls are displayed to the right, allowing you to view and edit
these statements.

Categories are organized into a hierarchic structure.
Book icon on a line indicates that this line has subcategories.
Subcategories list can be either opened (open book icon) or closed (closed book).
You can toggle open status of a subcategory list by clicking on a book icon
or double-clicking anywhere else on a line. You can also press {\bf Space} for this.
\else
Все директивы проектного файла выстроены в ряд категорий.
Левая часть диалога содержит список категорий.
Когда вы выбираете категорию установки из списка, справа отображаются 
соответствующие контроли, позволяя просматривать и редактировать эти установки.

Категории выстроены иерархически. 
Иконка-книга в строке показывает, что эта строка имеет подкатегории.
Список подкатегорий может быть либо открыт (закрытая книга), либо закрыт
(закрытая книга). Щелчок по иконке-книге открывает/закрывает ее. Для этого
можно также нажать {\bf Space} или дважды щелкнуть по любому другому месту в 
строке.
\fi
%-----------------------------------------------------------------------

\begin{popup}
\ifenglish
\caption{Statement categories}
\else
\caption{Категории установок}
\fi
\public{OPT\_LISTBOX}

\ifenglish
When you select a statement category from this list, the corresponding
controls are displayed to the right, allowing you to view and edit
these statements. Categories are organized into a tree-like structure.
Book icon on a line indicates that this line has subcategories.
Subcategories list can be either opened (open book icon) or closed (closed book).
You can toggle open status of a subcategory list by clicking on a book icon
or double-clicking anywhere else on a line. You can also press {\bf Space} for this.
\else
Когда вы выбираете категорию установок из списка, справа отображается 
соответствующий контроль, позволяя просматривать и редактировать эти установки.
Категории выстроены в древовидную структуру. Иконка-книга в строке показывает,
что эта категория имеет подкатегории. Список подкатегорий может быть либо открыт
(открытая книга), либо закрыт (закрытая книга). Щелчок по иконке-книге 
открывает/закрывает ее. Для этого можно также нажать {\bf Space} или дважды 
щелкнуть по любому другому месту в строке.
\fi
\end{popup}

\begin{popup}
\public{OPT\_FRAME}
\ifenglish
This is a frame page.
\else
Это - каркасная страница.
\fi
\end{popup}


\begin{popup}
\caption{Present}
\public{OPT\_PRESENT}
\ifenglish
{\bf Present} checkboxes indicate whether the corresponding equations
are present in the project of style being edited.
Options that are not present in the project have default values defined by
XDS compiler configuration file ({\bf xc.cfg} file for native-code compiler).
\else
Контроли {\bf Present} показывают, присутствуют ли в редактируемом стиле 
проекта соответсвующие параметры.
Опции, не присутствующие в проекте, задаются значениями, принимаемыми по 
умолчанию, указанными в файле настроек XDS-компилятора. 
(файл {\bf xc.cfg} для native-code компилятора).
\fi
\end{popup}

\begin{popup}
\caption{Restore page}
\public{OPT\_UNDOPAGE}

\ifenglish
Select the {\bf Restore page} pushbutton if you want statements from
the currently displayed category to be set to values they have before
you invoke the {\bf Project Editor}.
\else
Нажмите кнопку {\bf Restore page}, если хотите, чтобы установки из отображенной 
в данный момент категории приняли значения, которые они имели до того, как вы 
вызвали {\bf Project Editor}.
\fi
\end{popup}

\begin{popup}
\caption{Default value}
\public{OPT\_DEFPANE}

\ifenglish
The name of the option or equation corresponding to the control under
the mouse cursor is displayed in this area along with its default value.
\else
Название опции или параметра, соответствующих контролю, находящемуся под 
курсором мыши, отображаются в этой области вместе с их значением по умолчанию.
\fi
\end{popup}

\begin{popup}
\caption{Option value}
\public{OPT\_CHECK}

\ifenglish
Select or clear this checkbox to toggle the currently selected user option
setting.
\else
Включите или выключите этот контроль, чтобы включить/выключить установку 
текущей опции пользователя.
\fi
\end{popup}

\begin{popup}
\caption{Add}
\public{OPT\_ADDOPTION}

\ifenglish
Select the {\bf Add} pushbutton to add a new item to the list.
\else
Нажмите кнопку {\bf Add}, чтобы добавить в список новый элемент.
\fi
\end{popup}

\begin{popup}
\caption{Remove}
\public{OPT\_REMOVEOPTION}

\ifenglish
Select the {\bf Remove} pushbutton to remove the currently selected item
from the list.
\else
Нажмите кнопку {\bf Remove}, чтобы удалить выбранный элемент списка.
\fi
\end{popup}

\begin{popup}
\caption{Item list}
\public{OPT\_OPTLIST}

\ifenglish
This list contains user options, user equations, or modules from
the current project.
\else
Этот список содержит опции пользователя, параметры пользователя или модули
текущего проекта.
\fi
\end{popup}

\begin{popup}
\caption{OPT\_COMMENT}
\public{OPT\_COMMENT}

\ifenglish
This help topic is not available yet % !!!
\else
Этот раздел подзказки пока не доступен % !!!
\fi
\end{popup}

\begin{popup}
\ifenglish
\caption{Equation value}
\else
\caption{Значение параметра}
\fi
\public{OPT\_EQUEDIT}

\ifenglish
Type a new value for the currently selected user equation in this entry field.
\else
Наберите новое значение выбранного параметра пользователя в этом поле ввода.
\fi
\end{popup}

\begin{popup}
\caption{Browse}
\public{OPT\_BROWSE}

\ifenglish
Select the {\bf Browse} pushbutton to select a module using a standard file
dialog.
\else
Нажмите кнопку {\bf Browse}, чтобы выбрать модуль, используя стандартный 
файловый диалог.
\fi
\end{popup}

\begin{popup}
\caption{Modify}
\public{OPT\_SET}

\ifenglish
Select the {\bf Modify} pushbutton to replace the currently selected module
with the contents of the module name entry field.
\else
Нажмите кнопку {\bf Modify}, чтобы заменить выбранный модуль содержимым поля 
ввода имени модуля.
\fi
\end{popup}

\begin{popup}
\caption{OK}
\public{OPT\_OK}

\ifenglish
Select the {\bf OK} pushbutton to confirm the changes you've made to the
project file and close the {\bf Project Editor}.
\else
Нажмите кнопку {\bf OK}, чтобы подтвердить изменения, произведенные с файлом 
проекта и закрыть {\bf Project Editor}.
\fi
\end{popup}

\begin{popup}
\caption{Cancel}
\public{OPT\_CANCEL}

\ifenglish
Select the {\bf Cancel} pushbutton if you decided not to perform any
changes to the project file.
\else
Нажмите кнопку {\bf Cancel}, если решили не вносить изменения в файл проекта.
\fi
\end{popup}
